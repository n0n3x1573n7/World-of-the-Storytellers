\documentclass{report}

\begin{document}
	\world{작가는 이야기꾼의 세계 내에서 자신의 이야기를 찾을 수 있을 것이라는 사실을 알아야 합니다. 작가는 그 이야기의 지배권을 가지고 있어, 그 세계의 [진리]를 비틀고, [운명]을 조작할 수 있습니다. 이들은 자신의 이야기에 대해서는 저 이상의 권한을 가지고 있습니다.}
	\world{그렇다고 해서 이 권한이 이야기에 어떤 짓을 하더라도 유지되는 것은 아닙니다. 이야기가 알아볼 수 없을 정도로 손상되거나 오염되면 이야기가 작가의 지배를 거부하여 지배권을 잃게 될 수도 있는 것이죠.}
	\world{다른 이야기 속에서는 자신의 이야기의 [가호]를 받을 수 있습니다. 다른 이야기의 법칙을 자신의 이야기의 법칙을 이용해서 조금 구부릴 수 있게 되는 것이죠. 물론 일시적이고 효과가 미미하긴 하지만, 작가들의 대부분은 이야기를 이끌어나가는 뛰어난 이야기꾼의 기질을 가지고 있기 때문에, 이 미약한 구부림이 필요한 도움의 전부일 때가 많습니다.}
	
	이야기꾼의 세계에서 작가는 특별한 위치에 있습니다. 작가는 자신만의 이야기를 가지고 있는 존재로, 굳이 소설을 쓰거나 하지 않았더라도 자신이 창작한 세계가 있다면 이에 해당합니다.
	작가는 태초의 이야기에 들어왔을 때, 자신의 이야기에 대한 세계의 존재를 알게 됩니다. 이는 [설화] 이야기로, 반드시 능력을 가지고 있습니다. 이 이야기의 개연성 코스트는 별도 능력이 없는 한 기본(10)으로 취급됩니다.
	
	시스템 이상의 권한을 가질 수 있는 유일한 방법은 한 세계의 작가가 되는 것입니다. 작가는 자신이 만들어낸 세계에 한정하여 시스템 이상의 권능을 가집니다. 이야기의 미래를 계산하는 것 뿐 아니라 방향, 설정을 자신이 주무를 수 있게 되는거죠. 자신이 창조한 세계에 들어간 작가는 굳이 그 이야기의 역할 이야기를 얻거나 할 필요도 없이 세계의 법칙이 허용하는 한 모든 것을 할 수 있습니다.
	
	하지만, 작가가 이야기를 오염시키거나 파괴시켰을 때에는 조금 상황이 달라집니다. 이야기를 만든 이는는 작가더라도 그를 받아들이는 독자가 반드시 존재합니다. 독자들로 대표될 수 있는 비자명한 이들은 이야기 세계의 주민들입니다. 그 독자들이 보았을 때(즉, 개연성상) 이 작가가 추가/수정한 설정이 이야기를 너무 크게 바꾸면 작가가 이야기를 오염시킨 것으로 취급하고, 이야기의 진행 등이 너무 크게 바뀔 수 있으면 이야기가 손상/파괴된 것으로 취급합니다.
	작가 본인에 의하여 오염되거나 손상/파괴된 이야기는 작가의 권능을 거부합니다. 즉, 작가는 그 이야기에 대한 통제 권한을 잃게 되는 것이죠. J.K.롤링의 경우가 이에 해당합니다.
	
	다른 이야기 속에서는, 자신의 이야기의 “가호”를 받을 수 있습니다. 이 “가호”는 이 [설화] 이야기의 능력으로, 개연성 비용이 존재하지 않고, 이야기의 법칙을 구부릴 수 있습니다. 다만 이 능력은 잠시동안만 지속되거나, 횟수 제한이 존재하건, 효과가 미미한 등 매우 일시적이거나 미약한 효과를 제공합니다.

\end{document}