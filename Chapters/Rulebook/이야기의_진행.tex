\documentclass{report}

\begin{document}
	
	이야기는 크게 다음 네 가지 방식으로 진행할 수 있습니다.
	
	\section*{등장인물 vs. 이야기}
	\textbf{설명}: 다른 RPG들과 유사한 방식입니다. 이야기 속의 등장인물이 되어 이야기 속의 사건들을 헤쳐나가는 방식으로 진행됩니다.
	
	\textbf{변경점}: 시스템으로부터 받은 [권능]과 [제약]이 없다는 점이 다른 방식과의 가장 큰 차이점입니다. 한 세계관에 맞추어 캐릭터를 만들어야 하고, 침범 판정이 존재하지 않는 대신, 사망 등에 의한 페널티가 커지게 되어, 캐릭터가 회생 불능에 빠지게 될 수도 있습니다.
	
	\section*{이야기꾼 vs. 이야기}
	\textbf{설명}: 어떤 이야기 속으로 이야기꾼이 진입하여 그 세계 속의 문제 등을 해결하거나 회피하며 생존하는 것을 목표로 합니다.
	
	\textbf{변경점}: 사망 후 이야기에 재진입할 때의 공포증, 역할의 영구 소실, 역할 강제 변경 등의 특수한 페널티를 부여하며, 일부 상황에서는 재진입이 불가능할 수 있습니다.
	
	\section*{이야기꾼 vs. 타락한 자}
	\textbf{설명}: 어떠한 이야기, 특히 유명한 이야기 등을 타락한 자가 뒤트는 것을 바로잡는 이야기꾼들에 대한 이야기입니다.
	
	아래의 모든 룰은 이 진행 방식을 기준으로 서술되어 있습니다.
	
	\section*{타락한 자 vs. 시스템}
	\textbf{설명}: 이야기꾼이 타락한 자가 되어 시스템과 이야기의 운명의 방해로부터 이야기를 자신이 원하는 방향으로 이끌어나가고자 하는 이야기입니다.
	
	\textbf{변경점}: 이야기꾼의 [권능], 특히 [접근]의 권능이 제약을 받습니다. 그렇기 때문에 해당 이야기로부터 추방당한 경우, 재진입이 불가능한 경우가 대부분입니다.
	
	\bigskip
	
	여러 이야기 속을 여행하는 이야기꾼의 경우 여러 가지 진행 방식으로 이야기를 풀어나갈 수 있습니다. 가령,
	\begin{itemize}
		\item 이야기꾼이 배경 이야기 속에서 등장인물로서 깨달은 자가 되어가는 과정을 \emph{이야기 vs. 등장인물}로 진행한 뒤,
		\item 시스템으로부터 받은 첫 임무로 다른 이야기 속에서 이야기의 활용에 익숙해져가는 \emph{이야기 vs. 이야기꾼}을,
		\item 이야기의 활용에 익숙해지면 시스템을 도와 타락한 자를 잡고 징벌하는 \emph{이야기꾼 vs. 타락한 자}의 진행을,
		\item 자신이 필요한 이야기를 만들어 나가기 위해 시스템의 권고를 무시하고 이야기의 수정을 강행하는 \emph{타락한 자 vs. 시스템}의 진행을
	\end{itemize}
	모두 다양하게 활용할 수 있을 것입니다.
	
	\bigskip
	
	이야기는 여러개의 연속된 씬(Scene)으로 구성되어 있습니다. 씬이란, 이야기가 진행되면서 일어나는 사건들을 이야기꾼들이 해결해나가고자 노력하는 이야기의 구성 단위입니다. 이는 한 장소를 조사하거나, 한번의 전투를 치르는 등 뭔가 이야기꾼들이 이야기 속에서 작은 목표를 달성하기 위해서 노력하는 최소의 단위로 볼 수 있습니다. 소설이나 영화, 드라마 등에서의 씬의 개념과 크게 벗어나지 않습니다. 이들도 결국은 이야기니까요. 어디서부터 어디까지를 씬으로 정할지 모르겠다면, 뭔가 이야기 내에서 비교적 큰 전환점이 되는 곳을 씬의 전환으로 생각해도 좋습니다. 가장 대표적으로 전투의 시작과 종료, 장소의 이동 등을 들 수 있겠죠.

\end{document}