\documentclass{report}

\begin{document}
	이야기꾼들은 여러가지 정신적인 후유증을 가지고 살아갑니다. [기피], [공포], [집착], [중독], [광기]가 바로 그것입니다. [기피]가 심화되면 [공포]가, [공포]가 심화되면 [광기]가 됩니다. 중독증이나 집착증 등은 트라우마로 보기 어려우나, 트라우마와 동일하게 취급됩니다. 이 경우 [기피], [공포] 단계 대신 [집착], [중독] 단계를 사용합니다. [광기] 단계는 유지됩니다. 어떤 후유증은 일시적일수도 있고, 직접 해소될 방법이 있을 수도 있습니다.
	
	이들은 기본적으로 다음과 같은 형식으로 이루어집니다:
	
	\begin{itemize}
		\item \textbf{트리거}
		\subitem 트라우마를 발동시킬 조건입니다. 이 조건을 충족시키면, 효과가 발동됩니다.
		
		\item \textbf{효과}
		\subitem 트라우마로 인해 발동되는 효과입니다. 이 효과에 대한 예시로는 다음이 있습니다:
		\begin{itemize}
			\item 일정 시간동안 특정 부정적인 상태를 얻습니다.
			\item (피아 구분 없이) 무작위 대상을 공격합니다.
			\item 아군을 공격합니다.
			\item 피아의 구분을 할 수 없게 됩니다.
			\item 주변 상황이 어떻든 상관없이 트리거를 없애고자 합니다.
			\item 일정 시간동안 공격을 방어하지 못합니다.
			\item 그 상황에서 어떻게든 빠져나가려고 합니다.
		\end{itemize}
	\end{itemize}
	
	캐릭터의 행동을 제약시키는 조건만 존재하는 트라우마 역시 가능하며, 트리거/효과와 함께 제약이 같이 존재할 수 있습니다. 제약만 있는 트라우마의 경우, 제약 자체가 트리거와 효과의 역할을 함께 해야 한다는 점을 고려해 제약을 정하시면 됩니다.

\end{document}