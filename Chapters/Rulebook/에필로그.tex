\documentclass{report}

\begin{document}
	\world{짧다면 짧고, 길다면 길었을 이곳 [태초의 이야기]와 서사들, 그리고 여러분 이야기꾼에 대한 설명이 끝났습니다.}
	\world{여러분에게 저를 도와달라고 강요하고 싶지는 않습니다. [태초의 이야기]의 관리자로서 여러분이 원하는 서사 속으로 뛰어들어 서사를 해치지 않는 선에서 모험을 즐기는 것을 보는 것이 제 재미 중 하나니까요.}
	\world{하지만 이 [태초의 이야기]의 서사들이 오염되는 것을 막기 위해서는 여러분의 도움이 필요합니다.}
	\world{도와주시겠습니까?}
	
	이야기꾼의 세계는 "Fast Startup, Small Hiccup"\footnote{해석하자면, "세팅은 빠르게, 걸림돌은 작게"}를 목표로 만들어진 룰입니다.
	
	"Fast Startup"은 이야기꾼의 컨셉을 잡은 이후에는 빠르게 이야기꾼을 만들고 디자인할 수 있도록 능력에 대한 자유도를 크게 준 것으로, "Small Hiccup"은 서사를 진행하는 중 룰적인 문제가 발생하지 않도록, 발생하더라도 고치기 쉽도록 룰의 자유도를 높게 설정한 것과, 주사위로 인해 서사 안에서의 개연성적인 문제가 발생할 확률이 퍼지 다이스로 인해 낮아지도록 한 것으로 반영하고자 했습니다. 아직 비교적 새로 만들어진 룰이기 때문에 얼마나 이를 반영하고 있는지는 알 수 없지만, 앞으로 룰을 추가, 제거, 수정함에 있어 이를 최대한 반영할 수 있는 방향으로 수정하고자 노력할 것입니다.
	
	이야기꾼 여러분, 길고 지루한 룰 읽느라고 수고하셨습니다. 이제, 여러분의 이야기꾼과 함께 서사 속으로 뛰어들 때입니다.
	

\end{document}