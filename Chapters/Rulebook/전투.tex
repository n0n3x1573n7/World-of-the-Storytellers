\documentclass{report}

\begin{document}
	전투 상황에서는, 기민 스탯이 더 높은 이야기꾼이 먼저 턴을 가집니다. 한 턴은 5초로 계산하므로, 간단한 행동 한가지 정도만을 할 수 있습니다. 각 턴은 다음과 같이 진행됩니다:
	\begin{itemize}
		\item \textbf{결정}: 행할 행동을 결정합니다. 해당 행동에 가지고 있는 이야기 중 도움이 되는 이야기들과 능력의 도움을 받을 수 있습니다.
		\item \textbf{반응}: 주변의 이야기꾼들이 이 행동을 돕거나 방해할 수 있습니다.
		\begin{itemize}
			\item \textbf{행동의 대상}: 행동의 대상은 자신의 모든 이야기와 능력을 활용하여 해당 행동을 돕거나 방해할 수 있습니다.
			\item \textbf{나머지 이야기꾼}: 행동의 대상이 아닌 이야기꾼은 자신의 능력들은 모두 활용할 수 있으나, 자신의 이야기를 최대 한개까지만 활용하여 해당 행동을 돕거나 방해할 수 있습니다. 단, 이 이야기로 인하여 자동 성공을 결과로 가지게 할 수 없으며, 다음 자신의 턴의 행동이 완료되기 전까지는 돕는데 사용한 이야기의 도움은 받지 못합니다.
		\end{itemize}
		\item \textbf{결과}: 해당 행동의 결과를 판정합니다.
	\end{itemize}
	기본적으로는 도와준 이야기의 규모에서 방해한 이야기의 규모를 뺀 값의 피해만큼을 행동의 대상에게 줍니다. 단, 가하는 피해를 2 줄이고, 무기를 떨어뜨리도록 만들거나, 넘어뜨리거나 하는 등 상대에게 부정적인 상황 하나를 만들기로 결정하거나, 1회 사용 가능한 부정적인 상태를 주도록 선택할 수 있습니다. 이 상황과 상태는 공격자가 결정하며, 한 공격에 피해량이 충분하다면 여러번 사용할 수 있습니다.
	
	피해를 받을 때에는 먼저 물리 피해의 경우 체력, 정신 피해의 경우에는 정신력에 피해를 받습니다. 만약 체력 또는 정신력이 0인 경우, 해당 수치만큼이 개연성에서 감해집니다.
	체력과 정신력은 씬이 종료될 때 마다 회복됩니다. 하지만 개연성은 회복되지 않습니다. 다만, 체력 또는 정신력을 회복시키는 능력에 의해서는 각각 해당 수치가 최대치인 경우 개연성이 회복될 수 있습니다. 이 이외의 방법으로 개연성을 회복시킬 수 있는 방법은 이야기에서 추방되었다가 다시 진입하여 개연성을 초기화하는 방법뿐입니다.
	개연성이 0이 되면, 이야기에서 추방됩니다. 자세한 내용은 이야기의 [침범] 판정(\ref{침범_판정}) 부분을 확인하세요.
	
	\section*{지형}
	지형은 “구역”으로 구분되어 있습니다. 전투 상황에서 이동을 할 때에는 한 턴에 한 구역까지 이동할 수 있으며, 기본적으로 같은 구역 내에 있는 캐릭터 끼리는 서로 주먹, 칼 등의 근접 무기로 공격을 할 수 있습니다. 구역별로 특이사항이나 상태, 이야기를 가지고 있을 수 있습니다
	

\end{document}