\documentclass{report}

\begin{document}
	이야기꾼은 항상 변화하고, 성장합니다. 서사 속에서 죽음을 맞기도 하고, 서사 속에서 새로운 이야기를 얻으며 성장하기도 합니다.
	
	\section*{죽음}
	이야기꾼의 개연성이 0이 되면 해당 서사에서 죽음을 맞이하여, 추방됩니다. 해당 시점에서 다음이 모두 일어납니다:
	\begin{itemize}
		\item 죽은 원인 또는 상황에 대한 [기피]. 이미 유사한 [기피]나 [공포]가 있다면 해당 효과가 각각 [공포] 또는 [광기]로 심화됩니다. [광기]가 이미 있는 경우라 할지라도 관련 있는 새로운 트라우마를 얻거나 [광기]의 효과가 더 쉽게 트리거 되거나, 효과가 심화되는 등 페널티를 반드시 받습니다.
		\item 서사의 역할으로 인해 얻은 이야기를 모두 잃습니다.
		\item 받은 모든 임시 상태를 잃습니다.
		\item {}[태초의 이야기]에서 다시 나타납니다. 해당 씬이 끝난 뒤에, 재진입이 가능합니다.
	\end{itemize}
	
	\section*{미미한 성장}
	이야기꾼이 전투를 겪었거나, 이야기의 변화를 겪었다면 일어납니다. 시스템의 허가 하에 개연성을 소모하고 방금 있었던 전투나 변화에 어울리는 이야기 하나를 얻거나, 이미 존재하는 이야기 하나를 바꿀 수 있습니다. 미미한 성장으로 능력을 변화시키거나, 이야기를 잃을 수는 없습니다.
	
	\section*{작은 성장}
	서사 하나가 끝날 때 마다 이야기에서 얻는 보상과는 별개로 다음 중 하나를 할 수 있습니다:
	\begin{itemize}
		\item 개연성을 소모하고, 이야기를 하나 얻습니다.
		\item 개연성을 소모하고, 이미 있는 이야기에 대한 기술을 하나 얻습니다.
		\item 이야기 하나를 다른 이야기로 바꿉니다.
	\end{itemize}
	작은 성장과는 별개로, 서사에서 얻는 보상에 대한 가이드라인은 능력 가이드라인 챕터의 "서사의 보상"(\ref{서사_보상})을 참고하세요.
	
	\section*{중간 성장}
	시스템의 인정을 받는다면(보통 서사 두세개가 끝날때마다 한번) 다음을 모두 할 수 있습니다:
	\begin{itemize}
		\item 기본 최대 개연성 10을 얻습니다.
		\item 작은 성장을 합니다.\footnote{\label{medium-upgrade-small-upgrade}오타 아닙니다. 작은 성장의 선택지 중 한 가지를 선택해서 적용시키는 것을 총 두번 할 수 있습니다.}
		\item 작은 성장을 합니다.\footnoteref{medium-upgrade-small-upgrade}
	\end{itemize}
	
	\section*{큰 성장}
	시스템의 위기를 타파할 때 마다(보통 서사 대여섯개정도가 끝날때마다 한번) 다음을 모두 할 수 있습니다:
	\begin{itemize}
		\item 기본 최대 개연성 10을 얻습니다.\footnote{\label{big-upgrade-cost}즉, 기본 최대 개연성 20을 얻습니다.}
		\item 이야기와 그에 대한 기술을 하나 얻습니다.
		\item 중간 성장을 합니다.\footnoteref{big-upgrade-cost}
		\item 작은 성장을 합니다.
	\end{itemize}
	

\end{document}