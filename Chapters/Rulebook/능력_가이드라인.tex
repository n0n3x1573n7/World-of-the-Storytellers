\documentclass{report}

\begin{document}
	이야기에 따른 능력은 다양한 종류가 있을 수 있습니다.
	캐릭터의 서사에 따라서, 강력한 능력은 마나와 같은 자원을 소모할수도, 사용 대기 시간이 존재할수도, 시전 시간이 존재할수도, 사용 횟수 제한이 존재할수도, 까다로운 사용 조건이 존재할수도 있습니다. 하지만 모두 공통적으로, 다른 서사에 들어가면서 능력이 어느 정도 약화됩니다.
	이 장에서는 능력이 할 수 있는 일을 제시하고, 그 효과에 대한 범위에 대해 생각합니다.
	
	능력은 크게 [효과형], [활용형]으로 구분됩니다. 이 구분은 절대적이지 않습니다.
	
	[효과형] 능력은 아군 또는 적군에게 직접적으로 효과를 부여하는 등의 효과를 가진 능력입니다. 크게 [공격형]과 [지원형]으로 구분됩니다.
	[공격형] 능력은 상대에게 피해를 주거나, 기절 등 행동에 직접적인 제약을 주는 부정적인 상태를 주는 능력을 의미합니다.
	[지원형] 능력은 아군의 피해를 막거나 치유하고, 아군에게 도움이 되는 상태를, 적에게는 직접적인 제약이 되지는 않으나 능력 사용을 약화시키는 능력을 의미합니다.
	[효과형] 능력은 여러가지 속성을 가질 수 있습니다. 이 속성들은 능력의 성능과 개연성 비용에 직접적인 영향을 끼칩니다.
	
	아래의 수치는 성능에 따른 대략적인 개연성 비용입니다:
	%개연성 비용	 밸런스 패치
	\begin{itemize}
		\item 피해
		\begin{itemize}
			\item \textbf{직접 피해(Direct Damage)}: 피해량 1이 증가할때마다 개연성 비용 3이 추가됩니다.
			\item \textbf{지속 피해(Damage over Time)}: 턴당 피해량이 1 증가할때마다 개연성 비용 2가 추가됩니다. 지속시간 1턴마다 개연성 비용 2가 추가됩니다. 단, 최대 줄 수 있는 총 피해량의 세 배보다 이 코스트가 적다면 최대로 줄 수 있는 총 피해량의 세 배의 코스트를 가집니다.
		\end{itemize}
		
		\item 회복/방지
		\begin{itemize}
			\item \textbf{피해의 방지(Prevention)}: 최대로 방지할 수 있는 피해량 1당 개연성 비용 2가 추가됩니다.
			\item \textbf{피해의 회복(Healing)}: 최대로 치유할 수 있는 피해량 1당 개연성 비용 4가 추가됩니다.
			\item \textbf{지속 회복(Healing over time)}: 턴당 회복량이 1 증가할때마다 개연성 비용 2가 추가됩니다. 지속시간 1턴마다 개연성 비용 2가 추가됩니다. 5턴 이상은 추가되지 않습니다.
		\end{itemize}
		
		\item 사거리
		\begin{itemize}
			\item 아래의 내용을 계산할 때, 구역의 배치를 한 변이 10m인 무한 격자형 구역으로 생각합니다.
			\item \textbf{사거리(Ranged)}: 최대 사거리까지 떨어진 구역 수당 개연성 비용 1이 증가합니다. 5 이상으로 증가하지 않습니다.
			\item \textbf{범위(Area of Effect)}: 범위에 포함되는 구역당 개연성 비용 2가 증가합니다. 단, 아군 피해(Friendly Fire) 혹은 적군 회복 등 의도치 않은 결과를 불러올 수 있는 능력은 이 효과를 받지 않습니다.
			\item \textbf{회피 가능성(Dodgeability)}
				\begin{itemize}
					\item \textbf{조준(Aim)}: 조준이 필요하다면 개연성 비용 4가 감소합니다.
					\item \textbf{경감 가능(Reducable)}: 피해의 경감이 가능하다면 개연성 비용 2가 감소합니다.
					\item \textbf{회피 가능(Dodgeable)}: 회피가 가능하다면 개연성 비용 4가 감소합니다.
					\item \textbf{대상 지정(Targeted)/회피 불가(Undodgeable)}: 조준이 불필요하며 회피가 불가능하다면 개연성 비용 3이 증가합니다.
				\end{itemize}
			\item \textbf{거리에 따른 피해량 감소(Damage Falloff)}: 한 구역당 감소되는 피해량이 1 증가할때마다 개연성 비용 1이 감소합니다.
		\end{itemize}
		
		\item 사용 빈도
		\begin{itemize}
			\item \textbf{자원(Resource)}: 자원을 사용하는 경우 개연성 비용 3이 감소합니다. 추가로, 자원이 자동으로 충전되지 않는 등 자원을 찾기 어렵다면 개연성 비용 2가 감소합니다.
			\item \textbf{횟수 제한(Limited Use)}: 개연성 비용 10이 감소합니다. 씬당 사용 가능 횟수 1회당 개연성 비용 1이 증가합니다. 씬당 10회 이상 사용할 수 있는 능력은 횟수제한이 없는 것으로 간주합니다.
			\item \textbf{조건(Condition)}: 조건의 달성 수준에 따라 개연성 비용 1\textasciitilde10이 감소합니다.
			\item \textbf{준비 시간(Cooldown)}: 기술의 준비 시간 1턴당 개연성 비용 1이 감소합니다.
			\item \textbf{충전(Charge time)}: 기술의 충전 시간 1턴당 개연성 비용 1이 감소합니다. 만약 충전 시간동안 다른 행동에 제약을 받는다면, 1씩이 추가로 감소합니다.
			\item \textbf{다중 사용(Multiple Usage)}: 한 턴에 여러 번 사용 가능하다면, 기본 1회에서 추가되는 횟수당 개연성 비용 2가 추가됩니다.
			\item \textbf{중첩 사용과 복수 사용(Cumulative/Parallel Usage)}: 지속 효과형 능력에만 적용됩니다. 한 대상에게 중첩 사용이 가능하다면 개연성 비용 3이 추가됩니다. 여러 대상에게 동시에 다중 사용이 가능하다면 개연성 비용 3이 추가됩니다.
		\end{itemize}
		
		\item 효과
		\begin{itemize}
			\item \textbf{강제 이동(Forced Movement)}: 강제로 한 구역 이내에서 이동시킬 수 있다면 개연성 비용 2가 추가됩니다. 이 이상 이동시킬 수 있는 최대 구역당 개연성 비용 2가 추가됩니다. 만약 능력을 이용함으로써 자신이 강제로 이동되는 경우 이로 인해 추가되는 개연성 비용이 반으로 감소합니다.
			\item \textbf{긍정적/부정적 상태(Status Effects)}: 상태를 줄 수 있는 경우 기본적으로 개연성 비용 10이 증가합니다. 아래 사항에 해당한다면 개연성 비용이 감소합니다:
			\begin{itemize}
				\item 상태가 극복하기 쉬운 경우, 개연성 비용을 3 감합니다.
				\item 상태가 횟수 제한이 있는 경우, 5회에서 모자란 횟수당 1의 개연성 비용을 감합니다. 이 이상의 횟수인 경우 횟수 제한이 있는 것으로 취급하지 않습니다.
				\item 상태가 턴수 제한이 있는 경우, 4턴에서 모자란 턴수당 1의 개연성 비용을 감합니다. 이 이상의 턴수인 경우 턴수 제한이 있는 것으로 취급하지 않습니다.
			\end{itemize}
		\end{itemize}
		
		\item 트라우마
		\begin{itemize}
			\item {}[기피], [집착]급 트라우마 효과당 개연성 비용 10이 감소합니다.
			\item {}[공포], [중독]급 트라우마 효과당 개연성 비용 20이 감소합니다.
			\item {}[광기]급 트라우마 효과당 개연성 비용 30이 감소합니다.
		\end{itemize}
	\end{itemize}
	
	이런 능력의 효과를 결정할 때에, 최대 체력이나 최대 정신력 수치가 중요하게 작용할 것입니다. 기본적으로 양쪽 모두 10으로 시작하며, 최대 체력은 근력, 최대 정신력은 의지의 영향을 받습니다. 능력의 효과를 결정할 때, 평균적인 최대치를 10\textasciitilde20정도로 생각하시면 됩니다. 이 이상의 최대치를 가지고 있다면 탱커로 분류될 수 있을 것이고, 이 수치의 50\% 이상의 피해를 한번에 줄 수 있다면 궁극기에 해당하는 효과로 분류될 수 있을 것입니다.
	
	한 가지 능력은 여러가지 형태의 능력이 될 수 있습니다. 이러한 선택형 능력의 경우에는 모든 선택지가 한가지 테마로 연결되어야 한다는 제약이 존재하며, 개연성 비용의 계산이 다르게 적용됩니다. 이런 능력은 먼저 효과 각각에 대한 개연성 비용을 계산한 뒤, 최대로 선택 가능한 총 개연성 비용을 올림한 것이 비용이 됩니다.
	
	[활용형] 능력은 적을 공격하거나 아군을 지원하는 것 외의 다른 방향으로 활용할 수 있는 모든 능력을 의미합니다. 가령 순간이동이라던가, 공격에 사용하기는 어려우나 어딘가 쓸모있는 마법 물품 등이 여기에 포함됩니다. 이 능력들은 대부분의 경우 전투중에는 특정한 조건이 충족되지 않는다면 사용할 수 없습니다. [효과형] 능력을 직접적으로 보조하는 능력이라 할지라도 그 성능은 다양할 수 있으므로 그 성능을 통한 시스템의 판단에 따라 개연성 비용을 결정합니다. 기본적으로 추가 비용은 10으로 계산하나, [효과형] 능력에 해당하는 사항을 가지고 있다면 그에 해당하는 만큼 비용이 변할 수 있으며, 세션 진행 중 또는 후 그 효용성에 따라 시스템의 판단에 따라 개연성 비용이 변할 수 있습니다.
	
	한 가지 이야기가 [효과형]과 [활용형] 능력을 동시에 가질 수도 있고, 여러 [활용형] 능력이 섞여있을 수도 있습니다. 한 이야기가 가질 수 있는 능력의 종류와 수, 그리고 이야기의 코스트에는 제한이 없으나 코스트는 각 능력의 코스트의 합으로 계산되며 코스트가 큰 이야기일수록 침범 판정에 불리해진다는 것을 기억하세요.
	
	\bigskip
	
	다음과 같은 능력을 가진 이야기를 생각해봅시다:
	\begin{story}{독으로 만들어진 신체}{[생애][기피]}
		\entry{자신에게 접촉한 유기체 생명체인 모든 대상에게 물리적 상태 [중독됨 □□□]을 준다. 상태 [중독됨]을 가진 대상은 매 턴이 시작될 때, 상태 한 칸을 소모하고 개연성(등장인물이라면 체력) 1을 소모한다.}
		
		\entry{이 이외의 대상에 접촉했을 때에는 독을 묻힐 수 있다. 다음 중 하나를 선택적으로 적용한다:
		\begin{itemize}
			\item 물체를 즉시 부식시킨다. 해당 물체에 내구도가 있는 경우 내구도를 감소시킬 수 있다.
			\item 자신이 만졌던 곳에 다른 대상이 접촉한 경우, 해당 대상에게 접촉한 것처럼 취급한다.
		\end{itemize}
		두 가지 선택지 중 반드시 한가지만을 적용시킬 수 있다.}
		
		\limitedtrauma[\hline]{기피}{호의적인 타인에게 고의로 직간접적 신체 접촉을 할 수 없다.}
	\end{story}
	이 이야기의 경우, 기본 비용 10에 다음과 같이 개연성 비용을 계산할 수 있습니다:
	\begin{itemize}
		\item {}[중독됨]: 부정적 상태(Status Effect) 3회, 개연성 비용 +8
		\item 지속 피해: 3턴간 지속 피해 1\footnote{코스트 3턴*2+피해량 1*2=8}, 총 피해량 3\footnote{코스트 9} 개연성 비용 +9
		\item 선택형 능력: 양쪽 모두 [활용형]으로 각각 개연성 비용 +10, 따라서 개연성 비용 +10
		\item 기피증: 개연성 비용 -10
	\end{itemize}
	따라서 개연성 비용은 27, 침범도는 3\footnote{27/10을 올림한 값.}이 됩니다. 하지만 시스템의 판단에 따라 다음에서 개연성 비용이 변할 수도 있습니다:
	\begin{itemize}
		\item 부정적 상태의 부여가 지속 피해와 겹치므로 둘 중 하나의 개연성 비용 증가만을 적용시킨다.
		\item {}[효과형] 능력의 효과가 미미한 정도에 그치므로 개연성 비용을 감소시킨다.
	\end{itemize}
	
	\section*{이야기의 기본 비용}
	이야기의 기본 개연성 비용은 10입니다. 하지만, 특정 이야기들은 다른 이야기들과 연계되는 능력이 주가 되거나, 다른 이야기를 보조하는 능력이기도 하고, 또는 다른 이야기가 있어야만 사용할 수 있는 능력이기도 합니다. 이런 조건들을 \emph{필요 조건}이라고 합니다.
	
	이런 필요 조건이 있는 이야기들의 경우에는 기본 개연성 비용이 10에서 5로 줄어듭니다. 더불어, 어떤 이야기가 있어야만 사용할 수 있는 이야기의 경우, 기본 개연성 비용이 0으로 줄어듭니다. 필요 조건이 존재하더라도 기본 비용은 이 조건이 어떤 이야기에 기반해야만 변화하며, 이야기에 기반한다고 해서 단순히 어떤 부분을 공유하거나 사용할 수 있는 연계가 있기 때문에 비용이 변하지는 않습니다.
	
	\smallskip
	
	가령, 마나를 사용하는 마법사 이야기꾼을 만들기 위해 [마나 친화력]이라는 이야기로 마나를 사용할 수 있도록 했다고 가정하겠습니다. 이를 필요 조건으로 가지는, [마나 증폭기]라는 이름의 마나의 회복을 보조하는 이야기, [화염구]라는 마나를 사용하는 주문, 그리고 [마도사 자격증]이라는 이야기꾼의 생애 이야기를 생각해보겠습니다.
	
	[마나 증폭기]의 기본 개연성 비용은 [마나 친화력]과 연계되는 능력이 주가 되며 그를 보조하기 때문에, 5가 됩니다.
	
	[화염구]는 [마나 친화력] 없이는 사용할 수 없는 이야기이기 때문에, 기본 개연성 비용이 0이 됩니다.
	
	하지만[마도사 자격증] 이야기의 경우, 마나를 사용하거나 [마나 친화력]을 보조하는 능력이 아니기 때문에 이 이야기에 대한 기본 개연성 비용은 10이 될 것입니다.
	
	\section*{사용 후 버려지는 아이템}
	일정 횟수 또는 기간 사용 후 버려지거나 해당 서사에서 더 이상 사용할 수 없는 아이템의 경우라 할지라도, 개연성 비용은 동일하게 계산됩니다.
	
	하지만, 이 아이템을 더 이상 사용할 수 없게 되는 경우에\footnote{이는 아이템을 빼앗기거나 잃어버린 경우를 포함합니다.} 이야기꾼은 개연성의 부담을 경감받을 수 있습니다. 아이템이 모두 소모된 바로 다음 씬이 시작하기 전에, 해당 아이템의 개연성 코스트만큼의 최대 개연성과 개연성을 회복하지만, 해당 아이템의 스탯 증감을 포함한 모든 효과를 더 이상 사용할 수 없습니다.
	
	\section*{개연성 비용에 대한 가장 중요한 이야기}
	중요한 것은, 이 계산이 귀찮다면, 시스템과 상의하여 개연성 비용을 정하는 식으로 진행해도 된다는 것입니다. 이야기꾼의 세계의 규칙은 이야기 내적으로든 규칙 자체로든 의도적으로 수정이 쉽도록 만들어져 있기에, 시스템과 이야기꾼들의 판단이 가장 중요한 영향을 끼친다는 것을 명심하세요.
	
	\hypertarget{reward}{}
	\section*{서사의 보상}
	이야기꾼들이 서사를 헤쳐나가게 되면 등장인물이나 다른 깨달은 자, 또는 시스템에 의해 보상을 받을 수 있습니다. 이 보상은, 당연하게도, 이야기의 형태로 지급됩니다. 이 이야기들의 코스트는 기본적으로 0으로 취급됩니다. 보상으로 지급되는 이야기의 내용은 자유롭게 지급할 수 있지만, 많은 서사를 경험한 이야기꾼과 그렇지 못한 이야기꾼이 함께 존재하는 경우가 있을 수 있기 때문에 그 차이를 줄이기 위해 다음 중 하나 이상이 적용되도록 하는 것을 권장합니다:
	\begin{itemize}
		\item 개연성 코스트가 정상적으로 계산되어 적용된다.
		\item 능력이 없고 내용만 있어 도움만 받을 수 있는 이야기이다.
		\item 특정 조건이 만족되면 이야기를 잃게 된다. 예를 들어:
		\begin{itemize}
			\item 일정 기간동안만 적용된다.
			\item 해당 이야기의 신념에 반하는 행동을 할 경우 이야기를 잃는다.
		\end{itemize}
		\item 능력이 있다면, 발동 조건 등이 까다롭다. 예를 들어:
		\begin{itemize}
			\item 쿨타임이 길다.
			\item 사용시마다 강제로 침범판정을 행해야 한다.
			\item 특히 아이템의 경우, 발동 횟수 제한이 있다.
		\end{itemize}
	\end{itemize}
	만약 이 이야기로 인한 스탯의 변동이 있다면 이는 반드시 개연성 코스트에 영향을 끼치게 됩니다. 또한 트라우마류의 효과가 있다면 이 역시 개연성 코스트에 영향을 끼칩니다.
	
	물론, 이 서사를 얻은 세계로 다시 들어가게 된다면 위 제약이 사라지거나 약화될 수 있다는 것을 기억하세요.
	

\end{document}