\documentclass{report}

\begin{document}
		\world{이야기 속에 들어가면 여러분은 여러가지 역할과 이야기를 받게 됩니다. 어떤 경우에는 이야기 속에 들어갔을 때 받게 될 이야기를 선택할 수 있는 경우도 있을 것이고, 어떤 경우에는 이야기의 시공간적 배경으로 인해 강제로 이야기를 받게 되거나, 여러분이 가지고 있는 이야기가 변형될수도 있습니다.}
	
	이야기 속에 들어간 이야기꾼들은 이야기 속의 역할에 맞추어 추가적인 이야기를 받을 수 있습니다. 이 새로운 역할은 처음부터 주어지기도 하며, 나중에 필요에 의해 지급되거나 자신이 선택할 수도 있을 것입니다. 이 역할들은 간단하게 이야기만을 주기도 하지만, 새로운 능력이나 트라우마, 행동 제약 등을 제시할 수도 있습니다. 간단한 예시로, 사진 작가인 이야기꾼이 사진기가 발명된 직후의 세계의 이야기에에 카메라를 들고 들어간다면 카메라가 커다란 구형 카메라로 변한다거나 하는 페널티가 생길 수 있을 것입니다.
	
	\bigskip
	
	보다 구체적인 예시로, 이야기꾼이 생명을 지켜야 하는 신성한 직업인 사제가 된 경우, 다음 이야기들을 부여할 수 있습니다:
	
	\begin{story}{깨어난 자}{[역할:사제][신성]}
		어떤 상황에서도 이성을 잃지 않는다.
		
		모든 물리 공격에 [신성] 속성이 추가되어 타락한 자들에게 추가 정신력 또는 체력 피해를 준다.
		
		“비폭력"이 아닌 다른 트라우마의 트리거 조건을 만족했을 때 4df를 굴려 해당 트라우마의 트라우마 단계(기피 1, 공포 2, 광기 3) 이상의 수치가 나오면 효과를 무시한다.
	\end{story}
	
	\begin{story}{비폭력}{[역할:사제][기피][공포]}
		\textbf{트리거(기피)}: 생명체에게 피해를 입힌다.
		
		\textbf{효과}: 이번 씬 동안, 추방된다. “사제” 역할을 잃는다.
		
		\medskip
		
		\textbf{트리거(공포)}: 생명체를 사망에 이르게 한다.
		
		\textbf{효과}: 이 이야기에서 영구히 추방된다. “사제” 역할을 잃는다.
	\end{story}
	
	역할들은 강제로 부여될 수도 있습니다. 예를 들어, 불안정한 공간에서 벌어지는 이야기라면 다음 이야기가 부여될 수 있을 것입니다:
	\begin{story}{비정형의 공간}{[공간]}
		우주 상에 떠다니는 불규칙한 공간이기 때문에 명중 또는 회피 판정을 할 때에는 별도의 해당 페널티를 상쇄할 만한 이야기가 없다면 명중/회피 페널티로서 4df를 굴려 해당 수치를 판정치에 반드시 더해야 한다.
	\end{story}
	
	다른 많은 규칙에 존재하는 클래스 규칙을 사용하고 싶다면, 사용할 수 있습니다. 물론 이 이야기는 하나의 예시일 뿐이라는 것을 기억하세요:
	\begin{story}{클래스}{[세계]}
		각 이야기꾼은 이 이야기에 돌입하면 클래스 [물리] [마법] [신성] 중 한가지와 해당 클래스의 서브클래스 셋 중 하나를 선택하고, 해당 서브클래스 이야기의 능력과 제약을 받는다.
		
		\medskip
		
		\begin{tabularx}{\textwidth}{X|X|X}
			\multicolumn{1}{c|}{\textbf{[물리]}} & \multicolumn{1}{c|}{\textbf{[마법]}} & \multicolumn{1}{c}{\textbf{[신성]}} \\ \hline
			\textbf{[Rogue] 도둑의 미덕} & \textbf{[Conjurer] 힘이 흘러넘쳐} & \textbf{[Cleric] 의심스러운 믿음} \\
			{자신을 인지하지 못한 사람에 대한 판정 +2, 자신을 인지한 사람에 대한 피해량 1 감소(최소 1)} & {원소마법 계열의 능력의 위력 증가(피해 1 증가, 치유 1 증가, 유틸기 포함), 쿨다운 1턴 증가.} & {모든 능력의 치유량 50\% 증가, 최대 체력 20\% 감소.} \\\hline
			\textbf{[Barbarian] 나 아프다!} & \textbf{[Arcana] 미래를 보는 눈} & \textbf{[Paladin] 올곧은 의심} \\ 
			{생명력이 최대 생명력의 50\% 미만일 때 가하는 근접 공격 피해 50\% 증가, 지능 관련 판정 -1} & {조준 또는 회피를 할 때 행운 사용시 두 번 굴려 그 중 원하는 수치로 판정, 회피 실패시 받는 피해량 50\% 증가.} & {집중해야 하는 기술에 관련된 판정에서 위력 50\% 증가, 집중력 판정에 -1} \\ \hline
			\textbf{[Ranger] 흔들림 없는 손} & \textbf{[Diablerie] 계약의 흔들림} & \textbf{[Druid] 어색한 대자연} \\ 
			{명중 판정에 행운 사용시 +1} & {모든 능력의 체력 소모량 50\% 감소(최소 1), 위력 감소(피해 1 감소, 치유 1 감소, 유틸기 포함)} & {자연과 소통을 통해 사용하는 능력들이 발동하는데에 1턴 추가로 걸림, 위력 50\% 증가.}
		\end{tabularx}
	\end{story}
	
	이처럼 이야기 속에서 역할을 부여받으면 그 역할에 맞는 이야기와 능력을 얻음으로서 해당 이야기의 세계를 보다 생생하게 체험하게 해 줄 수 있습니다.
	

\end{document}