\documentclass{report}

\begin{document}
	\world{이야기꾼이 자신의 출신 이야기가 아닌 이야기에 들어갔을 때, 여러분은 이야기의 외부자로서 이야기의 흐름을 방해하거나 오염시켜서는 안됩니다. 이야기가 오염되었다는 것은 이야기 자체가 사라졌다는 것이 아니라, 그 이야기의 간섭받지 않은 결말, 즉 [종장]이 희석되어 알아보기 힘들게 되었다는 것을 의미합니다. 이렇게 되면 그 이야기의 본질이 바뀌게 되는 것이죠. 이를 막기 위해 [참견]의 제약으로, 너무 이야기에 큰 손상이 우려될 때에는 이야기꾼을 이야기에서 추방하게 됩니다.}
	\world{여러분은 이야기 속에서 저를 도와 이야기의 흐름을 되돌리는는 과정에서 그 이야기 속에서의 죽음을 맞이하게 될 수도 있습니다. 하지만, 이는 그 이야기 속에서 그 역할으로서의 당신이 죽었을 뿐이라는 것을 명심하세요. 새로운 역할을 부여받아 다시 이야기 속으로 들어가는것은 당연히 가능합니다.}
	
	이야기를 오염시키는 행동인지를 판단하기 위하여, 시스템은 행동에 따라 [침범] 판정을 하도록 강요합니다. 이야기의 [운명]이나 지식을 거스르는 행동, 부여받은 이야기의 역할에 맞지 않는 행동, 이야기에 어울리지 않는 능력을 사용하는 행동 등이 이에 포함됩니다.
	
	이야기꾼들이 가지고 있는 이야기들은 각각 개연성 코스트가 있습니다. 해당 이야기의 [침범도]는 개연성 코스트의 1/10(올림)으로 정의합니다.
	
	이야기꾼들에게는 각각 최대 개연성이 있습니다. 이야기꾼의 [저항도]는 최대 개연성의 1/10(버림)으로 정의합니다.
	
	\bigskip
	
	[침범] 판정은 판정을 할 때 마다 기본 난이도가 점점 증가합니다. [침범] 판정의 기본 난이도는 보통 최초에 -1으로 시작하여, 이야기꾼의 수만큼 판정을 할 때 마다 1씩 증가합니다. 다르게 생각하려면, 한 번 판정시마다 (1/이야기꾼)만큼 난이도가 상승한다고 생각해도 됩니다. 만약 누군가가 이야기를 이미 [침범]한 상태였다면, 시작할 때의 침범 판정의 난이도는 0보다 높을 수 있습니다. 이 기본 난이도에 침범 판정을 일으킨 이야기의 침범도를 더한 것이 난이도가 됩니다.
	
	[침범] 판정을 할 때에는 4df를 굴려 자신의 저항도를 더합니다. 이 값이 난이도보다 높으면 성공하고, 같거나 낮으면 실패합니다.
	
	[침범] 판정의 결과와는 상관 없이, 행동은 완료할 수 있습니다.
	
	[침범] 판정이 성공한다면 행동이 정상적으로 행해집니다.
	
	[침범] 판정이 실패한다면, 자신의 개연성을 해당 이야기의 개연성 코스트만큼 감합니다. 또는, 이야기에 따라 독립적인 실패의 대가가 따를 수 있습니다. 가령 이야기의 끝에 다다르고 있는 이야기는 오염될 가능성이 적기에, 다음 이야기의 내용으로 페널티가 대체될 수 있습니다:
	
	\begin{story}{시간의 끝}{[시간]}
		\entry[\hline]{이 이야기는 이미 시간의 끝을 향해 달려가고 있기 때문에 이야기의 개연성을 해칠 염려가 매우 적다. 따라서, [침범] 판정에 실패할 때, 다음 씬(비전투) 또는 두 턴 후(전투)까지 해당 판정에 실패한 이야기가 전면 봉쇄되어, 기술 뿐 아니라 해당 이야기로부터의 도움도 받을 수 없다(단, 스탯은 유지된다). 개연성 판정 난이도의 초기화는 이야기의 봉쇄가 일어나면 즉시 일어나나, 이야기가 불안정해짐에 따라 판정이 일어날 때 마다 난이도가 1 상승한다.}
	\end{story}
	
	어떠한 방법으로든 개연성이 0이 된다면 이야기에서 추방됩니다. 다음이 순서대로 진행됩니다:
	\begin{itemize}
		\item 개연성이 0이 된 이야기꾼이 이야기에서 추방됩니다. 해당 씬은 그대로 진행됩니다.
		\item 앞으로 등장인물이 아닌 모든 이야기꾼은 모든 이야기를 사용하거다 도움을 받는데에 있어, 반드시 [침범] 판정을 해야 합니다.
		\item 해당 씬이 종료되면 자동으로 모든 이야기꾼이 이야기에서 추방됩니다.
	\end{itemize}
	이야기꾼 vs 타락한 자의 경우, 시스템은 여러분이 돕고 있다는 사실을 알기에, 이 행위들을 적대 행위로서 간주하지는 않습니다. 시스템은 여러분의 권한을 다시 복구시켜 줄 것이며, 임무를 계속할 수 있습니다. 이로 인해 타락한 자가 추방되더라도 같이 추방되었기 때문에 일괄적으로 복구되어, 이야기를 계속 어지럽히는데에 일조합니다.
	
	[침범] 판정의 난이도는 모든 이야기꾼이 추방된 이후 다시 초기화됩니다.
	
	\section*{세계관 출신 인물의 이야기에 의한 침범 판정}
	가끔 이야기의 [깨달은 자]가 아닌 등장인물 중에는 먼치킨스러운 능력을 보유하고 있는 경우가 있습니다. 또, 이야기꾼들은 자신의 출신 이야기로 돌아가기도 하죠. 세계관과 연관 있는 이야기 또는 인물에 의한 [침범] 판정에 대한 세 가지 중요한 사실은 다음과 같습니다:
	\begin{enumerate}
		\item {}[깨달은 자]가 아닌 등장인물에 의해서는 [침범] 판정이 일어나지 않습니다.
		\item 그 세계 속에서 얻은 이야기 또는 능력에 의해서는 [침범] 판정이 일어나지 않습니다.
		\begin{itemize}
			\item 이는 누군가 개연성이 0이 되어 추방되었을 때의 강제 판정에도 포함됩니다.
		\end{itemize}
		\item 해당 세계관의 출신인 [깨달은 자]는 개연성이 0이 되어도 이야기에서 추방되지 않습니다.
		\begin{itemize}
			\item 추방되는 대신, 제대로 된 치료를 받기 전까지 체력과 정신력이 회복되지 않고 행동불능 상태가 됩니다.
			\item 이로 인해 이야기꾼이 사망하는 경우도 생길 수 있습니다.
		\end{itemize}
	\end{enumerate}
	
	\section*{\hypertarget{the-story-continues}{(선택 규칙)계속되는 이야기}}
	한 이야기의 세계 속에서 여러 이야기가 진행된다면, 이야기가 끝난 이후에, [침범] 판정의 난이도를 확인하여 다음 이야기에는 해당 난이도에서 [침범] 판정의 난이도가 시작하게 할 수 있습니다.
	
	서플리먼트의 \storyref{probability:the-story-continues}{계속되는 이야기}와 같은 이야기로 구현할 수 있습니다.

\end{document}