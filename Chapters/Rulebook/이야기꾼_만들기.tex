\documentclass{report}

\begin{document}
	\world{여러분이 저를 돕기 전에, 여러분에 대해 조금 더 자세히 알아야 여러분이 어디에서 도움을 주실 수 있는지 알 수 있을 것 같습니다. 먼저 여러분, 즉 [깨달은 자]들에 대해 조금 더 알아보죠.}
	
	\section*{[0] 이야기꾼이 가지게 될 권능과 제약을 확인합니다.}\label{권능_제약}
	\world{여러분이 [깨달은 자]가 되며 저는 여러분에게 다른 이야기꾼과 소통하고 이야기에 들어가는 데에 도움이 될 세 가지 권능을 드렸습니다.}
	\world{[소통]의 권능으로 여러분은 다른 이야기꾼이나 이야기 속의 등장인물들과 이야기하는데에 어려움을 느끼지 않을 것입니다. [접근]의 권능으로 이곳, [태초의 이야기]를 포함한 다른 이야기로의 이동이 가능할 것이고, [거래]의 권능으로 여러분이 가지고 있는 이야기를 다른 이들에게 전할 수 있게 될 것입니다.}
	\world{물론 여러분이 다른 이야기들을 망치지 않도록 두 가지 제약이 걸려 있습니다. [비밀]의 제약은 여러분이 이 곳에 대하여 말하는 것을 막습니다. 너무 많은 [깨달은 자]들이 생겨나 이를 악용하는 자가 생기는 것을 막고자 하는 것이죠. 또한 [참견]의 제약은 이야기를 너무 크게 침범하여 여러분이 이야기를 오염시키게 되면 그 이야기에서 추방됨과 동시에 [접근]의 권능에 제한을 받게 됩니다. 물론 예외적인 사항이 있고, 시스템으로서 판단 하에 권한을 복구시켜드릴 수 있다는 것을 알아두세요.}
	
	\begin{minipage}{\textwidth}
		\begin{tabularx}{\textwidth}{c|c|X}
			\hline
			\textbf{구분} & \textbf{이야기} & \textbf{설명} \\ \hline \hline
			[권능] & 소통\index{소통} & 자신의 생애 이야기가 아니라면 모든 언어를 이해할 수 있다. \\ \hline
			[권능] & 접근\index{접근} & [태초의 이야기]에서 접근 좌표를 아는 이야기로 이동하거나, 어떤 이야기에서든 [태초의 이야기]로 이동할 수 있다. \\ \hline
			[권능] & 거래\index{거래} & 자신의 모든 이야기를 이야기의 규모에 비례하는 적당한 시간을 사용하여 다른 [깨달은 자]들에게 전할 수 있다. \\ \hline
			[제약] & 비밀\index{비밀} & [태초의 이야기]와 관련된 그 어떠한 사항도 [깨달은 자]가 아닌 경우 발설할 수 없다. 발설한다면, [잊혀진 자]가 된다. \\ \hline
			[제약] & 참견\index{참견} & 이야기의 [침범] 판정을 실패하여 이야기를 오염시키면, 이야기에서 추방되고, 해당 이야기에 대한 권능 [접근]을 빼앗긴다. \\\hline
		\end{tabularx}
		
		\begin{center}
			이야기꾼의 권능과 제약
		\end{center}
	\end{minipage}
	
	\section*{[1] 이야기꾼의 배경 이야기를 정합니다.}
	
	이야기꾼의 “배경 이야기”란, 이야기꾼이 [태초의 이야기]로 들어오기 이전의 삶에 관한 간단한 요약입니다. 이 생애 이야기는 크게 [성격], [생애], [모험]으로 구분됩니다.
	\begin{itemize}[label={-}]
		\item {}[성격] 이야기는 이야기꾼의 성격에 관련된 이야기입니다.
		\item {}[생애] 이야기는 이야기꾼의 성장 배경과 생물학적 특성에 관련된 이야기입니다.
		\item {}[모험] 이야기는 이야기꾼의 출신 세계관에 관련된 이야기입니다.
	\end{itemize}
	
	배경 이야기를 최소한 세 개 정합니다. 배경 이야기는 이야기꾼의 성향과 성격을 결정하는 중요한 이야기들입니다.
	
	이야기의 속성에는 [마법], [신성], [기술], [종족], [경험], [직업] 등 여러가지가 있을 수 있으며, 둘 이상의 속성을 가진 이야기가 있을 수 있습니다. 예를 들어, 만약 어떤 경험으로 인하여 이야기꾼의 성격이 정해진 경우, [성격][경험]과 같이 복합적으로 표현할 수 있습니다.
	
	\section*{[2] 이야기꾼의 이야기들을 정합니다.}
	이야기꾼의 이야기는 이야기꾼이 살아온 생애, 경험한 모든 경험, 알고있는 지식 등을 정의해주는 내용들입니다. 배경 이야기가 아닌 이야기들을 정합니다.
	
	\section*{[3] 이야기의 능력과 스탯을 정합니다.}
	“능력”이란, 이야기꾼이 가지고 있는 이야기 중 고정적인 기능과 성능을 보이는 이야기입니다.
	
	모든 이야기꾼은 위에서 언급했듯(\ref{권능_제약}), 세 가지 권능, 즉 [소통], [접근], [거래] 권능을 가지고 있고, 두 가지 제약, 즉 [비밀]과 [참견]을 가지고 있습니다. 이들은 모두 능력으로 취급됩니다. 이 능력들은 이야기꾼들이 하나의 세계에만 종속되어 있다면 무시합니다.
	
	능력은 시스템과의 상의 하에 결정할 수 있습니다. 투입되는 이야기의 세계관에 따라 능력의 사용이 제한되거나, 약화되거나, 불가능해질 수 있습니다. 이런 능력을 사용하고자 한다면, 이야기의 [침범] 판정을 해야 할 수 있습니다.
	
	아이템 역시 이야기 또는 능력으로 취급합니다.
	
	\section*{[4] 트라우마를 정합니다.(선택)}
	캐릭터에게 트라우마가 있다면 정할 수 있습니다. 능력이 있는 이야기라고 해서 트라우마를 가질 수 없는 것은 아닙니다. 가령, 다음 이야기를 생각해보세요:
	
	\begin{story}{생명의 신의 사제}{[신성][기피]}
		모든 생명체에 의해 사랑을 받는다.
		\\
		\textbf{제약(기피)}: 다른 생명체를 구하기 위한 것이 아니라면 생명체에게 고의로 피해를 입힐 수 없다.
	\end{story}
	
	트라우마는 [기피], [공포], [집착], [중독]으로 구분되며 [기피]가 심화되면 [공포]로, [집착]이 심화되면 [중독]으로 발전할 수 있습니다. [공포] 또는 [중독]이 심화되면 [광기]로 변할 수 있으나, [광기]를 가진 채 시작하는 것은 추천드리지 않습니다. 위에서 이미 정한 이야기들을 트라우마로 지정할 수 있습니다. 트라우마 역시 이야기로서 활용될 수 있으며, 이 효과들은 자신뿐 아니라 다른 이야기꾼들로 인해 역이용당할 수 있습니다. 보다 자세한 설명을 위해서는 아래 “트라우마”(\ref{트라우마}) 부분을 참고하세요.
	
	\section*{[5] 각 이야기의 개연성 코스트와 침범도를 계산합니다.}
	
	이야기의 기본 개연성 코스트는 10으로 계산합니다.
	능력을 능력으로서 사용하는것 뿐 아니라 이야기로서 활용할 수도 있습니다.
	능력과 스탯에 대한 가이드라인과 개연성 코스트 계산 방법은 능력 가이드라인 챕터(\ref{능력_가이드라인})를 참고하세요. 침범도는 개연성 코스트의 1/10(올림)으로 계산합니다. 단, 음수인 경우 0으로 취급합니다.
	
	\section*{[6] 최대 개연성 수치와 저항도를 계산합니다.}
	%최대 개연성 수치 밸런스 패치
	기본 최대 개연성은 80으로 시작합니다. 80이라는 수치는 각 이야기의 "일반적인 인물"을 기준으로 합니다. 초능력이 있는 세계관이라면 이야기꾼이 가진 초능력, 마법이 있는 세계관이라면 이야기꾼이 사용 가능한 마법, 특별할 것이 없는 현실이라면 이야기꾼의 지식과 경험 등을 이 기본 최대 개연성 안에 녹여내야 합니다. 때로 이로 인해서 능력들의 위력이 감소하거나 사용 가능한 능력의 수가 감소할 수 있습니다. 이는 이야기꾼이 너무 강해 개연성을 해칠 염려가 있거나, 다른 이야기의 세계에서 해당 능력을 사용하기 어렵거나 불가능한 이유가 있음에도 불구하고 시스템의 도움으로 해당 능력을 개연성을 해칠 확률이 적어지는는 범위 안에서 이러한 이야기들을 사용할 수 있도록 배려해준 것으로 생각할 수 있습니다. 이야기꾼이 성장(\ref{성장})함에 따라 이야기꾼이 사용할 수 있는 이야기의 범위가 넓어지는 동시에 위력도 증가시킬 수 있다는 점을 기억하세요.
	
	물론 80이라는 수치는 시스템과 참여하는 모든 이야기꾼의 합의, 또는 때로는 이야기꾼들이 진입하는 이야기에 의해 변경될 수 있습니다. 이를 변경하려는 시스템은 이야기꾼이 처음에 선택하는 세 가지 배경 이야기를 얻는 데에 기본적으로 30의 수치가 소모된다는 것을 반드시 고려해주기를 부탁드립니다.
	
	예시 이야기꾼들은 모두 최대 개연성 80을 이용하여 만들어져 있습니다.
	
	최대 개연성 수치는 합의에 의한 경우가 아니라면 다음의 경우 변합니다:
	\begin{itemize}
		\item 능력과 스탯에 따른 개연성 비용당 최대 개연성 감소
		\item 성장으로 인한 개연성 획득
	\end{itemize}
	최대 개연성이 음수가 되도록 이야기를 가질 수 없습니다.
	
	저항도에는 최대 개연성 수치의 1/10(버림)을 기입합니다.
	
	\section*{[7] 게임 진행에 필요한 기타 수치들을 계산합니다.}
	10에 근력 스탯의 10배를 더한 값을 체력에, 10에 의지 스탯의 10배를 더한 값을 정신력에 각각 기입합니다. 이 값이 음수인 경우에는 0으로 취급합니다.
	이 이외에도 능력 등에 필요한 자원 등의 수치나 시스템의 판단 하에 필요한 기타 수치 등을 기입합니다. 
	
	\section*{등장인물과 이야기꾼}
	시스템이 이야기를 만들어나가기 위해 필요한 등장인물과 이야기꾼 역시 이와 같은 과정으로 만들어도 됩니다. 한 가지 중요한 점은 등장인물의 경우 개연성의 영향을 받지 않는다는 점입니다. 하지만 밸런스를 위해 적절한 개연성 수치를 사용하는 것을 추천드립니다. 예를 들어 스토리의 최종 흑막 등 특수한 경우에는 개연성 수치를 80이 아닌 120~160 정도로 하여 만들 수 있을 것입니다.
	
	\bigskip
	
	이야기꾼 시트와 예시 이야기꾼은 \href{https://docs.google.com/spreadsheets/d/1g3ZO-oALMVbytbE2tvSBdT6czxB32XHZ1crWIGavEhQ/edit?usp=sharing}{이 구글 시트}에서 확인하실 수 있습니다.

\end{document}