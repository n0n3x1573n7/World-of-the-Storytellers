\documentclass{report}

\begin{document}
	\world{[태초의 이야기]에 오신 여러분을 환영합니다. 여러분을 맞이하는 저는 시스템[System]이라고 불러 주시면 됩니다.}
	\world{이 곳에 온 여러분은 살고 있는 세계가 이야기라는 것을 깨닫게 되었을 것입니다. 저는 여러분들을 [깨달은 자]라고 부릅니다. 제가 가지고 있는 [계산]의 권능을 이용해 이야기 속에서 깨달음을 얻은 이들이 어디에서 나타났는지 알아내어, [접근]의 권능으로 여러분에게 다가가 [추출]의 권능으로 여러분을 이곳, [태초의 이야기]로 데려왔습니다. 여러분의 시간은 여러분이 원래 세계로 돌아가기 이전에는 멈추어 있습니다. 그렇기 때문에, 다시 돌아갔을 때에 다른 이들이 여러분을 못알아보지는 않을까 하는 걱정은 접어두셔도 괜찮습니다.}
	\world{[태초의 이야기]는 여러 이야기가 모여있는 도서관이나 박물관으로 생각하실 수 있습니다. 이 곳에서 저는 세상에 존재하는 수많은 이야기와 그들의 무한한 수의 가능성을 지켜보고 관리합니다. 작은 변수 하나로도 이야기의 흐름은 완전히 달라질 수 있고, 그렇기 때문에 무한한 가능성의 이야기들은 함께 존재하는 동시에 존재하지 않습니다.}
	\world{하지만 이곳을 지나친 수많은 이야기꾼들은 결국 자신의 이득을 탐하여 타락한 자가 되어갔습니다. 이런 자들은 이야기를 오염시키며 자신이 원하는 결말을 위해 이야기의 [운명]을 비틀고 [진리]를 어겨 이야기를 붕괴시키고 있습니다. 이들에게 제가 가할 수 있는 최대의 제재는 그들이 다른 이야기에 가할 수 있는 모든 영향력을 최소화하기 위해 그들의 [접근]의 권능을 빼앗아, 이야기를 서서히 잊혀지게 만들어 그들을 [잊혀진 자]가 되게 하여 존재를 이곳 [태초의 이야기]에서 지우는 것입니다.}
	\world{하지만 이야기들이 완전히 잊혀지도록 하는 데에는 꽤 오랜 시간이 걸리고, 그 동안 이들은 수많은 이야기를 붕괴시킬 수 있을 것입니다. 이를 막는 데에 여러분의 도움이 필요합니다.}
	
	이야기꾼의 세계는 RPG\footnote{Role Playing Game. TRPG/TTRPG(TableTop RPG), ORPG(Online RPG)로도 알려져 있다.}입니다. 여러명이 모여서 캐릭터를 만들고, 그들이 세계와 서로 상호작용을 하며 이야기를 만들어나가는 게임이죠. 이야기꾼의 세계에서는 다른 게임에서 GM\footnote{Game Master. DM(Dungeon Master)이라고도 부른다.}이라고 부르는 "이야기를 이끌어나가는 사람"을 \emph{시스템[System]}이라고 부릅니다. 플레이어들이 조종하는 PC\footnote{Player Character}라고 불리는 캐릭터들은 \emph{이야기꾼[Storytellers]}이라고 부릅니다. 여러분은 시스템이 되어 이야기의 틀을 잡고 이끌어 나가거나, 이야기꾼이 되어 그 이야기 속에서 여러분의 캐릭터들이 행동하도록 할 것입니다.
	다른 RPG와 이야기꾼의 세계가 다른 점은, 이야기꾼들이 단 하나의 세계에 종속되어있을 필요가 없다는 점입니다. 여러 세계에서 온 이야기꾼들이 같이 [태초의 이야기]로 들어와 풀어나가는 이야기를 볼 수 있다는 거죠.
\end{document}