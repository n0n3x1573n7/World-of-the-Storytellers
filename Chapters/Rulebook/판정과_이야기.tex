\documentclass{report}

\begin{document}
		\world{여러분이 이야기 속으로 들어가면 대부분은 개연성에 의해 이야기 속의 등장인물로서 편입이 되게 됩니다. 이런 이야기 속에서는, 여러분이 가지고 있는 이야기들의 도움을 받아야만 합니다. 여러분의 이야기는 여러분이 어떤 것을 잘 할 수 있고 어떤 것을 잘 하지 못하는지를 대변해줄 수 있을 뿐 아니라, 여러분이 어려운 상황에 처했을 때 직접적으로 도움을 줄 수도 있을 것입니다.}
	
	이야기 속에서 이야기꾼들이 행동을 할 때, 자신이 가진 스탯과 이야기의 도움을 받을 수 있습니다. 임무의 성공 여부는 보통 스탯의 수치로 결정되지만, 이야기의 도움을 받을 수도 있고, 특정 이야기가 해당 상황을 [재현] 할 수 있다면, 자동으로 성공하거나 큰 도움을 받을수도 있습니다.
	
	\bigskip
	
	이야기꾼의 세계에서는 이야기별로 다른 판정 방법을 사용할 수는 있겠지만, 스탯을 사용하고 이야기의 도움을 받는 등의 기본적인 판정을 할 때에는 기본적으로 주사위를 사용하지 않습니다. 하지만 아래의 선택 규칙인 [행운] 규칙을 사용할 때에는 "퍼지 다이스(fudge dice)"라고 불리는 주사위를 사용합니다. 퍼지 다이스는 +, 0, -의 세 가지 종류의 면이 같은 수로 존재하는 주사위를 사용합니다. 주사위를 새로 구매할 수도 있지만, 보통 많이 소지하고 있을 d6을 사용하여 1\textasciitilde2는 -, 3\textasciitilde4는 0, 5\textasciitilde6은 +로 취급하는 방법이 있습니다. 이 주사위를 사용하기로 선택한 이유는 평균이 0이고 분산이 작기 때문에 주사위 굴림의 운보다는 스탯 수치가 더 큰 영향을 주도록 하기 위함입니다. 반면, 능력의 경우에는 시스템이 허용하는 그 어떤 판정 방법을 사용해도 무방합니다. 예를 들어 d4, d6, d8, d10, d12, d20, d\%\footnote{d100으로도 알려져 있다.} 등의 다른 주사위들이나 플레잉카드, 타로카드 등의 카드들을 이용한 판정을 사용하는 능력을 만들었다면 시스템의 허가 하에 사용할 수 있을 것입니다.
	
	\bigskip
	
	이야기의 도움을 받을 때에는 이야기의 내용에 해당 상황에 도움이 되는 속성이 포함되어야 합니다. 예를 들어서, 무거운 돌을 옮겨야 한다면 “헤라클레스의 가호를 받은 자”라는 이야기는 도움을 줄 수 있지만, “여러 생을 살아온 구미호”라는 이야기는 도움을 줄 수 없을 것입니다.
	
	이야기의 도움을 통해 (해당 상황에 사용되는 스탯)+(도움을 준 이야기 수)가 성공치보다 얼마나 높거나 낮은지에 따라 다음이 일어납니다:
	\begin{itemize}
		\item \textbf{3 이상 작다}: 실패합니다. \emph{대실패}로 판정되어, 해당 상황에 대한 해소 전까지 지속되는 부정적 상태를 하나 받습니다.
		\item \textbf{1\textasciitilde2 작다}: \emph{실패}합니다. 단, 시스템이 지정해주는 페널티(대표적으로 1회 사용 후 소멸되는 부정적 상태)를 받는다면 성공할 수 있습니다.
		\item \textbf{같거나 1 크다}: \emph{통과}합니다. 원하는 바를 이룰 수 있으나, 그 이상도 이하도 일어나지 않습니다.
		\item \textbf{2\textasciitilde3 크다}: \emph{성공}합니다. 추가적으로, 1회 도움을 받은 후 소멸되는 긍정적 임시 상태를 받습니다.
		\item \textbf{4 이상 크다}: 성공합니다. \emph{대성공}으로 판정되어, 상태가 해소되기 전까지 계속해서 도움을 받을 수 있는 긍정적 상태를 받습니다.
	\end{itemize}
	상태를 받을 때에는 어떤 상태인지 시스템과 플레이어 양쪽이 모두 제안할 수 있습니다. 만약 부정적이든 긍정적이든 상관 없이 상태를 받을 수 있는 판정에도 불구하고 시스템과 플레이어가 둘 다 받을만한 상태를 생각할 수 없는 경우, 상태를 받을 수 없습니다.
	
	\section*{임시 상태}
	임시 상태란 길게는 해당 세션동안, 짧게는 한두번정도의 도움을 받을 수 있는 잠시동안 지속되는 상태입니다. 이는 여러가지 방법으로 받을 수 있으며, 특정 임시 상태를 부여하는 능력이 존재할 수 있습니다. 임시 상태는 그 효과에 따라 긍정적, 부정적, 중립적의 세 가지로, 그 종류에 따라 물리적, 정신적 등으로 구분됩니다. 종류에 따른 구분 방식은 이 외에도 여러가지가 있을 수 있으며, 한 이야기의 분류는 절대적이지 않을 수 있습니다. 예를 들어 [잠이 옴]이라는 상태는 일반적인 상황에서라면 부정적인 상태일지라도, 며칠간 잠을 자지 못한 불면증 환자에게는 축복과도 같은 긍정적인 상태로 다가올 수 있습니다.
	
	상태들은 효과나 종류에 상관없이 존재하는 동안은 이야기처럼 활용되어 판정에 도움을 줄 수 있습니다. 각 상태는 해소되기 전까지는 계속하여 유지되는데, 이를 해소하는 방법은 다음과 같습니다:
	\begin{itemize}
		\item 횟수가 제한된 상태의 경우, 횟수의 소모
		\item 시간이 제한된 상태의 경우, 시간의 흐름
		\item 치료, 상담 등으로 인한 상태의 직접적 해소
		\item 환경으로 인해 일어난 상태의 경우, 해당 환경에서 벗어나거나 환경의 해소
	\end{itemize}
	
	\section*{이야기의 [재현]}
	[재현]이란, 이야기 속의 상황과 현재 상황이 매우 유사한 경우 발생하는 예외사항입니다. 상황의 유사성에 따라 다음 중 하나가 선택적으로 적용됩니다:
	\begin{itemize}
		\item 어떤 상황을 극복해야 하는 경우:
		
		\begin{itemize}
			\item 상황이 세부 사항을 제외하고 거의 같은 경우, 자동으로 성공합니다.
			\item 상황에 유사성에 따라, 이야기의 도움을 크게 받아 한 번 이상의 도움을 받은 것으로 취급할 수 있습니다. 이 수치는 상황의 유사도에 따라서 시스템이 결정합니다.
		\end{itemize}
		
		\item 다른 이야기꾼와 대결하는 경우:
		
		\begin{itemize}
			\item 상황이 세부 사항을 제외하고 거의 같으며, 그 이야기에서도 해당 이야기꾼과 대결했었던 경우, 상대가 이와 동급의 이야기로 대항하지 않는다면 자동으로 성공합니다.
			\item 상황에 유사성에 따라, 이야기의 도움을 크게 받아 한 번 이상의 도움을 받은 것으로 취급할 수 있습니다. 이 수치는 상황의 유사도에 따라서 시스템이 결정합니다.
		\end{itemize}
	\end{itemize}
	예를 들어, 12과업을 행한 헤라클레스의 경우 \textbf{[네메아의 사자를 죽인 자]}라는 이야기를 가지고 있다면 다른 사자에 대한 대항 판정에 있어 자동으로 성공할 수 있고, 다른 맹수에 대한 대항 판정에 있어 자동으로 성공하지는 않겠지만 큰 보너스를 받게 될 수 있습니다. 만약 호랑이와 같이 사자와 유사한 맹수라면 +4 등 큰 보너스를, 악어와 같이 조금 거리가 있는 맹수라면 +2 등 작은 보너스를 받게 될 것입니다.
	
	\section*{이야기 효과의 우선순위}
	이따끔씩 여러 이야기가 한번에 적용되어야 할 때가 있을 수 있습니다. 예를 들어, 어떤 사건으로 인해 한 사람의 공포증의 트리거가 발동됨과 동시에 다른 사람의 패시브 능력의 조건이 충족된다거나 하는 식으로 말이죠. 이 경우에는, 다음과 같은 순서로 이야기의 우선순위가 정해집니다:
	\begin{enumerate}
		\item 이야기꾼이 들어가있는 세계 출신의 이야기
		\item 개연성 비용의 절대값이 높은 이야기
		\begin{itemize}
			\item 단 양수인 이야기와 음수인 이야기가 있다면, 양수인 이야기가 우선됨
		\end{itemize}
		\item 해당 이야기와 가장 관련 있는 이야기꾼의 스탯이 높은 이야기
		\begin{itemize}
			\item 기본적으로 스탯을 올려주는 이야기의 경우 해당 스탯, 또는 그 중 가장 높은 스탯
			\item 아닌 경우라 할지라도 이야기와 관련있는 스탯이 있다면 해당 스탯
			\item 전혀 없다고 생각된다면 이야기꾼의 의지 스탯을 이용함.
		\end{itemize}
		\item 위 모두가 동률이라면:
		\begin{itemize}
			\item 전투 상황이 아닐때, 
			\begin{itemize}
				\item 서로 다른 두 이야기꾼의 이야기라면 주사위를 굴려 결정.
				\item 한 이야기꾼의 이야기라면 이야기꾼의 마음대로 결정.
			\end{itemize}
			\item 전투 상황이라면 더 빠르게 자신의 턴이 돌아오는 이야기꾼의 이야기부터.
			\subitem 현재 턴인 이야기꾼 $\rightarrow$ 다음 턴인 이야기꾼 $\rightarrow$ ...
		\end{itemize}
	\end{enumerate}
	
	
	\section*{\hypertarget{pow-of-luck-unluck}{(선택 규칙)"행운"의 힘과 "불운"의 힘}}
	다음 중 전부 또는 일부를 선택하여 적용시킬 수 있습니다:
	\begin{itemize}
		\item 모든 행동을 할 때, “행운”의 힘을 빌리기로 결정할 수 있습니다. “행운”의 힘을 사용하고자 한다면, 이야기의 도움을 받은 횟수에 퍼지 주사위 네 개를 굴려 나온 값을 더해 판정합니다.
		
		\begin{itemize}
			\item “행운”을 사용한다면, 반드시 모든 판정에 “행운”을 적용시키기로 결정할 수 있습니다.
			
			\item ++++은 대성공, -{}-{}-{}-는 대실패로 판정을 지정할 수도 있습니다. 이 경우, 판정의 결과 수치에 상관없이 ++++ 또는 -{}-{}-{}-가 나오는 경우 대성공 또는 대실패합니다.
		\end{itemize}
		
		\item 세계 간의 괴리감과 함께 "불운의 힘"이 깃들어 "재현"이 실패할 확률이 존재할 수 있습니다. 이 경우, 퍼지 주사위 네 개를 굴립니다. 만약 -{}-{}-{}-가 나온다면, 이 이야기를 일반적인 이야기의 도움으로 취급합니다. -{}-{}-{}-가 나오지 않는다면, "재현"이 정상적으로 수행됩니다.
	\end{itemize}
	
	서플리먼트의 \hyperlink{story:luck-unluck}{행운과 불운} 챕터의 이야기들과 같은 이야기로 구현할 수 있습니다.
	
	\section*{판정의 예시\footnote{애스크를 통해 질문 주신 익명의 질문자분께 감사드립니다.}}
	
	예를 들기 위해 인식이 1인 이야기꾼이 있고, 어두운 통로에서 무언가 빠르게 휙 지나가는 것을 보기 위해 성공치가 4인 인식 판정을 해야 한다고 가정하겠습니다.
	
	이 이야기꾼에게 만약 [야간시]와 같은 어두운 곳에서 보는데에 도움을 주는 이야기나, [야구선수]와 같이 빠른 속도로 지나가는 물체를 보는데에 도움을 주는 이야기가 있다면, 이들로 인해 판정치에 각각 +1을 받을 수 있습니다. 이 경우 판정치, 즉 [(스탯)+(이야기)]는 3이 됩니다. 보통의 경우라면 성공치가 4이므로 1 차이로 이야기꾼은 인식 판정에 실패하거나, [눈이 아픔]과 같은 짧게 유지되는 임시 상태를 받음으로서 판정에 성공하게 될 수 있습니다. 즉, 기본적으로 판정을 할 때에는 다이스를 사용하지 않습니다. 오로지 이야기꾼의 스탯과 이야기, 즉 그들의 실력으로 판정이 결정되는 것이죠.
	
	하지만 [행운] 규칙을 사용하는 경우 퍼지 다이스 4개를 굴려 "행운"의 힘이 이야기꾼의 판정에 개입하도록 할 수 있습니다. 이 경우 판정치에 퍼지 다이스 네 개를 굴린 결과를 더한 것이 판정치가 됩니다. 예를 들어 위의 상황에서 [행운]의 힘을 빌렸을 때, 퍼지 다이스의 결과가 -{}-0+(-1)이라면 판정치는 2가 되어 실패하게 되나, 0++0(+2)라면 판정치는 5가 되어 "운이 좋게도" 열쇠를 물고 있는 쥐 한마리가 빠르게 지나갔다는 것을 보게 될수도 있게 되는 것이죠.
	
	만약 이 경우에 적용 가능한 능력이 존재한다면 상황은 역시 달라질 수 있습니다. 만약 위의 이야기 중 [야간시]가 다음과 같은 능력을 가지고 있다고 가정해봅시다:
	\begin{story}{야간시}{[종족]}
		어두운 곳에서 모든 판정을 할 때에 +1을 받는다. 특히, 시각에 의존한 판정을 할 때에 추가로 +1을 받는다.
	\end{story}
	이 경우, 판정치는 3에 [야간시]의 능력에 의한 +2가 추가되어 "행운"의 힘 없이도 5의 판정치로 판정에 성공하게 됩니다.
	

\end{document}