\documentclass{report}

\begin{document}
	이야기꾼의 세계는 룰적인 수정을 가하여 새로운 룰으로 만들어지기 쉬운 구조로 만들어져 있습니다. 그렇기 때문에 이야기꾼의 세계를 다른 RPG 규칙의 요소를 이용해 확장할수도 있고, 다른 RPG 규칙을 이야기꾼의 세계를 이용해 확장할수도 있습니다.
	
	이에 대한 예시를 들기 위해서 \href{https://twitter.com/shinhogoesreal/}{신호}님이 만드신 룰인 \href{https://twitter.com/shinhogoesreal/status/1165797377980035073}{미션스쿨백합}을 사용하도록 하겠습니다.
	
	\section*{이야기꾼의 세계를 다른 규칙으로 확장하기}
	이야기꾼의 세계를 다른 규칙으로 확장하는 방법은 간단합니다: 이야기꾼들이 진입하는 서사에 해당 규칙의 핵심에 해당하는 이야기를 [세계] 속성을 가진 이야기로 만들어 역할으로서 부여하거나 세계의 설명에 추가하는 것입니다.
	
	예를 들어 미션스쿨백합의 경우, 다음 네 가지 이야기를 추가하게 되면 간소화된 버전이긴 하나 미션스쿨백합의 규칙을 이용해 이야기꾼의 세계를 확장할 수 있을 것입니다:
	
	\bigskip
	
	\begin{story}{미션스쿨백합}{[세계][미션스쿨백합]}
		\entry{이야기꾼들은 미션스쿨의 학생으로 이 세계에 들어오게 된다. 각 이야기꾼은 정신적 상태인 죄책감 0/10을 추가로 가진다. 죄책감은 10 이상으로 증가할 수 없으나, 10 이상으로 증가하게 된다면 대신 적절한 정신적 상태를 얻는다.}
		
		\entry{새로운 주요 등장인물(이야기꾼 포함)이 등장할 때 마다, 해당 인물에 대한 끌림 상태 2/10을 가진다.}
		
		\entry[\hline]{각 이야기꾼은 \storyref{mission-school:action}{동성간 행위}, \storyref{mission-school:prayer}{기도}, \storyref{mission-school:confession}{고해성사}의 세 가지 행위 이야기를 사용할 수 있다.}
	\end{story}
	
	\begin{story}[mission-school:action]{동성간 행위}{[행위][미션스쿨백합]}
		\entry[\hline]{어떤 인물과 "동성간 행위"로 분류될 수 있는 행위를 하면 다음이 일어난다:
		
		\begin{enumerate}
			\item 양쪽 모두 죄책감이 1 증가한다.
			\item 행위를 받은 인물이 저항하기로 선택할 수 있다. 1d10을 굴려 끌림 미만의 값이 나오면 저항에 실패하여 아래 전투 판정으로 넘어간다. 저항에 성공했다면, 여기에서 해당 행위는 중지된다.
			\item 해당 행위를 시작한 인물이 사용한 스탯을 공격으로, 받은 인물의 의지를 수비로 하여 전투 판정을 한다.
			\item 공격 판정에 성공했다면, 해당 행위가 발생한다. 양쪽의 끌림이 자신의 죄책감의 1/3(소숫점 아래 버림) 증가한다. 단, 죄책감이 10이라면 끌림이 증가하지 않는다.
			\item 수비 판정에 성공했다면, 해당 행위를 거부한다. 행위를 시작한 인물의 죄책감이 1 추가로 증가한다.
		\end{enumerate}
		죄책감이 10인 경우 해당 행위를 할 수 없다.}
	\end{story}
	
	\begin{story}[mission-school:prayer]{기도}{[행위][미션스쿨백합]}
		\pre{혼자 있을때, 또는 조용한 곳에 있을때, 또는 성당 등에 있을 때 할 수 있다.}
		
		\entry{죄책감이 1 감소한다.}
		
		\entry[\hline]{4df를 굴려 의지를 더한 값이 0 이상이라면 죄책감이 1 추가로 감소한다.}
	\end{story}
	
	\begin{story}[mission-school:confession]{고해성사}{[행위][미션스쿨백합]}
		\pre{성당에서 4df를 굴려 2 이상의 수가 나오거나, 신부가 있음이 확실하다면 고해성사를 할 수 있다.}
		
		\entry{죄책감이 2 감소한다.}
		
		\entry[\hline]{4df를 굴려 의지를 더한 값이 1 이상이라면 고해성사를 받는 신부의 공감(최소 1)만큼의 죄책감이 추가로 감소한다. 단, 값이 -1 이하라면 고해성사를 받는 신부의 공감(최소 1)만큼의 죄책감을 얻는다.}
	\end{story}
	
	이렇게 간소화시킴으로서 룰에 있는 복잡한 주사위를 간소화할 수 있을 뿐 아니라, "죄책감"과 "끌림"을 상태로서 만들었기 때문에 이를 판정 등에 활용하는 등의 행위가 가능해졌습니다. 물론 완벽하게 똑같은 규칙을 사용하지는 않았지만 이를 사용함으로서 기본 규칙으로 이야기꾼의 세계를 사용하면서 미션스쿨백합의 핵심 규칙을 함께 즐길 수 있을 것입니다.
	
	\ifDLC
	\medskip
	
	또 다른 예시로 Call of Cthulhu의 광기 룰의 경우에는 다음 이야기를 역할로 부여할 수 있을 것입니다:
	\begin{story}{점점 미쳐가는 이야기}{[세계][설화][광기][Call of Cthulhu]}
		\entry{이성치 100을 얻는다. 일반적으로 이해할 수 없는 일(살인, 시체, 고대의 존재 등)을 목격할때마다, 그 수준에 따라 난이도와 피해량(성공/실패, 다이스 사용 가능)을 시스템이 지정한다. 그러면 1d(현재 이성치)를 굴려, 해당 수치가 난이도 이상이라면 이성치에 성공 피해량을, 미만이라면 실패 피해량을 받는다. (e.g. 60[1d4/2d8]의 경우 주사위의 결과가 60 이상이 나오면 성공하며, 성공시 1d4 정신력, 실패시 2d8 정신력을 잃는다.)}
		
		\entry[\hline]{현재 이성치의 10\% 이상에 달하는 피해를 한번에 받은 이야기꾼은 해당 상황에 대한 이 시나리오에서의 [기피]를, 20\% 이상인 경우 이 시나리오에서의 [공포]를 얻는다. 25\% 이상인 경우, [공포]를 얻는 것은 같으나 이는 시나리오에서 벗어나도 유지된다. 이를 방지하기 위해 이성치에 받는 피해의 전부 또는 일부를 1대1의 비율로 정신력에 받을 수 있다.}
	\end{story}
	\fi
	
	\section*{다른 규칙을 이야기꾼의 세계로 확장하기}
	이야기꾼의 세계의 규칙을 이용해 다른 규칙을 확장할수도 있을 것입니다. 이야기꾼의 세계를 이용하려는 이유로 생각되는 내용 각각에 대해서 어떻게 적용시킬 수 있을지 알아보도록 하겠습니다.
	
	\subsection*{크로스오버 가능성}
	크로스오버를 할 수 있는 가장 간단한 방법으로는 세계관만을 (일부) 차용해 사용하고, 나머지는 원 규칙을 사용하는 방법일 것입니다. "침범"에 관련된 규칙은 이야기꾼의 세계를 따라도 되지만, 이야기꾼들이 원래 모두 같은 세계 출신이고 해당 세계의 과거나 미래로 가는 것이라면 "침범"을 무시하고 진행해도 무방합니다. 그러나 같은 세계의 다른 시간이 아닌, 다른 세계로 가는 것이라면 "침범"을 적용시키는 것을 권장합니다. 이 경우 "침범" 판정의 성공치를 0으로 고정해두고 하는 것 역시 가능하고, "침범" 판정의 실패 페널티를 다른 것으로 고칠수도 있습니다.
	
	가령 중세 시대에 마나를 이용해 마법을 사용했으나, 현대로 오면서 기술이 발전하고 환경이 오염되며 마나의 농도가 낮아져 마법을 사용하기 힘든 세계의 경우, 다음과 같은 이야기로 침범 페널티를 정할 수도 있습니다:
	
	\begin{story}{낮은 농도의 마나}{[시간:미래]}
		\entry[\hline]{이 세계의 미래는 환경의 오염과 함께 마나의 농도가 매우 낮아져 있다. 이야기꾼이 살던 세계였기 때문에 [침범] 판정으로 인해 추방되지는 않을 것이나, 주문을 사용 할 때에는 정상적으로 [침범] 판정을 해야만 한다. 이 경우 성공치는 0으로 고정되나, [침범] 판정에 실패하면 해당 주문의 시전은 취소된다. 마나 등 자원을 소모하는 주문이었다면 자원들은 소모된다.}
	\end{story}
	
	\subsection*{능력 디자인의 자유성}
	능력 디자인을 편리하게 할 수 있도록 하기 위해, 이야기꾼의 세계에서는 능력이 주어졌을 때 개연성 코스트를 계산하는 방법이 주어져 있습니다. 이를 다른 규칙에 적용시켰을 때 그 규칙의 능력들의 개연성 코스트를 적용하는 방법으로 바꿀 수 있어야 합니다. 이렇게 계산하는 방법을 찾아낸 뒤 이를 사용하여 능력의 코스트의 상한선과 하한선을 두고 능력을 자유롭게 디자인할 수 있도록 할 수 있습니다.
	
	\subsection*{이야기꾼의 세계에서의 개연성 코스트와 확장 가능성}
	위 두 가지 상황에서는 공통적으로 크로스오버 가능성에서는 "침범" 판정을 위한 침범도와 저항도를, 능력 디자인의 자유성에서는 밸런스를 위한 개연성 코스트를 정해야 할 것입니다. 그렇기 때문에 이럴 때에는 GM의 판단에 따라 개연성 코스트를 결정해야 합니다. 그러기 위해서는 이야기꾼의 세계의 이야기에 비해 어떤 능력의 효과가 어느정도인지를 알아야 할 것입니다. 이야기꾼의 세계는 기본적으로 체력 10, 정신력 10, 개연성 80이 주어지는데 개연성의 대부분은 이야기와 능력을 얻고, 스탯을 높이는데에 사용될 것입니다. 이야기꾼의 세계의 이야기꾼의 실질 체력은 개연성 80을 기준으로 평균 약 20\textasciitilde30정도로 생각하여, 사용되는 규칙의 체력에 맞추어 개연성 코스트를 계산해도 됩니다. 물론 앞에서 얘기했듯, GM과 플레이어 간의 판단과 합의에 맞추어 이를 정하는 것 역시 가능합니다.

\end{document}