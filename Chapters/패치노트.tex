\documentclass{report}

\begin{document}
	\subsection*{2019. 02. 06. 이야기꾼의 세계 세계관 및 룰 작업 시작}
	룰을 개발하고 검수함에 있어 처음부터 아이디어를 주고받는 등 도움을 주고 지지해준 케시님께 큰 감사와 사랑을 보냅니다.
	
	알파 버전과 베타 버전 룰의 테스트 플레이와 검수를 도와주신 대두님, 소낙님, 신호님, 엠케님, 철화구야선생님께 감사를 드립니다.
\iffullchangelog
	\section*{2019. 09. 03. 최초 배포본 프로토타입 완성}
	\subsection*{2019. 09. 04. Version 1.0}
	이야기꾼의 세계의 최초 배포본입니다.
	\begin{itemize}
		\item 최초 배포본
	\end{itemize}
	
	\subsection*{2019. 09. 04. Version 1.0.1}
	\begin{itemize}
		\item "행운"의 힘 부분을 전투에서 판정으로 이동
		\item 판정에 관련된 예시 추가
	\end{itemize}
	
	\subsection*{2019. 09. 05. Version 1.0.2}
	\begin{itemize}
		\item 오타 수정
		\item 판정에 관련된 내용 추가
	\end{itemize}
	
	\subsection*{2019. 09. 05. Version 1.0.3}
	\begin{itemize}
		\item 이야기꾼에게 지급되는 최대 개연성에 대한 설명 추가
		\item 룰 비공개에 대한 조건 추가
	\end{itemize}
	
	\subsection*{2019. 09. 10. Version 1.0.4}
	\begin{itemize}
		\item 오타 수정, 애매한 예시 및 내용 수정 및 목록화된 내용의 일관성 확보
		\item 예시 트라우마의 트리거 및 제약에 기피, 공포, 광기 여부 추가
		\item 재현이 실패하는 경우를 "불운"의 힘으로 분리
	\end{itemize}
	
	\subsection*{2019. 09. 12. Version 1.0.5}
	\begin{itemize}
		\item 이야기의 보상에 관한 내용을 능력 가이드라인 챕터에 추가
	\end{itemize}
	
	\subsection*{2019. 09. 15. Version 1.0.6}
	\begin{itemize}
		\item 이야기꾼의 세계에 대한 짧은 소개를 서문으로 이동
		\item 이야기꾼의 세계 룰의 목적에 대한 서술을 에필로그에 추가
	\end{itemize}
	
	\subsection*{2019. 09. 23. Version 1.0.7}
	\begin{itemize}
		\item 이야기의 효과가 동시에 적용될 때의 우선순위에 대한 내용 추가
	\end{itemize}


	\subsection*{2019. 10. 07. Version 1.1.0}
	이야기꾼의 세계를 확장하는 방법에 대해 작업했습니다.

	\begin{itemize}
		\item 시스템의 권능과 이야기꾼의 권능과 제약에 대한 자세한 설명을 추가했습니다.
		\item 이야기꾼의 세계를 확장시키는 방법과 이야기꾼의 세계를 확장 룰로 사용하는 방법을 추가했습니다.
		\item 이야기를 만들며 알아두어야 할 점에 대해 추가했습니다.
		\item 스탯에 대한 설명을 추가했습니다.
	\end{itemize}


	\subsection*{2019. 10. 24. Version 1.2.0}
	소스코드 파일 정리 형식을 수정하며 룰북과 시나리오를 합치고, 서플리먼트 작업을 시작했습니다.

	\begin{itemize}
		\item 파일 정리 형식을 수정했습니다.
		\item 시간의 박물관 시나리오를 한 pdf에 합했습니다.
		\item 이후 업데이트의 서플리먼트를 위한 자리를 마련해 두었습니다.
	\end{itemize}
	
	\subsection*{2019. 11. 04. Version 1.2.1}
	CC 라이선스를 추가했습니다.
	
	\subsection*{2019. 11. 12. Version 1.2.2}
	이야기 찾아보기를 추가했습니다.
	
	\subsection*{2019. 11. 13. Version 1.2.3}
	\begin{itemize}
		\item 성공의 다섯 단계에 각각 대성공-성공-통과-실패-대실패의 이름을 붙였습니다.
		\item 한 이야기에서 다른 이야기를 언급할 때, 링크를 걸어 이동하기 쉽도록 만들었습니다.
	\end{itemize}
	
	\subsection*{2019. 12. 10. Version 1.2.4}
	종 포괄성을 위해 ``인간"과 ``사람"이라는 단어를 모두 제거했습니다.
	
	\subsection*{2019. 12. 13. Version 1.2.5}
	세계관을 의미하는 이야기를 모두 서사로 변경했습니다.
	
	\subsection*{2019. 12. 27. Version 1.2.6}
	지속 피해형 능력의 코스트가 일반 피해형 능력의 코스트와 피해량은 같으나 코스트가 낮게 나오는 경우를 방지하기 위해 코스트를 변경했습니다.
\fi

	\subsection*{2019. 12. 31. Version 1.3.0}
	서플리먼트의 1차적 작업이 완료되었습니다. 클래식 종족과 클래스를 포함한 다양한 이야기가 서플리먼트에 추가되었습니다. 버전 1.2.0과 비교했을 때의 차이점은 여러 버그 수정, 룰적/룰북적 최적화와 더불어 ``서사"용어의 수정과 종 포괄성 확대를 위한 ``인간", ``사람" 용어 제거 등이 있습니다.
\end{document}