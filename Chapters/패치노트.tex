\documentclass{report}

\begin{document}
	\section*{2019. 02. 06. 이야기꾼의 세계 세계관 및 룰 작업 시작}
	룰을 개발하고 검수함에 있어 처음부터 아이디어를 주고받는 등 도움을 주고 지지해준 케시님께 큰 감사와 사랑을 보냅니다.
	
	알파 버전과 베타 버전 룰의 테스트 플레이와 검수를 도와주신 대두님, 소낙님, 신호님, 엠케님, 철화구야선생님께 감사를 드립니다.
	
	\section*{2019. 09. 03. 최초 배포본 프로토타입 완성}
	\subsection*{2019. 09. 04. Version 1.0}
	이야기꾼의 세계의 최초 배포본입니다.
\iffalse
	\begin{itemize}
		\item 최초 배포본
	\end{itemize}
	
	\subsection*{2019. 09. 04. Version 1.0.1}
	\begin{itemize}
		\item "행운"의 힘 부분을 전투에서 판정으로 이동
		\item 판정에 관련된 예시 추가
	\end{itemize}
	
	\subsection*{2019. 09. 05. Version 1.0.2}
	\begin{itemize}
		\item 오타 수정
		\item 판정에 관련된 내용 추가
	\end{itemize}
	
	\subsection*{2019. 09. 05. Version 1.0.3}
	\begin{itemize}
		\item 이야기꾼에게 지급되는 최대 개연성에 대한 설명 추가
		\item 룰 비공개에 대한 조건 추가
	\end{itemize}
	
	\subsection*{2019. 09. 10. Version 1.0.4}
	\begin{itemize}
		\item 오타 수정, 애매한 예시 및 내용 수정 및 목록화된 내용의 일관성 확보
		\item 예시 트라우마의 트리거 및 제약에 기피, 공포, 광기 여부 추가
		\item 재현이 실패하는 경우를 "불운"의 힘으로 분리
	\end{itemize}
	
	\subsection*{2019. 09. 12. Version 1.0.5}
	\begin{itemize}
		\item 이야기의 보상에 관한 내용을 능력 가이드라인 챕터에 추가
	\end{itemize}
	
	\subsection*{2019. 09. 15. Version 1.0.6}
	\begin{itemize}
		\item 이야기꾼의 세계에 대한 짧은 소개를 서문으로 이동
		\item 이야기꾼의 세계 룰의 목적에 대한 서술을 에필로그에 추가
	\end{itemize}
	
	\subsection*{2019. 09. 23. Version 1.0.7}
	\begin{itemize}
		\item 이야기의 효과가 동시에 적용될 때의 우선순위에 대한 내용 추가
	\end{itemize}
\fi

	\subsection*{2019. 10. 07. Version 1.1.0}
	이야기꾼의 세계를 확장하는 방법에 대해 작업했습니다.
\iffalse
	\begin{itemize}
		\item 시스템의 권능과 이야기꾼의 권능과 제약에 대한 자세한 설명을 추가했습니다.
		\item 이야기꾼의 세계를 확장시키는 방법과 이야기꾼의 세계를 확장 룰로 사용하는 방법을 추가했습니다.
		\item 이야기를 만들며 알아두어야 할 점에 대해 추가했습니다.
		\item 스탯에 대한 설명을 추가했습니다.
	\end{itemize}
\fi
	\subsection*{2019. 10. 24. Version 1.2.0}
	소스코드 파일 정리 형식을 수정하며 룰북과 시나리오를 합치고, 서플리먼트 작업을 시작했습니다.
	\begin{itemize}
		\item 파일 정리 형식을 수정했습니다.
		\item 시간의 박물관 시나리오를 한 pdf에 합했습니다.
		\item 이후 업데이트의 서플리먼트를 위한 자리를 마련해 두었습니다.
	\end{itemize}

\end{document}