\documentclass{report}

\begin{document}
	이 챕터에서는 이야기의 방랑자들을 단순화하는 방법과 더불어, 이야기꾼의 세계와 이야기의 방랑자들간 이야기를 바꾸는 방법을 간략하게 설명합니다. 이야기꾼의 세계의 \hyperlink{expand}{확장하기} 챕터와 함께 사용하면, 다른 룰으로, 또는 다른 룰을 방랑하는 이야기꾼들을 이용하여 확장하는데에 큰 무리는 없을것입니다.
	
	\section*{이야기꾼의 세계에서 이야기의 방랑자들으로}
	이야기꾼의 세계에서 이야기의 방랑자들으로 이야기를 바꾸는 방법은 단순합니다. 이야기의 능력에서 모든 수치를 제거하고, 단순화하여 각각을 서술로 바꾸어 긍정적/부정적/중립적으로 분류하는 것입니다. 물론 개연성에 따라 조금 변화를 가해야 할 수도 있겠지만요. 시나리오의 경우 특히, \href{https://github.com/n0n3x1573n7/WoS-Scenarios}{시나리오집}에 수록된 시나리오처럼 이야기의 효과가 정확하게 지정되어 있다면 바꾸는 것이 크게 어렵지는 않을 것입니다.
	
	\section*{이야기의 방랑자들에서 이야기꾼의 세계로}
	이야기의 방랑자들에서 이야기꾼의 세계로 변환하는 것은 단순할수도, 복잡할수도 있습니다. 가장 단순한 방법은, 각 서술을 하나의 능력으로서 취급하는 것입니다. 이렇게 서술한 것을 굳이 수치화 하지 않은 상태에서 일반적인 능력으로서 사용하는 것도 가능하지만, 보다 더 구체적으로 서술들에 어울리는 수치를 부여하여 보다 더 구체화된 능력을 구현할 수도 있을 것입니다.
	
	\hypertarget{walking-storytellers}{}
	\section*{활보하는 이야기꾼들}
	이야기의 방랑자들은 이미 꽤나 단순한 형태의 규칙입니다. 하지만 확실한 이야기꾼의 테마를 잡고 있는 상태에서, 개연성 수치만을 기억하면서 플레이할 수 있도록 더욱 단순화 시킬 수도 있습니다.
	
	이렇게 하기 위해서는, 최초에 가지고 시작하는 개연성이 15가 아닌 30이 됩니다. 이 상태에서 서사를 진행해 나갑니다.
	
	이야기나 서술의 도움을 얻을 때에는, ``이야기꾼의 어떤 면이 이 장면에서 도움이 될 것이다"하는 것을 길잡이꾼에게 전달합니다. 길잡이꾼은 이를 듣고, 타당하다고 생각한다면 개연성의 소모 없이 도움을 받을 수 있습니다. 단, 애매하다고 생각한다면 개연성을 잃고 도움을 얻을 수 있다는 선택지를 제시합니다. 이 선택지를 받아들이는지 여부는 이야기꾼에게 달려 있습니다.
	
	이야기꾼은 임시 상태를 얻을 수 없고, 부상을 입을 때 부상 서술을 얻지 않고 직접 개연성을 잃습니다.
	
	이렇게 함으로서 이야기꾼의 테마와 개연성 수치만을 알고 있는 상태에서도 이야기의 방랑자들의 플레이가 가능하게 만들 수 있습니다.
\end{document}