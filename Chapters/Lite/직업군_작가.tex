\documentclass{report}

\begin{document}
	이야기꾼의 세계에서 작가는 특별한 위치에 있습니다. 작가는 자신만의 서사를 가지고 있는 존재로, 굳이 소설을 쓰거나 하지 않았더라도 자신이 창작한 세계가 있다면 이에 해당합니다.
	작가는 태초의 이야기에 들어왔을 때, 자신의 서사에 대한 세계의 존재를 알게 됩니다. 이는 이야기로서 존재하며, 그 세계에 대한 서술을 포함합니다.
	
	시스템 이상의 권한을 가질 수 있는 유일한 방법은 한 세계의 작가가 되는 것입니다. 작가는 자신이 만들어낸 세계에 한정하여 시스템 이상의 권능을 가집니다. 서사의 미래를 계산하는 것 뿐 아니라 방향, 설정을 자신이 주무를 수 있게 되는거죠. 자신이 창조한 세계에 들어간 작가는 굳이 그 서사의 역할 이야기를 얻거나 할 필요도 없이 세계의 법칙이 허용하는 한 모든 것을 할 수 있습니다.
	
	하지만, 작가가 서사를 오염시키거나 파괴시켰을 때에는 조금 상황이 달라집니다. 서사를 만든 이는 작가더라도 그를 받아들이는 독자가 반드시 존재합니다. 독자들로 대표될 수 있는 비자명한 이들은 서사 세계의 주민들입니다. 그 독자들이 보았을 때(즉, 개연성상) 이 작가가 추가/수정한 설정이 서사를 너무 크게 바꾸면 작가가 서사를 오염시킨 것으로 취급하고, 서사의 진행 등이 너무 크게 바뀔 수 있으면 서사가 손상/파괴된 것으로 취급합니다.
	작가 본인에 의하여 오염되거나 손상/파괴된 서사는 작가의 권능을 거부합니다. 즉, 작가는 그 서사에 대한 통제 권한을 잃게 되는 것이죠. \ifDLC J.K.롤링의 경우가 이에 해당합니다.\fi
	
	다른 서사 속에서는, 자신의 서사의 “가호”를 받을 수 있습니다. 이 “가호”는 해당 이야기의 서술으로, 들어가 있는 서사의 법칙을 조금 구부릴 수도 있을 것입니다.

\end{document}