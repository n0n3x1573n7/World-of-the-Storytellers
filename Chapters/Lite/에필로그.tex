\documentclass{report}

\begin{document}
	이야기의 방랑자들은 이야기꾼의 세계와 비교했을 때, 보다 단순화된 룰입니다. 특히 스탯의 경우 완전히 없앴고, 시트가 꽤 간소화되었죠. 또한, 주사위가 제거되고 그 랜덤성을 이야기꾼과 길잡이꾼의 심리전으로 대체하였습니다. \hyperlink{wandering-storytellers}{방랑하는 이야기꾼들} 챕터의 \hyperlink{walking-storytellers}{활보하는 이야기꾼} 단순화 규칙을 사용한다면, 걸어다니면서도 할 수 있을 정도로 간단한 룰이기도 합니다. 물론 엄청나게 단순화된 룰이기 때문에 지겨워지기도 쉬울 수 있으나, 이야기의 방랑자들으로 시작한 이야기꾼을 이야기꾼의 세계로 데려가는 것 역시 꽤 흥미로울 것이라 생각됩니다.
	
	이야기의 방랑자들은 다인으로도 할 수 있지만, 판정을 할 때 판정자와 방해자, 둘 간의 심리전으로 모든 판정이 이루어진다는 점 때문에 길잡이꾼과 이야기꾼 1:1, 또는 길잡이꾼은 상황을 전개하는 역할을 맡은 채로 두 이야기꾼 간의 관계를 중심으로 하는 서사를 진행하는 것을 특히 추천드립니다.
	
	다시 한번, 룰 읽느라 수고하셨습니다. 여러분에게 이야기의 가호가 있기를 바라겠습니다.
\end{document}