\documentclass{report}

\begin{document}
	\hypertarget{ability-limit-lite}{}
	\section*{이야기꾼의 권능과 제약}
	이야기꾼들은 다음 권능과 제약을 가지게 됩니다\footnote{이야기꾼의 세계에서의 것과 동일합니다.}.
	
	\smallskip
	
	\begin{minipage}{\textwidth}
		\begin{tabularx}{\textwidth}{c!{\color{black}\vrule}c!{\color{black}\vrule}X}
			\hline
			\textbf{구분} & \textbf{이야기} & \makecell{\centering\textbf{서술}} \\ \hline \hline
			권능 & 소통\index{소통} & 자신의 출신 서사가 아니라면 모든 언어를 이해할 수 있다. \\ \hline
			권능 & 접근\index{접근} & [태초의 이야기]에서 접근 좌표를 아는 서사로 이동하거나, 어떤 서사에서든 [태초의 이야기]로 이동할 수 있다. \\ \hline
			권능 & 거래\index{거래} & 자신의 모든 이야기를 이야기의 규모에 비례하는 적당한 시간을 사용하여 다른 [깨달은 자]들에게 전할 수 있다. \\ \hline
			제약 & 비밀\index{비밀} & [태초의 이야기]와 관련된 그 어떠한 사항도 [깨달은 자]가 아닌 경우 발설할 수 없다. 발설한다면, [잊혀진 자]가 된다. \\ \hline
			제약 & 참견\index{참견} & [침범] 판정을 실패하여 서사를 오염시키면, 서사에서 추방되고, 해당 서사에 대한 권능 [접근]을 빼앗긴다. \\\hline
		\end{tabularx}
		
		\smallskip
		
		\begin{tightcenter}
			\textbf{이야기꾼의 권능과 제약}
		\end{tightcenter}
	\end{minipage}
	
	\section*{이야기꾼의 이야기와 개연성}
	이야기꾼의 이야기는 이야기꾼이 살아온 생애, 경험한 모든 경험, 알고 있는 지식 등을 정의해주는 내용들입니다. 이야기꾼들은 최대 개연성 15\footnote{길잡이꾼의 판단에 따라, 추가로 더 주어져도 괜찮습니다. 하지만 이 이하로 떨어지는 것은 추천드리지 않습니다.}를 가지고 시작하며, 다음으로 최대 개연성을 소모하여 이야기를 얻을 수 있습니다:
	\begin{enumerate}
		\item 새로운 이야기를 얻는다.
			\begin{itemize}
				\item 새로운 이야기를 얻습니다. 이야기는 보통 짧은 구절으로 표현합니다.
				
				\item 이야기를 얻는 데에는 항상 최대 개연성 1을 소모합니다.
			\end{itemize}
		
		\item 가진 이야기에 [서술]을 추가한다.
			\begin{itemize}
				\item 그 이야기로 할 수 있는 일을 한 문장정도로 간단하게 서술하고, 이를 [긍정적] [중립적] [부정적]으로 분류합니다. 이는 이야기꾼의 입장에서 생각하여 결정합니다.
					\begin{itemize}
						\item{} [긍정적] 서술은 이야기꾼에게 많은 경우에 도움이 되는 효과를 가져다주는 서술입니다. 이 서술을 얻는 데에는 최대 개연성 1을 소모합니다.
						\item{} [부정적] 서술은 이야기꾼에게 많은 경우에 해가 되는 효과를 가져다주는 서술입니다. 이 서술을 얻으면, 최대 개연성 1을 얻습니다\footnote{이야기꾼의 세계에서, 트라우마가 이에 해당합니다.}.
						\item{} [중립적] 서술은 [긍정적]이면서 [부정적]인 서술, 또는 [긍정적]이지도 [부정적]이지도 않은 서술을 의미합니다. 이 서술을 얻는 데에는 최대 개연성을 소모하거나 얻지 않습니다.
					\end{itemize}
			\end{itemize}
	\end{enumerate}
	이렇게 얻은 이야기와 서술으로 변한 최대 개연성만큼의 개연성을 가지고 서사에 들어가게 됩니다. 그렇기 떄문에 이 시점에서 최소한 최대 개연성이 1 이상 남아있어야 합니다.
	
	이야기와 그 서술들은 해당하는 이야기꾼에게는 어떤 일이 있어도 사실입니다. 예를 들어, 다음과 같은 이야기를 생각해보겠습니다:
	
	\begin{lite}{엄청난 독서광}
		\positive{수많은 책을 읽어오며, 읽는 속도가 빠르다.}
		
		\neutral{머릿속에 들어 있는 지식을 말하고 싶어 입이 근질거린다.}
		
		\negative{책만 읽어오며 운동신경이 떨어졌다.}
	\end{lite}
	
	[엄청난 독서광]에서, 긍정적 서술은 자료가 주어져 있다면, 그를 분석하는 데에 걸리는 시간을 줄일 수 있는 도움이 되는 서술입니다. 중립적 서술은 아는 것이 많은 반면, 이를 말하고 싶어하는 잘난체하는 면을 보여주죠. 부정적 서술은 지식에 치중한 나머지 신체적인 면을 단련하지 못했음을 드러냅니다. 이처럼 서로 다른 서술은 한 이야기의 다른 면들을 보여줄 수 있습니다.
	
	\bigskip
	
	이야기꾼 시트와 예시 이야기꾼은 \hyperlink{lite-sheets}{캐릭터 시트} 챕터에서 확인하실 수 있습니다.
	
	\section*{등장인물과 이야기꾼}
	서사를 만들어나가기 위해 필요한 등장인물과 이야기꾼 역시 이와 같은 과정으로 만들어도 됩니다. 한 가지 중요한 점은 등장인물의 경우 개연성의 영향을 받지 않는다는 점입니다. 하지만 밸런스를 위해 적절한 최대 개연성 수치를 사용하는 것을 추천드립니다. 예를 들어 스토리의 최종 흑막 등 특수한 경우에는 최대 개연성 수치를 15가 아닌 20\textasciitilde30 정도로 하여 만들 수 있을 것입니다.
	
	물론 이런 이들은 이야기꾼에게 있는 개연성의 제약을 모종의 방법으로 덜었다는 설정을 주어, 이야기꾼과 같은 서술식 시트가 아닌 능력을 구체화하고 기믹을 지정하는 보다 전통적인 시트를 만들수도 있을 것입니다. 이러한 경우, 이야기꾼의 세계와는 다르게 이야기의 방랑자들에서는 어떻게든 무언가를 할 수 있는 구조로 되어 있으므로 밸런스 등을 맞추는 데에 큰 힘을 들이지 않고, 비교적 이야기꾼들이 즐기게 될 서사와 이야기에 집중할 수 있을 것입니다.
\end{document}