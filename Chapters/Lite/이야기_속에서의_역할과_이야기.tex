\documentclass{report}

\begin{document}
	서사 속에 들어간 이야기꾼들은 서사 속의 역할에 맞추어 추가적인 이야기를 받을 수 있습니다. 이 새로운 역할은 기본적으로 최대 개연성에 영향을 끼치지 않고, 처음부터 주어지기도 하며, 나중에 필요에 의해 지급되거나 자신이 선택할 수도 있을 것입니다. 이 역할들은 간단하게 이야기만을 주기도 하지만, 새로운 능력이나 트라우마, 행동 제약 등을 제시할 수도 있습니다. 간단한 예시로, 사진 작가인 이야기꾼이 사진기가 발명된 직후의 세계의 이야기에에 카메라를 들고 들어간다면 카메라가 커다란 구형 카메라로 변한다거나 하는 페널티가 생길 수 있을 것입니다.
	
	\bigskip
	
	보다 구체적인 예시로, 이야기꾼이 생명을 지켜야 하는 신성한 직업인 사제가 된 경우, 다음 이야기들을 부여할 수 있습니다:
	
	\begin{lite}{깨어난 자}
		\positive{어떤 상황에서도 이성을 잃지 않는다.}
		
		\negative{생명체에게 피해를 입힐 수 없다.}
	\end{lite}
	
	역할이 이야기를 부여하는 것 외에도, 서사 속에 들어가면 그 서사에 맞는 이야기들이 강제로 적용될 수 있습니다. 예를 들어 불안정한 공간 속에서 일어나는 서사의 경우, 다음 이야기를 얻게 될 수 있습니다:
	\begin{lite}{비정형의 공간}
		\negative{균형을 잡기 힘들다.}
	\end{lite}
	
	이처럼 서사 속에서 역할을 부여받으면 그 역할에 맞는 이야기와 능력을 얻음으로서 해당 이야기의 세계를 보다 생생하게 체험하게 해 줄 수 있습니다.
\end{document}