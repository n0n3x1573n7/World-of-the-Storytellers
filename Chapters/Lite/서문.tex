\documentclass{report}

\begin{document}
	이야기의 방랑자들은 RPG\footnote{Role Playing Game. TRPG/TTRPG(TableTop RPG), ORPG(Online RPG)로도 알려져 있다.}입니다. 여러명이 모여서 캐릭터를 만들고, 그들이 세계와 서로 상호작용을 하며 이야기를 만들어나가는 게임이죠. 이야기의 방랑자들에서는 다른 게임에서 GM\footnote{Game Master. DM(Dungeon Master)이라고도 부른다.}이라고 부르는 ``이야기를 이끌어나가는 이"를 \emph{길잡이꾼[Guide]}라고 부릅니다. 플레이어들이 조종하는 PC\footnote{Player Character}라고 불리는 캐릭터들은 \emph{이야기꾼[Storytellers]}이라고 부릅니다. 여러분은 길잡이꾼이 되어 서사의 틀을 잡고 이끌어 나가거나, 이야기꾼이 되어 그 서사 속에서 여러분의 캐릭터들이 행동하도록 할 것입니다.
	
	이야기의 방랑자들은 이야기꾼의 세계가 배경으로 하는 태초의 이야기에 기반하여, 보다 더 가볍게 즐길 수 있는 룰을 만드는 것을 목표로 합니다. 이야기꾼의 세계 룰을 모두 읽으신 후 읽는 것을 추천드리나, 다음 이야기꾼의 세계 챕터들의 내용은 이야기의 방랑자와 많은 부분을 공유하므로, 이 챕터들은 반드시 먼저 읽으시는 것을 추천드립니다:
	\begin{itemize}
		\item \hyperlink{story-progression}{서사의 진행}
		\item \hyperlink{power-limit}{권능과 제약}
		\item \hyperlink{author}{직업군: 작가}
	\end{itemize}
\end{document}