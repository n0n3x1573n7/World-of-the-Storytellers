\documentclass{report}

\begin{document}
	이야기의 방랑자들에서는 기본적으로 주사위, 카드 등을 사용하지 않고 판정합니다. 모든 판정은 \emph{판정자}와 \emph{방해자}간의 대결으로 구성됩니다. \emph{판정자}는 행동에 대한 판정을 하는 이를, \emph{방해자}는 그 행동에 대한 대응을 하는 이를 의미합니다. 판정자와 방해자는 전투의 상황에서는 각각 공격자와 방어자가, 이야기꾼이 행하는 일반적인 판정의 경우에는 각각 이야기꾼과 길잡이꾼이 판정자와 방해자의 역할을 맡게 됩니다.
	
	판정자와 방해자가 정해지면, 두 명은 동시에 1\textasciitilde10 중 한 숫자를 말합니다. 이는 두 명이 동시에 원하는 숫자의 손가락을 피는 등으로 행할 수 있고, 원한다면 d10이나 플레잉 카드 등을 사용하여 동시에 공개하는 것으로 할 수도 있습니다. 두 수를 합친 값이 10 이하라면 해당 숫자가 판정 결과값이 됩니다. 두 수를 합친 값이 11 이상이라면, 두 수에서 10을 뺀 값이 판정 결과값이 됩니다. 따라서, 판정치는 항상 1\textasciitilde10 중 한 숫자가 됩니다.
	
	보정치는 판정 결과값에 추가되는 값입니다. 이야기, 보다 정확하게는 이야기의 서술의 도움은 이 때에 받을수 있습니다\footnote{이야기의 제목 역시 하나의 서술로 취급할 수 있습니다.}. 해당 상황을 도와주는 서술 하나당 +1, 방해하는 서술 하나당 -1을 받습니다. 이 보정치에서 한 이야기는 한 개의 서술만으로만 도움을 줄 수 있지만, 한 이야기로부터 방해받을 수 있는 횟수는 제한되지 않습니다. 이 서술의 도움이나 방해는 거리나 상황이 허용한다면, 다른 이야기꾼으로부터 받을 수도 있습니다.
	
	서술의 도움뿐 아니라, 판정의 결과 합이 11 이상인 경우에는 추가 보정치가 주어집니다. 판정자가 방해자보다 작은 수를 제시했을 경우 판정에 +1을, 방해자가 판정자보다 작은 수를 제시했을 경우 판정에 -1을 가합니다. 같은 수를 제시했을 경우에 이 보정치는 주어지지 않습니다.
	
	판정치에 이런 모든 보정치를 더한 것이 최종 결과치가 되어, 이 수치를 이용해 판정을 진행하게 됩니다.
	
	\section*{판정치와 목표치}
	보정 없이 판정으로 얻을 수 있는 최대치인 ``10"은 해당 서사의 평균적인 등장인물이 낼 수 있는 극단적인 최대 위력을 의미합니다. 10 이상의 결과치는 매우 운이 좋거나 어떤 도움이 없다면 나올 수 없는 결과치입니다. 일반적인 경우, 7 이상의 결과치는 운이 좋지 않다면 나오기도 힘들 것입니다. 아래는 목표치에 대한 간략한 설명입니다:
	
	\begin{minipage}{\textwidth}
		\begin{tabularx}{\textwidth}{c!{\color{black}\vrule}X}
			\hline
			\textbf{판정치} & \makecell{\centering\textbf{난이도}} \\ \hline \hline
			0 이하 & 아무 힘도 들이지 않고도 할 수 있습니다. \\ \hline
			1 \textasciitilde 2 & 조금만 집중해도 할 수 있습니다. \\ \hline
			3 \textasciitilde 4 & 큰 힘을 들이지 않고도 가능합니다. \\ \hline
			5 \textasciitilde 6 & 힘은 들지만, 할 수는 있습니다. \\ \hline
			7 \textasciitilde 8 & 엄청난 힘이 듭니다. \\ \hline
			9 \textasciitilde 10 & 죽을힘을 다해야 가능합니다. \\ \hline
			10 이상 & 일반적으로 불가능합니다. \\ \hline
		\end{tabularx}
		
		\smallskip
		
		\begin{tightcenter}
			\textbf{목표치에 따른 행위의 난이도}
		\end{tightcenter}
	\end{minipage}
	
	\section*{임시 상태}
	임시 상태는 서사를 진행하는 중에 받을 수 있는 서술입니다.
	
	임시 상태는 판정에서 크게 성공하거나 실패했을때, 후술할 대결을 하면서, 또는 다른 여러 이유들으로 얻을 수 있습니다. 서술과 같은 방법으로 사용할 수 있으며, 판정에 보정치로 +1 또는 -1을 부여할 수 있지만, 빠르게는 한번 사용한 후, 아무리 늦어도 해당 씬이 종료될 때 사라집니다.
	
	\section*{(선택 규칙) ``0"의 판정}
	판정을 1\textasciitilde10으로 하는 대신, 0\textasciitilde10 또는 0\textasciitilde9로, 0을 포함시키는 방법을 생각해볼 수 있습니다.
	
	0을 냈다고 하더라도, 판정값의 결정은 동일하게 진행됩니다. 즉, 0을 낸 이는 해당 판정에 유의미한 기여를 할 수 없습니다. 하지만, 0을 낸 이는 판정이 종료된 이후, 판정자에게 해당 판정 중 일어난 일에 연관있는 유의미한 상태를 부여할 수 있습니다. 판정자가 냈다면 긍정적으로 사용할 수 있는 상태를, 방해자가 냈다면 부정적으로 사용할 수 있는 상태를 부여할 수 있습니다.
	
	예외적으로, 판정자와 방해자가 둘 다 모두 0을 냈다면, 양쪽이 서로 부여하는 상태는 상쇄되어 사라집니다. 하지만, 판정자는 해당 판정의 판정치를 10으로 취급합니다. 방해자가 완전히 등을 돌린 틈을 타 자신의 목적을 완수하는 것이죠.
	
	``0"을 판정치로 내면, 상대의 수치에 완전히 판정을 의존하게 됩니다. 하지만 이를 허용하게 되면 메타적인 측면에서는 심리전에 흥미로운 추가 규칙을 부여하고, 전략적인 포기 이후 판정에 사용할 수 있는 유의미한 상태를 받게 된다는 전략적인 면이 추가됩니다.
	
	\section*{(선택 규칙) 추가 보정치}
	한 이야기나 한 서술이 보정치를 +1까지만 가할 수 있다는 규칙을 무시할 수 있습니다. 특히, 크게 도움을 줄 수 있는 서술은 보정치를 그 이상으로 받거나, 보정치 대신 임시 상태를 받는 식으로 처리할 수 있습니다. 이 규칙은 이야기꾼의 세계의 \hyperlink{emersion}{재현}과 유사한 규칙입니다.
	
	\section*{부상}
	부상은 서사를 진행하는 중, 특히 전투를 하는 중에 받을 수 있는 서술입니다.
	
	부상은 심각도에 따라 작은 부상, 중간 부상, 큰 부상으로 구분됩니다. 이를 받으면 해당 심각도에 따라 개연성을 소진하게 됩니다. 아래 표를 참고하세요:
	
	\begin{minipage}{\textwidth}
		\begin{tabularx}{\textwidth}{c!{\color{black}\vrule}c!{\color{black}\vrule}X}
			\hline
			\textbf{부상의 종류} & \textbf{개연성 소진량} & \makecell{\centering\textbf{회복 시기}} \\ \hline \hline
			작은 부상 & 1 & 씬이 끝나면 회복됩니다. \\ \hline
			중간 부상 & 2 & 씬이 끝나면 작은 부상으로 경감되며, 서사에서 나오면 회복됩니다. \\ \hline
			큰 부상 & 3 & 씬이 끝나면 중간 부상으로 경감되며, 서사에서 나오면 반드시 부정적 서술로서 해당 부상을 받아야 합니다. \\ \hline
		\end{tabularx}
		
		\smallskip
		
		\begin{tightcenter}
			\textbf{부상}
		\end{tightcenter}
	\end{minipage}
	
	개연성이 0이 되면 서사 속에서 죽음을 맞게 되고, 이야기꾼은 서사에서 추방되게 됩니다.
	
	부상은 부정적인 서술과 같이 방해자가 역이용하여 방해하는데에 사용할 수 있습니다.
	
	\section*{대결}
	대결은 두 캐릭터 이상 사이에 발생한 갈등 상황을 의미합니다. 이는 전투를 포함합니다. 대결을 할 때의 순서는 판정 없이 해당 상황을 빠르게 타개할 수 있는 데에 도움이 되는 서술의 수로 결정합니다. 이 때에 한하여 한 이야기당 한 개의 서술만을 사용할 수 있다는 제한이 없습니다.
	
	대결을 할 때에 다른 인물과의 판정을 할 때에는, 공격과 방어에 대한 판정을 두 번 하는 대신, 한 번의 판정으로 피해량과 결과 등을 계산합니다. 공격을 하는 이가 판정자, 수비를 하는 이가 방해자가 되어 판정을 진행한 후, 해당 수치에서 5를 뺀 수치에 따라 대결의 결과가 정해집니다\footnote{대결의 목표치는 항상 5로 설정되어 있다고 생각해도 됩니다.}.
	
	대결의 결과 나온 수치가 0 미만, 즉 판정값이 4 이하라면, 방해자는 성공적으로 판정자의 행동을 방해한 것으로 취급합니다. 공격을 당한 경우 성공적으로 회피한 것입니다. 숫자가 작을수록 더 성공적으로 방해하거나 회피한 것으로, 방해자는 절대값을 취한 값의 반에 해당하는 수치만큼 다음 판정에 추가 보정치를 받습니다. 예를 들어 결과가 -3이라면, 소숫점 아래를 버린 +1의 보정치를 다음 판정에 받습니다. 이 보정치는 턴이 다시 돌아오기 전까지 쌓은 보정치 중 가장 높은 보정치 하나만 적용됩니다.
	
	대결의 결과 나온 수치가 0 초과, 즉 판정값이 6 이상이라면, 판정자가 방해에도 불과하고 성공적으로 행동을 이행한 것입니다. 숫자가 클수록 더 성공적으로 수행한 것으로, 해당 값을 사용하여 다음 행동 중 전부 또는 일부를 선택할 수 있습니다. 단, 중복 선택은 불가능합니다:
	
	\begin{minipage}{\textwidth}
		\begin{tabularx}{\textwidth}{c!{\color{black}\vrule}X}
			\hline
			\textbf{사용할 판정치의 값} & \makecell{\centering\textbf{할 수 있는 일}} \\ \hline \hline
			1 & 본인에게 해당 행동과 관련된 상태를 부여합니다. \\ \hline
			1 & 대상에게 해당 행동과 관련된 상태를 부여합니다. \\ \hline
			1 & 본인에게서 해당 행동과 관련된 상태를 제거합니다. \\ \hline
			1 & 대상에게서 해당 행동과 관련된 상태를 제거합니다. \\ \hline
			2 & 대상에게 작은 부상을 입힙니다.\\ \hline
			4 & 대상에게 중간 부상을 입힙니다. \\ \hline
			6 & 대상에게 큰 부상을 입힙니다. \\ \hline
		\end{tabularx}
		
		\smallskip
		
		\begin{tightcenter}
			\textbf{대결시 판정값에 따라 할 수 있는 일}
		\end{tightcenter}
	\end{minipage}
	
	대결의 결과 나온 수치가 0, 즉 판정값이 5라면, 판정자에게는 아래 중 하나를 선택합니다:
	\begin{enumerate}
		\item 대결의 결과가 나온 값을 1으로 취급하여 판정에 성공하는 대신, 상대 역시 위의 표에서 1에 해당하는 만큼의 효과를 적용시킬 수 있습니다.
		\item 또는, 판정에 실패한 것으로 취급합니다.
	\end{enumerate}
	
	\section*{부상의 회복}
	부상의 자연 회복이 아닌, 이야기꾼의 능력이나 치료 등으로 부상을 회복할 때 역시 판정을 하게 됩니다. 회복시키는 것을 시도하는 이야기꾼이 판정자, 길잡이꾼이 방해자가 되어 대결을 합니다. 해당 결과에 따라, 다음과 같이 부상을 회복할 수 있습니다:
	
	\begin{minipage}{\textwidth}
		\begin{tabularx}{\textwidth}{c!{\color{black}\vrule}X}
			\hline
			\textbf{사용할 판정치의 값} & \makecell{\centering\textbf{할 수 있는 일}} \\ \hline \hline
			1 & 대상에게서 해당 치료와 관련된 상태를 제거합니다. \\ \hline
			2 & 대상에게서 작은 부상을 치료하여 제거합니다.\\ \hline
			4 & 대상에게서 중간 부상을 치료하여 제거합니다. \\ \hline
			6 & 대상에게서 큰 부상을 치료하여 제거합니다. \\ \hline
		\end{tabularx}
		
		\smallskip
		
		\begin{tightcenter}
			\textbf{치료시 판정값에 따라 할 수 있는 일}
		\end{tightcenter}
	\end{minipage}
	
	음수가 나오게 되면, 치료 방법에 따라 상태 또는 부상이 심화되게 할 수도 있습니다.
	
	
	\ifprintout
	\section*{(선택 규칙) 발악\footnote{이 이야기를 제안해주신 소낙님께 감사드립니다.}}
	
	이야기가 늘어질 때 아래 규칙을 사용할 수 있습니다.
	
	\begin{lite}{모 아니면 도}
		
		\entry{대결을 하기 전, 직전의 대결에서 패배했다면 [발악]을 하기로 선택할 수 있다. 이 경우, 해당 패배한 대결을 포함하여 몇번이나 연속으로 패배했는지 그 수를 센다.}
		
		\entry{[발악]을 한 대결에서 승리한다면, 위에서 센 수만큼의 추가 보정치를 받는다. 이를 임시 상태 또는 부상을 주는 데에 사용할 수 있다.}
		
		\entry{[발악]을 한 대결에서 패배한다면, 상대는 위에서 센 수만큼의 추가 보정치를 받는다. 이를 임시 상태 또는 부상을 주는 데에 사용할 수 있다.}
	\end{lite}
	\fi
	
	\section*{판정의 예시}
	이야기꾼 A가 장애물을 뛰어넘는 상황을 생각해봅시다. 이 경우 A가 판정자, 길잡이꾼(B)이 방해자가 됩니다.
	
	A가 7, B가 2를 제시했다면 판정값은 9가 됩니다.
	
	A가 9, B가 2를 제시했다면 판정값은 1이 되며, 여기에 방해자가 더 작은 수를 제시했기 때문에 -1의 보정치가 추가됩니다.
	
	A가 9, B가 10을 제시했다면 판정값은 9가 되며, 여기에 판정자가 더 작은 수를 제시했기 때문에 +1의 보정치가 추가됩니다.
	
	A가 8, B가 8을 제시했다면 판정값은 6이 되며, 두 수가 같으므로 추가 보정치는 없습니다.
	
	A가 다음 이야기를 가지고 있다고 가정해보겠습니다:
	\begin{lite}[runner]{단거리 달리기 선수}
		\positive{짧은 시간동안 빠른 속도로 이동할 수 있다.}
	\end{lite}
	
	\storyref{runner}{단거리 달리기 선수}의 서술으로 도움닫기를 한다면 +1 보정치를 받을 수 있을 것입니다. 만약 A가 ``발목을 삐끗함"과 같은 부상을 가지고 있다면, -1 보정치가 가해질 것이고요.
	
	다른 이야기꾼 C가 다음과 같은 이야기를 가지고 있다고 생각해봅시다:
	\begin{lite}[magic-music]{음악의 마술사}
		\positive{[가속의 음악]을 통해 자신과 주위 이야기꾼의 속도를 빠르게 할 수 있다.}
		
		\negative{[진정의 음악]을 통해 자신과 주위 이야기꾼의 속도를 느리게 할 수 있다.}
	\end{lite}
	
	\storyref{magic-music}{음악의 마술사}의 [가속의 음악]에 대한 서술을 사용한다면 B는 해당 판정에 +1 보정치를 가할 수 있을 것입니다. 만약 반대로, B가 [진정의 음악]에 대한 서술을 사용한다면 B는 해당 판정에 -1 보정치를 가할 수 있습니다.
	
	\section*{늘어지는 전투}
	이야기의 방랑자들의 전투는 판정의 특성상 전투가 늘어지는 경우가 많습니다. 전투가 반드시 필요한 경우가 아니라면 전투를 지양하는 것을 추천드리나, 반드시 필요한 경우 아래 방법들을 사용하는 것을 추천드립니다:
	\begin{itemize}
		\item 적의 섬멸이 아닌 전투 종료 조건의 존재
		\begin{itemize}
			\item 예를 들어, 시간이 지나면 강제 종료되는 전투
		\end{itemize}
		\item \storyref{wot:all-or-nothing}{모 아니면 도}와 같은 가속용 이야기의 사용
	\end{itemize}
\end{document}