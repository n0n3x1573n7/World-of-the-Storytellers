\documentclass{report}

\begin{document}
	이야기꾼은 항상 변화하고, 성장합니다. 서사 속에서 죽음을 맞기도 하고, 서사 속에서 새로운 이야기를 얻으며 성장하기도 합니다.
	
	\section*{죽음}
	이야기꾼의 개연성이 0이 되면 해당 서사에서 죽음을 맞이하여, 추방됩니다. 해당 시점에서 다음이 모두 일어납니다:
	\begin{itemize}
		\item 죽은 원인 또는 상황에 대한 이야기 또는 서술을 얻습니다. 또는, 받은 부상 중 가장 큰 부상이나 가장 잦게 받은 부상에 대한 것이어도 괜찮습니다.
		\item 서사의 역할으로 인해 얻은 이야기를 모두 잃습니다.
		\item 받은 모든 임시 상태를 잃습니다.
		\item {}[태초의 이야기]에서 다시 나타납니다. 해당 씬이 끝난 뒤에, 재진입이 가능합니다.
	\end{itemize}
	
	\section*{미미한 성장}
	이야기꾼이 전투를 겪었거나, 이야기의 변화를 겪었다면 일어납니다. 길잡이꾼의 허가 하에 최대 개연성을 소모하고 방금 있었던 전투나 변화에 어울리는 이야기나 서술 하나를 얻거나, 이미 존재하는 이야기나 서술 하나를 바꿀 수 있습니다. 미미한 성장으로 서술을 변화시키거나, 이야기를 잃을 수는 없습니다.
	
	\section*{작은 성장}
	서사 하나가 끝날 때 마다 이야기에서 얻는 보상과는 별개로 다음 중 하나를 할 수 있습니다:
	\begin{itemize}
		\item 최대 개연성을 소모하고, 이야기를 하나 얻습니다.
		\item 최대 개연성을 소모하고, 이미 있는 이야기에 서술을 추가합니다.
		\item 서술 하나를 다른 서술로 바꾸거나, 이야기 하나를 다른 이야기로 바꿉니다.
	\end{itemize}
	
	\section*{중간 성장}
	시스템의 인정을 받는다면(보통 서사 두세개가 끝날때마다 한번) 다음을 모두 할 수 있습니다:
	\begin{itemize}
		\item 최대 개연성 2를 얻습니다.
		\item 작은 성장을 합니다.\footnote{\label{lite-medium-upgrade-small-upgrade}오타 아닙니다. 작은 성장의 선택지 중 한 가지를 선택해서 적용시키는 것을 총 두번 할 수 있습니다.}
		\item 작은 성장을 합니다.\footnoteref{lite-medium-upgrade-small-upgrade}
	\end{itemize}
	
	\section*{큰 성장}
	시스템의 위기를 타파할 때 마다(보통 서사 대여섯개정도가 끝날때마다 한번) 다음을 모두 할 수 있습니다:
	\begin{itemize}
		\item 최대 개연성 2를 얻습니다.\footnote{\label{lite-big-upgrade-cost}즉, 기본 최대 개연성 4를 얻습니다.}
		\item 중간 성장을 합니다.\footnoteref{lite-big-upgrade-cost}
		\item 작은 성장을 합니다.
	\end{itemize}
	
	\section*{서사의 보상}
	이야기꾼들은 서사 속에서 보상으로 이야기를 받을 수 있습니다. 단, 이 보상으로 인해서도 최대 개연성은 변화합니다. 따라서, 보상 이야기는 중립적인 서술 또는 개연성에 크게 변화를 가하지 않는 이야기로 하는 것으로 추천드립니다.
\end{document}