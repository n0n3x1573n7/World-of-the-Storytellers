\documentclass{report}

\begin{document}
	\begin{tabular}{|p{2cm}|p{2cm}|p{2cm}|p{2cm}|p{2cm}|p{2cm}|}
		\hline
		&     &  &     &  &    \\ \hline
		&     &  &     &  &    \\ \hline
		사다리 & 도적2 &  & 도적1 &  & 두목 \\ \hline
		&     &  &     &  &    \\ \hline
		&     &  &     &  &    \\ \hline
	\end{tabular}
	
	모든 칸에 해당하는 이야기입니다:
	\begin{story}{어지럽게 얽히고설킨 창고}{[장소]}
		두 번까지 이동할 수 있습니다.
		
		바닥을 통해 이동할 때, 4df를 굴립니다. -2 이하의 결과가 나오면 다음 자신의 턴까지 유지되는 물리적인 부정적 상태 [균형을 잃음]을 얻고, 이번 턴에는 더 이상 이동할 수 없습니다.
	\end{story}
	
	\bigskip
	
	사다리 칸에 해당하는 이야기입니다:
	\begin{story}{사다리}{[사물]}
		고정된 철제 사다리가 놓여있는 칸입니다. 한 턴을 소모해 로비로 이동할 수 있습니다.
	\end{story}
	
	만약 진입한 이야기꾼 중 무기를 [시간의 신 Chronos의 저주]에 의해 빼앗긴 이가 있다면, 지하 수장고 내의 무작위 칸에 무기가 숨겨져 있습니다. Chronos는 이 무기의 위치를 요청한다면 알려줄 것이지만 이 무기를 회수하는 것은 이야기꾼의 몫입니다.
\end{document}