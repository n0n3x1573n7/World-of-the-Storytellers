\documentclass{report}

\begin{document}
	
	\section*{지상 전시실 지도}
	\begin{tabular}{|p{3cm}|p{3cm}|p{3cm}|p{3cm}|p{3cm}|p{3cm}|}
		\hline
		\multirow{5}{*}{로비} & \multicolumn{2}{p{3cm}|}{\multirow{2}{*}{보석 전시실}} & \multicolumn{2}{p{3cm}|}{\multirow{2}{*}{마법 전시실}} & \multirow{5}{*}{서버실} \\
		& \multicolumn{2}{p{3cm}|}{}                        & \multicolumn{2}{p{3cm}|}{}                        &                      \\ \cline{2-5}
		& \multicolumn{4}{p{6cm}|}{복도}                                                                     &                      \\ \cline{2-5}
		& \multicolumn{2}{p{3cm}|}{\multirow{2}{*}{기술 전시실}} & \multicolumn{2}{p{3cm}|}{\multirow{2}{*}{종교 전시실}} &                      \\
		& \multicolumn{2}{p{3cm}|}{}                        & \multicolumn{2}{c|}{}                        &                      \\ \hline
	\end{tabular}
	
	\bigskip
	
	박물관의 네 가지 전시실에는 각각 네 가지씩의 아티팩트들이 특별 전시품으로서 전시되어 있습니다. 이 아티팩트들은 이야기꾼에 따라 변경하는 것을 권장하며, 이야기꾼들이 \emph{사용하고 싶도록} 만들어야 합니다. 예를 들어, 이야기꾼들의 능력의 페널티를 상쇄시킨다거나 하는 식으로요. 아래의 표에는 존재할만한 아티팩트들을 나열해두었습니다.
	
	\section*{보석 전시실}
	\begin{tabularx}{\textwidth}{l|l|X|l|l|l}
		\textbf{속성} & \textbf{명칭} & \textbf{능력} & \textbf{스탯 +} & \textbf{스탯 -} & \textbf{코스트}\\ \hline \hline
		[저주][보석]& 루비   & 소유자는 한 턴에 한 번, 개연성을 1 소모하고 한 구역 내의 모든 대상에게 회피 불가능의 물리 또는 정신 피해를 1 줄 수 있다.   & 자본     & & 0    \\ \hline
		[저주][보석]& 사파이어   & 소유자는 한 턴에 한 번, 개연성을 1 소모하고 한 구역 내의 모든 대상에게 [감전됨 □] 물리상태 또는 [멍해짐 □] 정신상태를 줄 수 있다.   & 자본     & & 0    \\ \hline
		[저주][보석]& 오팔   &  소유자는 한 턴에 한 번, 개연성을 1 소모하고 한 구역 내의 자신을 제외한 모든 대상의 체력 또는 정신력을 1 회복시킬 수 있다. 개연성은 회복시킬 수 없다.  & 자본     & & 0    \\ \hline
		[저주][보석]& 에메랄드   & 소유자는 한 턴에 한 번, 개연성을 1 소모하고 이동을 1회 추가로 할 수 있다.   & 자본     & & 0    \\ 
	\end{tabularx}
	
	\section*{기술 전시실}
	\begin{tabularx}{\textwidth}{l|l|X|l|l|l}
		\textbf{속성} & \textbf{명칭} & \textbf{능력} & \textbf{스탯 +} & \textbf{스탯 -} & \textbf{코스트}\\ \hline \hline
		[기술][생물]& 기계 공생체 & 한 턴을 소모해 혈액에 심을 수 있다.\newline 심기면 훔칠 수 없어지며, 이동을 포기하면 보호막 3을 얻을 수 있고, 다음 스탯에 변화를 준다: \newline \textbf{스탯+}: 기민, 근력 \newline \textbf{스탯-}: 의지, 공감, 인식  &   &     & 0 \\ \hline
		[기술][환상]& 테서렉트 & 누군가 개연성 판정에 실패할 때, 테서렉트의 코스트가 2 증가한다. \newline 테서렉트의 코스트가 0이 되면 테서렉트가 폭발하며 시간이 잠시 멈춘다. 즉시 한 턴을 진행한다. &  &      & -10 \\ \hline
		[기술][안정]& 댐퍼 & 자신의 턴이 종료될 때, 4df를 굴려 해당 값의 절대값에 1을 뺀 만큼의 개연성을 회복할 수 있다.  &  &      & 0 \\ \hline
		[기술][무기]& 죽음의 키스 & 단 한 번 발사할 수 있는 저격총. 사격 또는 사격:총기의 두 배 중 높은 쪽으로 판정하고, 인식과 기민 중 낮은 쪽으로 회피한다. 적중한다면, 해당 적의 체력을 1 남기고 모두 잃게 한다.  &  &      & 0 \\ \hline
		[기술][무기]& 레이저 건 & 턴당 한 번, 시야가 확보된 대상에게 회피 불가능한 피해 1을 주는 레이저를 발사한다. &  & & \\
	\end{tabularx}
	
	\section*{마법 전시실}
	\begin{tabularx}{\textwidth}{l|l|X|l|l|l}
		\textbf{속성} & \textbf{명칭} & \textbf{능력} & \textbf{스탯 +} & \textbf{스탯 -} & \textbf{코스트} \\ \hline \hline
		[마법][마나]& 불안정한 수정 & 마나를 사용한다면, 최대 마나가 10\% 증가한다. 이 아티팩트를 공중으로 던지면 폭발하여 자신 외의 같은 구역 안에 있는 모든 이에게 [실명됨: 1턴]을 준다. &  & & 0 \\ \hline
		[마법][목걸이]& 예지의 목걸이 & 착용자는 회피와 조준 판정에 +1을 받는다. 이 목걸이를 파괴함으로서 자동 성공을 결과로 가질 수 있다. &  & & 0 \\ \hline
		[흑마법][혈액] & 응고된 혈액 & 매 턴 정신력 1을 소모한다. 정신력이 0이 되면 이 아티팩트는 영구히 소실된다. &  & & -10 \\ \hline
		[마법][시계] & 시간의 회중시계 & 자신의 턴에 주사위에 의한 판정([행운] 등)을 할 때, 두 번 굴려 그 중 하나를 선택할 수 있다.& 속도 & & 0 \\ \hline
		[마법][목걸이]& 민첩의 목걸이 & 착용자는 회피 판정이나 조준 판정을 함에 있어 +1을 받는다. &  & & 0 \\
	\end{tabularx}
	
	\section*{종교 전시실}
	\begin{tabularx}{\textwidth}{l|l|X|l|l|l}
		\textbf{속성} & \textbf{명칭} & \textbf{능력} & \textbf{스탯 +} & \textbf{스탯 -} & \textbf{코스트}\\ \hline \hline
		[신성][십자가]& 순교자의 십자가 & 십자가를 소유한 상태로 이야기가 봉쇄되면, 방어막 3을 얻는다. &  & & 0 \\ \hline
		[신성][묵주]& 대주교의 묵주 & 묵주를 소유한 상태로 이야기가 봉쇄되면, 자신을 포함한 한 대상의 체력 2를 회복시킨다. &  & & 0 \\ \hline
		[신성][기도]& 성기사의 방패 & 구역 내에서 방패를 들고 무릎을 꿇은 채로 정신을 집중하고 있는 동안, 해당 구역에서 나갈수도 들어올 수도 없는 방벽이 생성된다. 이 방벽은 정신집중을 해제하거나, 안팎을 통틀어 10의 피해를 받으면 사라진다.  &  & & 0 \\ \hline
		[신성][토템]& 대정령의 토템 & 한 턴을 소모해 토템을 설치하거나 철거할 수 있다. 설치된 상태에서 같은 구역에 있는 모든 이들은 체력 1을 정신력 1, 또는 정신력 1을 체력 1으로 바꿀 수 있다. &  & & 0 \\
	\end{tabularx}
\end{document}