\documentclass{report}

\begin{document}
	이야기꾼들에게 비공개된 정보의 텍스트는 빨간색으로 표시되어 있습니다.
	
	\begin{spoiler}{시간의 끝}{[시간]}
		\entry[\hline]{이 이야기는 이미 시간의 끝을 향해 달려가고 있기 때문에 이야기의 개연성을 해칠 염려가 매우 적다. 따라서, [침범] 판정에 실패할 때, 다음 씬(비전투) 또는 두 턴 후(전투)까지 해당 판정에 실패한 이야기가 전면 봉쇄되어, 기술 뿐 아니라 해당 이야기로부터의 도움도 받을 수 없다(단, 스탯은 유지된다). 개연성 판정 난이도의 초기화는 이야기의 봉쇄가 일어나면 즉시 일어나나, 이야기가 불안정해짐에 따라 판정이 일어날 때 마다 난이도가 1 상승한다.}
	\end{spoiler}
	
	\begin{spoiler}{비정형의 공간}{[공간]}
		\entry[\hline]{우주 상에 떠다니는 불규칙한 공간이기 때문에 명중 또는 회피 판정을 할 때에는 별도의 해당 페널티를 상쇄할 만한 이야기가 없다면 명중/회피 페널티로서 4df를 굴려 해당 수치를 판정치에 반드시 더해야 한다.}
	\end{spoiler}
	
	\begin{spoiler}{시간의 신 Chronos의 저주}{[저주: 신]}
		\entry{시공간 이동에 관련된 모든 기술, 마법, 흑마법, 초능력 등이 이곳에서는 봉인된다.}
		
		\entry[\hline]{\textcolor{Red}{박물관의 시설물을 파괴할 수 있는 기술과 무기들은 즉시 압수되어 해당 인물들이 박물관 안에 있는 동안 전시 품목으로 추가되며, 허가받지 않고 이들을 훔치고자 한 이들은 모두 즉시 시간상에서 사라진다.}}
	\end{spoiler}
\end{document}