\documentclass{report}

\begin{document}
	아래의 모든 내용은 이 이야기에 들어오게 될 이야기꾼에 맞추어 개변하실 수 있습니다. 다만, 시나리오의 큰 흐름을 변경하거나, 개변한 시나리오를 재배포하지는 말아주세요.
	
	\section*{박물관 배경 지식}
	\world{
		크로노스 박물관.
		
		시간의 끝으로부터 멀지 않은 시간의 찰나에 존재하는, 모든 우주와 시간의 아티팩트를 모아둔, 시간 상에서 정지되어 있는 박물관이다. 모든 시간 여행자들의 일기장이자, 꿈이고, 고향과도 같은 곳이며, 수많은 도둑들이 호시탐탐 노리고 있는 곳이다.
		
		이 곳이 많은 침입을 받았으나 대부분의 경우 실패한 이유는 박물관 내부에서의 시공간에 관련된 모든 능력을 이 박물관의 가장 중요한 규칙, [시간의 신 Chronos의 저주]가 막아버리기 때문이다. 시공간 이동에 관련된 모든 기술, 마법, 흑마법, 초능력 등이 이곳에서는 봉인되며, 박물관의 시설물을 파괴할 수 있는 기술과 무기들은 즉시 압수되어 해당 인물들이 박물관 안에 있는 동안 전시 품목으로 추가된다.
		
		이 박물관은 여러 프로토콜들로 자동화되어 운영되고 있다. 그 중 대표적으로는 시간상에서 아티팩트를 수집해오는 Kairos 프로토콜, 그 아티팩트를 보존하는 Aion 프로토콜이 있다.
		
		아티팩트들은 위험도에 따라 Atropos, Clotho, Lachesis의 세 단계로 나뉘는 Moirai 척도로 분류된다. Moirai 척도의 기준은 해당 아티팩트가 도둑맞아 잘못된 시대로 갔을 때의 위험도를 판분류된다. 여기서의 위험도는 개연성을 해침으로서 시간패러독스가 발생할 위험도를 의미한다.
		
		Atropos는 전혀 위협적이지 않은 경우. 대부분의 보석류와 제한적인 고전 무기 등이 이에 속한다.
		
		Clotho는 해당 시대 이전에만 위협적인 경우. 대부분의 무기와 기술이 이에 속한다.
		
		Lachesis는 언제든지 위협을 가할 여지가 있는 경우.
		
		Lachesis급의 아티팩트가 도둑맞으면 우주의 안정성 자체가 위협을 받을 수 있으므로 가져가게 둘 바에는 박물관을 폐쇄 후 폭파하도록 되어 있으며, 이를 Moros 프로토콜이라 칭한다.
		
		아티팩트는 모두 홀로그램과도 같은 phasing 상태로 보관되어 있다. 이는 시간상의 한 조각을 떼어다가 저장한 것으로, 실물 크기의 실제 물건이지만 사용하거나 건드리지는 못하도록 되어 있다. 이는 Aion 프로토콜이 담당하고 있다.
	}
	
	\bigskip
	
	프로토콜명과 그 이름을 따온 그리스 신의 역할, 그리고 이 이야기에서 해당 프로토콜의 역할할을 정리하면 다음과 같습니다:
	
	\begin{itemize}
		\item Chronos: 시간 그 자체 - 도서관명, 총괄 인공지능
		\item Kairos: 기회의 신 - 수집 프로토콜
		\item Aion: 영원의 신 - 보존 프로토콜
		\item Moirai: 운명의 세 여신 - 위협 단계
		\begin{itemize}
			\item Atropos: 물레 - 평상 단계
			\item Clotho: 자 - 위협 단계
			\item Lachesis: 가위 - 위험 단계
		\end{itemize}
		\item Moros: 죽음의 신 - 자폭 프로토콜
	\end{itemize}
	
	\section*{[태초의 이야기](선택)}
	시스템과 이야기꾼이 처음으로 만나게 됩니다. 시스템은 [태초의 이야기]에 대해 설명을 해주고는, 이야기꾼들이 시간의 박물관에 가도록 유도합니다.
	
	\section*{도입부}
	박물관에 들어온 이야기꾼들은 처음으로 크로노스를 만나게 됩니다. 크로노스는 실체가 있어도 좋고, 텍스트와 음성 인터페이스로만 이루어져 있어도 좋습니다. 크로노스는 이 박물관의 관장이자 총괄 인공지능이고, 또 다른 [깨달은 자]입니다.
	
	이야기꾼들은 로비로 들어와 크로노스와 대화를 나눕니다. 적당한 시점에 끊고, 박물관을 관람하도록 합시다.
	
	\section*{박물관 관람}
	박물관의 지도를 공개할때에, 서버실은 아직 어떤 방인지 공개되지 않아야 합니다. 서버실 앞에는 [관리자 외 출입 금지] 표지판이 붙어 있다는 사실만을 알 수 있습니다. 서버실이 어떤 방인지를 알기 위해서는 크로노스에게 직접 물어보거나, 서버실에 직접 들어가야 합니다. 이 시점에서 서버실은 잠겨있습니다.
	
	이야기꾼들은 박물관의 네 가지 전시실을 모두 돌아볼 수 있습니다. 이 때에는 각 아티팩트의 개략적인 설명만을 해주도록 합시다. 이 아티팩트들의 실제 효과를 이 때에 알 수 있는 방법은 감정사 계열의 이야기를 가지고 있거나, 해당 아티팩트의 능력을 분석할 수 있는 적당한 지식이 있어야 합니다.
	
	두 번째 전시실에서 나올때에, 복도에서 조금 위화감이 든다는 언급을 해주세요. 만약 뭐가 위화감이 드는지 알아보고 싶다면, 이야기꾼들은 성공치가 3인 인식 판정을 해야합니다. 실패한다면 아무것도 알 수 없으나, 성공한다면 서버실의 문이 조금 열려있다는 사실을 알 수 있습니다. 지금 서버실에 대해 질의를 한다면, 크로노스는 "왜 열려있는지 모르겠다"는 반응입니다. 왜 그런지 확인해볼테니, 남은 전시실을 관람하라고 할 것입니다.
	
	이야기꾼들이 이 때 바로 서버실을 확인하든 전시실을 모두 관람한 후 확인하든 상관 없이 서버실을 확인할 수 있습니다. 서버실에서 성공치가 5인 인식 판정을 성공한다면 메모리카드가 본체 하나에서 빼꼼 삐져나와 있는 것을 확인할 수 있습니다. 만약 물어본다면 크로노스는 이 메모리카드에 대해 아는 바가 없으며, 이 박물관 안에는 이야기꾼들밖에 인식할 수 없다고 대답합니다. 이 메모리카드에는 Janus 바이러스가 들어있습니다.
	
	이 메모리카드는 제거되면 수장고에서 나와 있으면서 아티팩트를 소지하고 있지 않은 모든 도적단원들이 박물관 밖으로 퇴출되어 행동불능에 빠지는 매우 중요한 메모리카드입니다.
	
	\section*{헤르메스 도적단과의 조우}
	전시실을 모두 관람한 후 긴 시간동안 서버실을 확인하지 않거나, 서버실에 입장하고 잠시 후, 로비쪽에서 쿵 하는 커다란 소리가 들립니다. 로비로 달려나가면 지하의 빈 공간(지하 수장고입니다.)으로 이어져 있는 사다리를 볼 수 있으나, 지하는 너무 캄캄해서 잘 보이지 않습니다. 지하는 광원이 없다면 주변 한 구역까지만을 볼 수 있는 어둠에 잠겨 있습니다.
	
	크로노스는 그 후 Lachesis급의 도구의 무더기를 누군가 헤집는 것을 감지합니다. 이상하게도, 누가 헤집는지는 감지되지 않았지만요. 크로노스는 이야기꾼들에게 다음 이야기를 지급합니다:
	\begin{spoiler}{도슨트}{[박물관]}
		\entry{현재 특별 전시중인 아티팩트들의 효과를 정확하게 알 수 있다.}
		
		\entry{현재 특별 전시중인 아티팩트들을 사용할 수 있게 된다.}
		
		\entry{지하 수장고에 있는 무기를 되찾는다면 사용할 수 있다.}
	\end{spoiler}
	여기에서 전투에 들어가기 전, 이야기꾼들은 아티팩트를 집기 위해 전시실로 되돌아갈 수 있습니다. 하지만 이야기꾼들이 집은 아티팩트 하나당 도적단 두목이 한 턴을 먼저 진행할 수 있다는 점을 반드시 상기시켜 주세요. 또한 도적단이 행동을 할 때, 도적단 두목이 [명령] 이야기를 사용할 수 있다는 점을 기억하세요.
	
	\section*{헤르메스 도적단의 목표와 행동}
	헤르메스 도적단의 목적은 이 박물관의 수장고 안에 들어있는 Lachesis급의 도구인 [헤르메스의 지팡이]를 이 박물관 밖으로 반출하는 것입니다. 이들이 행해야 하는 행동은 다음과 같습니다:
	\begin{enumerate}
		\item 이야기꾼들이 박물관을 관람하는 동안:
		\begin{itemize}
			\item 도적 2가 박물관 서버실로 침투하여 Janus 바이러스를 심습니다.
			\begin{itemize}
				\item 이야기꾼들이 위화감을 느낄 때에는 도적 2가 바이러스를 심은 후 나올때입니다. 이때 서버실에 들어가도 아무도 없습니다.
			\end{itemize}
			\item 이 바이러스는 Chronos로부터 도적단을 숨겨줍니다.
		\end{itemize}
		\item 지하로 내려가는 타일을 깨트려 내려갑니다.
		\item 두목은 전투가 시작한 후 한 턴간 아티팩트를 찾아 획득합니다.
		\item 그 이후 서버실로 가서 봉쇄 상태에 빠진 주 출입구를 열어야 합니다.
		\begin{itemize}
			\item 최초에는 도적 2가 시도합니다. 도적 2가 행동불능에 빠진 후에는 도적 1이, 도적 1이 행동불능에 빠진 후에는 도적단 두목이 시도합니다.
			\begin{itemize}
				\item 시도하는 사람은 받는 모든 공격을 무시하고 서버실로 돌진합니다.
			\end{itemize}
			\item 서버실에 도착하면 한 턴을 소모해 도적은 주 출입구를 열기로 시도할 수 있습니다. 이 경우, 지식으로 판정합니다. 성공치는 3입니다.
			\begin{itemize}
				\item 대실패한 경우, 자폭 타이머가 시작하여 3라운드 후 박물관이 폭발합니다.
				\begin{itemize}
					\item 이 상태에서 다시 시도하여 성공한 경우, 자폭 타이머를 임시로 멈출 수 있습니다.
					\item 대성공한 경우, 자폭 타이머가 초기화됩니다.
				\end{itemize}
				\item 실패한 경우, 아무 일도 일어나지 않습니다.
				\item 통과 또는 성공한 경우, 주 출입구가 열리나 모든 전시실의 문이 닫힙니다.
				\item 대성공한 경우, 주 출입구가 열립니다.
			\end{itemize}
		\end{itemize}
		\item 주 출입구가 열린 후, 누군가가 아티팩트를 들고 빠져나가야 합니다.
		\begin{itemize}
			\item 아티팩트를 던져서 출입구 밖으로 반출하면, 아티팩트는 크로노스에 의해 다시 반입됩니다.
		\end{itemize}
	\end{enumerate}
	
	\section*{결말}
	세 가지 결론이 나올 수 있습니다.
	\begin{enumerate}
		\item 모든 도적단이 행동불능에 빠진다.
		\begin{itemize}
			\item 도적단은 모두 체포되어 처벌을 받습니다.
			\item 박물관은 이야기꾼들에게 감사하며 이야기꾼(들)에게 원하는 아티팩트 한가지를 지급합니다.
			\item 이야기꾼들은 능력이 없는 이야기 "헤르메스 도적단 체포"([설화])를 얻습니다.
		\end{itemize}
		
		\item 도적단이 [헤르메스의 지팡이]를 들고 도망치는데에 성공한다.
		\begin{itemize}
			\item 박물관은 보안 점검을 위해서 폐쇄에 들어갑니다.
			\item 이야기꾼들은 실망한채로 [태초의 이야기]로 되돌아갑니다.
		\end{itemize}
		
		\item 박물관의 자폭 시퀀스가 발동되어, 모두와 함께 폭발한다.
		\begin{itemize}
			\item 박물관은 시공간 상에서 사라집니다.
			\item 이야기꾼들은 다음 기피증을 얻고 [태초의 이야기]로 되돌아갑니다:
		\end{itemize}
	\end{enumerate}
	\begin{spoiler}{폭발}{[기피]}
		\triggertrauma{기피}{자신이 폭발물의 효과 반경 안에 존재한다는 사실을 알고 있다.}{해당 폭발물의 효과 반경에서 벗어나기 전까지지 모든 판정에 -1을 얻는다.}
	\end{spoiler}
\end{document}