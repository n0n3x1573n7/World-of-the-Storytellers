\documentclass{report}

\begin{document}
	\section*{도적단 두목}
	\textbf{체력}: 15(2인일 시 20), \textbf{정신력}: 10(2인일 시 15)
	
	\textbf{스탯}: \textbf{기민} 3, \textbf{도발} 1, \textbf{의지} 1, \textbf{전투} 1, 나머지 0
	
	\begin{spoiler}[thief-leader:steal]{훔치기}{[도적단]}
		\entry{같은 구역 안에 있는 한 대상을 지정한다. 그 대상이 가지고 있는 무작위 물체를 훔칠 수 있다. 단, 대상은 인식 판정을 하여 훔친 이의 기민 이상이 나온다면 이를 저지하여 훔친 이에게 피해 1을 입히고, 훔치기를 실패로 할 수 있다.}
	\end{spoiler}
	
	\begin{spoiler}{표적}{[도적단]}
		\entry{방 안에 있는 한 가지 물체를 표적으로 지정할 수 있다. 해당 물체를 \storyref{thief-leader:steal}{훔치기} 할 때에 피해를 입었더라도 \storyref{thief-leader:steal}{훔치기}에 성공한다. 들키지 않았다면 다른 물체를 즉시 다시 표적으로 지정할 수 있으며, 들켰다면 다음 씬에 지정할 수 있다.}
	\end{spoiler}
	
	\begin{spoiler}{명령}{[도적단]}
		\entry{자신의 턴 대신 같은 공간(지하층, 각 전시실, 복도, 서버실, 로비 각각을 한 공간으로 친다.)에 있는 모든 도적의 턴을 진행할 수 있다.}
	\end{spoiler}
	
	\begin{spoiler}{헤르메스의 지팡이}{[아티팩트]}
		\entry{공중에 떠서 이동할 수 있다. 아무도 없는 칸을 통해서라면 2회 이동할 수 있다.}
		
		\statchange{+}{기민\footnote{도적단 두목의 [기민] 수치는 [헤르메스의 지팡이]가 이미 적용된 수치입니다.}}
	\end{spoiler}
	
	\begin{spoiler}{도적 두목의 단도}{[아이템]}
		\entry{한 턴에 한 번, 칼을 휘둘러 같은 구역에 있는 대상에게 회피 불가능한 피해 4를 준다.}
	\end{spoiler}
	
	\begin{spoiler}{활}{[아이템]}
		\entry{한 턴 장전 후 발사한다. 사격으로 판정하고 기민으로 회피할 수 있다. 적중시, 개연성에 피해를 2 준다.}
	\end{spoiler}
	
	\section*{도적 1}
	\textbf{체력}: 15, \textbf{정신력}: 10
	
	\textbf{스탯}: \textbf{기민} 1, 나머지 0
	
	\begin{spoiler}[thief-1:steal]{훔치기}{[도적단]}
		\entry{같은 구역 안에 있는 한 대상을 지정한다. 그 대상이 가지고 있는 무작위 물체를 훔칠 수 있다. 단, 대상은 인식 판정을 하여 훔친 이의 기민 이상이 나온다면 이를 저지하여 훔친 이에게 피해 1을 입히고, 훔치기를 실패로 할 수 있다.}
	\end{spoiler}
	
	\begin{spoiler}{빠른 손발}{[도적단]}
		\entry{\storyref{thief-1:steal}{훔치기}를 할 때에 한해서 자신의 기민에 +1. 또한, 다른 행동을 하지 않는다면 한 턴에 아무도 없는 칸을 통해서(시작칸 기준) 2회 이동할 수 있다.\footnote{이 이야기로 인해 \storyref{thief-1:steal}{훔치기} 한정 \textbf{기민}이 2가 됩니다.}}
	\end{spoiler}
	
	\begin{spoiler}[thief-1:dagger]{도적의 단도}{[도적단]}
		\entry{한 턴에 한 번, 칼을 휘둘러 같은 구역에 있는 대상에게 회피 불가능한 피해 2를 준다.}
	\end{spoiler}
	
	\begin{spoiler}[thief-1:taunt]{도발}{[버서커]}
		\entry{같은 구역에 있는 한 대상을 선택하여 대상에게 의지 판정을 하게 한다. 만약 자신의 도발이 더 높다면, 다음 턴에는 해당 대상은 반드시 자신을 공격해야 한다.}
		
		\entry{\storyref{thief-1:berserker-change}{상태 변화: 버서커} 발동시 "\textbf{스탯+}: 도발"을 추가로 가진다.}
	\end{spoiler}
	
	\begin{spoiler}[thief-1:berserker-change]{상태 변화: 버서커}{[생애]}
		\entry{체력이 5 이하로 떨어지면, 이 능력과 \storyref{thief-1:taunt}{도발}의 스탯 변화가 해제되고 한번에 피해 1 이상을 받을 수 없게 되나, \storyref{thief-1:steal}{훔치기}와 \storyref{thief-1:dagger}{도적의 단도}가 봉인된다. 한 턴에 한 번, 같은 구역 안에 있는 대상에게 피해 (6-현재 체력)을 줄 수 있다. 피해량 이상의 기민으로 회피할 수 있다.}
		
		\entry{발동시 다음을 추가로 가진다:
		
		\statchange{+}{근력, 도발}
		
		\statchange{-}{기만, 은신, 공감, 의지\footnote{해당 능력이 발동되면, 스탯이 다음과 같이 변한다:  \textbf{기만} -1, \textbf{은신} -1, \textbf{공감} -1, \textbf{의지} -1, \textbf{근력} 1, \textbf{기민} 1, \textbf{도발} 2, 나머지 0}}}
	\end{spoiler}
	
	\section*{도적 2}
	\textbf{체력}: 10, \textbf{정신력}: 10
	
	\textbf{스탯}: \textbf{기민} 1, 나머지 0
	
	\textbf{상태}: \textbf{은신} □□
	
	\begin{spoiler}[thief-2:steal]{훔치기}{[도적단]}
		\entry{같은 구역 안에 있는 한 대상을 지정한다. 그 대상이 가지고 있는 무작위 물체를 훔칠 수 있다. 단, 대상은 인식 판정을 하여 훔친 이의 기민 이상이 나온다면 이를 저지하여 훔친 이에게 피해 1을 입히고, 훔치기를 실패로 할 수 있다.}
	\end{spoiler}
	
	\begin{spoiler}{완벽한 은신}{[도적단]}
		\entry{한 턴을 소모해 [은신 □□]를 얻는다. 이 상태가 있는 동안 다른 사람들에게 보이지 않는다. 피해를 받거나 이동하면 [은신] 상태 한 칸이 소모된다. 자신의 턴이 시작할 때 같은 칸에 있는 적 한 명당 [은신] 한 칸이 소모된다. \storyref{thief-2:steal}{훔치기}의 판정을 잔여 은신 상태+은신 스탯 또는 기민 스탯 중 높은 쪽으로 판정하나, 실패할 시 모든 [은신] 상태를 잃는다.}
	\end{spoiler}
	
	\begin{spoiler}{초심자의 행운}{[도적단]}
		\entry{매 턴 한 번, 주사위를 굴리는 판정에서 재굴림을 시도할 수 있다.}
	\end{spoiler}
	
	\begin{spoiler}{도적의 표창}{[아이템]}
		\entry{한 턴에 두 번, 표창을 던져 한 대상에게 피해 1을 줄 수 있으나, 기민 또는 인식 중 높은쪽이 자신의 기민보다 낮다면 회피한다. 떨어진 구역당 회피에 +1을 받는다.}
	\end{spoiler}
\end{document}