\documentclass{report}

\begin{document}
	이야기꾼에게 세 가지 권능과 두 가지 제약이 존재하듯이, 시스템에게도 세 가지 권능이 존재합니다. 이 챕터에서는 이 권능과 제약들에 대해 자세하게 설명하도록 하겠습니다.
	
	\bigskip
	
	먼저, 이야기꾼의 권능과 제약에 대해 살펴보도록 하겠습니다:
	
	\smallskip
	
	\begin{minipage}{\textwidth}
		\begin{tabularx}{\textwidth}{c!{\color{black}\vrule}c!{\color{black}\vrule}X}
			\hline
			\textbf{구분} & \textbf{이야기} & \makecell{\centering\textbf{설명}} \\ \hline \hline
			[권능] & 소통\index{소통} & 자신의 출신 서사가 아니라면 모든 언어를 이해할 수 있다. \\ \hline
			[권능] & 접근\index{접근} & [태초의 이야기]에서 접근 좌표를 아는 서사로 이동하거나, 어떤 서사에서든 [태초의 이야기]로 이동할 수 있다. \\ \hline
			[권능] & 거래\index{거래} & 자신의 모든 이야기를 이야기의 규모에 비례하는 적당한 시간을 사용하여 다른 [깨달은 자]들에게 전할 수 있다. \\ \hline
			[제약] & 비밀\index{비밀} & [태초의 이야기]와 관련된 그 어떠한 사항도 [깨달은 자]가 아닌 경우 발설할 수 없다. 발설한다면, [잊혀진 자]가 된다. \\ \hline
			[제약] & 참견\index{참견} & [침범] 판정을 실패하여 서사를 오염시키면, 서사에서 추방되고, 해당 서사에 대한 권능 [접근]을 빼앗긴다. \\\hline
		\end{tabularx}
		
		\smallskip
		
		\begin{tightcenter}
			\textbf{이야기꾼의 권능과 제약}
		\end{tightcenter}
	\end{minipage}
	
	\bigskip
	
	\emph{소통의 권능}은 이야기꾼들의 편의를 위해 지급되는 권능입니다. [태초의 이야기]에서 다른 이야기꾼들과 소통을 할 수 있도록 하기 위함도 있지만, 서사 속으로 들어갔을 때에 등장인물들과 이야기를 할 수 있도록 해주는 역할 역시 가지고 있습니다. 물론, 서사 속으로 들어가며 역할이 부여되거나 하는 등으로 인해 이해하지 못하는 언어가 존재할 수도 있다는 점은 염두에 두어야 합니다.
	
	\smallskip
	
	\emph{접근의 권능}은 이야기꾼의 세계의 크로스오버를 가능하게 만들어주는 핵심 이야기이자 [잊혀진 자]와 그렇지 않은 이야기꾼을 구분하는 핵심 권능입니다. 접근의 권능이 없는 이는 기본적으로 다른 서사로 들어갈 수 없습니다. 하지만 다른 이와의 계약을 통해 임시로 서사로의 접근 권한을 얻을 수는 있습니다. 예를 들어, 다음과 같은 이야기를 계약을 통해 얻었다면 서사 속으로 들어갈 수 있을 것입니다:
	
	\begin{story}{잭 더 리퍼의 가호}{[계약]}
		\entry[\hline]{[접근]의 권능 없이도 서사 속으로 들어갈 수 있다. 단, 들어간 서사 속에서 나오기 위해서는 잭 더 리퍼가 지정하는 등장인물들을 누구에게도 들키지 않고 죽여야만 한다.}
	\end{story}
	
	이 경우, 잭 더 리퍼 본인이 [잊혀진 자]라 할 지라도 많은 서사 속에서 잭 더 리퍼의 잔혹한 연쇄살인의 이야기가 유지됨으로서 잊혀지지 않고 [태초의 이야기] 속에서 자신의 존재를 드러낼 수 있을 것입니다.
	
	\smallskip
	
	\emph{거래의 권능}은 이야기꾼들이 자신이 가진 이야기를 다른 이들에게 이야기하여 그들에게 자신의 이야기를 알릴 수 있는 방법입니다. 물론 모든 이야기를 알릴 수 있는 것도 아니고, 이야기를 들은 이야기꾼은 자신의 생각대로 이야기를 해석하기 때문에 모든 이들에게 그대로 전해질 수 있는 것도 아니며, 이야기를 한번 전해줬다고 해서 그 이야기를 이야기꾼이 영원히 기억할 수 있는 것도 아닙니다. 하지만 이야기를 전해들은 이야기꾼은 그 이야기의 힘을 빌릴 수 있는 상황이 된다면 이야기의 힘을 빌릴 수 있게 될 것입니다. 이런 이야기를 배울 수 있는 것은 이야기꾼 뿐 아니라, 등장인물들로부터도 얻을 수 있을 것입니다.
	
	예를 들어, 헤라클레스가 \textbf{[네메아의 사자를 죽인 자]}의 이야기를 다른 이야기꾼에게 전해줄 때, 이 이야기를 듣고 죽어가면서도 끝까지 헤라클레스에게 대항한 사자의 용맹함에 감동을 받은 이야기꾼이라면 다음과 같은 이야기를 얻을 수 있을 것입니다:
	
	\begin{story}{죽음을 불사한 용맹}{[획득:거래]}
		\flavour{헤라클레스에게서 \textbf{[네메아의 사자를 죽인 자]}의 이야기를 [거래]의 권능으로 획득함}
		
		\entry[\hline]{전투에서 도주한다면 이 이야기를 잃는다.}
	\end{story}
	
	이렇게 거래의 권능으로 획득한 이야기는 \hyperlink{reward}{서사의 보상}으로 받은 것과 동일하게 취급합니다. 특히, 세 번째 제약인 "특정 조건 만족시 이야기를 잃는다"는 조건을 반드시 적용시키는 것을 권장드립니다. 만약 능력이 있는 이야기를 거래의 권능으로 획득한 경우, 일반적인 경우 능력을 획득할 수 없을 것이나, 교훈을 통해 자신이 깨달은 해당 이야기의 능력이 추가되는 경우, 원래 능력과는 상관 없는 완전히 다른 새로운 능력이 추가되거나, 원래 능력이 추가되는 경우 간접 경험일 뿐이기 때문에 약화된 능력만을 얻을 확률이 높을 것입니다.
	
	\smallskip
	
	\emph{비밀의 제약}과 \emph{참견의 제약}은 서사의 오염을 막고 그들을 보호하기 위한 제약들입니다. 비밀의 제약을 통해 너무 많은 이들이 [깨달은 자]가 되지 않도록 하고, 참견의 제약을 통해 서사의 오염을 직접적으로 막는 것이 이 제약들의 목적입니다.
	
	\bigskip
	
	시스템에게 주어진 권능은 크게 세가지로 구분됩니다. 이 권능은 여러분이 시스템으로서 이야기꾼들을 이끌어나가는 것을 정당화해주는 역할을 합니다.
	
	\smallskip
	
	\begin{minipage}{\textwidth}
		\begin{tabularx}{\textwidth}{c!{\color{black}\vrule}c!{\color{black}\vrule}X}
			\hline
			\textbf{구분} & \textbf{이야기} & \makecell{\centering\textbf{설명}} \\ \hline \hline
			[권능:시스템] & 계산\index{계산} & 서사의 과거 흐름을 바탕으로 서사의 흐름을 계산한다. \\ \hline
			[권능:시스템] & 접근\index{접근} & 서사에 접근하여 등장인물과 상호작용 할 수 있다. \\ \hline
			[권능:시스템] & 추출\index{추출} & 깨달은 자들을 [태초의 이야기]로 추출할 수 있다. \\ \hline
		\end{tabularx}
		
		\smallskip
		
		\begin{tightcenter}
			\textbf{시스템의 권능}
		\end{tightcenter}
	\end{minipage}
	
	\bigskip
	
	\emph{계산의 권능}은 시스템이 서사의 흐름을 계산할 수 있도록 하는 장치입니다. 이는 여러분이 시스템으로서 이야기꾼들을 이끌어나갈 때 여러분이 서사를 이끌어나갈 수 있게 하는 장치입니다.
	
	\smallskip
	
	\emph{접근의 권능}은 시스템이 서사 속의 등장인물에게 서사 속의 존재로서 나타나 등장인물과 상호작용할 수 있도록 해줍니다. 이는 이야기꾼들이 서사를 진행하다 어떻게 할 지 몰라 서사의 흐름이 막힌 경우나 서사가 너무 쉽게 흘러가는 경우, 시스템이 dei ex machina\footnote{deus/dea ex machina의 복수형.}적으로 직접적인 도움이나 시련을 줄 수 있도록 하는 장치입니다. 물론 좋은 서사는 이런 접근이 거의 없이도 해결될 수 있어야 할 것입니다.
	
	\smallskip
	
	\emph{추출의 권능}은 [깨달은 자]들을 이야기꾼으로 추출해내는, 이야기꾼의 세계를 위해 반드시 필요한 권능입니다. 이 권능을 이용해 시스템은 [태초의 이야기]로 깨달은 자들을 데려올 수 있습니다.

\end{document}