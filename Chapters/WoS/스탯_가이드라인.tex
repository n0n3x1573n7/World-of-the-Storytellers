\documentclass{report}

\begin{document}
	이야기꾼의 세계의 스탯은 여러가지로 나누어져 있습니다. 이미 계산했을 개연성을 제외하고도, 다양한 스탯이 있습니다. 이야기들은 연관성이 있다면 이 스탯을 높이는 데에 일조합니다. 물론, 스탯을 높이는 데에 일조한 이야기들은 개연성 코스트를 가지게 됩니다.
	스탯은 특화가 가능한 경우가 있습니다. 이 경우, 해당 스탯에 관련되어 있는 세부 분야 하나의 스탯을 올리기로 결정할 수 있습니다. 세부 분야의 경우 해당 분야에 관련있는 판정을 할 때의 판정치는 낮아지나 해당하지 않는 경우 판정이 불가능할 수 있습니다.
	
	스탯에는 다음이 있습니다:
	
	\smallskip
	
	\begin{minipage}{\textwidth}
		\begin{tabularx}{\textwidth}{c!{\color{black}\vrule}X}
			\hline
			\textbf{스탯} & \textbf{설명}\\ \hline \hline
			도발          & 상대를 언어, 행동으로 평정심을 잃게 만듦\\\hline
			인식          & 상황 변화에 따른 인지력          \\\hline
			기민          & 특정 상황에 대한 신속한 반응       \\\hline
			근력          & 순간적인 강력한 힘, 지속적인 근지구력  \\\hline
			은신          & 상대가 자신을 인지하기 어렵게 만듦    \\\hline
			기만          & 지성체를 거짓이나 과장, 은닉을 이용하여 속임 \\\hline
			공감          & 상대방의 감정, 의견에 대한 포용과 인정 \\\hline
			의지          & 물리적, 정신적 충격을 견딜 수 있는 마음이나 믿음\\\hline
		\end{tabularx}
		
		\smallskip
		
		\begin{tightcenter}
			\textbf{특화 불가능한 스탯}
		\end{tightcenter}
	
	\medskip
	
		\begin{tabularx}{\textwidth}{c!{\color{black}\vrule}X!{\color{black}\vrule}l}
			\hline
			\textbf{스탯} & \multicolumn{1}{c!{\color{black}\vrule}}{\textbf{설명}} & \multicolumn{1}{c}{\textbf{특화 예시}} \\ \hline \hline
			제작          & 특정한 물건, 프로그램 등의 제작, 설계          & 컴퓨터 바이러스, 예술품, 총          \\\hline
			운전          & 탑승물에 탑승하고 조종할 수 있는 능력           & 자동차, 말, 우주선         \\\hline
			전투          & 물리적 충격이 존재하는 모든 형태의 근접전         & 단도, 맨손, 개머리판         \\\hline
			사격          & 물리적 충격이 존재하는 모든 형태의 중장거리 전투     & 활, 총기, 저격총         \\\hline
			지식          & 특화된 내용에 대한 다양한 정보               & 상식, 종교, 역사, 과학, 오컬트         \\\hline
			인맥          & 지성체 사이의 관계 형성, 대화와 타협           & 과학자, 정치인, 부랑자         \\\hline
			자본          & 돈, 자산 등 금전을 획득하거나 사용할 수 있는 능력   & 예술품, 금, 현찰  \\\hline
		\end{tabularx}
		
		\smallskip
		
		\begin{tightcenter}
			\textbf{특화 가능한 스탯}
		\end{tightcenter}
	\end{minipage}
	
	\bigskip
	
	필요하다면 이야기에 따라 스탯을 추가, 제거, 변형하거나, 특화 스탯을 비특화로, 비특화 스탯을 특화로 변형할 수 있습니다.
	
	\bigskip
	
	예를 들어, [지식:역사] 스탯을 올린 경우, 어떤 종교에 관한 지식을 알고 있는지에 대한 판정에서 해당 종교의 역사에 대한 지식을 얻을 수는 있습니다. 이 경우 [지식] 스탯을 사용한 판정에서 사용될 판정치보다는 [지식:역사]에서 사용될 판정치가 낮을 것입니다. 하지만 [지식:과학]에 해당할 쿼크의 종류에는 무엇이 있고, 그들이 각각 어떤 식으로 상호작용하는지를 알 수는 없습니다.
	
	\bigskip
	
	스탯을 직관적으로 생각하자면, 평균적인 이들이 가진 능력은 스탯 0으로 생각할 수 있습니다. 각 기준은 스탯에 따라 다를 것인데, 예를 들어 [제작:총기]의 경우 스탯 0인 이들은 시도조차 할 수 없을 것이지만, [지식:상식]의 경우 스탯 0인 이들은 운이 좋다면 어디선가 얻어들었을 가능성이 있는 것이죠. 따라서 "전문가"라고 불리는 수준에 도달하기 위해 필요한 스탯 역시 다를 것입니다. [지식:상식]의 경우에는 1\textasciitilde2 정도만 있어도 전문가로 불릴 수 있을 것이나, [제작:총기]의 경우 1\textasciitilde2로는 총기의 원리를 이해하는 정도이고 4\textasciitilde5 정도가 되어야 간단한 총기를 제작할 수 있게 될 것입니다.
	
	\bigskip
	
	어떤 이야기가 스탯 하나를 높이는 데에 일조한다면 개연성 비용이 5 증가합니다. 스탯 하나를 낮추는 데에 일조한다면 개연성 비용이 5 감소합니다.
	단, 특화 분야의 스탯을 올리거나 내리는 데에 일조한다면 개연성 비용이 5가 아닌 2 증가하고 감소합니다.
	
	\bigskip
	
	스탯의 수치에는 최대/최소 제한이 없습니다. 하지만, 개연성 비용에 의한 제한을 받기 때문에 너무 많은 스탯을 올리기는 힘들다는 점을 기억하세요. 또한 등장인물의 경우, 아직 자신이 가진 이야기를 자각하지 못한 경우가 대부분이므로 이야기와 관계 없이 스탯을 설정받을 수 있다는 점 역시 기억해두세요.
	

\end{document}