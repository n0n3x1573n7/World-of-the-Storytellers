\documentclass{report}

\begin{document}
	\begin{story}{폭발하는 얼음}{[포션: 투척][폭발][냉기]\hyperlink{fd46}{☆}}
		\entry{이 포션이 든 병을 시야에 보이는 최대 두 구역 떨어진 구역까지 던져 깨트릴 수 있다. 이 포션이 든 병이 깨지면, 해당 구역에 폭발이 일어나는 동시에 해당 구역에 [얼음 바닥: 3턴] 상태를 적용시킨다. [얼음 바닥] 상태가 유지되는 동안 이 구역을 통해 이동할 때에는 별도의 도움이 되는 이야기가 없다면 미끄러운 바닥으로 인해 해당 구역을 통과해서 이동하게 된다.}
		
		\entry{폭발에 휘말린 이들 중 [냉기 저항] 속성이 없는 이들은 피해를 3 받고, [얼어붙음 □□] 상태를 받는다. [얼어붙음] 상태가 있는 동안, 이동을 할 때 한 구역 덜 움직인다.}
		
		\entry{폭발에 휘말린 이들 중 [냉기 저항] 속성이 있는 이들은 피해를 1 받는다.}
		
		\cost{31}
	\end{story}
	
	\begin{story}{성불(成佛)}{[부적: 부착]}
		\pre{주먹, 무기 등에 이 부적을 부착한다.}
		
		\entry{일반적으로 피격하거나 피해를 줄 수 없는 대상\footnote{\hyperlink{species:ghost}{[종족: 귀신]} 등을 예시로 들 수 있습니다.}에게 이 부적이 부착된 것을 통해 접촉하거나 피해를 줄 수 있다.}
		
		\cost{15}
	\end{story}
\end{document}