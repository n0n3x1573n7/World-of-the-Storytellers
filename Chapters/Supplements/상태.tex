\documentclass{report}

\begin{document}
	상태는 횟수나 시간(특히 턴 수) 제한이 있는 상태로서 주는 방법도 있지만, 특정한 효과를 가진 이야기를 통해 줄 수도 있습니다. \hyperlink{species:skeleton}{스켈레톤 종족}의 \storyref{skeleton:creaky-joint}{삐그덕거리는 관절} 이야기가 \storyref{skeleton:living-dead}{산송장} 이야기를 주는 것이나, \hyperlink{species:formless-spawn}{형태 없는 자 종족}의 \storyref{formless-spawn:empty-impression}{공허한 첫인상} 이야기가 \storyref{formless-spawn:no-head}{머리가... 없어?} 이야기를 주는 것을 예로 들 수 있습니다. ``상태를 가진 동안 특정 효과가 적용됩니다" 종류의 이야기의 경우, 이러한 상태 이야기를 주는 이야기로서 할 수 있으나, 간단한 효과의 경우에는 이야기의 능력에 적어두는 것으로 가독성을 높일 수도 있습니다 \footnote{서플리먼트에서는 위 두 경우를 제외하고, 짧은 시간동안만 적용되거나, 일회성이거나, 상태를 주거나 받은 이 하나에게만 적용되기 때문에 이야기의 능력에 적어두는 식으로 적용했습니다.}.
	
	이런 상태 이야기를 주는 이야기를 만들 때에는 횟수나 시간 제한을 두는 것을 추천드리며, 상태 이야기가 개연성에 직접적인 타격을 주지 않도록 하는 것을 권장드립니다. 코스트는 다른 상태를 주는 이야기와 동일하게 계산합니다.
	
	영구적인 상태의 경우, 특히 현대인 캐릭터에게 어울리는 \hyperlink{status:chronic-diseases}{현대인의 만성질환}을 주는 경우에는 캐릭터의 이야기를 보다 풍성하게 만들어줄 수 있고, 메타적인 측면에서는 개연성에 대한 부담을 조금 덜어줄 수 있을 것입니다.
	
	\section{물리적 상태}
		\documentclass{report}

\begin{document}
	\begin{story}{출혈}{[상태이상:물리]}
		\entry[\hline]{피해를 받을 때 마다, 추가로 피해를 1 받는다.}
	\end{story}
	
	\begin{story}{중독}{[상태이상:물리]}
		\entry[\hline]{매 턴, 피해를 1 받는다. 이 피해로 인해 개연성에 피해를 받을 수 없다.}
	\end{story}
\end{document}
	
	\section{정신적 상태}
		\documentclass{report}

\begin{document}
	\begin{story}{두려움}{[상태이상:정신]}
		\entry[\hline]{정신적인 저항 판정에 대해 -2를 받는다.}
	\end{story}
\end{document}
	
	\hypertarget{status:chronic-diseases}{}
	\section{현대인의 만성질환}
		\documentclass{report}

\begin{document}
	\begin{story}{스마트폰 중독}{[만성질환][중독]}
		\limitedtrauma{중독}{아이템 [스마트폰]을 의도적으로 양도할 수 없으며, [스마트폰]과 다른 구역에 있다면 이를 다시 획득하기 전까지 모든 행동은 [스마트폰]을 획득하기 위해서만 행할 수 있다.}
		
		\cost{-10}
	\end{story}
\end{document}
\end{document}