\documentclass{report}

\begin{document}
	도박에 죽고 도박에 사는 이들입니다.
	
	\subsubsection*{도박사 공통 이야기}
	
	\begin{story}[gambler:high_highs_low_lows]{높은 고점, 낮은 저점}{[클래스: 도박사][집착]}
		\limitedtrauma{집착}{자동으로 성공하는 판정이라 할지라도, 반드시 \storyref{luck:pow-of-luck}{행운의 힘}을 활용한 판정을 진행해야 한다.}
		
		\entry{자동으로 성공하는 판정에 \storyref{luck:pow-of-luck}{행운의 힘}을 활용한 판정을 했다면 해당 판정에 +2를 받는다.}
		
		\entry{\storyref{luck:pow-of-luck}{행운의 힘}을 활용한 판정에서, 2df를 추가로 굴리고, 판정에 +1을 받을 수 있다. 한 판정에 2번까지 사용할 수 있다.}
		
		\entry{\storyref{luck:pow-of-luck}{행운의 힘}을 활용한 판정에 실패할 경우, 1df를 추가로 굴릴 수 있다. 한 판정에 4번까지 사용할 수 있다.}
		
		\cost{12}
	\end{story}
	
	\subsubsection*{세부 클래스: 다이스 블랙잭 도박사}
	
	\begin{story}[gambler:dice_blackjack]{다이스 블랙잭}{[클래스: 도박사][다이스 블랙잭]}
		\entry{매 턴 한번, 같은 구역에 있는 이를 하나 지목해, \storyref{gambler:dice_blackjack}{다이스 블랙잭}을 진행할 수 있다:
		\begin{enumerate}
			\setlength\itemsep{0.2em}
			\item 각자, 2d6을 굴린다.
			\item 도박사부터 돌아가며 1d6을 추가로 굴릴 수 있다.
			\item 두 명이 모두 주사위를 굴리지 않겠다고 선언하면 게임을 종료한다.
			\item 주사위 결과의 총합이 정확히 12라면, \textcolor{red}{[스페셜]}으로 취급한다.
			\item 주사위 결과의 총합이 12를 넘겼다면, \textcolor{blue}{[버스트]}로 취급하여 다음이 일어난다:
			\begin{itemize}
				\item 상대의 수가 6 이하라면 상대보다 1 낮은 수로 취급하고, 즉시 게임을 종료한다.
				\item 상대의 수가 7 이상이라면 6으로 취급하고, 즉시 게임을 종료한다.
			\end{itemize}
		\end{enumerate}
		게임 종료 시점에 두 숫자가 같다면(둘 다 스페셜인 경우 포함) 승패 없이 종료된다. 그렇지 않다면, 더 숫자가 높은 이가 승자, 낮은 이가 패자가 된다. 승자는 물리/정신피해 중 하나를 선택하여 패자에게 다음 피해량 계산법에 따라 피해를 준다:
		\begin{itemize}
			\item 승자가 \textcolor{red}{[스페셜]}이 아니라면, 두 수의 차이만큼의 피해를 준다.
			\item 승자가 \textcolor{red}{[스페셜]}이라면, 두 수의 차이만큼에 1을 더한 만큼의 피해를 준다. 추가되는 피해량 1은 \textcolor{red}{[스페셜 피해량]}이라고 부른다.
			\item 패자가 \textcolor{blue}{[버스트]}라면, 위의 피해량의 두배에 해당하는 피해를 물리 또는 정신에 한번에 주거나, \textcolor{green}{[스플릿]} 하여 물리/정신에 각각 나누어 줄 수 있다.
		\end{itemize}
		피해량을 계산한 후, 게임에 참여한 이야기꾼의 이야기에 따라 피해량이 조정될 수 있다. 이 피해량은 게임에 참여하지 않은 이야기꾼에 의해 조정될 수 없다.}
		
		\cost{20\footnote{주사위 운에 따라, 자신도 상대도 공평하게 피해를 받을 수 있는 능력이기 때문에 일반적인 활용형 능력으로 취급됩니다.}}
	\end{story}
	
	\begin{story}{도박 중독}{[클래스: 도박사][다이스 블랙잭][중독]}
		\pre{\storyref{gambler:dice_blackjack}{다이스 블랙잭}}
		
		\limitedtrauma{중독}{다른 이에게 피해를 가하고 싶을 때 \storyref{gambler:dice_blackjack}{다이스 블랙잭}을 할 수 있다면, 반드시 \storyref{gambler:dice_blackjack}{다이스 블랙잭}을 통해서만 피해를 줄 수 있다.}
		
		\entry{\statchange{-}{근력, 의지, 자본, 인맥}}
		
		\cost{-35}
	\end{story}
	
	\begin{story}[gambler:all-in]{올인}{[클래스: 도박사][다이스 블랙잭][집착]}
		\pre{\storyref{gambler:dice_blackjack}{다이스 블랙잭}}
		
		\entry{\storyref{gambler:dice_blackjack}{다이스 블랙잭}의 결과 자신이 \textcolor{red}{[스페셜]}일 때, 즉시 [멍해짐: 1턴] 상태를 주는 것으로 게임으로 인한 피해량 판정 없이 게임을 종료할 수 있다. [멍해짐] 상태를 받은 상대는 다음 자신의 턴에 아무 행동도 할 수 없다.}
		
		\limitedtrauma{집착}{\storyref{gambler:dice_blackjack}{다이스 블랙잭}의 결과가 \textcolor{red}{[스페셜]}이라도, \textcolor{green}{[스플릿]} 할 수 없다.}
		
		\cost{5}
	\end{story}
	
	\begin{story}[gambler:cheating]{밑장빼기}{[클래스: 도박사][다이스 블랙잭][기피]}
		\pre{\storyref{gambler:dice_blackjack}{다이스 블랙잭}}
		
		\entry{전투 씬에서, 세 라운드에 한 번, \storyref{gambler:dice_blackjack}{다이스 블랙잭}을 할때 자신이 \textcolor{blue}{[버스트]}가 난 경우, \textcolor{blue}{[버스트]}를 낸 주사위 굴림을 취소할 수 있다.}
		
		\entry{비전투 씬에서, 한 씬에 한 번, 판정을 할 때에 다시 판정을 해 원래 결과와 새 결과 중 원하는 결과를 선택할 수 있다.}
		
		\triggertrauma{기피}{\storyref{gambler:dice_blackjack}{다이스 블랙잭}을 할때, 자신이 주사위를 더 굴리지 않겠다고 선언하지 않았고, 상대가 \textcolor{red}{[스페셜]}이다.}{반드시 10 이상이 나올때까지 주사위를 추가로 굴려야 한다.}
		
		\cost{5\footnote{전투 씬 또는 비전투 씬에서만 사용할 수 있는 능력이 있기 때문에 두 능력의 코스트(각각 10) 중 높은 쪽을 적용시켰습니다.}}
	\end{story}
	
	\subsubsection*{도박사 심화 이야기}
		\begin{story}{위협적인 근육}{[클래스: 도박사][다이스 블랙잭][심화]}
			\pre{\storyref{gambler:dice_blackjack}{다이스 블랙잭}}
			
			\entry{\storyref{gambler:dice_blackjack}{다이스 블랙잭}에서, 상대가 \textcolor{blue}{[버스트]}인 경우, 해당 수치가 12에서 넘어선 만큼 추가로 피해를 준다.}
			
			\entry{\statchange{+}{근력}}
			
			\cost{20}
		\end{story}
		
		\begin{story}{참을성}{[클래스: 도박사][심화]}
			\pre{\storyref{gambler:cheating}{밑장빼기}}
			
			\entry{\storyref{gambler:cheating}{밑장빼기}의 기피증을 무시한다. 즉, 상대가 \textcolor{red}{[스페셜]}이고 자신이 10 미만이며 주사위를 굴리지 않겠다고 선언하기 전이라 할지라도 주사위를 더 이상 굴리지 않아도 된다.}
			
			\entry{\statchange{+}{인내}}
			
			\cost{20}
		\end{story}
		
		\begin{story}{성공적인 도박사}{[클래스: 도박사][심화]}
			\pre{\storyref{gambler:all-in}{올인}}
			
			\entry{\storyref{gambler:all-in}{올인}을 게임 종료시 동점일때에도 발동시킬 수 있다. 즉, 게임 종료시 동점일 때에도 상대에게 [멍해짐: 1턴]을 줄 수 있다.}
			
			\entry{\statchange{+}{자본}}
			
			\cost{20}
		\end{story}
		
		\begin{story}{카지노의 VIP 고객}{[클래스: 도박사][심화]}
			\entry{인맥 판정에 자본 스탯을 보너스로 받아 판정한다.}
			
			\entry{\statchange{+}{인맥}}
			
			\cost{20}
		\end{story}
\end{document}