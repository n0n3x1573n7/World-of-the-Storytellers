\documentclass{report}

\begin{document}
	\world{`머리에 한 방!'}
	
	검, 도끼, 활, 철퇴 등 자신만의 무기에 적응해, 적들과 물리적인 싸움을 하는 데 특화된 직업입니다.
	
	\begin{story}[warrior:basic]{전사의 소양}{[클래스:전사]\hyperlink{celesteela}{ⓣ}}
		\pre{자신에게 특화된 무기를 하나 선택한다. 아래 내용은 전사가 해당 무기로 공격할때에만 적용된다.}
		
		\entry{양손무기를 사용한다면 전사는 명중 보정을 +1 받고, 피해량이 1 증가한다.}
		
		\entry{방패를 사용한다면 전사는 사용하는 방패에 따라 다른 효과를 받는다.}
		
		\entry{자신의 공격 차례에, 명중 보정치를 1 감소시킬 때마다 무기로 가하는 공격의 피해량을 2만큼 증가시킬 수 있다.}
		
		\cost{20}
	\end{story}
	
	\begin{story}{최선의 공격은...}{[클래스:전사]\hyperlink{celesteela}{ⓣ}}
		\entry{피해량 X 이하의 공격의 피해량이 절반(반올림)으로 감소한다. X는 최초에 2로 시작하며, 중간 성장 또는 큰 성장을 할 때 마다 1 증가시킬 수 있다. 이 이외의 방법으로는 X를 증가시킬 수 없다.}
		
		\cost{10+X/2}
	\end{story}
	
	\subsection{전사 심화 이야기}
		\begin{story}{전사의 기술}{[클래스:전사][심화]\hyperlink{celesteela}{ⓣ}}
			\pre{\storyref{warrior:basic}{전사의 소양}, 아래 내용은 \storyref{warrior:basic}{전사의 소양}에서 선택한 무기로 공격할 때에만 적용된다.}
			
			\entry{이도류를 사용한다면 전사는 명중 보정을 -4만큼 받고 피해량이 4 감소하지만, 1턴에 2회 공격할 수 있다.}
			
			\entry{근접 무기를 사용한다면, 전사는 자신의 주위 전범위를 휩쓸어 공격할 수 있다. 단, 사용 후 다음 자신의 차례까지 자신을 대상으로 하는 공격에 대해 회피 보정치가 2 감소한다.}
			
			\entry{원거리 무기를 사용한다면, 전사는 사거리 내 일직선상의 모든 적을 관통하여 공격할 수 있다. 단, 적을 관통할 때마다 명중 보정치가 추가로 2 감소한다.}
			
			\entry{방패를 사용한다면, 전사는 같은 구역 내에 있는 적에게 방패를 밀쳐내어 1의 피해를 주고 집중을 취소시킬 수 있다. 이 공격은 회피할 수 있다.}
			
			\cost{20}
		\end{story}
\end{document}