\documentclass{report}

\begin{document}
	\world{`은신 암살자’}
	
	누구보다도 은밀하고 날렵하게 접근하여 가공할 만한 피해를 주거나, 다양한 상황에서의 임기응변이 좋은 다재다능한 클래스입니다.
	
	\begin{story}{비열한 습격}{[클래스:도적]\hyperlink{celesteela}{ⓣ}}
		\pre{도적은 단검, 한손무기 등 가벼운 무기에 특화되어 있기 때문에, 두손무기나 무거운 무기로는 도적의 공격 기술들을 사용할 수 없다.}
		
		\entry{도적이 대상에게 선제 공격을 하거나, 상대가 도적을 인식하지 못하였을 때 공격하거나, 아군과 함께 대상을 포위하였을 때 공격한다면 [기습]이 발동한다. [기습]했을 때, 도적은 대상에게 피해를 1만큼 1d4회 준다. [기습]의 피해량은 도적이 중간 성장이나 큰 성장을 했을 때 1만큼 증가한다.}
		
		\cost{22}
	\end{story}
	
	\begin{story}[rogue:allrounder]{만능 캐릭터?}{[클래스:도적]\hyperlink{celesteela}{ⓣ}}
		\entry{\statchange{+}{도발, 인식, 기민, 은신, 기만}}
		
		\entry{\statchange{-}{근력[3], 위 스탯 증가 중 하나 선택}}
		
		\entry{도적 이야기꾼의 배경 이야기 중 지식, 인맥, 자본 중 어울리는 한 가지 스탯을 결정해, \statchange{+}{(해당 스탯)}을 받는다. 어떤 상황에서든, 이 세 스탯 중 하나로 판정하여 상황을 타개할 수 있다.}
		
		\cost{30}
	\end{story}
	
	\subsubsection*{도적 심화 이야기}
		\begin{story}{이도류 도적}{[클래스:도적][심화]\hyperlink{celesteela}{ⓣ}}
			\entry{도적은 양손에 무기를 들고 공격할 수 있다. 이 때 각각의 공격에 대해 명중 판정 에 -2를 받고, 주는 피해가 2 감소하지만 한 턴에 두 번 공격할 수 있다. [기습]은 각 공격에 대해 따로 적용된다.}
			
			\cost{20}
		\end{story}
	
	\subsection{음유시인}
		\documentclass{report}

\begin{document}
	\world{`5인 파티인데 나 혼자 살았어, 그럼 난 뭘 할 수 있지?'}
	
	노래를 바탕으로 아군을 강화하며, 적들을 약화시키며 도적과 같은 다재다능함을 무기로 하는 클래스입니다.
	
	도적의 이야기 중 \storyref{rogue:allrounder}{만능 캐릭터?}를 공유하고, 다음 이야기를 추가로 가집니다:
	
	\begin{story}[bard:music]{음악의 치유사}{[클래스:음유시인]}
		\entry{다양한 종류의 음악을 매개로 한 마법을 사용할 수 있다. 마법의 기능과 개연성 코스트에 대해서는 마스터와 상의 후 결정한다\footnote{음유시인의 마법 사용의 메커니즘은 노래이기에, \storyref{magic:focus}{집중}형 마법, 그 중에서도 범위형 효과가 있는 마법을 사용하는 것을 권고드립니다.}.}
	\end{story}
	
	\subsubsection*{음유시인 심화 이야기}
		\begin{story}{제대로 놀아 볼까!}{[클래스:음유시인][심화]}
			\pre{\storyref{bard:music}{음악의 치유사}}
			
			\entry{다섯 턴에 한 번 사용할 수 있다. 한 턴 동안, \storyref{bard:music}{음악의 치유사}의 주문의 반경이 한 구역 넓어지고, 효과가 두 배로 증가한다.}
			
			\cost{15}
		\end{story}
\end{document}
\end{document}