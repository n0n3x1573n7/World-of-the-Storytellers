\documentclass{report}

\begin{document}
	\world{`은신 암살자’}
	
	누구보다도 은밀하고 날렵하게 접근하여 가공할 만한 피해를 주거나, 다양한 상황에서의 임기응변이 좋은 다재다능한 클래스입니다.
	
	\begin{story}{비열한 습격}{[클래스:도적]\hyperlink{celesteela}{ⓣ}}
		\pre{도적은 단검, 한손무기 등 가벼운 무기에 특화되어 있으며, 두손무기나 무거운 무기로는 도적의 공격 기술들을 사용할 수 없다.}
		
		
		\entry{도적이 대상에게 선제 공격을 하거나, 상대가 도적을 인식하지 못하였을 때 공격하거나, 아군 캐릭터와 대상을 포위하였을 때 공격에 성공하였으면 [기습]이 발동한다. [기습]했을 때, 도적은 대상에게 피해를 1만큼 1d4회 준다. [기습]의 피해량은 도적이 중간 성장이나 큰 성장을 했을 때 1만큼 증가한다.}
		
		\cost{22}
	\end{story}
	
	\begin{story}{만능 캐릭터?}{[클래스:도적]\hyperlink{celesteela}{ⓣ}}
		\entry{\statchange{+}{도발, 인식, 기민, 은신, 기만}}
		
		\entry{\statchange{-}{근력[3], 위 스탯 증가 중 하나 선택}}
		
		\entry{도적 이야기꾼의 배경 이야기 중 지식, 인맥, 자본 중 어울리는 한 가지 스탯을 결정한다. 해당 스탯에 +1. 답이 나오지 않는 상황에서, 이 세 스탯 중 하나로 판정하여 상황을 타개할 수 있다.}
		
		\cost{25}
	\end{story}
	
	\subsubsection{도적 심화 이야기}
		\begin{story}{이도류 도적}{[클래스:도적][심화]\hyperlink{celesteela}{ⓣ}}
			\entry{도적은 양손에 무기를 들고 공격할 수 있다. 이 때 각각의 공격에 대해 명중 판정 에 -2를 받고, 주는 피해가 2 감소하지만 한 턴에 두 번 공격할 수 있다. [기습]은 각 공격에 대해 따로 적용된다.}
			
			\cost{20}
		\end{story}
\end{document}