\documentclass{report}

\begin{document}
	\world{`팀의 체력을 책임진다'}
	팀의 체력을 책임지지는 않지만 튼튼하고, 튼튼하며, 강합니다. 대신 엄청나게 답답합니다.
	
	\begin{story}[paladin:holy-fury]{성스러운 빛을 믿고 분노를 제어하십시오}{[클래스:성기사]\hyperlink{celesteela}{ⓣ}}
		\pre{성기사는 자신이 믿는 신을 선택해야 한다. 그 신은 선한 성향을 가지고 있어야 한다.}
		
		\entry{성기사의 모든 행동은 선하며 세상에 질서를 가져와야 신념에 기반해야 한다.}
		
		\triggertrauma{기피}{신념에 어긋나는 행동을 한다.}{해당 행동에 대한 판정에 -1을 받는다.}
		
		\entry{성기사는 자신에게 특화된 무기를 정할 수 있다. 성기사는 미래지향적 무기를 제외하고 모든 무기에 특화 가능하다.}
		
		\entry{성기사는 이동하는 대신, 다음과 같은 네 가지 속성 중 한 가지를 자신의 무기에 부여할 수 있다. 무기에 피격당한 대상은 의지로 해당하는 효과를 저항할 수 있다. 해당 효과는 한 대상에 대해 둘 이상 적용되지 않는다.
		\begin{tightcenter}
			\begin{tabularx}{\linewidth}{l|l}
				\textbf{속성}    & \textbf{효과}                                                                                 \\\hline\hline
				불꽃             & 적중한 대상에게 3턴간 턴당 1의 피해를 준다.                                                   \\\hline
				냉기             & 적중한 대상은 다음 3턴간 동전을 던져 앞이 나와야만 이동할 수 있다.                            \\\hline
				전격             & 적중한 대상은 다음 3턴간 한 턴에 공격 또는 이동 중 한가지만 할 수 있다. \\\hline
				신성             & 적중한 대상이 3턴간 [실명] 상태를 받아, 필중하는 기술을 제외한 기술의 방향이 무작위가 된다.
			\end{tabularx}
	\end{tightcenter}}
	\end{story}
	
	\begin{story}[paladin:holy-power]{축복받은 힘}{[클래스:성기사]\hyperlink{celesteela}{ⓣ}}
		\entry{악한 성향의 대상을 공격할 때, 대상을 [심판]하여 두 배의 피해를 줄 수 있다. 이 효과는 24시간에 1회만 사용 가능하다. 성기사가 중간 성장이나 큰 성장을 할 때마다 사용 가능 회수가 1회 추가된다.}
		
		\entry{성기사는 24시간에 3회 사용 가능한 신성 에너지를 가진다. 신성 에너지는 자신이나 언데드가 아닌 아군을 1d6만큼 회복시킬 수 있고, 언데드에게는 1d6만큼 피해를 줄 수 있다. 이 피해는 회피할 수 없다.}
	\end{story}
	
	\subsection{성기사 심화 이야기}
		\begin{story}{Elemental Charge}{[클래스:성기사][심화]\hyperlink{celesteela}{ⓣ}}
			\pre{\storyref{paladin:holy-fury}{성스러운 빛을 믿고 분노를 제어하십시오}}
			
			\entry{불꽃 공격의 효과를 받고 있는 대상에게 냉기 공격을 적중시키거나, 냉기 공격의 효과를 받고 있는 대상에게 전격 공격을 적중시키거나, 전격 공격을 받고 있는 대상에게 신성 공격을 적중시키면 피해량이 1d6만큼 증가한다.}
		\end{story}
		
		\begin{story}{스마이트}{[클래스:성기사][심화]\hyperlink{celesteela}{ⓣ}}
			\pre{\storyref{paladin:holy-power}{축복받은 힘}}
			
			\entry{[심판]의 대상을 1턴간 기절시켜 행동 불능 상태로 만든다. 단, [심판]을 연속으로 사용할 수 없게 된다.}
		\end{story}
\end{document}