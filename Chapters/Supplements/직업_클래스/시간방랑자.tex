\documentclass{report}

\begin{document}
	시간 속을 여행하며 떠돌아다니는 방랑자들을 의미합니다.
	
	\begin{story}{완벽한 시간적응}{[클래스: 시간 방랑자]ⓣ}
		\textit{다양한 시간대에서 다양한 경험을 한 시간 방랑자들은 해당 이야기의 사람들에게 더 쉽게 다가갈 수 있다.}
		
		자신이 살던 시간대에서 벗어나 있음으로서 받는 모든 페널티를 무시할 수 있다.
		
		\cost{15}
	\end{story}
	
	\begin{story}{타임 패러독스}{[클래스: 시간 방랑자][광기]ⓣ}
		\textbf{제약(광기)}: 행운 등의 적용으로 인하여 주사위를 굴려 판정을 했을 때, 대성공에 해당하는 판정이 나오면, 다음 주사위를 굴리는 판정에 대해서 해당 주사위를 뒤집은 결과\footnote{주사위를 뒤집은 판정이란, 일반적인 주사위에서는 특정 결과치에 대해 (최대치)-(결과치)+1의 값이 나온 것으로 취급하고, 퍼지 다이스에서는 각각의 주사위 중 +를 모두 -로, -를 모두 +로 바꾼 결과로 취급합니다.}가 나온 것으로 취급한다.
		
		\cost{-20}
	\end{story}
	
	\begin{story}{되돌리기}{[클래스: 시간 방랑자]ⓣ}
		하루에 한 번, 판정 하나를 성공한 것으로 취급할 수 있다.
		
		\cost{15}
	\end{story}
	
	\begin{story}{현실 조작}{[클래스: 시간 방랑자]ⓣ}
		행운 등의 사용으로 인하여 네 개의 퍼지 다이스를 굴릴때, 개연성을 지불하여 주사위를 조작할 수 있다. 조작하는 주사위의 개수에 따라 지불해야하는 개연성의 양이 다음과 같이 변화한다:
		\begin{center}
			\begin{tabular}{c|c}
				\textbf{조작하는 주사위 개수} & \textbf{개연성 코스트} \\\hline\hline
				1                             & 0                      \\\hline
				2                             & 1                      \\\hline
				3                             & 4                      \\\hline
				4                             & 9                      \\\hline
				
			\end{tabular}
		\end{center}
		
		\cost{30}
	\end{story}
\end{document}