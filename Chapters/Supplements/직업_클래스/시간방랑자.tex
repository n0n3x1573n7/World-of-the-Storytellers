\documentclass{report}

\begin{document}
	시간 속을 여행하며 떠돌아다니는 방랑자들을 의미합니다.
	
	\begin{story}{현실 조작}{[클래스: 시간 방랑자]\hyperlink{celesteela}{ⓣ}}
		\entry{\storyref{luck:pow-of-luck}{행운의 힘} 등의 사용으로 인하여 4df를 굴릴때, 개연성을 지불하여 주사위를 조작할 수 있다. 조작하는 주사위의 개수에 따라 지불해야하는 개연성의 양이 다음과 같이 변화한다:
		\begin{center}
			\begin{tabular}{c|c}
				\textbf{조작하는 주사위 개수} & \textbf{개연성 코스트} \\\hline\hline
				1                             & 0                      \\\hline
				2                             & 1                      \\\hline
				3                             & 4                      \\\hline
				4                             & 9                      \\\hline
				
			\end{tabular}
		\end{center}}
		
		\cost{30}
	\end{story}
	
	\begin{story}{완벽한 시간적응}{[클래스: 시간 방랑자]\hyperlink{celesteela}{ⓣ}}
		\flavour{다양한 시간대에서 다양한 경험을 한 시간 방랑자들은 해당 서사의 이들에게 더 쉽게 다가갈 수 있다.}
		
		\entry{자신이 살던 시간대에서 벗어나 있음으로서 받는 모든 페널티를 무시할 수 있다.}
		
		\cost{15}
	\end{story}
	
	\begin{story}{되돌리기}{[클래스: 시간 방랑자]\hyperlink{celesteela}{ⓣ}}
		\entry{하루에 한 번, 판정 하나를 성공한 것으로 취급할 수 있다.}
		
		\cost{15}
	\end{story}
	
	\begin{story}{타임 패러독스}{[클래스: 시간 방랑자][광기]\hyperlink{celesteela}{ⓣ}}
		\limitedtrauma{광기}{행운 등의 적용으로 인하여 주사위를 굴려 판정을 했을 때, 대성공에 해당하는 판정이 나오면, 다음 주사위를 굴리는 판정에 대해서 해당 주사위를 뒤집은 결과\footnote{주사위를 뒤집은 판정이란, 일반적인 주사위에서는 특정 결과치에 대해 (최대치)-(결과치)+1의 값이 나온 것으로 취급하고, 퍼지 다이스에서는 각각의 주사위 중 +를 모두 -로, -를 모두 +로 바꾼 결과로 취급합니다.}가 나온 것으로 취급한다.\footnote{이 결과가 다시 대성공이라면 [타임 패러독스]는 여러번 연속으로 발동될 수도 있습니다.}}
		
		\cost{-20}
	\end{story}
	
	\subsection{시간방랑자 심화 이야기}
	
	\begin{story}{시간 가속}{[클래스: 시간 방랑자][심화]}
		\entry{한 씬에 두 번, 자신의 턴이 종료될 때 시간을 가속하여 한 턴을 추가로 진행한다. 이 추가로 진행하는 턴에는 다른 모든 이들의 판정에 -1을 가한다. 두 번을 모두 사용하여 두 턴을 추가로 진행하는 것은 불가능하다.}
		
		\entry{\statchange{+}{기민}}
		
		\cost{30}
	\end{story}
	
	\begin{story}{주마등}{[클래스: 시간 방랑자][심화]}
		\entry{한 서사 속에서 한 번, [침범] 판정이 아닌 이유로 개연성이 0이 되는 피해를 받았을때, 해당 피해를 무시할 수 있다.}
		
		\cost{15}
	\end{story}
\end{document}