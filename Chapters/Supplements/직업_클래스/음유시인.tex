\documentclass{report}

\begin{document}
	\world{`5인 파티인데 나 혼자 살았어, 그럼 난 뭘 할 수 있지?'}
	
	노래를 바탕으로 아군을 강화하며, 적들을 약화시키며 도적과 같은 다재다능함을 무기로 하는 클래스입니다.
	
	도적의 이야기 중 \storyref{rogue:allrounder}{만능 캐릭터?}를 공유하고, 다음 이야기를 추가로 가집니다:
	
	\begin{story}[bard:music]{음악의 치유사}{[클래스:음유시인]}
		\entry[\hline]{다양한 종류의 음악을 매개로 한 마법을 사용할 수 있다. 마법의 기능과 개연성 코스트에 대해서는 마스터와 상의 후 결정한다\footnote{음유시인의 마법 사용의 메커니즘은 노래이기에, \storyref{magic:focus}{집중}형 마법, 그 중에서도 범위형 효과가 있는 마법을 사용하는 것을 권고드립니다.}.}
	\end{story}
	
	\subsubsection*{음유시인 심화 이야기}
		\begin{story}{제대로 놀아 볼까!}{[클래스:음유시인][심화]}
			\pre{\storyref{bard:music}{음악의 치유사}}
			
			\entry{다섯 턴에 한 번 사용할 수 있다. 한 턴 동안, \storyref{bard:music}{음악의 치유사}의 주문의 반경이 한 구역 넓어지고, 효과가 두 배로 증가한다.}
			
			\cost{15}
		\end{story}
\end{document}