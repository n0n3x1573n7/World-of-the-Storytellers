\documentclass{report}

\begin{document}
	\pre[]{언데드 이외의 종족}
	
	\begin{story}{성물함}{[클래스: 리치][기피][공포]\hyperlink{celesteela}{ⓣ}}
		\flavour{리치는 죽음에 이르러도 성물함에서 부활할 수 있다.}
		
		\entry{한 턴을 소모하여 성물함을 설치할 수 있다. 성물함의 내구도는 리치의 최대 체력의 25\%(버림)이다.}
		
		\limitedtrauma{기피}{성물함이 설치되어 있는 동안, 최대 체력이 성물함의 내구도만큼 감소한다.}
		
		\entry{리치가 한 서사 속에서 추방되려 할 때, 성물함이 설치되어 있고 파괴되지 않았다면 해당 위치에서 체력, 정신력, 개연성을 모두 회복한 채로 부활하고, 성물함은 파괴된다.}
		
		\triggertrauma{공포}{자신을 제외한 누군가가 성물함을 파괴하려고 함을 알고 있다.}{앞으로의 모든 행동은 해당 행위를 방해하기 위해서만 행할 수 있다.}
		
		\entry{성물함은 [태초의 이야기]로 돌아가지 않는 한, 한 서사 속에서 파괴되기 전, 한 번만 사용할 수 있다. 단, 파괴되지 않았을 때 설치 후, 재설치를 하면 이동이 가능하다.}
		
		\cost{25}
	\end{story}
	
	\begin{story}{미신}{[클래스: 리치][광기]\hyperlink{celesteela}{ⓣ}}
		\limitedtrauma{광기}{물리적 피해로 인해 받는 피해량이 두 배로 증가한다.}
		
		\entry{마법적 피해로 인해 받는 피해량을 25\%(올림) 감소시킨다. 단, 이로 인해 받는 피해량이 0이 되도록 하거나, 최대 정신력의 10\%에 해당하는 양 이상을 감소시킬 수 없다.}
		
		\cost{-10}
	\end{story}
\end{document}