\documentclass{report}

\begin{document}
	\begin{story}{신의 축복}{[클래스:사제][집착]\hyperlink{celesteela}{ⓣ}}
		\pre{자신이 믿는 신을 설정한다. 사제의 성향과 신의 성향은 선 성향이어야 한다.}
		
		\entry{사제는 24시간에 6회 사용 가능한 [신성력]을 가진다. [신성력]으로는 대상의 생명력을 1d6만큼 회복시키거나, 언데드에게 1d6만큼의 피해를 줄 수 있다. 언데드에게 주는 피해는 회피할 수 없다. 중간 성장이나 큰 성장을 할 때마다 [신성력]의 사용 가능 횟수가 2회 증가한다.}
		
		\entry{[신성력]을 1회분 사용하여, 언데드가 아닌 대상에게 보호막을 부여할 수 있다. 보호막은 최대 6의 피해를 흡수할 수 있다. 사제의 보호막은 서로 중첩되지 않는다.}
		
		\entry{[신성력]을 1회분 사용하여, 대상의 전투, 사격, 기민, 근력 중 하나에 4턴간 +1을 부여할 수 있다. 대상은 자신이 될 수 있으며, 이 버프는 서로 다른 능력치를 증가시킨다는 제한 내에서 중첩이 가능하다.}
		
		\triggertrauma{집착}{사제는 [신성력]을 유지하기 위해 자신이 믿는 신에게 하루에 10분간 기도를 해야 한다. 이 기도를 하지 않는다.}{다시 기도를 하기 전까지 [신성력]을 사용할 수 없다.}
		
		\cost{21}
	\end{story}
	
	\begin{story}{신의 권능}{[클래스:사제]\hyperlink{celesteela}{ⓣ}}
		\entry{사제는 24시간에 1회, 대상의 부정적인 상태 이상을 해제할 수 있다. 상태 이상의 해제는 대상의 의지에 +2 보정을 받아 시도한다.}
		
		\cost{20}
	\end{story}
	
	\subsubsection{선한 성향 초월자의 사제 심화 이야기}
		\begin{story}{부활}{[클래스:사제][심화]\hyperlink{celesteela}{ⓣ}}
			\entry{자신의 [신성력]을 최대 4회분 사용해서 대상을 부활시킨다. 단, 사망이 확인된 지 1일이 넘지 않은 대상만 가능하다. 소모한 신성력 1회분당 최대 체력의 25\%에 해당하는 양의 체력을 회복시킨다. 이야기꾼인 경우, 회복량은 같으나 체력 대신 개연성을 회복시키고, 초과량에 대해 체력을 회복시킨다.}
			
			\cost{20}
		\end{story}
\end{document}