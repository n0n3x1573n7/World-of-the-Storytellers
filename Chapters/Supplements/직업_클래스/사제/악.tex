\documentclass{report}

\begin{document}
	\begin{story}[cleric:bad:bless]{신의 축복}{[클래스:사제][집착]\hyperlink{celesteela}{ⓣ}}
		\pre{자신이 믿는 신을 설정한다. 사제의 성향과 신의 성향은은 악 성향이어야 한다.}
		
		\entry{사제는 24시간에 6회 사용 가능한 [공허의 힘]을 가진다. [공허의 힘]을 사용해 언데드의 생명력을 1d6만큼 회복시키거나, 언데드가 아닌 대상에게 1d6만큼의 피해를 줄 수 있다. 언데드가 아닌 대상에게 주는 피해는 회피할 수 없다. 중간 성장이나 큰 성장을 할 때마다 [공허의 힘]의 사용 가능 회수가 2회 증가한다.}
		
		\entry{[공허의 힘]을 1회분 사용하여, 대상 언데드에게 보호막을 부여할 수 있다. 보호막은 최대 6의 피해를 흡수할 수 있다. 사제의 보호막은 서로 중첩되지 않는다.}
		
		\entry{[공허의 힘]을 1회분 사용하여, 대상의 전투, 사격, 기민, 근력 중 하나에 4턴간 +1을 부여할 수 있다. 대상은 자신이 될 수 있으며, 이 버프는 서로 다른 능력치를 증가시킨다는 제한 내에서 중첩이 가능하다.}
		
		\triggertrauma{집착}{사제는 [공허의 힘]을 유지하기 위해 자신이 믿는 신에게 하루에 10분간 기도를 해야 한다. 이 기도를 하지 않는다.}{다시 기도를 하기 전까지 [공허의 힘]을 사용할 수 없다.}
		
		\cost{21}
	\end{story}
	
	\begin{story}{신의 권능}{[클래스:사제]\hyperlink{celesteela}{ⓣ}}
		\entry{사제는 24시간에 1회, 대상의 모든 부정적인 상태 이상의 지속 시간을 2배로 늘리거나 효과를 2배로 증가시킬 수 있다. 이 효과는 대상의 의지에 -2 보정을 받아 시도한다.}
		
		\cost{20}
	\end{story}
	
	\subsubsection*{악한 성향 초월자 사제 심화 이야기}
		\begin{story}{최후의 의식}{[클래스:사제][심화]\hyperlink{celesteela}{ⓣ}}
			\pre{\storyref{cleric:bad:bless}{신의 축복}(악한 성향)}
			
			\entry{자신의 [공허의 힘]을 모두 사용한다. 소모된 [공허의 힘] 1회분당 대상의 최대 체력의 25\%에 달하는 피해를 준다. 이야기꾼의 경우, 이 의식으로 인해 추방되지 않고, 항상 개연성이 1은 남는다.}
			
			\cost{20}
		\end{story}
\end{document}