\documentclass{report}

\begin{document}
	이 파트에서는 이야기꾼 여러분과 시스템 여러분이 이야기를 진행하는 데에 도움을 주고자 여러 상황에서 사용할 수 있는 예시 이야기를 제시하고 새로운 이야기를 만드는 데에 도움을 주는 것을 목표로 합니다. 이 파트의 이야기들에는 적용이 가능하다면 이야기들의 개연성 코스트를 제시합니다. 이 개연성 코스트는 획득을 위한 기본 개연성 코스트 10이 적용되어 있는 수치임을 감안해 주세요.
	
	중요한 점은, 이 파트에 있는 이야기는 어디까지나 \emph{예시일 뿐이라는 것}을 명시하세요.
	
	이 파트의 이야기들을 만들며 여러 분들께 도움을 받았습니다. 그 목록과 도움을 받은 내용입니다:
	\begin{itemize}
		\item 소낙님: 이야기 대분류 중 몇 가지 제안
		
		\item 철화구야선생님(\href{https://twitter.com/Celesteela_S}{@Celesteela\_S}): 이야기 내용, 효과 및 개연성 코스트 점검\\
		철화구야선생님이 만들어주신 이야기들의 속성 항목 마지막에 ⓣ 표기를 해두었습니다.
		
		\item 대두님(\href{https://twitter.com/_DAEDU_}{@\_DAEDU\_}): 이야기 내용, 효과 및 개연성 코스트 점검\\
		대두님이 만들어주신 이야기들의 속성 항목 마지막에 Ⓓ  표기를 해두었습니다.
		
		\item FD(퍼디)님(\href{https://twitter.com/firstdelusion46}{@firstdelusion46}): 마법학교 세계관 룰 제작에 이야기꾼의 세계를 사용하는 것을 제안해주셨습니다.\\
		마법학교 세계관에 대해 생각하며 나온 이야기들의 속성 항목의 마지막에 ☆ 표기를 해두었습니다.
	\end{itemize}
\end{document}