\documentclass{report}

\begin{document}
	\hypertarget{story:luck-unluck}{여기}에서는 앞의 룰에서 언급한 선택 규칙인 \hyperlink{pow-of-luck-unluck}{행운의 힘과 불운의 힘}을 이야기의 형태로 구현해 두었습니다.
	
	\begin{story}[luck:pow-of-luck]{행운의 힘}{[도움: 행운]}
		\entry{모든 판정을 함에 있어, \storyref{luck:pow-of-luck}{행운의 힘}을 빌리기로 선택할 수 있다. 이 힘을 빌리기로 선택한 경우, 판정치에 4df를 굴려 합한 값이 최종 판정치가 된다.}
	\end{story}
	
	\begin{story}{랜덤의 신}{[중립: 행운]}
		\entry{모든 판정을 함에 있어, \storyref{luck:pow-of-luck}{행운의 힘}을 반드시 빌려야 한다.}
	\end{story}
	
	\begin{story}{랜덤의 신의 가호}{[중립: 행운]}
		\entry{\storyref{luck:pow-of-luck}{행운의 힘}을 빌렸다면, 판정치에 상관없이 4df의 값이 ++++이라면 대성공, -{}-{}-{}-이라면 대실패로 판정한다.}
	\end{story}
	
	\begin{story}[luck:pow-of-unluck]{불운의 힘}{[방해: 행운]}
		\entry{이야기의 "재현"이 일어날 때, 4df를 굴려 -{}-{}-{}-이라면 재현이 실패하여, 일반적인 이야기의 도움으로 취급한다. 이 이외의 경우에는 재현이 정상적으로 수행된다.}
	\end{story}
\end{document}