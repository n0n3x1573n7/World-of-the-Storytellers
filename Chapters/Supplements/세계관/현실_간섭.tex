\documentclass{report}

\begin{document}
	\begin{story}{나만 힘드냐?}{[현실]}
		\flavour{플레이어, 너도 좀 힘들어 봐라!}
		
		\entry[\hline]{4df 판정을 기준으로, 다음 운동을 하면 판정치에 +1을 부여받는다:
		\begin{itemize}
			\item 팔굽혀펴기 4\footnote{\label{random-number}임의의 수치이므로 수정이 가능합니다.}회
			\item 윗몸일으키기 5\footnoteref{random-number}회
			\item 플랭크 5\footnoteref{random-number}초 유지
		\end{itemize}
		한 판정에서는 한 종류의 운동만 할 수 있고, 연속적이어야 하며, 자세가 풀리면 그 이전까지만 인정한다.\footnote{운동의 종류를 추가할 수 있습니다.}}
	\end{story}
	
	\begin{story}{뇌물}{[현실][현금(주의)]}
		\flavour{시스템님 잘 좀 부탁드립니다(굽신)}
		
		\entry{4df 판정을 기준으로, 주사위를 \storyref{positive-dice}{긍정적 주사위}로 바꾸기 위해 현실 화폐를 지불할 수 있다.
		\begin{story}{긍정적 주사위}{[현실][물체]\hyperlink{daedu}{Ⓓ}}
			\entry[\hline]{50\% 확률로 +가 나오고, 50\% 확률로 0이 나오는 주사위. 이는 다음과 같이 구현할 수 있다:
				\begin{itemize}
					\item 동전을 던져 앞은 +, 뒤는 0으로 취급
					\item 일반적인 d6을 던져 홀수는 +, 짝수는 0으로 취급
					\item df를 던져, -가 나오면 재굴림
				\end{itemize}
				이 외에도 50\%의 확률을 구현할 수 있으면 무방하다.}
		\end{story}
		주사위를 1개 바꾸기 위해 드는 금액은 5원이며, 이 이상 추가로 바꾸기 위해서는 한개당 10배의 금액을 지불해야 한다.\footnote{즉 1개를 대체하기 위해서는 5원, 2개는 50원, 3개는 500원, 4개 전부를 대체하기 위해서는 5,000원이 필요합니다.}\footnote{\label{money-warning}이야기의 내용 상 실제 현금을 의미하기는 하나, 다른 자원과 비율으로 대체 가능하며, 이 이야기를 위하여 너무 많은 돈을 사용하지는 않으시기를 부탁드립니다.}}
		
		\entry[\hline]{4df 판정을 기준으로, 성공치를 낮추기 위해 현실 화폐를 지불할 수 있다. 성공치를 1 낮추기 위해 드는 금액은 10원이며, 이 이상 추가로 낮추기 위해서는 1당 10배의 금액을 지불해야 한다.\footnote{즉, 1을 낮추기 위해서 10원, 2를 낮추기 위해서 100원, 4를 낮추기 위해서 10,000원이 필요합니다.}\footnoteref{money-warning}}
	\end{story}
\end{document}