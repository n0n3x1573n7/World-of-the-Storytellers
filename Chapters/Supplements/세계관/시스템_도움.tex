\documentclass{report}

\begin{document}
	\begin{story}[system:tutorial]{튜토리얼}{[도움: 시스템]}
		\entry{이야기 속에서 만나는 중요해보이는 인물, 사물, 상황들을 묘사하는 이야기를 알 수 있다. 이런 이야기들을 직접 조사하여 해당 이야기에 가려서 숨겨져있던 사실을 밝혀내고, 상황을 진행시킬 수 있다. 이런 조사를 하기 위해서는 상황에 맞는 어떠한 방법을 사용해도 무방하다. 이 세계의 이야기를 풀어나가기 위해서는, 이 이야기들만을 조사해도 충분하다.}
	\end{story}
	
	\begin{story}[system:wildcard]{와일드카드}{[도움: 시스템]}
		\entry{각 이야기꾼이 이야기 속으로 들어갈 때, 상태 [와일드카드: 2]를 부여한다.}
		
		\entry{이야기꾼(들)이 본인의 이야기로 인해 역경에 처했을 때 한 개의 [와일드카드]를 얻을 수 있다. 예를 들어, 다음과 같은 경우가 있을 수 있다:
		\begin{itemize}
			\item 트라우마의 트리거가 발동된 경우.
			\item 이야기꾼이 자신의 이야기에 따른 자발적인 행동(특히 트라우마의 제약 등)으로 인해 이야기 속에서의 문제 또는 갈등이 심화될 경우.
			\item 시스템이 이야기꾼의 이야기 중 하나를 이용하여 이야기꾼에게 역경을 제시하는 경우.
		\end{itemize}
		트라우마로 인한 와일드카드 획득은 한 씬에 한 이야기당 한 개만을 얻을 수 있다.}
		
		\entry{[와일드카드] 1개를 소모하여 다음과 같이 사용할 수 있다:
		\begin{itemize}
			\item 개연성을 3 회복한다.
			\item 스탯을 사용한 판정에 +2를 부여한다.
			\item {}[침범] 판정에 +1을 부여한다.
			\item 시스템이 이야기꾼의 이야기 중 하나를 이용하여 이야기꾼에게 역경을 제시할 때, 이를 거부할 수 있다.
			\item \storyref{luck:pow-of-luck}{행운의 힘}이 적용된 판정에서 주사위의 재굴림이 가능하다.
		\end{itemize}
		판정당 사용할 수 있는 와일드카드의 수의 제한은 없다.}
		
		\entry{[와일드카드]는 [태초의 이야기]로 돌아가면 초기화되나, 한 이야기 속에 지속적으로 있을 때의 최대 소지 개수에는 제한이 없다.}
	\end{story}
\end{document}