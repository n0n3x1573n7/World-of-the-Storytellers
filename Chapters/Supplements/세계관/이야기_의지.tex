\documentclass{report}

\begin{document}
	
	\begin{story}{하나의 거대한 카지노}{[세계]\footnote{하나의 거대한 카지노, 재력, 자본 대신 다른 이름과 스탯을 사용할 수 있습니다.}}
		\flavour{이 세계는 하나의 거대한 카지노다.}
		
		\entry{체력과 정신력을 계산하는 방식처럼, [재력]을 자본을 이용해 계산한다. [재력] 역시 체력과 정신력처럼 초과 피해를 개연성으로 흡수할 수 있으나, [재력]은 최초 수치만 존재할 뿐 최대치는 존재하지 않는다.}
	\end{story}
	
	\begin{story}{열린 이야기}{[이야기: 열린 결말]}
		\flavour{이 서사는 완성된 서사가 아니기 때문에, 무수히 많은 가능성으로 가득차있다.}
		
		\entry{이 서사는 다양한 결말을 자신의 범주 하에 두고 있다. 따라서 다른 결과를 내더라도 서사가 오염된 것으로 보지 않는다.}
		
		\entry{이 서사에서는 침범 판정에서 실패해도 감소되어야 하는 개연성의 반만 감소된다.}
	\end{story}
	
	\begin{story}{적대적 이야기}{[이야기: 붕괴]}
		\flavour{이 서사는 여러분을 반기지 않는다.}
		
		\entry{서사 밖에서 들어온 이야기를 사용할 때 마다, 그 수준에 따라 d6을 굴린다\footnote{개연성 비용 5당 주사위 대략 1개를 굴리는 것으로 생각하면 됩니다.}. 각 눈별로 지금까지 누적 몇 번이 나왔는지를 기록해둔다.}
		
		\entry{이야기를 사용하지는 않았으나, 서사의 의지에 반하는 행동을 하면 그 심각도에 따라 지금까지 가장 적게 나온 눈이 일정 횟수 더 나온 것으로 취급한다. 반대로, 서사의 의지에 부합하는 행동을 하면 지금까지 가장 많이 나온 눈이 일정 횟수 덜 나온 것으로 취급할 수 있다.}
		
		\entry{한 주사위의 눈이 일정 횟수 이상 나왔다면, 해당 서사 출신이 아닌 모든 이야기꾼은 추방된다. 이 이야기로의 재진입은 시스템의 중재 없이는 불가능하다.}
	\end{story}
	
\ifDLC
	\begin{story}{소원이 있다면 검을 들라}{[세계][설화][은검의 스텔라나이츠]\footnote{이 이야기를 사용하기 위해서는, \href{http://cympub.kr/stellarknights}{은검의 스텔라나이츠} 룰북이 반드시 필요합니다.}}
		\entry{시스 또는 브링거 한 쪽만 있어도 없는 상대방의 이야기의 일부를 빌려와, 전투에 참여할 수 있다.
		
		스텔라나이트는 최대 체력과 최대 정신력이 각각 5 낮은 상태로 시작하나, 내구력과 방어력을 가진다. 내구력은 개연성과 같은 역할이나, 체력과 정신력 이후에 깎이는 대신 그 이전에 깎인다. 스텔라나이트의 내구력을 변화시키는 기술은 비-스텔라나이트의 체력과 정신력을 변화시킬 수 있으며, 그 반대 역시 가능하다.}
		
		\entry{이 이야기의 개연성 비용은 20이며, 스텔라나이트 스킬은 각각 개연성 비용 10의 단일 이야기로 취급한다.
		
		시스의 경우 브링거가 사용할 수 있는 스킬 최대 2종류를 사용할 수 있으며, 한 턴에 최대 하나만 사용할 수 있다.
		
		브링거의 경우 스킬 최대 3종류를 사용할 수 있다. 턴이 시작할 때 행운 판정을 하여, 성공한다면 두 번, 실패한다면 한 번 사용할 수 있다.
		
		둘 모두 있으며 시스와 브링거가 모두 동의하여 스텔라나이트로 변신한 경우, 스탯은 브링거의 것으로, 이야기와 개연성은 합친 것으로 취급하나, 시스의 턴은 브링거와 함께 진행한다. 스킬 최대 6종류를 사용할 수 있고, 스킬을 한 턴에 최대 두 개까지 사용할 수 있다.}
		
		\entry{비-스텔라나이트 캐릭터의 방어력은 3+(근력 또는 의지 중 높은 쪽)으로 취급한다.
		
		누군가의 방어력이 증가하거나 감소했을 때, 비-스텔라나이트 캐릭터의 공격을 받으면 방어력이 증가한 경우 해당 수치만큼 피해량이 경감되지만 1 미만으로는 감소되지 않으며, 감소한 경우 해당 수치만큼 피해량이 추가되지만 두 배 초과로는 증가되지 않는다.}
		
		\entry{
			\begin{story}{부케}{[스텔라나이트]}
				\entry{전투가 시작할 때, 스텔라나이트 이야기꾼은 부케 3개+(동료 이야기꾼 수)를 받는다. 부케 하나를 소모해서 다음을 할 수 있다:
				
				\begin{itemize}
					\item 스킬 1회 추가 사용, ``자기 차례"가 아닌 스킬 사용, 또는 스킬에 주사위 하나를 세트.
					\item 공격 판정의 주사위 수 하나 증가. 한 공격 판정에 최대 3번까지만 사용할 수 있다.
					\item 어떤 판정이든 주사위를 모두 다시 굴린다. 한 판정에 한번만 사용할 수 있다.
				\end{itemize}
				
				스텔라나이트 이야기꾼의 턴이 종료되었을 때 사용하지 않은 스킬 1회당 부케 1개를 추가로 획득한다.}
			\end{story}
			}
		
		\entry{
			\begin{story}{왜곡}{[스텔라나이트]}
				\entry{스텔라나이트 이야기꾼은 서사당 단 한번, 왜곡을 사용할 수 있다. 이는 서사에서 추방되기 직전에도 사용할 수 있으며, 다음 중 하나를 할 수 있다:
				
				\begin{itemize}
					\item 개연성이 0이라면 개연성을 1 회복하고, 내구력을 2d6 회복한다.
					\item 개연성이 0이라면 개연성을 1 회복하고, 내구력을 4 회복하며, 다음 공격 판정에 주사위 3개를 추가하는 동시에, 1이 나온 주사위를 모두 다시 굴릴 수 있다.
				\end{itemize}}
			\end{story}
		}
	\end{story}
\fi
	
\ifDLC
	\begin{story}{외상성 충격}{[트라우마][이야기: 방해][inSANe]}
		\entry{이 서사 속에 들어오면, [공포도] (7-의지)/13를 받는다. 또한, 이 서사 속에서만 적용되는 [전용 트라우마]가 준비되어 있다. 전용 트라우마는 하나의 이야기로 취급하며, 코스트는 0이며, 보통 획득 즉시 정신력에 피해를 주는 동시에 특정 효과를 준다.}
		
		\entry{서사 속에서, 정신적인 충격을 받을 만한 일이 일어나면, 2d6을 굴린다. 이 수치와 [공포도]를 비교했을때:
			
			\begin{center}
				\begin{tabular}{l|l}
					비교         & 결과 \\\hline\hline
					2d6 $\geq$ 공포도 & [공포도]가 1 증가한다. \\\hline
					2d6 $<$ 공포도    & [공포도]가 현재 2d6에서 나온 수치로 감소하고, [전용 트라우마]를 한 개 받는다.\\
				\end{tabular}
			\end{center}
			
			[공포도]는 이 외의 방법으로 변화시킬 수 없다.
		}
	\end{story}
\fi

\ifDLC
	\begin{story}{점점 미쳐가는 이야기}{[이야기: 방해][설화][Call of Cthulhu]}
		\entry{이성치 100을 얻는다. 일반적으로 이해할 수 없는 일(살인, 시체, 고대의 존재 등)을 목격하거나 정신적인 충격을 받을 만한 일이 일어날때마다, 그 수준에 따라 난이도와 피해량(성공/실패, 다이스 사용 가능)을 시스템이 지정한다. 그러면 1d(현재 이성치)를 굴려, 해당 수치가 난이도 이상이라면 이성치에 성공 피해량을, 미만이라면 실패 피해량을 받는다. (e.g. 60[1d4/2d8]의 경우 주사위의 결과가 60 이상이 나오면 성공하며, 성공시 1d4 정신력, 실패시 2d8 정신력을 잃는다.)}
		
		\entry{현재 이성치의 10\% 이상에 달하는 피해를 한번에 받은 이야기꾼은 해당 상황에 대한 이 시나리오에서의 [기피]를, 20\% 이상인 경우 이 시나리오에서의 [공포]를 얻는다. 25\% 이상인 경우, [공포]를 얻는 것은 같으나 이는 시나리오에서 벗어나도 유지된다. 이를 방지하기 위해 이성치에 받는 피해의 전부 또는 일부를 1대1의 비율로 정신력에 받을 수 있다.}
	\end{story}
\fi
\end{document}