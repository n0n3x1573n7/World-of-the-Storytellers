\documentclass{report}

\begin{document}
	
\ifDLC
	\begin{story}{외상성 충격}{[트라우마][이야기: 방해][inSANe]}
		\entry{이 서사 속에 들어오면, [공포도] (7-의지)/13를 받는다. 또한, 이 서사 속에서만 적용되는 [전용 트라우마]가 준비되어 있다. 전용 트라우마는 하나의 이야기로 취급하며, 코스트는 0이며, 보통 획득 즉시 정신력에 피해를 주는 동시에 특정 효과를 준다.}
		
		\entry[\hline]{서사 속에서, 정신적인 충격을 받을 만한 일이 일어나면, 2d6을 굴린다. 이 수치와 [공포도]를 비교했을때:
		
		\begin{center}
			\begin{tabular}{l|l}
				비교         & 결과 \\\hline\hline
				2d6 $\geq$ 공포도 & [공포도]가 1 증가한다. \\\hline
				2d6 $<$ 공포도    & [공포도]가 1 감소하고, [전용 트라우마]를 한 개 받는다.\\
			\end{tabular}
		\end{center}
		
		[공포도]는 이 외의 방법으로 변화시킬 수 없다.
		}
	\end{story}
\fi
	
\ifDLC
	\begin{story}{점점 미쳐가는 이야기}{[이야기: 방해][설화][Call of Cthulhu]}
		\entry{이성치 100을 얻는다. 일반적으로 이해할 수 없는 일(살인, 시체, 고대의 존재 등)을 목격하거나 정신적인 충격을 받을 만한 일이 일어날때마다, 그 수준에 따라 난이도와 피해량(성공/실패, 다이스 사용 가능)을 시스템이 지정한다. 그러면 1d(현재 이성치)를 굴려, 해당 수치가 난이도 이상이라면 이성치에 성공 피해량을, 미만이라면 실패 피해량을 받는다. (e.g. 60[1d4/2d8]의 경우 주사위의 결과가 60 이상이 나오면 성공하며, 성공시 1d4 정신력, 실패시 2d8 정신력을 잃는다.)}
		
		\entry[\hline]{현재 이성치의 10\% 이상에 달하는 피해를 한번에 받은 이야기꾼은 해당 상황에 대한 이 시나리오에서의 [기피]를, 20\% 이상인 경우 이 시나리오에서의 [공포]를 얻는다. 25\% 이상인 경우, [공포]를 얻는 것은 같으나 이는 시나리오에서 벗어나도 유지된다. 이를 방지하기 위해 이성치에 받는 피해의 전부 또는 일부를 1대1의 비율로 정신력에 받을 수 있다.}
	\end{story}
\fi
	
	\begin{story}{하나의 거대한 카지노}{[세계]\footnote{하나의 거대한 카지노, 재력, 자본 대신 다른 이름과 스탯을 사용할 수 있습니다.}}
		\flavour{이 세계는 하나의 거대한 카지노다.}
		
		\entry[\hline]{체력과 정신력을 계산하는 방식처럼, [재력]을 자본을 이용해 계산한다. [재력] 역시 체력과 정신력처럼 초과 피해를 개연성으로 흡수할 수 있으나, [재력]은 최초 수치만 존재할 뿐 최대치는 존재하지 않는다.}
	\end{story}
	
	\begin{story}{열린 이야기}{[이야기: 열린 결말]}
		\flavour{이 서사는 완성된 서사가 아니기 때문에, 무수히 많은 가능성으로 가득차있다.}
		
		\entry{이 서사는 다양한 결말을 자신의 범주 하에 두고 있다. 따라서 다른 결과를 내더라도 서사가 오염된 것으로 보지 않는다.}
		
		\entry[\hline]{이 서사에서는 침범 판정에서 실패해도 감소되어야 하는 개연성의 반만 감소된다.}
	\end{story}
\end{document}