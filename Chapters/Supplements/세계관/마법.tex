\documentclass{report}

\begin{document}
	\begin{story}[magic:mana]{마나}{[마법]☆}
		\entry{마법을 사용할 때에 재료가 필요하지 않다고 하더라도, [마력]을 소모해야만 한다. 마법을 사용할 수 있는 이들은 [마력]을 가지고 있고, 이는 특정 정신과 관련된 스탯\footnote{마력의 운용 방식에 따라, 정신적 침습을 버티며 사용해야 하는 경우에는 의지, 마나에 대한 지식을 사용한다면 지식:마나, 노래를 통해 운용한다면 제작:음악 등 다양한 스탯에 연관지을 수 있습니다.}에 비례한다. 이는 마력의 운영 방법에 따라 다른 스탯을 사용할 수 있다. 마력은 시간이 지나면 충전된다.}
	\end{story}

	\begin{story}[magic:focus]{집중}{[마법][기도]☆}
		\entry{마법을 사용할 때에 재료가 필요하지 않다면, 별도의 자원 소모가 없는 대신, 모든 마법에는 부작용이 있다. 마법을 사용하는 매 턴 의지 판정을 해 0 초과의 수치가 나오지 않으면 마법의 부작용이 적용된다.}
	\end{story}
	
	\begin{story}[magic:chi]{기}{[흑마법]☆\footnote{통상적인 체력 또는 피 만을 소진하는 흑마법의 경우에는 [피의 마법]으로, 정신력을 소모하는 마법의 경우에는 [타락의 마법]으로 [기]의 이름을 바꿀 수도 있을 것입니다.}}
		\entry{마법을 사용할 때에 재료가 필요하지 않다고 하더라도, 체력 또는 정신력을 소모해야만 한다. 소모된 체력 또는 정신력 1당 [기]를 일정량 충전해 사용할 수 있다.}
	\end{story}
\end{document}