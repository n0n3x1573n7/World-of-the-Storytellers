\documentclass{report}

\begin{document}
	\begin{story}{호감도}{[관계]}
		다른 인물에게 가지는 호감도를 수치로 나타낸다. 숫자별로 다음을 나타낸다:
		
		\begin{tabularx}{\textwidth}{l|X}
			호감도 수치 & 관계의 예시\\
			\hline
			\hline
			-2 미만     & 적\\
			\hline
			-1          & 첫 인상이 좋지 않았던 사람\\
			\hline
			0           & 모르는 사람\\
			\hline
			1           & 아는 사람\\
			\hline
			2 이상      & 친구\\
		\end{tabularx}
		
		호감도 수치가 존재하지 않는 인물에 대한 호감도는 0으로 취급한다.
		
		[인맥] 판정을 할 때에는 이야기꾼의 상대에 대한 호감도 수치를 더해 판정한다. 아래 두 경우를 제외하고, [인맥] 판정에 실패한 경우 호감도를 1 소모하고 성공한 것으로 취급할 수 있다.
		
		해당 인물이 이야기꾼의 호감을 사는 행동을 하거나, 이야기꾼의 신념에 맞는 행동을 한다면 [인맥] 판정을 한다. 인맥 판정의 수치가 호감도 수치보다 높게 나왔다면, 이야기꾼의 상대에 대한 호감도 수치가 1 증가한다.
		
		해당 인물이 이야기꾼의 원한을 살 행동을 하거나, 이야기꾼의 신념에 반하는 행동을 한다면 [인맥] 판정을 한다. 인맥 판정의 수치가 호감도 수치보다 낮거나 같게 나왔다면, 이야기꾼의 상대에 대한 호감도 수치가 1 감소한다.
	\end{story}

	\begin{story}{관계의 이야기}{[관계]}
		관계를 이야기로서 표현한다. 일방적인 관계라면 일반적인 이야기로 취급하나, 대칭적인 관계라면 기본 개연성 코스트는 10이 아닌 5로 계산하고, 추가적으로 서로 동일하거나 유사한 능력이 존재한다면 능력으로 인한 개연성 코스트 추가량은 반으로 감소한다.
	\end{story}
\end{document}