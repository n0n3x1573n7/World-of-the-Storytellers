\documentclass{report}

\begin{document}
	\begin{story}{부상 상태}{[상태]}
		\entry{체력 또는 정신력에 피해를 입을 때, 받는 피해량이 2 이상인 경우, 피해량 2를 감소시키는 대신 관련된 임시 상태 한개를 받는다. 적합한 내용이 없는 경우, ``부상"으로만 적어도 무방하다. 이는 해당 씬이 종료될때까지 유지되며, 관련된 판정에 계속해서 -1 페널티를 받는다.}
	\end{story}
	
	\begin{story}{체력과 정신력}{[스탯]}
		\entry{근력과 의지 스탯이 체력과 정신력에 영향을 주지 않는다.\footnote{구버전 룰에 사용하여 현재 버전의 규칙과 동일하게 체력과 정신력을 계산할 수 있는 확장 이야기입니다.}}
		
		\entry{[신체]와 [정신] 스탯을 가진다. 이 스탯을 사용한 판정은 불가능하며, 특화 스탯으로 취급한다. [신체]는 체력에, [정신]은 정신력에 기존의 근력과 의지처럼 영향을 주나, 스탯 1당 체력/정신력 5를 증감시킨다.}
	\end{story}
\end{document}