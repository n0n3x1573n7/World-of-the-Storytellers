\documentclass{report}

\begin{document}
	\begin{story}[probability:the-story-continues]{계속되는 이야기}{[이야기: 연속]\footnote{룰북에서 언급한 선택 규칙인 \hyperlink{the-story-continues}{계속되는 이야기}를 이야기의 형태로 구현해 두었습니다.}}
		\flavour{한 서사 속에서 여러 연속적인 서사가 진행되기 때문에, 한 서사가 끝난 이후에도 침범 판정의 난이도의 초기화가 진행되지 않는다.}
		
		\entry[\hline]{한 서사가 끝났을 때, [침범] 판정의 난이도를 확인하여, 다음 서사가 시작할 때, [침범] 판정의 난이도는 해당 난이도에서 시작한다.}
	\end{story}
	
	\begin{story}{시간의 끝}{[시간]}
		\entry[\hline]{이 서사는 이미 시간의 끝을 향해 달려가고 있기 때문에 서사의 개연성을 해칠 염려가 매우 적다. 따라서, [침범] 판정에 실패할 때, 다음 씬(비전투) 또는 두 턴 후(전투)까지 해당 판정에 실패한 이야기가 전면 봉쇄되어, 기술 뿐 아니라 해당 이야기로부터의 도움도 받을 수 없다(단, 스탯은 유지된다). 개연성 판정 난이도의 초기화는 이야기의 봉쇄가 일어나면 즉시 일어나나, 서사가 불안정해짐에 따라 판정이 일어날 때 마다 난이도가 1 상승한다.}
	\end{story}
	
	\begin{story}{1인칭 등장인물 시점}{[세계: 몰입][등장인물: (이름)]\footnote{1인칭 등장인물 시점 대신, 1인칭 주인공/관찰자/주연/조연 시점 등으로 바꿀 수 있습니다.}}
		\entry{이 서사에 돌입할 때, 반드시 역할을 하나 지급받는다. 이 서사 속에서는 이야기꾼 본인보다는 이 역할에 몰입하게 된다.}
		
		\entry[\hline]{추방 판정의 난이도는 개개인별로 측정되며, 최초에는 0에서 시작한다. 역할로 지급받은 이야기가 아닌 이야기를 사용하기 위해서는 반드시 추방 판정이 필요하다. 판정이 성공한다면 판정의 난이도가 1 증가한다. 판정이 실패한다면 해당 이야기는 다음 씬이 종료될때까지 봉인되며, 판정의 난이도가 해당 이야기의 침범도만큼 낮아진다. 매 씬이 종료될 때 추방 판정의 난이도가 반(버림)으로 감소한다.}
	\end{story}
	
	\begin{story}{벌점과 징계}{[학교]☆}
		\entry[\hline]{이 서사는 학교 내에서 벌어지는 일을 다루고 있다. 대신, 각 등장인물과 이야기꾼은 [악명: 0]과 [벌점: 0] 상태를 가진다. [침범] 판정의 조건은 ``학교 내의 평화를 어지럽히는 행위"로 바뀌며, 판정의 난이도는 이야기 전체에 적용되지 않고, 개인의 [악명] 수치를 이용한다. [침범] 판정에 성공한다면 [악명] 수치가 1 증가하고, [침범] 판정에 실패한다면 [악명] 수치가 0으로 초기화되는 대신 벌점이 1 증가한다. 벌점에 따라 학교 내에서의 행위에 다음 제약이 걸린다\footnote{벌점에 따른 징계 단계를 세분화하거나, 벌점 최대치를 변경하는 등의 수정이 가능합니다.}:
		\begin{itemize}
			\item \textbf{10점}: 학교에서 퇴학되어, 이야기꾼이라면 서사에서 추방당한다.
		\end{itemize}
		[악명]과 [벌점]은 학교 내의 권위자(교사 등)에 의해 감소될 수 있다.}
	\end{story}
\end{document}