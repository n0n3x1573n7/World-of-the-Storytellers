\documentclass{report}

\begin{document}
	\begin{story}{벌점과 징계}{[학교]☆}
		이 이야기는 학교 내에서 벌어지는 일을 다루고 있다. 대신, 각 등장인물과 이야기꾼은 [악명: 0]과 [벌점: 0] 상태를 가진다. [침범] 판정의 조건은 "학교 내의 평화를 어지럽히는 행위"로 바뀌며, 판정의 난이도는 이야기 전체에 적용되지 않고, 개인의 [악명] 수치를 이용한다. [침범] 판정에 성공한다면 [악명] 수치가 1 증가하고, [침범] 판정에 실패한다면 [악명] 수치가 0으로 초기화되는 대신 벌점이 1 증가한다. 벌점에 따라 학교 내에서의 행위에 다음 제약이 걸린다\footnote{벌점에 따른 징계 단계를 세분화하거나, 벌점 최대치를 변경하는 등의 수정이 가능합니다.}:
		\begin{itemize}
			\item \textbf{10점}: 학교에서 퇴학되어, 이야기꾼이라면 이야기에서 추방당한다.
		\end{itemize}
		[악명]과 [벌점]은 학교 내의 권위자(교사 등)에 의해 감소될 수 있다.
	\end{story}
	
	\begin{story}{시간의 끝}{[시간]}
		이 이야기는 이미 시간의 끝을 향해 달려가고 있기 때문에 이야기의 개연성을 해칠 염려가 매우 적다. 따라서, [침범] 판정에 실패할 때, 다음 씬(비전투) 또는 두 턴 후(전투)까지 해당 판정에 실패한 이야기가 전면 봉쇄되어, 기술 뿐 아니라 해당 이야기로부터의 도움도 받을 수 없다(단, 스탯은 유지된다). 개연성 판정 난이도의 초기화는 이야기의 봉쇄가 일어나면 즉시 일어나나, 이야기가 불안정해짐에 따라 판정이 일어날 때 마다 난이도가 1 상승한다.
	\end{story}
\end{document}