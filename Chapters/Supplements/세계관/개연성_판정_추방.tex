\documentclass{report}

\begin{document}
	\begin{story}[probability:the-story-continues]{계속되는 이야기}{[이야기: 연속]\footnote{룰북에서 언급한 선택 규칙인 \hyperlink{the-story-continues}{계속되는 이야기}를 이야기의 형태로 구현해 두었습니다.}}
		\flavour{한 서사 속에서 여러 연속적인 서사가 진행되기 때문에, 한 서사가 끝난 이후에도 침범 판정의 난이도의 초기화가 진행되지 않는다.}
		
		\entry[\hline]{한 서사가 끝났을 때, [침범] 판정의 난이도를 확인하여, 다음 서사가 시작할 때, [침범] 판정의 난이도는 해당 난이도에서 시작한다.}
	\end{story}
	
	\begin{story}{시간의 끝}{[시간]}
		\entry[\hline]{이 서사는 이미 시간의 끝을 향해 달려가고 있기 때문에 서사의 개연성을 해칠 염려가 매우 적다. 따라서, [침범] 판정에 실패할 때, 다음 씬(비전투) 또는 두 턴 후(전투)까지 해당 판정에 실패한 이야기가 전면 봉쇄되어, 기술 뿐 아니라 해당 이야기로부터의 도움도 받을 수 없다(단, 스탯은 유지된다). 개연성 판정 난이도의 초기화는 이야기의 봉쇄가 일어나면 즉시 일어나나, 서사가 불안정해짐에 따라 판정이 일어날 때 마다 난이도가 1 상승한다.}
	\end{story}
	
	\begin{story}{1인칭 등장인물 시점}{[세계: 몰입][등장인물: (이름)]\footnote{1인칭 등장인물 시점 대신, 1인칭 주인공/관찰자/주연/조연 시점 등으로 바꿀 수 있습니다.}}
		\entry{이 서사에 돌입할 때, 반드시 역할을 하나 지급받는다. 이 서사 속에서는 이야기꾼 본인보다는 이 역할에 몰입하게 된다.}
		
		\entry[\hline]{추방 판정의 난이도는 개개인별로 측정되며, 최초에는 0에서 시작한다. 역할로 지급받은 이야기가 아닌 이야기를 사용하기 위해서는 반드시 추방 판정이 필요하다. 판정이 성공한다면 판정의 난이도가 1 증가한다. 판정이 실패한다면 해당 이야기는 다음 씬이 종료될때까지 봉인되며, 판정의 난이도가 해당 이야기의 침범도만큼 낮아진다. 매 씬이 종료될 때 추방 판정의 난이도가 반(버림)으로 감소한다.}
	\end{story}
	
	\begin{story}{벌점과 징계}{[학교]\hyperlink{fd46}{☆}}
		\entry[\hline]{이 서사는 학교 내에서 벌어지는 일을 다루고 있다. 대신, 각 등장인물과 이야기꾼은 [악명: 0]과 [벌점: 0] 상태를 가진다. [침범] 판정의 조건은 ``학교 내의 평화를 어지럽히는 행위"로 바뀌며, 판정의 난이도는 이야기 전체에 적용되지 않고, 개인의 [악명] 수치를 이용한다. [침범] 판정에 성공한다면 [악명] 수치가 1 증가하고, [침범] 판정에 실패한다면 [악명] 수치가 0으로 초기화되는 대신 벌점이 1 증가한다. 벌점에 따라 학교 내에서의 행위에 다음 제약이 걸린다\footnote{벌점에 따른 징계 단계를 세분화하거나, 벌점 최대치를 변경하는 등의 수정이 가능합니다.}:
		\begin{itemize}
			\item \textbf{10점}: 학교에서 퇴학되어, 이야기꾼이라면 서사에서 추방당한다.
		\end{itemize}
		[악명]과 [벌점]은 학교 내의 권위자(교사 등)에 의해 감소될 수 있다.}
	\end{story}
	
	\begin{story}{추격자}{[범죄자]}
		\flavour{이 세계에 들어온 당신들은 모두 범법자다. 추격을 따돌리고 살아남아라.}
		
		\entry{\textbf{[추격] 판정}: 4df를 굴린다. 이 판정의 결과에 따라 다음 일이 일어난다:
		\begin{itemize}
			\item \textbf{성공치 이상}: 성공치가 1 증가한다.
			\item \textbf{성공치 미만}: 성공치가 -4로 초기화된다. 즉시 해당 장소에 판정치가 성공치보다 낮은 만큼의 수의 추격자가 나타난다\footnote{성공치가 2이고, 주사위가 -1인 경우, 3(=2-(-1))명의 추격자가 나타납니다.}. 추격자가 아닌 모든 이는 기민을 사용해 추격자와 추격전을 벌인다. 추격자의 기민은 0으로 취급하고, 수적으로 우위인 만큼 추가 보너스를 받으나, 그렇지 않더라도 페널티를 받지는 않는다. 단, 수적으로 열세인 경우 나타난 추격자 만큼만 체포할 수 있으며, 판정값이 낮은 사람을 우선으로 체포한다\footnote{예를 들어 등장인물이 1명이고 추격자가 3명인 경우 추격자는 2의 보너스를, 등장인물이 2명이고 추격자가 1명인 경우 추격자는 보너스도 페널티도 받지 않습니다. 단, 이 경우 둘 중 한명만 체포할 수 있습니다.}. 이 판정의 결과에 따라:
				\begin{itemize}
					\item 판정 실패시, 추격자는 더 고도의 기술력을 가지고 있기 때문에 체포되어 사살된다. 이야기꾼의 경우, 추방된다. 이 추방으로 인해서 추방 시퀀스가 발동되지는 않으나, 한 씬 동안 재진입이 불가능하다.
					\item 판정 성공시, 도망에 성공한다.
				\end{itemize}
		\end{itemize}
		최초의 성공치는 -4이다\footnote{이 성공치는 세계관이 얼마나 위험한가에 따라 조정할 수 있습니다. 가령, 공권력이 검문이 심한 서사의 경우 최초 성공치를 조금 높게 잡을 수 있고, 제약이 약한 서사의 경우 성공치를 낮춰 한번 등장 한 후 몇 씬간은 등장하지 않도록 할수도 있습니다.}. 이 수치는 모두에게 적용된다.}
		
		\entry{매 씬, 등장 여부를 선택할 수 있다. 등장하지 않기로 선택했다면, [추격] 판정을 피할 수 있다.}
		
		\entry[\hline]{매 씬이 종료될 때 반드시, 해당 씬에 등장했다면 [추격] 판정을 한다.}
	\end{story}
\end{document}