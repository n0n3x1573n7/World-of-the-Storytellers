\documentclass{report}

\begin{document}
	\begin{story}{무아지경}{[마법:집중]}
		\pre{\storyref{magic:focus}{집중} 메커니즘을 사용하는 마법을 사용할 수 있다.}
		
		\entry{\storyref{magic:focus}{집중} 메커니즘을 사용하는 마법을 사용할 때, 집중 해제 판정에 의지 대신 인식으로 판정할 수 있다. 인식으로 판정하는 경우, \emph{실패 또는 대실패}해야 집중을 이어갈 수 있다. 인식 판정에 성공한 경우, 집중은 해제되나[위협을 느낌 □: 1턴]을 얻는다.}
		
		\cost{20}
	\end{story}
	
	\begin{story}{능숙한 이야기꾼}{[이야기꾼]}
		\flavour{이야기꾼의 극의에 달한 자.}
		
		\entry{한 서사에서 단 한 번, 이야기의 진리, 즉 물리적인 사실 등을 무시한 일을 한 번 행할 수 있다. 이를 사용한 판정은 자동으로 대성공한다.}
		
		\entry{이야기로부터 도움을 받을 때, 일반적인 도움일 때에도 추가 +1을 받는다.}
		
		\cost{30}
	\end{story}
\end{document}