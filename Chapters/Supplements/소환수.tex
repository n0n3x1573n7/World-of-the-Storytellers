\documentclass{report}

\begin{document}
	\begin{story}[summon]{소환수}{[소환수: (소환수 이름)]\footnote{소환수 대신, 퍼밀리어, 펫, 컴패니언, 동물 친구 등 다른 용어를 사용해도 무방합니다.}}
		\pre{소환수를 위해 소모할 이야기꾼의 코스트 \textnormal{X}를 정한다. \textnormal{X}는 양수일 필요는 없다.}
		
		\entry{(소환수 이름)을 소환수로 부릴 수 있다. 소환수는 기본 최대 개연성 수치의 시작이 2X로 시작하는 별도의 이야기꾼으로 취급한다. 이 때 이 수치는 이야기꾼의 기본 최대 개연성 수치를 넘어설 수 없으며, 소환수의 최대 개연성 수치 역시 이야기꾼의 최대 개연성 수치를 넘어설 수 없다. 소환수를 만들 때 역시 이야기꾼 생성 규칙 중 배경 이야기를 3개가 아닌 1개만 정해도 된다는 사실을 제외하면 해당 규칙을 모두 따라야 한다\footnote{예를 들어, 최대 개연성이 음수가 되도록 이야기를 정할수는 없습니다.}.}
		
		\entry{소환수는 \storyref{summon}{소환수}를 이야기로 가질 수 없다.}
		
		\entry{소환수는 이야기꾼이 최대 개연성을 소모하여 소환수에게 개연성을 더 할당하지 않는 이상은 성장할 수 없다.}
		
		\cost{10+X\footnote{소환수의 소환 해제가 가능하다면, 해제된 동안의 코스트는 10으로 취급합니다.}}
	\end{story}
	
	\section{드루이드(Druid)}
		\documentclass{report}

\begin{document}
	\begin{story}{}{[클래스:드루이드]\hyperlink{celesteela}{ⓣ}}
		
	\end{story}
	
	\subsection{드루이드 심화 이야기}
		\begin{story}{}{[클래스:드루이드][심화]\hyperlink{celesteela}{ⓣ}}
			
		\end{story}
\end{document}
	
	\section{해적 선장}
		\documentclass{report}

\begin{document}
	\begin{story}[captain:pirate]{해적 선장}{[직업][소환수]}
		\flavour{해적선의 주인이자 선원들의 고용주.}
		
		\entry{이 이야기는 아래 사항들을 제외하고, \storyref{summon}{소환수}로 취급한다.}
		
		\entry{해적선과 무기, 선원들을 소환수로서 사용할 수 있다. 소환수를 소환한 동안 자금에 다음 페널티를 받는다:
		
		\begin{center}
			\begin{tabular}{c|c}
			소환 대상 & 페널티 \\\hline\hline
			\hyperlink{captain:sailor}{선원} & -1 \\\hline
			\hyperlink{captain:cannon}{대포} & -2 \\\hline
			\hyperlink{captain:ship}{해적선} & -3
			\end{tabular}
		\end{center}
		이로 인해 해적 선장의 자금이 음수가 될 수 없다. 자금이 허락하는 한 여러 개체를 소환할 수 있다.}
		
		\entry{\statchange{+}{자금[3]}}
		
		\cost{25}
	\end{story}
	
	\hypertarget{captain:sailor}{}
	\subsubsection*{해적 선원}
		체력 10, 정신력 10, 개연성 0
		
		모든 스탯 0
		
		\begin{story}{어진가 조금 모자란 친구들}{[생애][소환수]}
			\flavour{선장의 명령을 절대적으로 따르는 충직한 선원.}
			
			\entry{자의적인 소환 해제가 불가능하나, 소환된 동안 명령이 없으면 자의적으로 행동할 수 있으며, 씬이 종료될 때 소환을 해제할 수 있다.}
			
			\entry{다음 신체적 결함 중 한 가지를 지정한다.
			\begin{tightcenter}
				\begin{tabular}{c|c}
					신체적 결함 & 스탯 변화 \\\hline\hline
					의안 & \statchange{-}{관찰} \\\hline
					의족 & \statchange{-}{기민} \\\hline
					의수 & \statchange{-}{근력}
				\end{tabular}
			\end{tightcenter}
			[의수]를 선택해도 체력은 변하지 않는다.}
			
			\cost{10}
		\end{story}
		
		\begin{story}{고용 관계}{[기피]}
			\entry{선원이 사망하면 소환은 해제되며, 선장의 \storyref{captain:pirate}{해적 선장} 이야기에 ``\statchange{-}{자금}"이 추가된다. 이는 [태초의 이야기]로 되돌아가면 초기화된다.}
			
			\cost{-10}
		\end{story}
	
	\hypertarget{captain:cannon}{}
	\subsubsection*{대포}
		\begin{story}{대포}{[생애][소환수]}
			\entry{내구 10을 가진다.}
			
			\entry{자의적인 소환 해제가 불가능하다. 소환된 씬이 종료되면 즉시 소환이 해제된다.}
			
			\entry{내구 1을 소모한다. 한 구역에 피해 3을, 주변 구역에 방사 피해 1을 준다.}
			
			\cost{20}
		\end{story}
		
		\begin{story}{파괴된 무기}{[공포]}
			\entry{무기가 파괴되면 소환은 해제되며, 선장의 \storyref{captain:pirate}{해적 선장} 이야기에 ``\statchange{-}{자금[2]}"이 추가된다. 이는 [태초의 이야기]로 되돌아가면 초기화된다.}
			
			\cost{-20}
		\end{story}
	
	\hypertarget{captain:ship}{}
	\subsubsection*{해적선}
		\begin{story}{해적선}{[생애][소환수]}
			\entry{내구 40을 가진다.}
			
			\entry{자의적인 소환 해제가 불가능하다. 씬이 종료되어야 소환을 해제하기로 선택할 수 있다.}
			
			\entry{바다 위를 자유롭게 항해할 수 있다.}
			
			\entry{해적선 밖에서 안으로 피해를 줄 때, 내부에 있는 인원에게 피해를 주지 못하는 대신, 피해량의 반 만큼 해적선의 내구도가 감소한다.}
			
			\entry{해적선상의 전투에서 누군가 피해를 받으면 해적선의 내구가 1 감소한다.}
			
			\entry{해적선에 타 있는 동안, 피해를 주는 대상을 해적선을 설정하여 직접 해적선의 내구에 피해를 줄 수 있다.}
			
			\cost{30}
		\end{story}
		
		\begin{story}{파괴된 해적선}{[광기]}
			\entry{해적선의 내구도가 0이 되면 소환은 해제되며, 선장의 \storyref{captain:pirate}{해적 선장} 이야기에 ``\statchange{-}{자금[3]}"이 추가된다. 이는 [태초의 이야기]로 되돌아가면 초기화된다.}
			
			\entry{해적선이 파괴될 때 선상에 있던 모든 이들은 즉시 해적선이 존재하던 표면 상(바다 위, 땅 등)으로 낙하하며, 해적선을 파괴시킨 이를 제외한 모든 선상에 있던 이들은 상태이상 [패닉]을 얻는다.}
			
			\cost{-30}
		\end{story}
\end{document}
	
	\section{네크로맨서(Necromancer)}
		\documentclass{report}

\begin{document}
	\begin{story}[necromancer:half-dead]{죽음을 반쯤 경험한 자}{[종족: (이야기꾼의 종족)][종족: (언데드 종족 중 하나)]}
		\pre{언데드 종족 중 하나를 선택한다.\footnote{서플리먼트를 기준으로, \hyperlink{species:vampire}{[종족: 뱀파이어]}, \hyperlink{species:ghost}{[종족: 귀신]}, \hyperlink{species:zombie}{[종족: 좀비]}, \hyperlink{species:skeleton}{[종족: 스켈레톤]}이 있습니다.}}
		
		\entry{종족이 반은 본래의 종족, 반은 해당 언데드 종족이 된다. 해당 언데드 종족의 이야기를 전부 가진다. 단, 해당 이야기들로 인해 다른 살아있는 이들의 종족을 변하게 할 수는 없다.}
		
		\cost{20}
	\end{story}
	
	\begin{story}{언데드 부활}{[클래스: 네크로맨서]}
		\pre{클래스: 네크로맨서, \storyref{necromancer:half-dead}{죽음을 반쯤 경험한 자}}
		
		\entry{시체가 있다면, 그를 전투중이 아닐 때 \storyref{necromancer:half-dead}{죽음을 반쯤 경험한 자}에서 선택한 종족으로 변화시킬 수 있다. 해당 시체에게 선택한 언데드의 모든 이야기를 준다. 해당 시체는 생전의 이야기를 모두 가진 채로 이야기꾼의 \storyref{summon}{소환수}로서 행동한다. 이 이야기의 코스트에 소환수로 만들어질 때의 비용이 더해지나, 이미 존재하는 시체를 소환수로서 사용한 것이기 때문에 소환수의 비용이 25\% 감소한다.}
		
		\cost{20}
	\end{story}
\end{document}
\end{document}