\documentclass{report}

\begin{document}
	\begin{story}[summon]{소환수}{[소환수: (소환수 이름)]\footnote{소환수 대신, 퍼밀리어, 펫, 컴패니언, 동물 친구 등 다른 용어를 사용해도 무방합니다.}}
		\pre{소환수를 위해 소모할 이야기꾼의 코스트 \textnormal{X}를 정한다. \textnormal{X}는 양수일 필요는 없다.}
		
		\entry{(소환수 이름)을 소환수로 부릴 수 있다. 소환수는 기본 최대 개연성 수치의 시작이 2X로 시작하는 별도의 이야기꾼으로 취급한다. 이 때 이 수치는 이야기꾼의 기본 최대 개연성 수치를 넘어설 수 없으며, 소환수의 최대 개연성 수치 역시 이야기꾼의 최대 개연성 수치를 넘어설 수 없다. 소환수를 만들 때 역시 이야기꾼 생성 규칙 중 배경 이야기를 3개가 아닌 1개만 정해도 된다는 사실을 제외하면 해당 규칙을 모두 따라야 한다\footnote{예를 들어, 최대 개연성이 음수가 되도록 이야기를 정할수는 없습니다.}.}
		
		\entry{소환수는 \storyref{summon}{소환수}를 이야기로 가질 수 없다.}
		
		\entry{소환수는 이야기꾼이 최대 개연성을 소모하여 소환수에게 개연성을 더 할당하지 않는 이상은 성장할 수 없다.}
		
		\cost{10+X\footnote{소환수의 소환 해제가 가능하다면, 해제된 동안의 코스트는 10으로 취급합니다.}}
	\end{story}
	
	\section{드루이드(Druid)}
		\documentclass{report}

\begin{document}
	\begin{story}{}{[클래스:드루이드]\hyperlink{celesteela}{ⓣ}}
		
	\end{story}
	
	\subsection{드루이드 심화 이야기}
		\begin{story}{}{[클래스:드루이드][심화]\hyperlink{celesteela}{ⓣ}}
			
		\end{story}
\end{document}
	
	\section{해적 선장}
		\documentclass{report}

\begin{document}
	\begin{story}[captain:pirate]{해적 선장}{[직업][소환수]}
		\flavour{해적선의 주인이자 선원들의 고용주.}
		
		\entry{이 이야기는 아래 사항들을 제외하고, \storyref{summon}{소환수}로 취급한다.}
		
		\entry{해적선과 무기, 선원들을 소환수로서 사용할 수 있다. 소환수를 소환한 동안 자금에 다음 페널티를 받는다:
		
		\begin{center}
			\begin{tabular}{c|c}
			소환 대상 & 페널티 \\\hline\hline
			\hyperlink{captain:sailor}{선원} & -1 \\\hline
			\hyperlink{captain:cannon}{대포} & -2 \\\hline
			\hyperlink{captain:ship}{해적선} & -3
			\end{tabular}
		\end{center}
		이로 인해 해적 선장의 자금이 음수가 될 수 없다. 자금이 허락하는 한 여러 개체를 소환할 수 있다.}
		
		\entry{\statchange{+}{자금[3]}}
		
		\cost{25}
	\end{story}
	
	\hypertarget{captain:sailor}{}
	\subsubsection*{해적 선원}
		체력 10, 정신력 10, 개연성 0
		
		모든 스탯 0
		
		\begin{story}{어진가 조금 모자란 친구들}{[생애][소환수]}
			\flavour{선장의 명령을 절대적으로 따르는 충직한 선원.}
			
			\entry{자의적인 소환 해제가 불가능하나, 소환된 동안 명령이 없으면 자의적으로 행동할 수 있으며, 씬이 종료될 때 소환을 해제할 수 있다.}
			
			\entry{다음 신체적 결함 중 한 가지를 지정한다.
			\begin{tightcenter}
				\begin{tabular}{c|c}
					신체적 결함 & 스탯 변화 \\\hline\hline
					의안 & \statchange{-}{관찰} \\\hline
					의족 & \statchange{-}{기민} \\\hline
					의수 & \statchange{-}{근력}
				\end{tabular}
			\end{tightcenter}
			[의수]를 선택해도 체력은 변하지 않는다.}
			
			\cost{10}
		\end{story}
		
		\begin{story}{고용 관계}{[기피]}
			\entry{선원이 사망하면 소환은 해제되며, 선장의 \storyref{captain:pirate}{해적 선장} 이야기에 ``\statchange{-}{자금}"이 추가된다. 이는 [태초의 이야기]로 되돌아가면 초기화된다.}
			
			\cost{-10}
		\end{story}
	
	\hypertarget{captain:cannon}{}
	\subsubsection*{대포}
		\begin{story}{대포}{[생애][소환수]}
			\entry{내구 10을 가진다.}
			
			\entry{자의적인 소환 해제가 불가능하다. 소환된 씬이 종료되면 즉시 소환이 해제된다.}
			
			\entry{내구 1을 소모한다. 한 구역에 피해 3을, 주변 구역에 방사 피해 1을 준다.}
			
			\cost{20}
		\end{story}
		
		\begin{story}{파괴된 무기}{[공포]}
			\entry{무기가 파괴되면 소환은 해제되며, 선장의 \storyref{captain:pirate}{해적 선장} 이야기에 ``\statchange{-}{자금[2]}"이 추가된다. 이는 [태초의 이야기]로 되돌아가면 초기화된다.}
			
			\cost{-20}
		\end{story}
	
	\hypertarget{captain:ship}{}
	\subsubsection*{해적선}
		\begin{story}{해적선}{[생애][소환수]}
			\entry{내구 40을 가진다.}
			
			\entry{자의적인 소환 해제가 불가능하다. 씬이 종료되어야 소환을 해제하기로 선택할 수 있다.}
			
			\entry{바다 위를 자유롭게 항해할 수 있다.}
			
			\entry{해적선 밖에서 안으로 피해를 줄 때, 내부에 있는 인원에게 피해를 주지 못하는 대신, 피해량의 반 만큼 해적선의 내구도가 감소한다.}
			
			\entry{해적선상의 전투에서 누군가 피해를 받으면 해적선의 내구가 1 감소한다.}
			
			\entry{해적선에 타 있는 동안, 피해를 주는 대상을 해적선을 설정하여 직접 해적선의 내구에 피해를 줄 수 있다.}
			
			\cost{30}
		\end{story}
		
		\begin{story}{파괴된 해적선}{[광기]}
			\entry{해적선의 내구도가 0이 되면 소환은 해제되며, 선장의 \storyref{captain:pirate}{해적 선장} 이야기에 ``\statchange{-}{자금[3]}"이 추가된다. 이는 [태초의 이야기]로 되돌아가면 초기화된다.}
			
			\entry{해적선이 파괴될 때 선상에 있던 모든 이들은 즉시 해적선이 존재하던 표면 상(바다 위, 땅 등)으로 낙하하며, 해적선을 파괴시킨 이를 제외한 모든 선상에 있던 이들은 상태이상 [패닉]을 얻는다.}
			
			\cost{-30}
		\end{story}
\end{document}
	
	\section{네크로맨서(Necromancer)}
		\documentclass{report}

\begin{document}
	\begin{story}[necromancer:half-dead]{죽음을 반쯤 경험한 자}{[종족: (이야기꾼의 종족)][종족: (언데드 종족 중 하나)]}
		\pre{언데드 종족 중 하나를 선택한다.\footnote{서플리먼트를 기준으로, \hyperlink{species:vampire}{[종족: 뱀파이어]}, \hyperlink{species:ghost}{[종족: 귀신]}, \hyperlink{species:zombie}{[종족: 좀비]}, \hyperlink{species:skeleton}{[종족: 스켈레톤]}이 있습니다.}}
		
		\entry{종족이 반은 본래의 종족, 반은 해당 언데드 종족이 된다. 해당 언데드 종족의 이야기를 전부 가진다. 단, 해당 이야기들로 인해 다른 살아있는 이들의 종족을 변하게 할 수는 없다.}
		
		\cost{20}
	\end{story}
	
	\begin{story}{언데드 부활}{[클래스: 네크로맨서]}
		\pre{클래스: 네크로맨서, \storyref{necromancer:half-dead}{죽음을 반쯤 경험한 자}}
		
		\entry{시체가 있다면, 그를 전투중이 아닐 때 \storyref{necromancer:half-dead}{죽음을 반쯤 경험한 자}에서 선택한 종족으로 변화시킬 수 있다. 해당 시체에게 선택한 언데드의 모든 이야기를 준다. 해당 시체는 생전의 이야기를 모두 가진 채로 이야기꾼의 \storyref{summon}{소환수}로서 행동한다. 이 이야기의 코스트에 소환수로 만들어질 때의 비용이 더해지나, 이미 존재하는 시체를 소환수로서 사용한 것이기 때문에 소환수의 비용이 25\% 감소한다.}
		
		\cost{20}
	\end{story}
\end{document}
	
	\section{꿈꾸는 자(The Dreamer)}
		\documentclass{report}

\begin{document}
	\begin{story}[dreamer:sleep]{잠}{[클래스:꿈꾸는 자]}
		\flavour{잠을 자는 것을 너무나도 좋아한다.}
		
		\entry{언제 어디서든, 한 턴을 소모하여 잠을 잘 수 있다. 잠을 자는 동안, 모든 상태이상의 작용이 멈추며, 정신력에 피해를 받을 수 없다. 잠에서 깨어나면 상태이상의 작용이 재개된다.}
		
		\cost{20}
	\end{story}
	
	\begin{story}[dreamer:dream]{꿈}{[클래스: 꿈꾸는 자]}
		\flavour{꿈 속에서는 무엇이든 할 수 있었다.}
		
		\entry{잠을 자는 동안, 한 턴을 소모해 꿈을 꾸기로 선택할 수 있다. 꿈을 꾸면, 다음 \storyref{summon}{소환수} 중 하나를 소환할 수 있다:
		\begin{center}
			\begin{tabular}{c!{\color{black}\vrule}c}
				소환 대상 & 코스트 \\\hline\hline
				\hyperlink{dreamer:sheep}{양 한마리} & 10 \\\hline
				\hyperlink{dreamer:unicorn}{유니콘} & 15 \\\hline
				\hyperlink{dreamer:bad-dream}{그림자 괴물} & 15 \\
			\end{tabular}
		\end{center}
		소환수의 기본 스탯은 꿈꾸는 자의 스탯과 동일하나, 소환수에 따라 추가 스탯을 받게 될 수 있다. 단, 개연성은 0으로 취급하며 체력 또는 정신력이 0에 다다르면 강제로 디스폰된다.}
		
		\entry{소환수의 시야는 꿈꾸는 자와 공유된다.}
		
		\entry{꿈을 꾸는 동안이라도 소환자는 피해를 받을 수 있으나, 피해를 받아 소환자가 추방되더라도 소환수는 사라지지 않는다.}
		
		\cost{10+X\footnote{소환수에 따라 코스트가 달라집니다.}}
	\end{story}
	
	\begin{story}[dreamer:nightmare]{악몽}{[클래스: 꿈꾸는 자][공포]}
		\pre{\storyref{dreamer:sleep}{잠}, \storyref{dreamer:dream}{꿈}}
		
		\triggertrauma{공포}{\storyref{dreamer:dream}{꿈}에 의한 소환수가 타의에 의해 디스폰된다.}{즉시 \storyref{dreamer:sleep}{잠}에서 깨어난다. 이번 씬 동안, \storyref{dreamer:sleep}{잠}들기 위해 필요한 턴이 한 턴 증가하고, \storyref{dreamer:dream}{꿈}꾸기 위해 필요한 턴도 한 턴 증가한다. 이가 여러번 일어나면 이 턴 수는 누적된다.}
		
		\cost{-15}
	\end{story}
	
	\hypertarget{dreamer:sheep}{}
	\subsubsection*{양 한마리}
		\begin{story}{양 한마리}{[꿈]}
			\entry{소환될 때, 꿈꾸는 자가 있는 구역에 소환된다. 꿈꾸는 자가 있는 구역에 다른 누군가가 있다면 해당하는 이들은 강제로 즉시 인접한 구역으로 이동해야 한다.}
			
			\entry{\begin{story}{양 두마리, 양 세마리}{[꿈:양 한마리]}
				\entry{한 턴을 소모하여, 양이 있는 구역과 인접한 빈 구역 중 하나에 추가로 양을 소환한다. 또는, 한 턴을 소모하여 소환되어 있는 양 전부 또는 일부를 디스폰한다.}
				
				\entry{양이 모두 디스폰되면, \storyref{dreamer:dream}{꿈}에서 깨어나지만, \storyref{dreamer:sleep}{잠}에서는 깨어나지 않는다.}
			\end{story}}
			
			\entry{양은 꿈꾸는 자와 구역을 공유할 수 있으나, 다른 이들과는 구역을 공유할 수 없다. 꿈꾸는 자와 구역을 공유한 동안에는, 꿈꾸는 자를 향한 모든 공격은 그 구역에 있는 양을 향한다.}
			
			\entry{양은 체력을 모두 공유하며, 공격 또는 방어할 수 없다.}
			
			\entry{\begin{story}{편안한 꿈}{[꿈:양 한마리]}
				\entry{타인의 공격으로 인해 소멸되더라도 \storyref{dreamer:nightmare}{악몽}의 트리거를 발동시키지 않는다.}
			\end{story}}
			
			\cost{20}
		\end{story}
	
	\hypertarget{dreamer:unicorn}{}
	\subsubsection*{유니콘}
	\begin{story}{상상속의 말}{[꿈]}
		\entry{소환될 때, 꿈꾸는 자가 있는 구역에서 꿈꾸는 자를 태운 채로 소환된다.}
		
		\entry{\begin{story}{가속하는 말}{[꿈:유니콘]}
			\entry{유니콘은 소환될 때와 매 턴이 시작될 때, [가속] 상태를 2회 받는다. [가속] 상태는 최대 7회까지 누적되며, 한 턴에 얼마든지 소모할 수 있다.}
			
			\entry{유니콘은 [가속] 상태 1회를 소모함으로서 한 구역을 이동할 수 있다.}
		\end{story}}
		
		\entry{\begin{story}[unicorn:flying]{비행하는 말}{[꿈:유니콘]}
			\entry{유니콘은 [가속] 상태 2회를 소모함으로서 공중으로 떠오를 수 있다. 다시 바닥으로 내려오것은 언제든지 가능하다.}
		
			\entry{유니콘은 떠 있지 않는 동안 꿈꾸운 자를 태우거나, 내리게 할 수 있다. 이를 하는 데에 한 턴이 소요된다. 소환한 꿈꾸는 자 본인이 아닌 이는 태울 수 없다.}
			
			\entry{공중에 떠오른 동안, [가속] 상태 1회를 소모하여 두 구역까지 이동할 수 있다.}
		\end{story}}
		
		\entry{\begin{story}{말발굽 차기}{[꿈:유니콘]}
			\entry{유니콘은 [가속] 상태 1회를 소모함으로서 1회당 같은 구역에 있는 한 대상에게 피해 2를 줄 수 있다. \storyref{unicorn:flying}{비행하는 말}으로 떠오른 동안은 공격을 할 수 없으나, 바닥으로 내려오는 동시에 사용한다면 피해량이 3으로 증가한다.}
		\end{story}}
		
		\entry{\begin{story}{회피 기동}{[꿈:유니콘]}
			\entry{유니콘은 방어할 수 없다. 대신, [가속] 상태를 소모하여 1회당 피해량 2를 경감시킬 수 있다. \storyref{unicorn:flying}{비행하는 말}으로 떠오른 동안은 피해량 3이 경감되나, 꿈꾸는 자를 태운 동안이라면 피해량 경감량이 1 감소한다.}
		\end{story}}
		
		\cost{30}
	\end{story}
	
	\hypertarget{dreamer:bad-dream}{}
	\subsection*{악몽의 산물}
	\begin{story}{그림자 괴물}{[꿈]}
		\entry{소환될 때, 꿈꾸는 자를 집어삼킨 채로 등장한다.}
		
		\entry{\begin{story}{악몽}{[꿈: 그림자 괴물]}
				\limitedtrauma{기피}{자의적으로 소환을 해제하더라도 \storyref{dreamer:nightmare}{악몽}의 트리거를 발동시킨다.}
				
				\limitedtrauma{공포}{소환을 한 후부터 해제하기 전까지 매 턴, 꿈꾸는 자는 의지 판정을 해 0 이상의 값이 나오지 않으면 정신력을 0보다 낮은만큼 잃는다.}
			\end{story}}
		
		\entry{\begin{story}{퍼져가는 악몽}{[꿈: 그림자 괴물]}
				\entry{그림자 괴물은 시야 안에 들어오는 어떤 곳으로도 이동할 수 있다. 이동한 직후, 누군가가 눈치채기 전까지 \statchange{+}{은신[2]}를 얻는다.}
				
				\entry{누군가를 \storyref{shadow-beast:gulp}{집어삼킨}채로도 이동할 수 있다. 이 경우, 스탯의 추가는 없다.}
		\end{story}}
		
		\entry{\begin{story}{발목 붙잡기}{[꿈: 그림자 괴물]}
				\entry{그림자 괴물은 같은 구역에 있는 자신을 인식하지 못한 모든 이의 발목을 붙잡아 놀래키기로 시도할 수 있다. 이 경우, 그림자 괴물의 은신에 대항하여 모든 대상은 인식 판정을 한다. 인식 판정에 실패한 모든 대상에게 그림자 괴물은 다음 중 두 개를 적용시킨다:
				\begin{itemize}
					\item 체력에 피해 1
					\item 물리적 상태 [넘어짐 □]
					\item 정신력에 피해 1
					\item 정신적 상태 [깜짝 놀람 □]
			\end{itemize}}
		\end{story}}
		
		\entry{\begin{story}[shadow-beast:gulp]{집어삼키기}{[꿈: 그림자 괴물]}
				\entry{그림자 괴물은 같은 구역에 있는 대상 하나를 집어삼키기로 할 수 있다. 이 경우, 그림자 괴물의 은신에 대항하여 대상은 인식과 기민 중 낮은 쪽으로 판정한다. 그림자 괴물이 판정에서 이겼다면, 대상은 악몽의 세계로 집어삼켜진다. 누군가를 집어삼킨 동안, \statchange{-}{은신[2]}를 얻는다.}
				
				\entry{집어삼켜진 대상은 그림자 괴물이나 악몽의 세계 밖의 이에게 영향을 줄 수 없으며, 자신의 턴이 시작할 때 체력과 정신력에 피해를 1 입는다. 매 자신의 턴에 한번씩, 그림자 괴물에게서 다음 방법으로 빠져나가고자 시도할 수 있다:
				\begin{itemize}
					\item 그림자 괴물의 근력에 대항한 근력 판정에 성공시, 악몽의 세계를 힘으로 찢으며 탈출한다.
					\item 그림자 괴물의 의지에 대항한 의지 판정에 성공시, 악몽의 세계를 정신으로 부정하며 탈출한다.
					\item 한번에 그림자 괴물의 남은 체력과 정신력의 합의 반 이상의 피해를 주는 능력을 사용하면, 악몽의 세계가 불안정해지며 집어삼켜진 이를 토해낸다.
				\end{itemize}}
					
				\entry{꿈꾸는 자는 자의적으로 집어삼켜질 수 있으며, 이로 인한 추가적인 부정적인 효과는 없다.}
				
				\entry{그림자 괴물은 턴이 시작할때 누군가를 집어삼키고 있었다면 자의적으로 뱉어낼 수 있다. 이 경우, 이미 이번 턴에 한 번 이동했더라도 다시 이동할 수 있다.}
		\end{story}}
		
		\cost{29}
	\end{story}
	
	\subsubsection*{꿈꾸는 자 심화 이야기}
		\begin{story}{기면증}{[클래스:꿈꾸는 자][심화][기피]}
			\pre{\storyref{dreamer:sleep}{잠}, \storyref{dreamer:dream}{꿈}}
			
			\triggertrauma{기피}{씬이 시작한다.}{\storyref{dreamer:sleep}{잠}에 빠져든다.}
			
			\cost{0}
		\end{story}
		
		\begin{story}{몽유병}{[클래스:꿈꾸는 자][심화]}
			\pre{\storyref{dreamer:sleep}{잠}, \storyref{dreamer:dream}{꿈}}
			
			\entry{\storyref{dreamer:sleep}{잠}을 자거나, \storyref{dreamer:dream}{꿈}을 꾸면서도 이동할 수 있으나, 두 턴에 한 구역만을 이동할 수 있다.}
			
			\cost{15}
		\end{story}
		
		\begin{story}{깊은 잠}{[클래스:꿈꾸는 자][심화][기피]}
			\pre{\storyref{dreamer:sleep}{잠}, \storyref{dreamer:dream}{꿈}}
			
			\limitedtrauma{기피}{전투 상황이 아니라면, \storyref{dreamer:sleep}{잠}에서 깨어날 수 없고, 한번 잠들면 \storyref{dreamer:dream}{꿈}을 통해서만 다른 이들과 소통할 수 있다.}
			
			\cost{0}
		\end{story}
\end{document}
\end{document}