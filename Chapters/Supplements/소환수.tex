\documentclass{report}

\begin{document}
	\begin{story}[summon]{소환수}{[소환수: (소환수 이름)]\footnote{소환수 대신, 퍼밀리어, 펫, 컴패니언, 동물 친구 등 다른 용어를 사용해도 무방합니다.}}
		\pre{소환수를 위해 소모할 이야기꾼의 코스트 \textnormal{X}를 정한다. \textnormal{X}는 양수일 필요는 없다.}
		
		(소환수 이름)을 소환수로 부릴 수 있다. 소환수는 기본 최대 개연성 수치의 시작이 2X로 시작하는 별도의 이야기꾼으로 취급한다. 이 때 이 수치는 이야기꾼의 기본 최대 개연성 수치를 넘어설 수 없으며, 소환수의 최대 개연성 수치 역시 이야기꾼의 최대 개연성 수치를 넘어설 수 없다. 소환수를 만들 때 역시 이야기꾼 생성 규칙 중 배경 이야기를 3개가 아닌 1개만 정해도 된다는 사실을 제외하면 해당 규칙을 모두 따라야 한다\footnote{예를 들어, 최대 개연성이 음수가 되도록 이야기를 정할수는 없습니다.}.
		
		소환수는 \storyref{summon}{소환수}를 이야기로 가질 수 없다.
		
		소환수는 이야기꾼이 최대 개연성을 소모하여 소환수에게 개연성을 더 할당하지 않는 이상은 성장할 수 없다.
		
		\cost{10+X\footnote{소환수의 소환 해제가 가능하다면, 해제된 동안의 코스트는 10으로 취급합니다.}}
	\end{story}
\end{document}