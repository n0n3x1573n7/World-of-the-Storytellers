\documentclass{report}

\begin{document}
	\begin{story}{더 이상 살아있지 못한 자}{[종족: 귀신][광기]}
		\textbf{효과(광기)}: [빙의] 또는 [폴터가이스트]가 발동되고 있지 않은 동안, 물리적인 효과를 주거나 받을 수 없다.
		
		\textbf{효과(광기)}: 영체의 [실체화], [빙의] 또는 [폴터가이스트]가 발동되고 있지 않은 동안, 타인에게 시각적으로 인식될 수 없다.
		
		\smallskip
		
		\textbf{개연성 코스트}: -50
	\end{story}
	
	\begin{story}{빙의}{[종족: 귀신]}
		지성체에 빙의할 수 있다. 귀신은 공감으로 공격하고, 대상은 도발 또는 의지로 수비한다. 공격이 성공하면, 대상에 빙의하여 대상을 조종하고, 대상의 이야기 등을 모두 사용할 수 있다.
		
		받은 물리적 피해는 대상에게 피해를 주나, 신성 속성이 있다면 귀신의 영체에도 동일한 양의 피해를 준다. 받은 정신적 피해는 대상에게는 피해를 주지 않고, 귀신에게만 영체 피해를 준다.
		
		빙의된 대상은 귀신이 자신 또는 귀신 본인의 이야기를 사용할 때 마다 저항할 수 있다. 빙의시와 같은 판정을 한다. 귀신이 승리한다면 차이만큼의 정신적 피해를 대상에게 주며, 대상이 승리한다면 귀신은 차이만큼의 영체 피해를 받고 빙의가 해제된다.
		
		대상의 정신력이 0이 되면 귀신은 대상의 신체를 빼앗아, 대상의 종족이 [귀신]이 되게 만든다. 귀신은 대상의 성격적인 이야기를 제외한 모든 이야기를 빼앗고 종족이 대상의 종족으로 변한다.
		
		빙의는 귀신이 소멸하면 자동으로 해제되며, 귀신이 자의적으로 해제할 수 있다.
		
		\smallskip
		
		\textbf{개연성 코스트}: 20
	\end{story}
	
	\begin{story}{폴터가이스트}{[종족: 귀신]}
		비지성체를 조종할 수 있다. 이를 이용해 직접적인 피해를 가할 수 없다.
		
		모든 피해는 신성 속성이 아니라면 폴터가이스트에 걸린 비지성체를 손상시킬 수 없다. 신성 속성인 경우 비지성체에 손상을 줄 수 있음과 동시에 귀신에게 영체 피해를 준다.
		
		\smallskip
		
		\textbf{개연성 코스트}: 15
	\end{story}
	
	\begin{story}{영체}{[종족: 귀신]}
		신성 속성이 없는 모든 물리적 효과에 면역이다. 체력과 정신력 수치를 더해 [영체] 수치로 취급한다. 영체 수치는 체력/정신력과 마찬가지로 매 씬마다 회복되며, 초과하는 피해를 받으면 마찬가지로 개연성에 피해를 받는다.
		
		개연성 수치가 0이 되면 이야기에서 추방되는 대신, [소멸]되어 영원히 사라진다. 대신, [태초의 이야기]에 돌아가면 개연성 수치를 모두 회복할 수 있다.
		
		영체의 [실체화]를 발동하면 타인에게 생전의 모습을 가진 반투명한 영체의 형태로 인식될 수 있다.
		
		\smallskip
		
		\textbf{개연성 코스트}: 20
	\end{story}
\end{document}