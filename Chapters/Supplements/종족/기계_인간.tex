\documentclass{report}

\begin{document}
	\begin{story}{양철 나무꾼}{[종족: 기계 인간][기피]Ⓓ}
		\textit{오즈의 마법사에 나오는 양철 나무꾼처럼, 인간의 마음을 가지고 싶어하는, 녹슬기 쉬운 기계 종족입니다.}
		
		\textbf{트리거(기피)}: 폭우, 강, 바다 등 다량의 물을 보거나, 그런 곳에 존재한다.
		
		\textbf{효과}: 물이 있는 곳으로 자발적으로 이동할 수 없다.
		
		\smallskip
		
		\textbf{스탯 -}: 공감
		
		\cost{-5}
	\end{story}
	
	\begin{story}[humanoid:iron-body]{철로 만들어진 몸체}{[종족: 기계 인간][공포]Ⓓ}
		\textit{강도는 높지만, 내구성이 떨어지면 점점 효율이 떨어지는 기계 몸체를 가지고 있습니다.}
		
		\smallskip
		
		최대 체력이 20 증가한다.
		
		\smallskip
		
		\textbf{트리거(공포)}: 체력이 모두 소진된다.
		
		\textbf{효과}: 모든 물리적 스탯(기민, 근력, 제작, 운전, 전투, 사격)이 -1 된 것으로 취급한다.
		
		\cost{-5\footnote{최대 체력의 증가는 근력 +2의 하위호환입니다. 그렇기 때문에 해당 스탯 증가에 해당하는 양보다 적은 개연성 코스트를 할당했습니다.}}
	\end{story}
	
	\begin{story}{블랙박스}{[종족: 기계 인간]Ⓓ}
		한 씬에 한 번, 관찰 판정에 자동으로 성공할 수 있다. 이 능력은 관찰 판정에 이미 실패한 상태에서도 사용할 수 있다.
		
		\cost{20}
	\end{story}
	
	\subsection{기계 인간 심화 능력}
		\begin{story}{기계 모듈}{[종족: 기계 인간][심화]}
			매 씬이 시작할 때, 물리적 스탯(기민, 근력, 제작, 운전, 전투, 사격) 중 하나에 +1을 하고, 다른 하나에 -1을 할 수 있다. 이로 인해서 최대 체력이 변하지는 않는다.
			
			\cost{15}
		\end{story}
		
		\begin{story}{자가 수리}{[종족: 기계 인간][심화]}
			전투 씬에서, 매 턴이 시작할 때, 최대 체력의 20\%에 해당하는 양의 체력을 회복한다. \storyref{humanoid:iron-body}{철로 만들어진 몸체}의 공포증 트리거가 발동되어 있는 동안, 최대 체력의 10\%로 회복량이 감소한다.
			
			\cost{38\footnote{최대 체력에 20이 추가로 주어지고, 기본 최대 체력이 10 주어지며, 근력을 0 또는 1 올렸다고 가정했을 때, 최대 체력은 30\textasciitilde40이 됩니다. 이 경우의 회복량은 6\textasciitilde8으로, 평균치인 7 회복을 가정했을 때의 코스트입니다.}}	
		\end{story}
\end{document}