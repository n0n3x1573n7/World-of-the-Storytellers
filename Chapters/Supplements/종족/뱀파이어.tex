\documentclass{report}

\begin{document}
	\begin{story}[vampire:kiss]{뱀파이어의 키스}{[종족: 뱀파이어]}
		\pre{종족: 뱀파이어}
		
		다른 이야기꾼 또는 등장인물이 뱀파이어를 보고 있다면 대상을 유혹할 수 있다. 뱀파이어는 인맥으로 공격 판정을 하고, 대상은 의지 또는 기만으로 수피 판정을 한다. 공격이 성공한다면 대상은 판정치의 차이의 수치에 해당하는 턴만큼 유지되는 상태 [유혹: (뱀파이어 이름)]을 얻고, 뱀파이어를 직접 공격할 수 없다. 방어가 성공한다면 대상은 판정치의 차이의 수치에 해당하는 턴만큼 \storyref{vampire:kiss}{뱀파이어의 키스}에 면역이 된다.
		
		\smallskip
		
		같은 구역 안에 있는 [유혹: (뱀파이어 이름)] 상태를 가진 이야기꾼 또는 등장인물의 목을 물 수 있다. 같은 구역 안에 있는 다른 이들에 의해 기회공격을 받을 수 있으나, 뱀파이어가 그로 인해 사망하거나 \storyref{vampire:transform}{변신}을 통해 도망가지 않았다면 대상의 종족에 뱀파이어가 추가되고, [유혹: (뱀파이어 이름)] 상태가 [지배: (뱀파이어 이름)]으로 변화하여 영구히 유지된다.
		
		\smallskip
		
		\textbf{스탯 +}: 인맥
		
		\smallskip
		
		\cost{30}
	\end{story}
	
	\begin{story}[vampire:transform]{변신: 뱀파이어}{[종족: 뱀파이어]}
		\pre{종족: 뱀파이어}
		
		다음 중 하나로 변신해 해당 효과를 받을 수 있다:
		\begin{itemize}
			\item \textbf{안개}: 물리 공격 면역, 마법 및 정신 공격 회피 또는 방어 불가
			\item \textbf{박쥐}: 비행 가능, 어둠 속에서 시야 확보 가능
		\end{itemize}
		
		변신은 \storyref{vampire:kiss}{뱀파이어의 키스}로 인한 기회공격을 받았을 때 한 번의 물리 피해를 모두 자동으로 회피하는데에 사용할 수 있다.
		
		\smallskip
		
		\cost{30}
	\end{story}
	
	\begin{story}[vampire:fang]{커다란 송곳니}{[종족: 뱀파이어][공포]}
		\pre{종족: 뱀파이어}
		
		\textbf{제약(공포)}: 얼굴을 본 상대가 \storyref{vampire:kiss}{뱀파이어의 키스}의 희생양이 아니라면 종족을 숨길 수 없다.
		
		\smallskip
		
		\cost{-10}
	\end{story}
	
	\begin{story}{영면}{[종족: 뱀파이어][광기]}
		\pre{종족: 뱀파이어}
		
		\textbf{트리거(광기)}: 태양빛을 직접 쬔다.
		
		\textbf{효과}: 태양빛을 피하기 전까지 \storyref{vampire:kiss}{뱀파이어의 키스}와 \storyref{vampire:transform}{변신}이 비활성화된다. 비전투중이라면, 태양빛을 가릴 물체를 찾기 전까지 다른 모든 행동을 할 수 없다. 전투중이라면, 매 턴이 시작할 때 최대체력의 10\%에 해당하는 피해(최소 1)를 받는다.
		
		\smallskip
		
		\textbf{트리거(광기)}: [신성]속성의 공격(성수, 신성력 등)을 받는다.
		
		\textbf{효과}: 공격자가 허용하기 전까지 \storyref{vampire:kiss}{뱀파이어의 키스}와 \storyref{vampire:transform}{변신}이 비활성화된다. [신성]속성의 추가 공격력 보너스가 상대에게 없다면, 피해량의 50\%에 달하는 양의 정신력 피해를 받는다.
		
		\smallskip
		
		\cost{-50}
	\end{story}
	
\end{document}