\documentclass{report}

\begin{document}
	\begin{story}{뼈 무더기}{[종족: 스켈레톤]}
		\pre{종족: 스켈레톤}
		
		\flavour{죽은 채로 움직이는, 뼈만 남은 앙상한 종족이다.}
		
		\entry{몸을 자유자재로 해체할 수 있다. 특히, 전투 중에 자신의 턴이 아닐 때에도 몸을 완전히 해체하여 뼈 무더기가 된다면 모든 공격을 회피할 수 있다. 단, 다음 턴이 끝날때까지 아무것도 할 수 없으며, 턴이 끝났을 때까지 해당 구역 등에 지속되는 효과 또는 [신성] 속성의 공격에 대해서는 회피가 불가능하다.}
		
		\cost{25}
	\end{story}
	
	\begin{story}[skeleton:creaky-joint]{삐그덕거리는 관절}{[종족: 스켈레톤][공포]}
		\pre{종족: 스켈레톤}
		
		\triggertrauma{공포}{누군가가 자신이 스켈레톤임을 처음으로 눈치챈다.}{해당 인물은 하루동안 기피증 \storyref{skeleton:living-dead}{산송장}을 받는다.
		
		\begin{story}[skeleton:living-dead]{산송장}{[기피: 스켈레톤]}
			\limitedtrauma{기피}{이 공포증을 준 대상에게 호의적으로 행동할 수 없다.}
			
			\cost{0}
		\end{story}}
		
		\entry{\statchange{-}{근력[2], 민첩[2], 은신}}
		
		\cost{-35}
	\end{story}
	
	\begin{story}{분리 가능한 뼈}{[종족: 스켈레톤]}
		\pre{종족: 스켈레톤}
		
		\entry{뼈를 분리/조합할 수 있다. 특히, 신체의 일부를 분리해도 자신의 의지대로 움직일 수 있으며, 다른 뼈를 구할 수 있다면, 자신의 몸에 연결해 자신의 것 처럼 사용할 수 있다. 특히, 이로 인해 추가 이야기나 스탯을 받게 될 수 있다.\footnote{예를 들어, 공룡 뼈를 연결했다면 근력을 추가로 받을 수 있고, 새의 날개뼈를 팔 대신 연결했다면 비행 능력을 가진 이야기 등을 얻게 될 수 있습니다.}}
		
		\cost{20}
	\end{story}
\end{document}