\documentclass{report}

\begin{document}
	자연에 동화되어 생활하는 자연친화적인, 옅은 갈색의 피부와 큰 키, 뾰족한 귀가 특징인 종족입니다.
	
	\begin{story}{육감}{[종족: 숲 엘프]ⓣ}
		\entry{하루에 최대 두 번, 인식 판정에 +1을 받을 수 있다.}
		
		\cost{15}
	\end{story}
	
	\begin{story}{날랜 추적자}{[종족: 숲 엘프]ⓣ}
		\statchange{+}{기민}
		
		\statchange{-}{근력, (기민을 제외한 특화되지 않은 스탯 중 하나 선택)}
		
		\cost{5}
	\end{story}
	
	\begin{story}[forest-elfnature-warrior]{대자연의 전사}{[종족: 숲 엘프][기피][공포]ⓣ}
		\pre{거주하던 자연환경 하나를 미리 선택한다.}
		
		\flavour{자연 속에서 거주하며, 세상 물정에 어둡다.}
		
		\limitedtrauma{기피}{도시 사람들과 대화할 때, 세상 물정에 어둡기 때문에 관련 지식과 인맥 판정에 -1을 받는다.}
		
		\triggertrauma{공포}{무차별적인 자연 파괴 현장을 목격한다.}{1d6을 굴려, 다음을 적용시킨다:
		
		\begin{center}
			\begin{tabularx}{\linewidth}{c|X}
				\textbf{주사위} & \textbf{효과}                                        \\\hline\hline
				1               & 정신적 상태이상 [절망]을 얻고, 행동불가 상태가 된다. \\\hline
				2/3            & 정신적 상태이상 [이성을 잃음]을 얻고, 자연을 파괴한 대상이 무력화될 때 까지 대상에게 할 수 있는 가장 강한 공격을 한다. \\\hline
				3/4            & 대상에게 협상을 시도한다. 협상이 실패한다면, 2/3의 눈이 나온것과 동일하게 취급한다. \\\hline
				6               & 이성적으로 생각하여 분노할 뿐, 아무 일도 일어나지 않는다. \\
			\end{tabularx}
		\end{center}}
		
		\statchange{+}{지식:(거주지)[2]}
		
		\cost{-14}
	\end{story}
	
	\begin{story}{향수병}{[종족: 숲 엘프][트라우마: 가변]ⓣ}
		\pre{\storyref{forest-elf:nature-warrior}{대자연의 전사}}
		
		\textbf{트리거}: \storyref{forest-elf:nature-warrior}{대자연의 전사}에서 정한 자연환경 또는 [태초의 이야기]에 가지 않는 날이 3일 이상 지속된다.
		
		\textbf{효과}: 다음 효과가 누적으로 적용된다. 효과가 누적됨에 따라 코스트 역시 변한다:
		\begin{center}
			\begin{tabular}{c|l|c}
				\textbf{일수}      & \textbf{효과}                                & \textbf{코스트}\\\hline\hline
				0-2일              & 없음                                         & 0              \\\hline
				3-6일              & [집착] 모든 대성공 결과를 성공으로 취급한다. & 0              \\\hline
				7-9일              & [중독] 모든 성공 결과를 통과로 취급한다.     & -10            \\\hline
				10일 이상          & [광기] 모든 통과 결과를 실패로 취급한다.     & -20            \\
			\end{tabular}
		\end{center}
		
		해당 자연환경에 연관된 사진, 물체 등을 접함으로서 일시적으로 한 씬 동안 효과를 한 단계 완화시킬 수 있다.
	\end{story}
\end{document}