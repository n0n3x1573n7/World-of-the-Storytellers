\documentclass{report}

\begin{document}
	\begin{story}{죽어 있지 못한 자}{[종족: 좀비]}
		이미 한 번 죽음을 경험했기 때문에, 죽음으로 인한 공포를 느끼지 못한다.
		
		받는 모든 정신 피해량이 1 감소(최소 1까지만 감소됨)한다.
		
		"공포"에 관련된 상태이상의 지속 시간이 반으로 감소하고, 횟수 역시 반으로 감소한다.
		
		\smallskip
		
		\cost{20}
	\end{story}
	
	\begin{story}{살아 있지 못한 자}{[종족: 좀비]}
		살아 움직이고 있으며 제정신인 것 처럼 보이나, 실제로는 죽어있는 채로 움직인다.
		
		받는 모든 물리 피해량이 1 감소(최소 1까지만 감소됨)한다.
		
		물리적 부상에 관련된 상태이상의 지속 시간이 반으로 감소하고, 횟수 역시 반으로 감소한다.
		
		\cost{20}
	\end{story}
	
	\begin{story}{충족되지 못하는 굶주림}{[종족: 좀비][광기]}
		\textbf{제약(광기)}: 상태이상 [굶주림] 0/50을 얻는다. 전투 중에는 자신의 턴이 종료될 때 굶주림이 5 증가하고, 타인에게 피해를 1 줄때마다, 또는 피해를 1 받을때마다 굶주림 1을 잃는다. 굶주림은 50 이상으로 올라가지 않는다. 굶주림의 수치에 따라 다음 효과가 발생한다:
		\begin{itemize}
			\item 굶주림이 25 이상이면, 매 전투에서 턴이 시작할 때, 의지 또는 근력 판정을 한다. 만약 0 이상의 결과가 나오지 않는다면, 행동을 할 수 없다.
			\item 굶주림이 50이면 모든 행동이 불가능하다.
		\end{itemize}
		굶주림은 전투가 종료된 후 초기화된다.
		
		\cost{-20}
	\end{story}
	
	\begin{story}{죽음의 손길}{[종족: 좀비][광기]}
		맨손으로 종족이 좀비가 아닌 자를 공격할 때, 피해량이 1 증가한다.
		
		\textbf{트리거(광기)}: 맨손으로 좀비가 아닌 이를 공격했을 때, 해당 대상이 사망하거나 추방된다.
		
		\textbf{효과}: 대상의 종족이 좀비로 변한다. 대상의 체력, 정신력, 굶주림이 최대치가 된다.
		
		\cost{-17}
	\end{story}
\end{document}