\documentclass{report}

\begin{document}
	\begin{story}{더 이상 살아있지 못한 자}{[종족: 귀신][광기]}
		\pre{종족: 귀신}
		
		\limitedtrauma{광기}{\storyref{ghost:possess}{빙의} 또는 \storyref{ghost:poltergeist}{폴터가이스트}가 발동되고 있지 않은 동안, 물리적인 효과를 주거나 받을 수 없다.}
		
		\limitedtrauma{광기}{\storyref{ghost:ghost-body}{영체}의 실체화, \storyref{ghost:possess}{빙의} 또는 \storyref{ghost:poltergeist}{폴터가이스트}가 발동되고 있지 않은 동안, 타인에게 시각적으로 인식될 수 없다.}
		
		\cost{-50}
	\end{story}
	
	\begin{story}[ghost:ghost-body]{영체}{[종족: 귀신]}
		\pre{종족: 귀신}
		
		\entry{[신성] 속성이 없는 모든 물리적 효과에 면역이다. 체력과 정신력 수치를 더해 [영체] 수치로 취급한다. 영체 수치는 체력/정신력과 마찬가지로 매 씬마다 회복되며, 초과하는 피해를 받으면 마찬가지로 개연성에 피해를 받는다.}
		
		\entry{개연성 수치가 0이 되면 이야기에서 추방되는 대신, 소멸되어 영원히 사라진다. 대신, [태초의 이야기]에 돌아가면 개연성 수치를 모두 회복할 수 있다.}
		
		\entry{영체의 [실체화]를 발동하면 타인에게 생전의 모습을 가진 반투명한 영체의 형태로 인식될 수 있다.}
		
		\cost{20}
	\end{story}
	
	\begin{story}[ghost:possess]{빙의}{[종족: 귀신]}
		\pre{종족: 귀신}
		
		\entry{지성체에 빙의할 수 있다. 귀신은 공감으로 공격하고, 대상은 도발 또는 의지로 수비한다. 공격이 성공하면, 대상에 빙의하여 대상을 조종하고, 대상의 이야기 등을 모두 사용할 수 있다.}
		
		\entry{받은 물리적 피해는 대상에게 피해를 주나, [신성] 속성이 있다면 귀신의 영체에도 동일한 양의 피해를 준다. 받은 정신적 피해는 대상에게는 피해를 주지 않고, 귀신에게만 영체 피해를 준다.}
		
		\entry{빙의된 대상은 귀신이 자신 또는 귀신 본인의 이야기를 사용할 때 마다 저항할 수 있다. 빙의시와 같은 판정을 한다. 귀신이 승리한다면 차이만큼의 정신적 피해를 대상에게 주며, 대상이 승리한다면 귀신은 차이만큼의 영체 피해를 받고 빙의가 해제된다.}
		
		\entry{대상의 정신력이 0이 되면 귀신은 대상의 신체를 빼앗아, 대상의 종족이 [귀신]이 되게 만든다. 귀신은 대상의 성격적인 이야기를 제외한 모든 이야기를 빼앗고 종족이 대상의 종족으로 변한다.}
		
		\entry{빙의는 귀신이 소멸하면 자동으로 해제되며, 귀신이 자의적으로 해제할 수도 있다.}
		
		\cost{20}
	\end{story}
	
	\begin{story}[ghost:poltergeist]{폴터가이스트}{[종족: 귀신]}
		\pre{종족: 귀신}
		
		\entry{비지성체를 조종할 수 있다. 이를 이용해 직접적인 피해를 가할 수 없다.}
		
		\entry{모든 피해는 [신성] 속성이 아니라면 폴터가이스트에 걸린 비지성체를 손상시킬 수 없다. [신성] 속성인 경우 비지성체에 손상을 줄 수 있음과 동시에 귀신에게 영체 피해를 준다.}
		
		\cost{15}
	\end{story}
	
	\subsection{귀신 심화 이야기}
	
	\begin{story}{두 개의 영혼}{[종족: 귀신][종족: (대상)][심화]}
		\pre{\storyref{ghost:possess}{빙의}, \storyref{ghost:ghost-body}{영체}}
		
		\entry{\storyref{ghost:possess}{빙의}의 대상이 호의적이고, 귀신과 합의가 되었다면 대상의 신체에 귀신이 공존할 수 있다. 이렇게 된 경우, 귀신과 대상의 개연성 수치가 합쳐져 하나로 취급된다.}
		
		\entry{\storyref{ghost:ghost-body}{영체} 수치는 유지되며, 용도는 다음과 같다:
		\begin{itemize}
			\item {}[신성] 속성 공격을 받으면 대상의 체력/정신력과 함께 [영체] 수치가 감소된다.
			\item 대상의 체력/정신력이 감소될 때, 귀신의 [영체] 수치로 피해를 흡수할 수 있다.
		\end{itemize}
		귀신의 개연성 수치가 0이 되었을 때 귀신이 소멸되어 사라짐은 동일하다.}
		
		\entry{귀신은 더 이상 대상과 별개로 타게팅될 수 없고, 대상을 타게팅하면 귀신은 함께 타게팅 된 것으로 취급하며, 대상이나 귀신이 받은 상태는 둘 모두에게 공유된다.}
		
		\cost{20}
	\end{story}
\end{document}