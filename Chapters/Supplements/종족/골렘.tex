\documentclass{report}

\begin{document}
	\begin{story}{움직이는 돌더미}{[종족: 골렘]\hyperlink{daedu}{Ⓓ}}
		\pre{종족: 골렘}
		
		\flavour{돌더미가 마법의 힘으로 엉겨붙으며 생명체의 형태를 띄며 생겨난 종족이다.}
		
		\entry{전투 씬에서, 독 혹은 불에 의한 지속 피해량을 2 감소시킨다. 이로 인해 피해량이 0 보다 작아질 수는 없다.}
		
		\cost{15}
	\end{story}
	
	\begin{story}{바위손}{[종족: 골렘][공포]\hyperlink{daedu}{Ⓓ}}
		\pre{종족: 골렘, 여러 개의 돌덩이들이 뭉쳐져 만들어져 있어야 함}
		
		\triggertrauma{공포}{도구를 손에 들어 소지하려 한다.}{4df를 굴린다. 0 이하의 값이 나오면, 다음 턴이 되기 전까지 해당 도구는 사용할 수 없다. -3 이하의 값이 나오면, 해당 도구는 파괴된다.}
		
		\cost{-10}
	\end{story}
	
	\begin{story}{크기가 곧 힘}{[종족: 골렘]\hyperlink{daedu}{Ⓓ}}
		\flavour{골렘의 크기는 골렘의 체력과 비례한다. 돌이 떨어져 나갈수록 체력과 힘은 줄어들지만 민첩해진다.}
	\end{story}
	
	\begin{story}{자연주의}{[종족: 골렘][집착]\hyperlink{daedu}{Ⓓ}}
		\flavour{몸에 이끼가 잔뜩 껴있는 착한 골렘을 생각해보세요.}
		
		\triggertrauma{집착}{???}{???}
		
		\entry{???}
		
		\cost{???}
	\end{story}
\end{document}