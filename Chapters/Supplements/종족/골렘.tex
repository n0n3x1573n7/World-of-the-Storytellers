\documentclass{report}

\begin{document}
	\begin{story}{움직이는 돌더미}{[종족: 골렘]\hyperlink{daedu}{Ⓓ}}
		\pre{종족: 골렘}
		
		\flavour{돌더미가 마법의 힘으로 엉겨붙으며 생명체의 형태를 띄며 생겨난 종족이다.}
		
		\entry{전투 씬에서, 독 혹은 불에 의한 지속 피해량을 2 감소시킨다. 이로 인해 피해량이 0 보다 작아질 수는 없다.}
		
		\cost{15}
	\end{story}
	
	\begin{story}{바위손}{[종족: 골렘][공포]\hyperlink{daedu}{Ⓓ}}
		\pre{종족: 골렘, 여러 개의 돌덩이들이 뭉쳐져 만들어져 있어야 함}
		
		\triggertrauma{공포}{도구를 손에 들어 소지하려 한다.}{4df를 굴린다. 0 이하의 값이 나오면, 다음 턴이 되기 전까지 해당 도구는 사용할 수 없다. -3 이하의 값이 나오면, 해당 도구는 파괴된다.}
		
		\cost{-10}
	\end{story}
	
	\begin{story}{크기가 곧 힘}{[종족: 골렘]\hyperlink{daedu}{Ⓓ}}
		\pre{종족:골렘}
	
		\flavour{골렘의 크기는 골렘의 체력과 비례한다.}
		
		\entry{골렘의 최대 체력에 대한 현재 체력의 비율에 따라 골렘의 스탯이 다음과 같이 변화한다:
		
		\begin{tightcenter}
			\begin{tabular}{c!{\color{black}\vrule}c}
				\textbf{잔여 체력}      & \textbf{효과}\\\hline\hline
				100\% 이하\textasciitilde50\% 초과              & 변화 없음\\\hline
				50\% 이하\textasciitilde0\% 초과              & \makecell{\statchange{+}{기민}\\\statchange{-}{근력}}\\\hline
				0\%\footnote{체력이 모두 소모되고, 개연성에 피해를 받고 있을때를 의미합니다.}              & \makecell{\statchange{+}{기민, 은신}\\\statchange{-}{근력, 전투}}\\
			\end{tabular}
		\end{tightcenter}}
		
		\cost{10}%코스트 전부 어차피 똑같아서 하나로 합쳤고 기본 코스트 10 추가했음
	\end{story}
	
	\begin{story}{자연주의}{[종족: 골렘][기피]\hyperlink{daedu}{Ⓓ}}
		\flavour{몸에 이끼가 잔뜩 껴있는 착한 골렘을 생각해보세요.}
		
		\limitedtrauma{기피}{자연물을 직접 파괴하거나, 자연물로 이루어진 개체를 먼저 공격할 수 없다.}
		
		\cost{0}
	\end{story}
\end{document}
