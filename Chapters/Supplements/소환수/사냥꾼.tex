\documentclass{report}

\begin{document}
	\world{`명치절단기'}
	주변환경을 잘 활용하여 자신만의 사냥감을 순식간에 잡아내는 클래스입니다.
	
	
	\begin{story}{사냥감 추적}{[클래스:사냥꾼]}
		\flavour{사냥꾼은 자연을 이용하는 데에 특화되어 있다.}
		
		\entry{숲이나 산, 동굴 등의 자연 엄폐물이 많은 곳에서 헌터는 은신 보정 +1을 받고, 장애물으로 인한 기민 페널티를 받지 않는다.}
		
		\entry{자신이 찾는 대상이 해당 지역의 자연물과 거리가 있다면 인식 또는 관련된 특화 지식 판정으로 해당 대상의 향방을 알 수 있다.}
		
		\cost{20}
	\end{story}
	
	\begin{story}[hunter:one-shot]{한 발이면 충분하지}{[클래스:사냥꾼]}
		\entry{사냥꾼은 미래지향적 무기, 무거운 두손무기를 제외한 모든 무기에 특화되어 있고, 둘 이상의 무기를 전투에 사용할 수 있다. 무기를 전환하는 데 한 턴을 소모해야 한다.}
		
		\entry{전투 중, 사냥꾼이 적보다 먼저 자리잡았다면, 전투 또는 사격 판정을 성공했을 때 1d6을 굴린다. 1d6에서 6의 눈이 나왔다면 [치명타] 판정이 되어, 해당 공격의 피해량이 두 배가 된다.}
		
		\entry{전투 중, 사냥꾼이 적보다 나중에 자리잡았지만 자연 엄폐물이 많은 지역에서 사냥꾼은 회피 판정에 +1을 받는다.}
		
		\cost{20}
	\end{story}
	
	\subsubsection*{사냥꾼 심화 이야기}
		\begin{story}{숲의 전령}{[클래스:사냥꾼][심화]}
			\pre{동물 동료의 종을 선택한다. 동물 동료는 중형 육식 포유류\footnote{개/늑대 등을 포함합니다.}에서 대형 육식 포유류\footnote{곰 크기정도까지라고 생각하시면 됩니다.}, 또는 중령 이상의 조류\footnote{까마귀 정도의 크기 이상이라고 생각하시면 됩니다.}}
			
			\entry{동물 동료를 \storyref{summon}{소환수}로서 사용한다. \storyref{summon}{소환수} 생성 방법에 따라 소환수를 만든다. 동물 동료는 디스폰이 불가능하며, 매우 단편적 의사 소통이 가능하고, 전투에 적극 참여가 가능하다.}
		\end{story}
		
		\begin{story}{돌진 사냥꾼}{[클래스:사냥꾼][심화]}
			\pre{\storyref{hunter:one-shot}{한 발이면 충분하지}}
			
			\entry{적보다 먼저 자리잡을 필요 없이, 사격 판정을 성공했을 때 1d6을 굴려 6의 눈이 나오면 [치명타] 판정이 된다.}
			
			\cost{20}
		\end{story}
\end{document}