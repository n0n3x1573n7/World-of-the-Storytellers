\documentclass{report}

\begin{document}
	\begin{story}[dreamer:sleep]{잠}{[클래스:꿈꾸는 자]}
		\flavour{잠을 자는 것을 너무나도 좋아한다.}
		
		\entry{언제 어디서든, 한 턴을 소모하여 잠을 잘 수 있다. 잠을 자는 동안, 모든 상태이상의 작용이 멈추며, 정신력에 피해를 받을 수 없다. 잠에서 깨어나면 상태이상의 작용이 재개된다.}
		
		\cost{20}
	\end{story}
	
	\begin{story}[dreamer:dream]{꿈}{[클래스: 꿈꾸는 자]}
		\flavour{꿈 속에서는 무엇이든 할 수 있었다.}
		
		\entry{잠을 자는 동안, 한 턴을 소모해 꿈을 꾸기로 선택할 수 있다. 꿈을 꾸면, 다음 \storyref{summon}{소환수} 중 하나를 소환할 수 있다:
		\begin{center}
			\begin{tabular}{c|c}
				소환 대상 & 코스트 \\\hline\hline
				\hyperlink{dreamer:sheep}{양 한마리} & 10 \\\hline
				\hyperlink{dreamer:summon2}{소환수 2} &  \\
			\end{tabular}
		\end{center}
		소환수의 기본 스탯은 꿈꾸는 자의 스탯과 동일하나, 소환수에 따라 추가 스탯을 받게 될 수 있다. 단, 개연성은 0으로 취급하며 체력이 0에 다다르면 강제로 디스폰된다.}
		
		\entry{소환수의 시야는 꿈꾸는 자와 공유된다.}
		
		\entry{꿈을 꾸는 동안이라도 소환자는 피해를 받을 수 있으나, 피해를 받아 소환자가 추방되더라도 소환수는 사라지지 않는다.}
		
		\cost{10+X\footnote{소환수에 따라 코스트가 달라집니다.}}
	\end{story}
	
	\begin{story}[dreamer:nightmare]{악몽}{[클래스: 꿈꾸는 자][공포]}
		\pre{\storyref{dreamer:sleep}{잠}, \storyref{dreamer:dream}{꿈}}
		
		\triggertrauma{공포}{\storyref{dreamer:dream}{꿈}에 의한 소환수가 타의에 의해 디스폰된다.}{즉시 \storyref{dreamer:sleep}{잠}에서 깨어난다. 이번 씬 동안, \storyref{dreamer:sleep}{잠}들기 위해 필요한 턴이 한 턴 증가하고, \storyref{dreamer:dream}{꿈}꾸기 위해 필요한 턴도 한 턴 증가한다. 이가 여러번 일어나면 이 턴 수는 누적된다.}
		
		\cost{-10}
	\end{story}
	
	\hypertarget{dreamer:sheep}{}
	\subsection{양 한마리}
	꿈꾸는 자가 있는 구역에 소환된다. 꿈꾸는 자가 있는 구역에 다른 누군가가 있다면 해당하는 이들은 강제로 즉시 인접한 구역으로 이동해야 한다.
	
	\begin{story}{양 두마리, 양 세마리}{[꿈]}
		\entry{한 턴을 소모하여, 양이 있는 구역과 인접한 빈 구역 중 하나에 추가로 양을 소환한다. 또는, 한 턴을 소모하여 소환되어 있는 양 전부 또는 일부를 디스폰한다.}
		
		\entry{양이 모두 디스폰되면, \storyref{dreamer:dream}{꿈}에서 깨어나지만, \storyref{dreamer:sleep}{잠}에서는 깨어나지 않는다.}
		
		\entry{양은 꿈꾸는 자와 구역을 공유할 수 있으나, 다른 이들과는 구역을 공유할 수 없다. 꿈꾸는 자와 구역을 공유한 동안에는, 꿈꾸는 자를 향한 모든 공격은 그 구역에 있는 양을 향한다.}
		
		\entry{양은 체력을 모두 공유하며, 공격 또는 방어할 수 없다. 대신, 타인의 공격으로 인해 소멸되더라도 \storyref{dreamer:nightmare}{악몽}의 트리거를 발동시키지 않는다.}	
		
		\cost{20}
	\end{story}
	
	\hypertarget{dreamer:summon2}{}
	\subsection{소환수 2}
\end{document}