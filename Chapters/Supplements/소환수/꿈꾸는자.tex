\documentclass{report}

\begin{document}
	\begin{story}[dreamer:sleep]{잠}{[클래스:꿈꾸는 자]}
		\flavour{잠을 자는 것을 너무나도 좋아한다.}
		
		\entry{언제 어디서든, 한 턴을 소모하여 잠을 잘 수 있다. 잠을 자는 동안, 모든 상태이상의 작용이 멈추며, 정신력에 피해를 받을 수 없다. 잠에서 깨어나면 상태이상의 작용이 재개된다.}
		
		\cost{20}
	\end{story}
	
	\begin{story}[dreamer:dream]{꿈}{[클래스: 꿈꾸는 자]}
		\flavour{꿈 속에서는 무엇이든 할 수 있었다.}
		
		\entry{잠을 자는 동안, 한 턴을 소모해 꿈을 꾸기로 선택할 수 있다. 꿈을 꾸면, 다음 \storyref{summon}{소환수} 중 하나를 소환할 수 있다:
		\begin{center}
			\begin{tabular}{c|c}
				소환 대상 & 코스트 \\\hline\hline
				\hyperlink{dreamer:sheep}{양 한마리} & 10 \\\hline
				\hyperlink{dreamer:unicorn}{유니콘} & 15 \\
			\end{tabular}
		\end{center}
		소환수의 기본 스탯은 꿈꾸는 자의 스탯과 동일하나, 소환수에 따라 추가 스탯을 받게 될 수 있다. 단, 개연성은 0으로 취급하며 체력이 0에 다다르면 강제로 디스폰된다.}
		
		\entry{소환수의 시야는 꿈꾸는 자와 공유된다.}
		
		\entry{꿈을 꾸는 동안이라도 소환자는 피해를 받을 수 있으나, 피해를 받아 소환자가 추방되더라도 소환수는 사라지지 않는다.}
		
		\cost{10+X\footnote{소환수에 따라 코스트가 달라집니다.}}
	\end{story}
	
	\begin{story}[dreamer:nightmare]{악몽}{[클래스: 꿈꾸는 자][공포]}
		\pre{\storyref{dreamer:sleep}{잠}, \storyref{dreamer:dream}{꿈}}
		
		\triggertrauma{공포}{\storyref{dreamer:dream}{꿈}에 의한 소환수가 타의에 의해 디스폰된다.}{즉시 \storyref{dreamer:sleep}{잠}에서 깨어난다. 이번 씬 동안, \storyref{dreamer:sleep}{잠}들기 위해 필요한 턴이 한 턴 증가하고, \storyref{dreamer:dream}{꿈}꾸기 위해 필요한 턴도 한 턴 증가한다. 이가 여러번 일어나면 이 턴 수는 누적된다.}
		
		\cost{-15}
	\end{story}
	
	\hypertarget{dreamer:sheep}{}
	\subsubsection*{양 한마리}
		\begin{story}{양 한마리}{[꿈]}
			\entry{소환될 때, 꿈꾸는 자가 있는 구역에 소환된다. 꿈꾸는 자가 있는 구역에 다른 누군가가 있다면 해당하는 이들은 강제로 즉시 인접한 구역으로 이동해야 한다.}
			
			\entry{\begin{story}{양 두마리, 양 세마리}{[꿈:양 한마리]}
				\entry{한 턴을 소모하여, 양이 있는 구역과 인접한 빈 구역 중 하나에 추가로 양을 소환한다. 또는, 한 턴을 소모하여 소환되어 있는 양 전부 또는 일부를 디스폰한다.}
				
				\entry[\hline]{양이 모두 디스폰되면, \storyref{dreamer:dream}{꿈}에서 깨어나지만, \storyref{dreamer:sleep}{잠}에서는 깨어나지 않는다.}
			\end{story}}
			
			\entry{양은 꿈꾸는 자와 구역을 공유할 수 있으나, 다른 이들과는 구역을 공유할 수 없다. 꿈꾸는 자와 구역을 공유한 동안에는, 꿈꾸는 자를 향한 모든 공격은 그 구역에 있는 양을 향한다.}
			
			\entry{양은 체력을 모두 공유하며, 공격 또는 방어할 수 없다.}
			
			\entry{\begin{story}{편안한 꿈}{[꿈:양 한마리]}
				\entry[\hline]{타인의 공격으로 인해 소멸되더라도 \storyref{dreamer:nightmare}{악몽}의 트리거를 발동시키지 않는다.}
			\end{story}}
			
			\cost{20}
		\end{story}
	
	\hypertarget{dreamer:unicorn}{}
	\subsubsection*{유니콘}
	\begin{story}{상상속의 말}{[꿈]}
		\entry{소환될 때, 꿈꾸는 자가 있는 구역에서 꿈꾸는 자를 태운 채로 소환된다.}
		
		\entry{\begin{story}{가속하는 말}{[꿈:유니콘]}
			\entry{유니콘은 소환될 때와 매 턴이 시작될 때, [가속] 상태를 2회 받는다. [가속] 상태는 최대 7회까지 누적되며, 한 턴에 얼마든지 소모할 수 있다.}
			
			\entry[\hline]{유니콘은 [가속] 상태 1회를 소모함으로서 한 구역을 이동할 수 있다.}
		\end{story}}
		
		\entry{\begin{story}[unicorn:flying]{비행하는 말}{[꿈:유니콘]}
			\entry{유니콘은 [가속] 상태 2회를 소모함으로서 공중으로 떠오를 수 있다. 다시 바닥으로 내려오것은 언제든지 가능하다.}
		
			\entry{유니콘은 떠 있지 않는 동안 꿈꾸운 자를 태우거나, 내리게 할 수 있다. 이를 하는 데에 한 턴이 소요된다. 소환한 꿈꾸는 자 본인이 아닌 이는 태울 수 없다.}
			
			\entry[\hline]{공중에 떠오른 동안, [가속] 상태 1회를 소모하여 두 구역까지 이동할 수 있다.}
		\end{story}}
		
		\entry{\begin{story}{말발굽 차기}{[꿈:유니콘]}
			\entry[\hline]{유니콘은 [가속] 상태 1회를 소모함으로서 1회당 같은 구역에 있는 한 대상에게 피해 2를 줄 수 있다. \storyref{unicorn:flying}{비행하는 말}으로 떠오른 동안은 공격을 할 수 없으나, 바닥으로 내려오면서 공격을 한다면 피해량이 3으로 증가한다.}
		\end{story}}
		
		\entry{\begin{story}{회피 기동}{[꿈:유니콘]}
			\entry[\hline]{유니콘은 방어할 수 없다. 대신, [가속] 상태를 소모하여 1회당 피해량 2를 경감시킬 수 있다. \storyref{unicorn:flying}{비행하는 말}으로 떠오른 동안은 피해량 3이 경감되나, 떠오른 동시에 꿈꾸는 자를 태운 동안이라면 피해량 1만을 경감시킬 수 있다.}
		\end{story}}
		
		\cost{30}
	\end{story}
	
	\subsubsection*{꿈꾸는 자 심화 이야기}
		\begin{story}{기면증}{[클래스:꿈꾸는 자][심화][기피]}
			\pre{\storyref{dreamer:sleep}{잠}, \storyref{dreamer:dream}{꿈}}
			
			\triggertrauma{기피}{씬이 시작한다.}{\storyref{dreamer:sleep}{잠}에 빠져든다.}
			
			\cost{0}
		\end{story}
		
		\begin{story}{몽유병}{[클래스:꿈꾸는 자][심화]}
			\pre{\storyref{dreamer:sleep}{잠}, \storyref{dreamer:dream}{꿈}}
			
			\entry{\storyref{dreamer:sleep}{잠}을 자거나, \storyref{dreamer:dream}{꿈}을 꾸면서도 이동할 수 있으나, 두 턴에 한 구역만을 이동할 수 있다.}
			
			\cost{15}
		\end{story}
		
		\begin{story}{깊은 잠}{[클래스:꿈꾸는 자][심화][기피]}
			\pre{\storyref{dreamer:sleep}{잠}, \storyref{dreamer:dream}{꿈}}
			
			\limitedtrauma{기피}{전투 상황이 아니라면, \storyref{dreamer:sleep}{잠}에서 깨어날 수 없고, 한번 잠들면 \storyref{dreamer:dream}{꿈}을 통해서만 다른 이들과 소통할 수 있다.}
			
			\cost{0}
		\end{story}
\end{document}