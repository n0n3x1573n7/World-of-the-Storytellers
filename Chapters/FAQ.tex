\documentclass{report}

\begin{document}
	내용에 대한 질문 사항이 있으시다면, 트위터 \href{https://www.twitter.com/n0n3x1573n7_WS}{@n0n3x1573n7\_WS}의 DM, 혹은 해당 계정에 연결되어있는 \href{https://ask.fm/n0n3x1573n7_WS}{ask.fm}으로 문의 주시면 확인하는대로 답변드리겠습니다. 질문 중 중요한 내용은 본문에 추가하고, 자주 나오는 질문을 취합하여 이곳에 추가할 예정입니다.
	
	\bigskip
	
	\begin{faq}{비공개 또는 유료화 계획이 있나요?}
		아직까지는 없습니다. 단, 추후 비공개 또는 유료화를 하게 된다면 사전 안내를 드릴 예정입니다. 특히 유료화의 경우 하게 되더라도 본 계정에서 질의응답은 계속 받을 예정입니다.
		
		다만, 만약 본 룰로 작성된 시나리오 또는 플레이 로그 중 다음과 같은 사항이 발생한다면 \href{https://twitter.com/n0n3x1573n7_WS}{본 계정}으로 해당 시나리오 또는 플레이 로그의 이름, 계정명 등을 언급하거나 해당 계정을 멘션하지 않고 사유와 함께 퍼블릭 트윗으로 경고를 드릴 예정입니다. 경고가 과도하게 누적된다면 본 룰을 즉시 비공개 처리할 예정입니다.
		\begin{itemize}
			\item 성적인 컨텐츠에 따른 나이 제한 표기 또는 트리거 워닝이 제대로 표기되지 않은 경우. 시나리오의 내용을 읽지 않고서도 알 수 있도록 배포 트윗 내지는 시나리오의 전체 공개 부분 또는 사전 고지 부분에 반드시 표기 되어야 합니다. 해당 트리거 워닝의 내용이 시나리오의 핵심 반전 등 스포일러성이라 할지라도 예외는 없습니다.
			\item 여성, 성소수자, 장애인 등 사회적 약자 또는 소수자에 대한 혐오 또는 차별을 조장하는 경우. 세계관이나 이야기의 특성상 이가 묵인되는 경우라 할지라도 정도를 넘어설 경우 역시 포함됩니다. 특히 이를 조장하는 시나리오를 배포한 경우 \href{https://help.twitter.com/ko/rules-and-policies/hateful-conduct-policy}{트위터의 운영원칙} 등 배포된 플랫폼(들)의 운영원칙에 의해 신고할 예정입니다.
			\item 위 항목들에 해당하지는 않으나 위 항목들에 준하는 경우. 특히, 위 항목에 적혀있지 않다고 해서 허점을 찾았다고 생각하지는 않으시기를 부탁드립니다.
		\end{itemize}
		해당 사안들을 발견하셨을때 본 계정에 해당 내용에 대한 DM을 주신다면 감사하겠습니다.
		
		만약에 이 룰이 유료화 된다면 그 이후, 본 룰을 이용하여 게임을 즐기고 싶으시다면 룰북을 어떤 경로로든(온라인 pdf, 오프라인 책자 등) 구매하셔서 플레이를 즐겨주시기를 부탁드립니다.
	\end{faq}
	
	\begin{faq}{이야기꾼은 반드시 [깨달은 자]인가요?}
		이야기꾼이 반드시 [깨달은 자]일 필요는 없습니다. 예를 들어, 이야기 vs. 등장인물의 흐름에서 이야기꾼은 [깨달은 자]가 아닌 등장인물로서 이야기 속의 역경을 헤쳐나갑니다. 그 외에도 외부의 [깨달은 자]들이 이야기꾼의 세계에 들어와서 일어나는 일들을 이야기할수도 있습니다.
	\end{faq}

\end{document}