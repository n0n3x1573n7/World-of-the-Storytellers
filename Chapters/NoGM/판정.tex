\documentclass{report}

\begin{document}
	[판정] 흐름은 이야기꾼이 어떤 일을 극복하기 위해, 또는 어떤 집단의 이야기꾼이 서로 대결하기 위해 필요한 흐름입니다. 이 챕터에서는 판정을 하는 방법에 대해 설명하겠습니다.
	
	먼저, [판정] 흐름에는 난이도와 판정의 내용, 그리고 판정의 대상이 주어집니다. 난이도는 1, 2, 3, 4, 5 중 하나이며, 별다른 언급이 없다면 기본 난이도는 3입니다. 판정의 난이도가 성공과 실패에 서로 다르게 주어질수도 있습니다. 판정의 내용은 이 판정을 하게 되는 이유가 서술되어 있습니다. 판정의 대상은 이 판정에 참여할 수 있는 이야기꾼들입니다. 이야기꾼 중 일부만 참여할 수도 있고, 전부 참여할 수도 있습니다.
	
	다음으로, 판정에 참여하는 이야기꾼들은 판정에 대해 서술에 의해 도움이나 방해를 받습니다. 이야기꾼이 아닌 관련있는 소재의 서술 역시 빌릴 수 있지만, 한 서술은 판정에 도움을 주거나 방해를 하는 두 가지 중 한 가지만 할 수 있습니다. 판정에 이야기꾼의 일부만 참여하는 경우, 참여하지 않는 이야기꾼들은 구경만 해도 되지만, 소재들의 서술으로 도움을 주거나 방해를 할 수 있습니다.
	
	선택적으로, 한 이야기꾼 당 한 번, 이 이야기의 관중이 되어 이 판정이 성공했을 때 더 재밌을지, 실패했을 때 더 재밌을지를 결정할 수 있습니다. 굳이 선택하지 않아도 좋으나, ``성공했으면 좋겠다"고 한다면 한 번 도와준 것으로, ``실패했으면 좋겠다"고 한다면 한 번 방해한 것으로 취급하여 이야기의 진행을 흥미롭게 만들 수 있습니다.
	
	이제 도움을 받은 만큼 도움 주사위를, 방해를 받은 만큼 방해 주사위를 굴립니다. 이를 위해서 위의 단계에서 도와준 서술에 도움 주사위를, 방해한 서술에 방해 주사위를 올려두는 것으로 표시를 해두어도 좋습니다. 주사위를 굴릴때, 난이도보다 큰 수가 나오면 해당 주사위는 성공 또는 실패에 기여한 것입니다. 즉, 1은 항상 실패하고, 6은 항상 성공합니다. 또한 주사위를 굴린 결과 숫자 6이 나오면 난이도 이하의 눈이 나온 주사위 중 하나를 다시 굴릴 수 있습니다! 다시 굴린 주사위에서도 6이 나온다면 난이도 이하의 눈이 나온 주사위가 남아있는 한 계속해서 다시 굴릴 수 있습니다. 여기에서 도움 주사위 중 (성공) 난이도 이상의 눈이 나온 수를 [성공 수치], 방해 주사위 중 (실패) 난이도 이상의 눈이 나온 수를 [실패 수치]라고 하겠습니다.
	
	이제 마지막으로 [성공 수치]와 [실패 수치]의 값에 따라 결과가 달라집니다. 기본적으로는 [성공 수치]가 높으면 성공, [실패 수치]가 높으면 실패할 것입니다. 두 수치가 같다면 판정의 대상이 되는 이야기꾼이 판정의 결과를 결정합니다.
\end{document}