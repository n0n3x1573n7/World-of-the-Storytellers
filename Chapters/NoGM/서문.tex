\documentclass{report}

\begin{document}
	이야기의 생성자들은 RPG\footnote{Role Playing Game. TRPG/TTRPG(TableTop RPG), ORPG(Online RPG)로도 알려져 있다.}입니다. 여러명이 모여서 캐릭터를 만들고, 그들이 세계와 서로 상호작용을 하며 이야기를 만들어나가는 게임이죠. 플레이어들이 조종하는 PC\footnote{Player Character}라고 불리는 캐릭터들은 \emph{이야기꾼} \emph{[Storytellers]}이라고 부릅니다.
	
	비교적 완성된 형태의 서사에 들어가는 이야기꾼의 세계나 이야기의 방랑자들과는 다르게 이야기의 생성자들에서는 \emph{완성되지 않은 상태의} 이야기 속으로 들어가게 됩니다. 서사 속에서 이야기꾼들은 이야기 속의 등장인물이 되어 서사를 체험하게 되며, 이 과정에서 이야기꾼이 아닌 이의 간섭은 사실상 없다시피 합니다. 있다 하더라도 이야기꾼 본인들이 이야기를 만들어나가는 과정 중에서 등장하게 되죠. 즉, 이야기꾼의 세계나 이야기의 방랑자들과는 다르게 시스템이나 길잡이꾼 등 이야기를 이끌어나가는 이는 존재하지 않습니다. 굳이 이끌어나가는 이를 찾자면, 서사 그 자체가 이야기꾼들을 이끌어나가는 것이라고 할 수 있습니다.
	
	이야기의 생성자들은 이야기꾼의 세계가 배경으로 하는 태초의 이야기에 기반하여, 이야기꾼들이 자신의 이야기를 직접 개척해 나가는 것을 목표로 만들어졌습니다.
	
	다음 이야기꾼의 세계 챕터들의 내용은 이야기의 생성자들과 많은 부분을 공유하므로, 이 챕터들을 먼저 읽으시는 것을 추천드립니다:
	\begin{itemize}
		\item \hyperlink{story-progression}{서사의 진행}
		\item \hyperlink{power-limit}{권능과 제약}
	\end{itemize}
	
	또한 다음 이야기의 방랑자들 챕터들은 이야기의 생성자들과 많은 부분을 공유하므로, 이 챕터들 역시 먼저 읽으시는 것을 추천드립니다. 특히, 이야기꾼을 만드는 방법이 완벽하게 동일하기 때문에 \hyperlink{lite-character-creation}{이야기꾼 만들기} 챕터는 반드시 읽으시는 것을 추천드립니다.
	\begin{itemize}
		\item \hyperlink{lite-character-creation}{이야기꾼 만들기}
		\item \hyperlink{lite-sheets}{캐릭터 시트}
		\item \hyperlink{lite-roles}{이야기 속에서의 역할과 이야기}
	\end{itemize}
	
	이야기의 생성자들을 하기 위해서는 서로 구분되는 두 종류의 육면체 주사위가 많이 필요합니다. 많으면 많을수록 좋습니다! 이들은 판정을 할 때 사용되며, 후에 각각 도움 주사위와 방해 주사위로 부르겠습니다.
\end{document}