\documentclass{report}

\begin{document}
	``메타"는 이야기꾼이 이 이야기 밖에서 온, 이곳이 이야기일 뿐이라는 것을 알고 있는 이야기꾼이기 때문에 존재하는 특별한 능력입니다.
	
	각 이야기꾼별로, 장이 시작할 때 메타를 1회 받습니다. 메타는 언제든지 발현할 수 있으며, 발현한다면 다음과 같은 능력을 사용할 수 있습니다:
	\begin{itemize}
		\item 자신에게 온 흐름을 타인에게 넘깁니다.
		\item (맥락에 맞는) 새로운 흐름을 만듭니다.
		\item 타인의 흐름에 이야기꾼이 간섭해 흐름을 뒤틉니다.
		\item 판정을 조작하여 주사위 전부 또는 일부 재굴림을 할 수 있습니다.
		\item 이 이외에도 다른 이야기꾼들의 동의를 받는다면 무엇이든지 가능합니다!
	\end{itemize}
	즉, 이야기꾼들이 이야기에 직접 간섭하는 \emph{메타적인} 능력을 행할 수 있는 것입니다! 메타는 어떤 곳에나 사용할 수 있지만, 예외적으로 가지에 해당하는 부분, 또는 다른 이야기꾼의 메타에는 메타를 사용할 수 없습니다\footnote{물론, 다른 이야기꾼들의 동의가 있다면 가능합니다.}.
	
	메타는 사용하지 않는다면 다음 장으로 진행할 때 누적됩니다. 즉, 다음 장에서 한번 더 사용할 수 있는 것이죠.
	
	만약 이야기를 만들어나감에 있어 \ifprintout X 카드\else\href{http://tinyurl.com/x-card-rpg}{X 카드}\fi\footnote{John Stavropoulos가 만든 도구로, 이 링크에서 확인할 수 있습니다: \url{http://tinyurl.com/x-card-rpg} \\ 꼭 X 카드가 아니라 할지라도 자신과 타 플레이어의 안전을 위해서 이러한 도구를 사용하는 것을 강력하게 권장드립니다.}와 같은 플레이어의 안전을 위한 수단을 사용한다면, 해당 수단을 사용하는 것은 그 \emph{어떤 경우에도 메타의 발현으로 취급하지 않습니다}. 이는 모든 플레이어들의 안전을 위해 존재하는 수단으로, 룰적인 부분보다도 반드시 우선시되어야 한다는 것을 꼭 기억하세요.
\end{document}