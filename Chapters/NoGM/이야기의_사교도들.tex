\documentclass{report}

\begin{document}
	앞선 모든 설명에서는 시나리오로서 서사의 흐름을 누군가 미리 정리해두어야만 했습니다. 하지만 이 서사의 흐름을 미리 정해두지 않고도 진행할 수 있는 방법 역시 있습니다. 이 챕터에서는 이 방법에 대해 설명하도록 하겠습니다.
	
	이 선택 규칙을 사용한다면 주어진 시나리오가 아닌 어떤 설정이 주어졌을 때 새로운 이야기를 만드는 것을 목표로 합니다. 이야기꾼의 세계나 이야기의 방랑자들에서, 이런 이야기를 활용할 수 있는 것은 이 이야기를 만드는 데에 기여한 ``작가"들 뿐입니다. 이 목적을 알 수는 없기에 [이야기의 사교도들]이라는 이름이 붙은 것이죠.
	
	먼저 이야기꾼들은 하나의 배경 설정을 함께 정합니다. 시공간적 배경과 함께 이야기꾼당 한개 정도의 소재를 함께 정해둡니다. 각자 하나씩 정하는 것도 좋습니다. 또한, 장수 제한을 정합니다. 장수 제한은 최대 6장 정도로 하는 것을 추천드리며, 이는 플롯을 여섯 단계로 나누는 것을 기준으로 하는 것을 추천드립니다. 물론 여기에 몇 단계를 더 추가하거나 생략하는 것도 괜찮습니다. 플롯의 여섯 단계는 다음과 같습니다:
	
	\medskip
	
	\noindent\fbox{%
		\parbox{\linewidth}{%
			[발단(Exposition)] $\rightarrow$ [전개(Complication)]
			\begin{tightcenter}  $\searrow$ [위기(Crisis)] $\rightarrow$ [절정(Climax)] $\searrow$ \end{tightcenter}
			\hspace*{0pt}\hfill [하강(Falling Action)] $\rightarrow$ [결말(Resolution)]
		}%
	}
	
	\medskip
	
	이 이후, 각 이야기꾼은 이 배경과 소재, 다른 이야기꾼에 대한 흐름을 일정 개수(이야기꾼당 3\textasciitilde 5개 정도, 또는 합쳐서 15\textasciitilde20개 정도가 적당합니다.) 작성합니다. 첫 장에는 모두 질문을 적는 것을 추천합니다. 각 장이 시작하기 전에 해당 장에 해당하는 단계를 언급하고 그를 고려하여 흐름을 작성하는 것 역시 추천합니다. 이 이후, 이 흐름들을 모두 섞어 일반적인 진행 방법대로 진행합니다. 중요한 것은, 흐름이 현재 정해진 이야기들에 비추어 봤을 때 부적절한 경우 메타의 발현 없이 다수의 동의로 흐름을 삭제할 수 있다는 점입니다.
	
	두 번째 장 부터는 흐름을 뽑는 대신 판정을 하기 위한 적절한 사건을 발생시켜 진행하기로 결정할 수 있습니다.
	
	모든 흐름이 종료된 후,  또는 ``더 공개되었을 때 흥미로울 만한 흐름이 없다"고 모두가 동의할 때 [가지]가 나온 것으로 취급합니다. 가지는 이번 장을 진행하며 흐름에서 나온 이야기(특히 분기) 중 하나로 다수의 동의하에 결정하며, 이가 이 이후의 주된 서사의 흐름이 됩니다.
	
	가지가 종료되면 현재 진행된 장수가 장수 제한을 넘기지 않았다면, 다시 흐름을 적는 단계로 돌아가 다음 장을 진행합니다. 이 단계에서 원한다면, 모두의 동의를 얻어 서사를 종료할 수 있습니다! 반대로, 장수 제한을 넘겼다면 이야기를 종료하는 것이 원칙이지만, 모두가 동의한다면 이야기를 더 연장할 수 있습니다.
	
	\medskip
	
	[메타] 규칙을 적용중이라면, 자신이 적은 흐름의 대상이 자신이 될 수도 있습니다. 흐름의 대상을 크게 제한하지 않은 상태에서 다른 이야기꾼이 흐름을 적은 이를 지명했다면, 이 경우를 [카르마]라고 칭합니다. 카르마 상황에서, 해당 흐름에 본인이 [메타]를 적용하지 않았다면 답변이 끝난 후 [메타]를 1회 받습니다.
\end{document}