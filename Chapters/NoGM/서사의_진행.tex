\documentclass{report}

\begin{document}
	서사에 들어간 이야기꾼들은 서사에 맞는 역할을 먼저 부여받고, 그 다음에 흐름에 따라 질문에 대답하고 서사적으로 중요한 선택을 하고, 어떨 때는 판정을 통해 행동을 강요받기도 하며 서사 속의 역할에 녹아들게 됩니다.
	
	\section{이야기꾼 만들기}
		이야기꾼을 만드는 방법은 이야기의 방랑자들에서의 \hyperlink{lite-character-creation}{이야기꾼 만들기}와 거의 동일합니다. 하지만, 한 가지 큰 차이점이 있습니다. 바로 이야기꾼을 만들 때, 개연성의 최대 제한을 계산하는 대신, [서사 포화도]라고 불리는 \emph{소모된 개연성}을 계산한다는 점입니다. 이를 위해 서술은 그 내용의 구분 없이 모든 서술을 긍정적으로 취급합니다.
	
	\section{서사 속의 역할 부여}
		이야기의 방랑자들에서의 \hyperlink{lite-roles}{이야기 속에서의 역할과 이야기} 부분을 참고해주세요.
		
		이야기의 방랑자들과 다른 점은, 많은 경우 \emph{모든 이야기꾼}이 서사 속으로 진입할 때 서사에 맞는 역할을 받게 되며, 역할은 반드시 그 역할에 맞는 이야기들을 지급한다는 점입니다. 이 이야기들은 [서사 포화도]에 영향을 끼칩니다. 이야기꾼의 세계나 이야기의 방랑자들에서의 역할이 개연성에 영향을 끼치지 못하는 것과는 다르다는 점을 기억하세요.
	
	\section{이야기꾼 자기소개 및 소재 공개}
		이야기꾼들은 각자 자기 소개를 합니다. 정해진 형식은 없지만, ``얼마나 \textasciitilde 할지는 두고봐야겠지만" 등으로 끝내는 것을 추천드립니다! 이는 이야기꾼의 이야기나 역할 등에서 눈여겨보아야 할 포인트를 직접 짚어준다는 점에서 후의 이야기의 진행 방향을 정하는 데에 도움을 줄 수 있습니다.
		
		또한, 이야기 속에는 미리 공개된 여러 소재가 존재합니다. 이들 역시 그들만의 이야기를 가지고 있습니다. 이들에 관한 내용이 알려집니다.
		
	\section{서사의 흐름}
		서사는 여러 장(Chapter)으로 구성되어 있습니다. 각 장은 특정한 수의 [흐름]으로 구성되어 있습니다. [흐름]은 크게 다음으로 구분할 수 있습니다:
		
		\subsection*{질문}
			이야기꾼의 느낌이나 과거 등에 대한 질문입니다. 이 질문은 서사를 풍부하게 하는데에 사용합니다. 가장 처음 서사에 들어왔을때에는, 이야기꾼과 이야기꾼 또는 소재 간의 관계를 정립하기 위하여 질문을 활용합니다. 이는 서사를 풍부하게 하기 위한 장치로 사용됩니다.
		
		\subsection*{분기}
			분기는 이야기꾼들의 선택에 따라 흐름을 좌우할 수 있는 질문입니다. 이 선택에 따라 이후 일부 흐름이 무시될 수 있습니다. 이는 서사의 흐름을 통제하기 위한 장치로 사용됩니다.
		
		\subsection*{판정}
			판정은 이야기꾼들이 가진 이야기로 헤쳐나가야 할 장애물입니다. 판정의 방법에 대해서는 \hyperlink{nogm-diceroll}{판정} 챕터에서 더 자세히 설명하겠습니다.
		
		\bigskip
		
		한 흐름의 결과나 답변에 따라, 이야기꾼 또는 소재에 새로운 서술이나 이야기가 추가됩니다. 이 흐름 중 다음 장의 결과를 좌지우지 하는 핵심이라고 할 수 있는 흐름 하나를 [가지]라고 합니다. [가지]는 분기, 질문, 판정 중 어떤 것도 될 수 있고, 혹은 이들을 혼합한 것이 될 수도 있습니다. [가지]가 지나면 한 장이 종료됩니다.
		
		한 장이 시작될 때, 흐름을 잘 섞고 [가지]를 흐름의 가장 아래, 또는 특별한 지시가 있다면 해당 위치에 넣습니다. 그 후, [서사 포화도]가 가장 높은 이야기꾼이 [서술자]가 됩니다. 서술자부터 시작해 [흐름]을 무작위로 하나 뽑습니다. 이 흐름의 종류에 따라 다음이 일어납니다:
		
		\smallskip
		
		\begin{itemize}
			\item 흐름이 질문 또는 분기인 경우:
			\begin{itemize}
				\item 흐름의 응답 대상이 정해져 있는 경우, 해당하는 대상 중 하나에게 흐름을 넘깁니다.
				\item 흐름의 응답 대상이 정해져 있지 않은 경우, 서술자가 원하는 이야기꾼에게 흐름을 넘깁니다.
			\end{itemize}
			\item[] 흐름을 넘겨받은 이야기꾼은 질문이나 분기 선택지에 대한 대답을 합니다.
			\begin{itemize}
				\item 이 대답이 이야기꾼의 새로운 면모를 보여준다면, 이 내용에 관련된 이야기나 서술을 추가합니다.
				\item 무언가 새로운 사건이 발생하였다면 해당 사건에 관련된 이야기나 서술을 추가합니다.
				\item 위에 해당하지 않는다면 이야기나 서술을 추가하지 않아도 됩니다.
			\end{itemize}
			\item[] 그 후, 흐름을 넘겨받은 이야기꾼이 다음 서술자가 됩니다.
			\item[] 자기 자신이 흐름을 받아도 되지만, 이 경우 반드시 대답이 끝난 후 서술자를 다른 이야기꾼에게 넘겨야 합니다.
			
			\item[]
			
			\item 흐름이 판정인 경우:
			\begin{itemize}
				\item 흐름의 판정 대상에 해당하는 대상 중 하나에게 흐름을 넘깁니다. 이 이야기꾼이 다음 서술자가 됩니다.
			\end{itemize}
			\item[] 판정 대상에 해당하는 모든 이야기꾼은 이 판정에 가세할 수 있습니다. 자세한 판정 방법은 \hyperlink{nogm-diceroll}{판정} 챕터를 참고하세요!
			\item[] 판정 결과에 따라 이야기 등이 변화하거나, 없어지거나, 추가될 수 있습니다.
			
			\item[] 
			
			\item 흐름이 가지인 경우:
			\begin{itemize}
				\item 가지의 지시를 따릅니다. 대부분의 경우 모든 이야기꾼이 참여해야만 합니다.
				\item 이후, 이야기의 한 장이 종료됩니다. 가지의 지시에 따라 다음 장을 준비합니다.
			\end{itemize}
		\end{itemize}
		
		각 질문의 대답에 따라 이야기를 얻을 수 있습니다. 이 내용에 따라 자신이 선택해도 괜찮지만, 다른 이야기꾼들이 제안해준다면 그를 따를 수도 있습니다.
		
		어떤 분기에 의해 무시되는 흐름이 발생된다면, 그 흐름을 버리고 다음 흐름을 가져옵니다. 이 경우에 서술자는 바뀌지 않습니다.
		
		준비된 흐름을 진행한 후 가지가 종료된 경우, 한 장이 종료됩니다.
		
\end{document}