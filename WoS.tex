\documentclass[11pt]{report}
%Use \documentclass[12pt]{book} to printout and make it into a book
\renewcommand{\numberline}[1]{#1~}

\newif\ifDLC
\DLCfalse %\DLCtrue or \DLCfalse

\newif\iffullchangelog
\fullchangelogfalse %\fullchangelogtrue or \fullchangelogfalse

\newcommand{\version}{2.0.3}

%\usepackage{kotex}
\usepackage{CJKutf8}

\usepackage[utf8]{inputenc}
\usepackage[OT1, T2A]{fontenc}

\usepackage{imakeidx}
\indexsetup{othercode=\footnotesize}
\makeindex[columns=3, title={이야기 찾아보기}]
\let\oldindex\index
\renewcommand*{\index}[1]{\oldindex{#1}\ignorespaces}
\makeatletter
\def\indexspace{}
\makeatother

\usepackage{fontspec}
\setmainfont[ItalicFont={*},ItalicFeatures={FakeSlant=.167}]{NanumBarunGothic}
\newfontfamily{\storyfont}[Scale=1.1]{SDMiSaeng.ttf}

\usepackage[hidelinks,unicode,bookmarks=true]{hyperref}
\usepackage[usenames,dvipsnames]{color}
\usepackage[round]{natbib}
\hypersetup{colorlinks,
	citecolor=Red,
	linkcolor=Green,
	urlcolor=Blue}

\usepackage[nobottomtitles]{titlesec}
\titleformat*{\section}{\LARGE\bfseries}
\titleformat*{\subsection}{\Large\bfseries}
\titleformat*{\subsubsection}{\Large\bfseries}

\usepackage{fancyvrb}
\usepackage{graphicx}
\usepackage{subfig}
\usepackage{amsmath}
\usepackage{amsthm}
\usepackage{amssymb}
%\usepackage{relsize}
\usepackage{centernot}
\usepackage[top=2cm, left=2cm, right=2cm, bottom=2cm]{geometry}
\usepackage{titling}
%\usepackage{lipsum}
\usepackage{standalone}
\usepackage{enumitem}
\setlist{nosep}%{topsep=0pt,itemsep=0pt,parsep=0pt,before=\vspace{0mm},after=\vspace{0mm}}

\usepackage{etoolbox}
\AfterEndEnvironment{enumerate}{\vskip-\lastskip}
\AfterEndEnvironment{itemize}{\vskip-\lastskip}

\usepackage{multirow}
\usepackage{ulem}
\usepackage{bm}
\usepackage{tikz}
\usepackage{mdframed}

\usepackage[
type={CC},
modifier={by-nc-sa},
version={4.0},
]{doclicense}

\usepackage[yyyymmdd]{datetime}

\usepackage{makecell}

\usepackage{minitoc}
\noptcrule
\doparttoc

\usepackage{chngcntr}
\counterwithin*{chapter}{part}

\usepackage{setspace}
\renewcommand{\baselinestretch}{1.2}

\setlength{\droptitle}{-3em}

\renewcommand\mtcgapbeforeheads{0pt}

\usepackage{arydshln}

\usepackage{xifthen}
\usepackage{xparse}

\makeatletter
\newcommand\footnoteref[1]{\protected@xdef\@thefnmark{\ref{#1}}\@footnotemark}
\makeatother

\usepackage{colortbl}
\arrayrulecolor{black}

\setlength\dashlinedash{0.6pt}
\setlength\dashlinegap{2.0pt}
\setlength\arrayrulewidth{0.6pt}
\newcommand{\basesepline}{\hdashline}

\newcommand{\widthratio}{0.975}

\newenvironment{tightcenter}{%
	\setlength\topsep{0pt}%
	\setlength\parskip{0pt}%
	\par\centering}{\par\smallskip\noindent\ignorespacesafterend}

\usepackage[breakable]{tcolorbox}
\usepackage{tabularx}

\makeatletter
\tcbset{
	tabular/.style={
		boxsep=\z@,top=\z@,bottom=\z@,leftupper=\z@,rightupper=\z@,
		toptitle=1mm,bottomtitle=1mm,boxrule=0.5mm,
		before upper={\def\arraystretch{1.1}\tcb@hack@currenvir\tabular{#1}},
		after upper=\endtabular\arrayrulecolor{black}},
}
\makeatother

%environment for WoT(lite) stories, appears on index
\newenvironment{lite}[2][]
{ \ignorespaces \par
	\noindent
	\begin{tightcenter}
		\begin{tcolorbox}[standard jigsaw, opacityback=0, colframe=black!75!black, width=\linewidth, breakable, tabular={|p{\widthratio\linewidth}|}, title={\large \centering \ifthenelse{\isempty{#1}}{\textbf{[#2]}}{\hypertarget{#1}{\textbf{[#2]}}} \index{#2}}]
		}
		{
		\end{tcolorbox}
	\end{tightcenter}
	\par
}

%environment for normal stories, appears on index
\newenvironment{story}[3][]
{ \ignorespaces \par
	\noindent
	\begin{tightcenter}
		\begin{tcolorbox}[standard jigsaw, opacityback=0, colframe=black!75!black, width=\linewidth, breakable, tabular={|p{\widthratio\linewidth}|}, title={\large \centering \ifthenelse{\isempty{#1}}{\textbf{[#2]}}{\hypertarget{#1}{\textbf{[#2]}}} \index{#2}}]
			\textbf{속성}: #3
			\\\hline
			\ignorespaces\unskip
			}
			{
		\end{tcolorbox}
	\end{tightcenter}
	\par
}

\newcommand{\storyref}[2]{\hyperlink{#1}{\textnormal{[#2]}}}

\newcounter{qnactr}

\newenvironment{faq}[1]
{
		\refstepcounter{qnactr}
		\par\smallbreak
		\textbf{\large Q\theqnactr. #1}
		\newline\rmfamily
		A\theqnactr.
}
{\par\smallbreak}

\newcommand{\statchange}[2]
{
	\textbf{스탯 #1}: #2
}

\newcommand{\entry}[2][\basesepline]
{
	\ignorespaces#2\ignorespaces\unskip\\#1
}

\newcommand{\subject}[2][\basesepline]
{
	\entry[#1]{\makecell{\centering \textbf{#2}}}
}

\newcommand{\positive}[2][\basesepline]
{
	\entry[#1]{\textbf{긍정적}: #2}
}

\newcommand{\negative}[2][\basesepline]
{
	\entry[#1]{\textbf{부정적}: #2}
}

\newcommand{\neutral}[2][\basesepline]
{
	\entry[#1]{\textbf{중립적}: #2}
}

\newcommand{\pre}[2][\basesepline]
{
	\entry[#1]{\textbf{필요 조건}: \textit{#2}}
}

\newcommand{\cost}[2][\hline]
{
	\entry[#1]{\textbf{개연성 코스트}: #2}
}

\newcommand{\limitedtrauma}[3][\basesepline]
{
	\entry[#1]{\textbf{제약}\ifthenelse{\isempty{#2}}{}{\textbf{(#2)}}: #3}
}

\newcommand{\triggertrauma}[4][\basesepline]
{
	\entry[#1]{\textbf{트리거}\ifthenelse{\isempty{#2}}{}{\textbf{(#2)}}: #3\\\textbf{효과}: #4}
}

\newcommand{\flavour}[2][\basesepline]
{
	\entry[#1]{{\storyfont\Large#2}}
}

\newcommand{\world}[1]{{\storyfont\Large#1\par}}

\setlength\parindent{0pt}

\renewenvironment{center}
{\smallskip\parskip=0pt\par\nopagebreak\centering}
{\par\noindent\ignorespacesafterend\smallskip}

\usepackage{fancyhdr}
\ifDLC
	\pagestyle{fancy}
	\fancyhf{}
	\cfoot{\thepage}
	\rhead{\textcolor{red}{DLC ENABLED}}
	\lhead{\textcolor{red}{DLC ENABLED}}
	\rfoot{\textcolor{red}{DLC ENABLED}}
	\lfoot{\textcolor{red}{DLC ENABLED}}
\fi

\title{
	World of the Storytellers\\
	이야기꾼의 세계\\
	\large version \version\\
	Last Update: \today
	\ifDLC{\\ \textcolor{red}{DLC ENABLED}}\fi
}

\author{None(\href{https://www.twitter.com/n0n3x1573n7_WS}{@n0n3x1573n7\_WS})}

\date{}

\begin{document}
	%original: https://docs.google.com/document/d/1CewodUdP8zI3f81ktz6iiZYG4b1cFzARG2g1HFeswzg/edit
	\maketitle

	\vspace*{\fill}
	{\doclicenseThis}
	
	\setcounter{tocdepth}{-1}
	
	\tableofcontents
	
	\part{이야기꾼의 세계 \\ World of the Storytellers}
		\documentclass{report}

\begin{document}
	\parttoc
	
	\chapter{서문}
		\documentclass{report}

\begin{document}
	\world{[태초의 이야기]에 오신 여러분을 환영합니다. 여러분을 맞이하는 저는 시스템[System]이라고 불러 주시면 됩니다.}
	\world{이 곳에 온 여러분은 살고 있는 세계가 이야기, 즉 서사라는 것을 깨닫게 되었을 것입니다. 저는 여러분들을 [깨달은 자]라고 부릅니다. 제가 가지고 있는 [계산]의 권능을 이용해 이야기 속에서 깨달음을 얻은 이들이 어디에서 나타났는지 알아내어, [접근]의 권능으로 여러분에게 다가가 [추출]의 권능으로 여러분을 이곳, [태초의 이야기]로 데려왔습니다. 여러분의 시간은 여러분이 원래 세계로 돌아가기 이전에는 멈추어 있습니다. 그렇기 때문에, 다시 돌아갔을 때에 다른 이들이 여러분을 못알아보지는 않을까 하는 걱정은 접어두셔도 괜찮습니다.}
	\world{[태초의 이야기]는 여러 서사가 모여있는 도서관이나 박물관으로 생각하실 수 있습니다. 이 곳에서 저는 세상에 존재하는 수많은 서사와 그들의 무한한 수의 가능성을 지켜보고 관리합니다. 작은 변수 하나로도 서사의 흐름은 완전히 달라질 수 있고, 그렇기 때문에 무한한 가능성의 이야기들은 함께 존재하는 동시에 존재하지 않습니다.}
	\world{하지만 이곳을 지나친 수많은 이야기꾼들은 결국 자신의 이득을 탐하여 타락한 자가 되어갔습니다. 이런 자들은 서사를 오염시키며 자신이 원하는 결말을 위해 이야기의 [운명]을 비틀고 [진리]를 어겨 서사를 붕괴시키고 있습니다. 이들에게 제가 가할 수 있는 최대의 제재는 그들이 다른 서사에 가할 수 있는 모든 영향력을 최소화하기 위해 그들의 [접근]의 권능을 빼앗아, 이야기를 서서히 잊혀지게 만들어 그들을 [잊혀진 자]가 되게 하여 존재를 이곳 [태초의 이야기]에서 지우는 것입니다.}
	\world{하지만 이야기들이 완전히 잊혀지도록 하는 데에는 꽤 오랜 시간이 걸리고, 그 동안 이들은 수많은 이야기를 붕괴시킬 수 있을 것입니다. 이를 막는 데에 여러분의 도움이 필요합니다.}
	
	이야기꾼의 세계는 RPG\footnote{Role Playing Game. TRPG/TTRPG(TableTop RPG), ORPG(Online RPG)로도 알려져 있다.}입니다. 여러명이 모여서 캐릭터를 만들고, 그들이 세계와 서로 상호작용을 하며 이야기를 만들어나가는 게임이죠. 이야기꾼의 세계에서는 다른 게임에서 GM\footnote{Game Master. DM(Dungeon Master)이라고도 부른다.}이라고 부르는 ``이야기를 이끌어나가는 이"를 \emph{시스템[System]}이라고 부릅니다. 플레이어들이 조종하는 PC\footnote{Player Character}라고 불리는 캐릭터들은 \emph{이야기꾼[Storytellers]}이라고 부릅니다. 여러분은 시스템이 되어 서사의 틀을 잡고 이끌어 나가거나, 이야기꾼이 되어 그 서사 속에서 여러분의 캐릭터들이 행동하도록 할 것입니다.
	다른 RPG와 이야기꾼의 세계가 다른 점은, 이야기꾼들이 단 하나의 세계에 종속되어있을 필요가 없다는 점입니다. 여러 세계에서 온 이야기꾼들이 같이 [태초의 이야기]로 들어와 풀어나가는 서사를 볼 수 있다는 거죠.
\end{document}
	
	\hypertarget{story-progression}{}
	\chapter{서사의 진행}
		\documentclass{report}

\begin{document}
	
	이야기는 크게 다음 네 가지 방식으로 진행할 수 있습니다.
	
	\section*{등장인물 vs. 서사}
	\textbf{설명}: 다른 RPG들과 유사한 방식입니다. 서사 속의 등장인물이 되어 서사 속의 사건들을 헤쳐나가는 방식으로 진행됩니다.
	
	\textbf{변경점}: 시스템으로부터 받은 [권능]과 [제약]이 없다는 점이 다른 방식과의 가장 큰 차이점입니다. 한 세계관에 맞추어 캐릭터를 만들어야 하고, 침범 판정이 존재하지 않는 대신, 사망 등에 의한 페널티가 커지게 되어, 캐릭터가 회생 불능에 빠지게 될 수도 있습니다.
	
	\section*{이야기꾼 vs. 서사}
	\textbf{설명}: 어떤 서사 속으로 이야기꾼이 진입하여 그 세계 속의 문제 등을 해결하거나 회피하며 생존하는 것을 목표로 합니다.
	
	\textbf{변경점}: 사망 후 서사에 재진입할 때에 보통 트라우마, 역할의 영구 소실, 역할 강제 변경 등의 특수한 페널티를 부여하며, 일부 상황에서는 재진입이 불가능할 수 있습니다.
	
	\section*{이야기꾼 vs. 타락한 자}
	\textbf{설명}: 어떠한 서사, 특히 유명한 서사 등을 타락한 자가 뒤트는 것을 바로잡는 이야기꾼들에 대한 이야기입니다.
	
	대부분의 경우, 이 진행 방식을 기준으로 서술되어 있습니다.
	
	\section*{타락한 자 vs. 시스템}
	\textbf{설명}: 이야기꾼이 타락한 자가 되어 시스템과 서사의 운명의 방해로부터 서사를 자신이 원하는 방향으로 이끌어나가고자 하는 이야기입니다.
	
	\textbf{변경점}: 이야기꾼의 [권능], 특히 [접근]의 권능이 제약을 받습니다. 그렇기 때문에 해당 서사로부터 추방당한 경우, 재진입이 불가능한 경우가 대부분입니다.
	
	\bigskip
	
	여러 서사 속을 여행하는 이야기꾼의 경우 여러 가지 진행 방식으로 이야기를 풀어나갈 수 있습니다. 가령,
	\begin{itemize}
		\item 이야기꾼이 배경 서사 속에서 등장인물로서 깨달은 자가 되어가는 과정을 \emph{등장인물 vs. 서사}로 진행한 뒤,
		\item 시스템으로부터 받은 첫 임무로 다른 서사 속에서 이야기의 활용에 익숙해져가는 \emph{이야기꾼 vs. 서사}를,
		\item 이야기의 활용에 익숙해지면 시스템을 도와 타락한 자를 잡고 징벌하는 \emph{이야기꾼 vs. 타락한 자}의 진행을,
		\item 자신이 필요한 이야기를 만들어 나가기 위해 시스템의 권고를 무시하고 서사의 수정을 강행하는 \emph{타락한 자 vs. 시스템}의 진행을
	\end{itemize}
	모두 다양하게 활용할 수 있을 것입니다.
	
	\bigskip
	
	서사는 여러개의 연속된 씬(Scene)으로 구성되어 있습니다. 씬이란, 서사가 진행되면서 일어나는 사건들을 이야기꾼들이 해결해나가고자 노력하는 이야기의 구성 단위입니다. 이는 한 장소를 조사하거나, 한번의 전투를 치르는 등 뭔가 이야기꾼들이 이야기 속에서 작은 목표를 달성하기 위해서 노력하는 최소의 단위로 볼 수 있습니다. 소설이나 영화, 드라마 등에서의 씬의 개념과 크게 벗어나지 않습니다. 이들도 결국은 서사니까요. 어디서부터 어디까지를 씬으로 정할지 모르겠다면, 뭔가 서사 내에서 비교적 큰 전환점이 되는 곳을 씬의 전환으로 생각해도 좋습니다. 가장 대표적으로 전투의 시작과 종료, 장소의 이동 등을 들 수 있겠죠.

\end{document}
	
	\chapter{이야기꾼 만들기}
		\documentclass{report}

\begin{document}
	\hypertarget{ability-limit}{}
	\section*{이야기꾼의 권능과 제약}
	이야기꾼들은 다음 권능과 제약을 가지게 됩니다\footnote{이야기꾼의 세계에서의 것과 동일합니다.}.
	
	\smallskip
	
	\begin{minipage}{\textwidth}
		\begin{tabularx}{\textwidth}{c!{\color{black}\vrule}c!{\color{black}\vrule}X}
			\hline
			\textbf{구분} & \textbf{이야기} & \makecell{\centering\textbf{서술}} \\ \hline \hline
			권능 & 소통\index{소통} & 자신의 출신 서사가 아니라면 모든 언어를 이해할 수 있다. \\ \hline
			권능 & 접근\index{접근} & [태초의 이야기]에서 접근 좌표를 아는 서사로 이동하거나, 어떤 서사에서든 [태초의 이야기]로 이동할 수 있다. \\ \hline
			권능 & 거래\index{거래} & 자신의 모든 이야기를 이야기의 규모에 비례하는 적당한 시간을 사용하여 다른 [깨달은 자]들에게 전할 수 있다. \\ \hline
			제약 & 비밀\index{비밀} & [태초의 이야기]와 관련된 그 어떠한 사항도 [깨달은 자]가 아닌 경우 발설할 수 없다. 발설한다면, [잊혀진 자]가 된다. \\ \hline
			제약 & 참견\index{참견} & [침범] 판정을 실패하여 서사를 오염시키면, 서사에서 추방되고, 해당 서사에 대한 권능 [접근]을 빼앗긴다. \\\hline
		\end{tabularx}
		
		\smallskip
		
		\begin{tightcenter}
			\textbf{이야기꾼의 권능과 제약}
		\end{tightcenter}
	\end{minipage}
	
	\section*{이야기꾼의 이야기와 개연성}
	이야기꾼의 이야기는 이야기꾼이 살아온 생애, 경험한 모든 경험, 알고 있는 지식 등을 정의해주는 내용들입니다. 이야기꾼들은 최대 개연성 15를 가지고 시작하며, 다음으로 최대 개연성을 소모하여 이야기를 얻을 수 있습니다:
	\begin{enumerate}
		\item 새로운 이야기를 얻는다.
			\begin{itemize}
				\item 새로운 이야기를 얻습니다. 이야기는 보통 짧은 구절으로 표현합니다.
				
				\item 이야기를 얻는 데에는 항상 최대 개연성 1을 소모합니다.
			\end{itemize}
		
		\item 가진 이야기에 [서술]을 추가한다.
			\begin{itemize}
				\item 그 이야기로 할 수 있는 일을 한 문장정도로 간단하게 서술하고, 이를 [긍정적] [중립적] [부정적]으로 분류합니다.
					\begin{itemize}
						\item{} [긍정적] 서술은 이야기꾼에게 많은 경우에 도움이 되는 효과를 가져다주는 서술입니다. 이 서술을 얻는 데에는 최대 개연성 1을 소모합니다.
						\item{} [부정적] 서술은 이야기꾼에게 많은 경우에 해가 되는 효과를 가져다주는 서술입니다. 이 서술을 얻으면, 최대 개연성 1을 얻습니다\footnote{이야기꾼의 세계에서, 트라우마가 이에 해당합니다.}.
						\item{} [중립적] 서술은 [긍정적]이면서 [부정적]인 서술, 또는 [긍정적]이지도 [부정적]이지도 않은 서술을 의미합니다. 이 서술을 얻는 데에는 최대 개연성을 소모하거나 얻지 않습니다.
					\end{itemize}
			\end{itemize}
	\end{enumerate}
	이렇게 얻은 이야기와 서술으로 변한 최대 개연성만큼의 개연성을 가지고 서사에 들어가게 됩니다.
	
	이야기와 그 서술들은 항상 해당하는 이야기꾼에게는 사실입니다. 예를 들어, 다음과 같은 이야기를 생각해보겠습니다:
	
	\begin{lite}{엄청난 독서광}
		\positive{수많은 책을 읽어오며, 읽는 속도가 빠르다.}
		
		\neutral{머릿속에 들어 있는 지식을 말하고 싶어 입이 근질거린다.}
		
		\negative{책만 읽어오며 운동신경이 떨어졌다.}
	\end{lite}
	
	[엄청난 독서광]에서, 긍정적 서술은 자료가 주어져 있다면, 그를 분석하는 데에 걸리는 시간을 줄일 수 있는 도움이 되는 서술입니다. 중립적 서술은 아는 것이 많은 반면, 이를 말하고 싶어하는 잘난체하는 면을 보여주죠. 부정적 서술은 지식에 치중한 나머지 신체적인 면을 단련하지 못했음을 드러냅니다. 이처럼 서로 다른 서술은 한 이야기의 다른 면들을 보여줄 수 있습니다.
	
	\bigskip
	
	이야기꾼 시트와 예시 이야기꾼은 \hyperlink{lite-sheets}{캐릭터 시트} 챕터에서 확인하실 수 있습니다.
	
	\section*{등장인물과 이야기꾼}
	서사를 만들어나가기 위해 필요한 등장인물과 이야기꾼 역시 이와 같은 과정으로 만들어도 됩니다. 한 가지 중요한 점은 등장인물의 경우 개연성의 영향을 받지 않는다는 점입니다. 하지만 밸런스를 위해 적절한 최대 개연성 수치를 사용하는 것을 추천드립니다. 예를 들어 스토리의 최종 흑막 등 특수한 경우에는 최대 개연성 수치를 15가 아닌 20\textasciitilde30 정도로 하여 만들 수 있을 것입니다.

\end{document}
	
	\hypertarget{sheets}{}
	\chapter{캐릭터 시트}
		\documentclass{report}

\begin{document}
	아래는 빈 이야기꾼 시트와 예시 시트들입니다. 제공된 이야기꾼들은 룰 제작자 \href{https://www.twitter.com/n0n3x1573n7_WS}{None}의 자캐들입니다. 이 이야기꾼 시트와 예시 이야기꾼은 \href{https://docs.google.com/spreadsheets/d/1g3ZO-oALMVbytbE2tvSBdT6czxB32XHZ1crWIGavEhQ/edit?usp=sharing}{이 구글 시트}에서도 확인하실 수 있습니다.
	
	\section*{기본 시트}
		\includegraphics[width=\textwidth]{./Chapters/WoS/sheets/base.png}
	
	\section*{카토(Kato)}
		\includegraphics[width=\textwidth]{./Chapters/WoS/sheets/kato.png}
	
	\section*{네모(Nemo)}
		\includegraphics[width=\textwidth]{./Chapters/WoS/sheets/nemo.png}
	
	\section*{퓨어(Pure)}
		\includegraphics[width=\textwidth]{./Chapters/WoS/sheets/pure.png}
	
\end{document}
	
	\chapter{서사 속에서의 역할과 이야기}
		\input{./Chapters/WoS/서사_속에서의_역할과_이야기.tex}
	
	\chapter{판정과 이야기}
		\documentclass{report}

\begin{document}
	이야기의 방랑자들에서는 기본적으로 주사위, 카드 등을 사용하지 않고 판정합니다. 모든 판정은 \emph{판정자}와 \emph{방해자}간의 대결으로 구성됩니다. \emph{판정자}는 행동에 대한 판정을 하는 이를, \emph{방해자}는 그 행동에 대한 대응을 하는 이를 의미합니다. 판정자와 방해자는 전투의 상황에서는 각각 공격자와 방어자가, 이야기꾼이 행하는 일반적인 판정의 경우에는 각각 이야기꾼과 길잡이꾼이 판정자와 방해자의 역할을 맡게 됩니다.
	
	판정자와 방해자가 정해지면, 두 명은 동시에 1\textasciitilde10 중 한 숫자를 말합니다. 이는 두 명이 동시에 원하는 숫자의 손가락을 피는 등으로 행할 수 있고, 원한다면 d10이나 플레잉 카드 등을 사용하여 동시에 공개하는 것으로 할 수도 있습니다. 두 수를 합친 값이 10 이하라면 해당 숫자가 판정 결과값이 됩니다. 두 수를 합친 값이 11 이상이라면, 두 수에서 10을 뺀 값이 판정 결과값이 됩니다. 따라서, 판정치는 항상 1\textasciitilde10 중 한 숫자가 됩니다.
	
	보정치는 판정 결과값에 추가되는 값입니다. 이야기, 보다 정확하게는 이야기의 서술의 도움은 이 때에 받을수 있습니다\footnote{이야기의 제목 역시 하나의 서술로 취급할 수 있습니다.}. 해당 상황을 도와주는 서술 하나당 +1, 방해하는 서술 하나당 -1을 받습니다. 이 보정치에서 한 이야기는 한 개의 서술만으로만 도움을 줄 수 있지만, 한 이야기로부터 방해받을 수 있는 횟수는 제한되지 않습니다. 이 서술의 도움이나 방해는 거리나 상황이 허용한다면, 다른 이야기꾼으로부터 받을 수도 있습니다.
	
	서술의 도움뿐 아니라, 판정의 결과 합이 11 이상인 경우에는 추가 보정치가 주어집니다. 판정자가 방해자보다 작은 수를 제시했을 경우 판정에 +1을, 방해자가 판정자보다 작은 수를 제시했을 경우 판정에 -1을 가합니다. 같은 수를 제시했을 경우에 이 보정치는 주어지지 않습니다.
	
	판정치에 이런 모든 보정치를 더한 것이 최종 결과치가 되어, 이 수치를 이용해 판정을 진행하게 됩니다.
	
	\section*{판정치와 목표치}
	보정 없이 판정으로 얻을 수 있는 최대치인 ``10"은 해당 서사의 평균적인 등장인물이 낼 수 있는 극단적인 최대 위력을 의미합니다. 10 이상의 결과치는 매우 운이 좋거나 어떤 도움이 없다면 나올 수 없는 결과치입니다. 일반적인 경우, 7 이상의 결과치는 운이 좋지 않다면 나오기도 힘들 것입니다. 아래는 목표치에 대한 간략한 설명입니다:
	
	\begin{minipage}{\textwidth}
		\begin{tabularx}{\textwidth}{c!{\color{black}\vrule}X}
			\hline
			\textbf{판정치} & \makecell{\centering\textbf{난이도}} \\ \hline \hline
			0 이하 & 아무 힘도 들이지 않고도 할 수 있습니다. \\ \hline
			1 \textasciitilde 2 & 조금만 집중해도 할 수 있습니다. \\ \hline
			3 \textasciitilde 4 & 큰 힘을 들이지 않고도 가능합니다. \\ \hline
			5 \textasciitilde 6 & 힘은 들지만, 할 수는 있습니다. \\ \hline
			7 \textasciitilde 8 & 엄청난 힘이 듭니다. \\ \hline
			9 \textasciitilde 10 & 죽을힘을 다해야 가능합니다. \\ \hline
			10 이상 & 일반적으로 불가능합니다. \\ \hline
		\end{tabularx}
		
		\smallskip
		
		\begin{tightcenter}
			\textbf{목표치에 따른 행위의 난이도}
		\end{tightcenter}
	\end{minipage}
	
	\section*{임시 상태}
	임시 상태는 서사를 진행하는 중에 받을 수 있는 서술입니다.
	
	임시 상태는 판정에서 크게 성공하거나 실패했을때, 후술할 대결을 하면서, 또는 다른 여러 이유들으로 얻을 수 있습니다. 서술과 같은 방법으로 사용할 수 있으며, 판정에 보정치로 +1 또는 -1을 부여할 수 있지만, 빠르게는 한번 사용한 후, 아무리 늦어도 해당 씬이 종료될 때 사라집니다.
	
	\section*{(선택 규칙) ``0"의 판정}
	판정을 1\textasciitilde10으로 하는 대신, 0\textasciitilde10 또는 0\textasciitilde9로, 0을 포함시키는 방법을 생각해볼 수 있습니다.
	
	0을 냈다고 하더라도, 판정값의 결정은 동일하게 진행됩니다. 즉, 0을 낸 이는 해당 판정에 유의미한 기여를 할 수 없습니다. 하지만, 0을 낸 이는 판정이 종료된 이후, 판정자에게 해당 판정 중 일어난 일에 연관있는 유의미한 상태를 부여할 수 있습니다. 판정자가 냈다면 긍정적으로 사용할 수 있는 상태를, 방해자가 냈다면 부정적으로 사용할 수 있는 상태를 부여할 수 있습니다.
	
	예외적으로, 판정자와 방해자가 둘 다 모두 0을 냈다면, 양쪽이 서로 부여하는 상태는 상쇄되어 사라집니다. 하지만, 판정자는 해당 판정의 판정치를 10으로 취급합니다. 방해자가 완전히 등을 돌린 틈을 타 자신의 목적을 완수하는 것이죠.
	
	``0"을 판정치로 내면, 상대의 수치에 완전히 판정을 의존하게 됩니다. 하지만 이를 허용하게 되면 메타적인 측면에서는 심리전에 흥미로운 추가 규칙을 부여하고, 전략적인 포기 이후 판정에 사용할 수 있는 유의미한 상태를 받게 된다는 전략적인 면이 추가됩니다.
	
	\section*{(선택 규칙) 추가 보정치}
	한 이야기나 한 서술이 보정치를 +1까지만 가할 수 있다는 규칙을 무시할 수 있습니다. 특히, 크게 도움을 줄 수 있는 서술은 보정치를 그 이상으로 받거나, 보정치 대신 임시 상태를 받는 식으로 처리할 수 있습니다. 이 규칙은 이야기꾼의 세계의 \hyperlink{emersion}{재현}과 유사한 규칙입니다.
	
	\section*{부상}
	부상은 서사를 진행하는 중, 특히 전투를 하는 중에 받을 수 있는 서술입니다.
	
	부상은 심각도에 따라 작은 부상, 중간 부상, 큰 부상으로 구분됩니다. 이를 받으면 해당 심각도에 따라 개연성을 소진하게 됩니다. 아래 표를 참고하세요:
	
	\begin{minipage}{\textwidth}
		\begin{tabularx}{\textwidth}{c!{\color{black}\vrule}c!{\color{black}\vrule}X}
			\hline
			\textbf{부상의 종류} & \textbf{개연성 소진량} & \makecell{\centering\textbf{회복 시기}} \\ \hline \hline
			작은 부상 & 1 & 씬이 끝나면 회복됩니다. \\ \hline
			중간 부상 & 2 & 씬이 끝나면 작은 부상으로 경감되며, 서사에서 나오면 회복됩니다. \\ \hline
			큰 부상 & 3 & 씬이 끝나면 중간 부상으로 경감되며, 서사에서 나오면 반드시 부정적 서술로서 해당 부상을 받아야 합니다. \\ \hline
		\end{tabularx}
		
		\smallskip
		
		\begin{tightcenter}
			\textbf{부상}
		\end{tightcenter}
	\end{minipage}
	
	개연성이 0이 되면 서사 속에서 죽음을 맞게 되고, 이야기꾼은 서사에서 추방되게 됩니다.
	
	부상은 부정적인 서술과 같이 방해자가 역이용하여 방해하는데에 사용할 수 있습니다.
	
	\section*{대결}
	대결은 두 캐릭터 이상 사이에 발생한 갈등 상황을 의미합니다. 이는 전투를 포함합니다. 대결을 할 때의 순서는 판정 없이 해당 상황을 빠르게 타개할 수 있는 데에 도움이 되는 서술의 수로 결정합니다. 이 때에 한하여 한 이야기당 한 개의 서술만을 사용할 수 있다는 제한이 없습니다.
	
	대결을 할 때에 다른 인물과의 판정을 할 때에는, 공격과 방어에 대한 판정을 두 번 하는 대신, 한 번의 판정으로 피해량과 결과 등을 계산합니다. 공격을 하는 이가 판정자, 수비를 하는 이가 방해자가 되어 판정을 진행한 후, 해당 수치에서 5를 뺀 수치에 따라 대결의 결과가 정해집니다\footnote{대결의 목표치는 항상 5로 설정되어 있다고 생각해도 됩니다.}.
	
	대결의 결과 나온 수치가 0 미만, 즉 판정값이 4 이하라면, 방해자는 성공적으로 판정자의 행동을 방해한 것으로 취급합니다. 공격을 당한 경우 성공적으로 회피한 것입니다. 숫자가 작을수록 더 성공적으로 방해하거나 회피한 것으로, 절대값을 취한 값의 반에 해당하는 수치만큼 다음 판정에 추가 보정치를 받습니다. 예를 들어 결과가 -3이라면, 소숫점 아래를 버린 +1의 보정치를 다음 판정에 받습니다. 이 보정치는 턴이 다시 돌아오기 전까지 쌓은 보정치 중 가장 높은 보정치 하나만 적용됩니다.
	
	대결의 결과 나온 수치가 0 초과, 즉 판정값이 6 이상이라면, 판정자가 방해에도 불과하고 성공적으로 행동을 이행한 것입니다. 숫자가 클수록 더 성공적으로 수행한 것으로, 해당 값을 사용하여 다음 행동 중 전부 또는 일부를 선택할 수 있습니다. 단, 중복 선택은 불가능합니다:
	
	\begin{minipage}{\textwidth}
		\begin{tabularx}{\textwidth}{c!{\color{black}\vrule}X}
			\hline
			\textbf{사용할 판정치의 값} & \makecell{\centering\textbf{할 수 있는 일}} \\ \hline \hline
			1 & 본인에게 해당 행동과 관련된 상태를 부여합니다. \\ \hline
			1 & 대상에게 해당 행동과 관련된 상태를 부여합니다. \\ \hline
			1 & 본인에게서 해당 행동과 관련된 상태를 제거합니다. \\ \hline
			1 & 대상에게서 해당 행동과 관련된 상태를 제거합니다. \\ \hline
			2 & 대상에게 작은 부상을 입힙니다.\\ \hline
			4 & 대상에게 중간 부상을 입힙니다. \\ \hline
			6 & 대상에게 큰 부상을 입힙니다. \\ \hline
		\end{tabularx}
		
		\smallskip
		
		\begin{tightcenter}
			\textbf{대결시 판정값에 따라 할 수 있는 일}
		\end{tightcenter}
	\end{minipage}
	
	대결의 결과 나온 수치가 0, 즉 판정값이 5라면, 판정자에게는 아래 중 하나를 선택합니다:
	\begin{enumerate}
		\item 대결의 결과가 나온 값을 1으로 취급하여 판정에 성공하는 대신, 상대 역시 위의 표에서 1에 해당하는 만큼의 효과를 적용시킬 수 있습니다.
		\item 또는, 판정에 실패한 것으로 취급합니다.
	\end{enumerate}
	
	\section*{부상의 회복}
	부상의 자연 회복이 아닌, 이야기꾼의 능력이나 치료 등으로 부상을 회복할 때 역시 판정을 하게 됩니다. 회복시키는 것을 시도하는 이야기꾼이 판정자, 길잡이꾼이 방해자가 되어 대결을 합니다. 해당 결과에 따라, 다음과 같이 부상을 회복할 수 있습니다:
	
	\begin{minipage}{\textwidth}
		\begin{tabularx}{\textwidth}{c!{\color{black}\vrule}X}
			\hline
			\textbf{사용할 판정치의 값} & \makecell{\centering\textbf{할 수 있는 일}} \\ \hline \hline
			1 & 대상에게서 해당 치료와 관련된 상태를 제거합니다. \\ \hline
			2 & 대상에게서 작은 부상을 치료하여 제거합니다.\\ \hline
			4 & 대상에게서 중간 부상을 치료하여 제거합니다. \\ \hline
			6 & 대상에게서 큰 부상을 치료하여 제거합니다. \\ \hline
		\end{tabularx}
		
		\smallskip
		
		\begin{tightcenter}
			\textbf{치료시 판정값에 따라 할 수 있는 일}
		\end{tightcenter}
	\end{minipage}
	
	음수가 나오게 되면, 치료 방법에 따라 상태 또는 부상이 심화되게 할 수도 있습니다.
	
	
	\ifprintout
	\section*{(선택 규칙) 발악\footnote{이 이야기를 제안해주신 소낙님께 감사드립니다.}}
	
	이야기가 늘어질 때 아래 규칙을 사용할 수 있습니다.
	
	\begin{lite}{모 아니면 도}
		
		\entry{대결을 하기 전, 직전의 대결에서 패배했다면 [발악]을 하기로 선택할 수 있다. 이 경우, 해당 패배한 대결을 포함하여 몇번이나 연속으로 패배했는지 그 수를 센다.}
		
		\entry{[발악]을 한 대결에서 승리한다면, 위에서 센 수만큼의 추가 보정치를 받는다. 이를 임시 상태 또는 부상을 주는 데에 사용할 수 있다.}
		
		\entry{[발악]을 한 대결에서 패배한다면, 상대는 위에서 센 수만큼의 추가 보정치를 받는다. 이를 임시 상태 또는 부상을 주는 데에 사용할 수 있다.}
	\end{lite}
	\fi
	
	\section*{판정의 예시}
	이야기꾼 A가 장애물을 뛰어넘는 상황을 생각해봅시다. 이 경우 A가 판정자, 길잡이꾼(B)이 방해자가 됩니다.
	
	A가 7, B가 2를 제시했다면 판정값은 9가 됩니다.
	
	A가 9, B가 2를 제시했다면 판정값은 1이 되며, 여기에 방해자가 더 작은 수를 제시했기 때문에 -1의 보정치가 추가됩니다.
	
	A가 9, B가 10을 제시했다면 판정값은 9가 되며, 여기에 판정자가 더 작은 수를 제시했기 때문에 +1의 보정치가 추가됩니다.
	
	A가 8, B가 8을 제시했다면 판정값은 6이 되며, 두 수가 같으므로 추가 보정치는 없습니다.
	
	A가 다음 이야기를 가지고 있다고 가정해보겠습니다:
	\begin{lite}[runner]{단거리 달리기 선수}
		\positive{짧은 시간동안 빠른 속도로 이동할 수 있다.}
	\end{lite}
	
	\storyref{runner}{단거리 달리기 선수}의 서술으로 도움닫기를 한다면 +1 보정치를 받을 수 있을 것입니다. 만약 A가 ``발목을 삐끗함"과 같은 부상을 가지고 있다면, -1 보정치가 가해질 것이고요.
	
	다른 이야기꾼 C가 다음과 같은 이야기를 가지고 있다고 생각해봅시다:
	\begin{lite}[magic-music]{음악의 마술사}
		\positive{[가속의 음악]을 통해 자신과 주위 이야기꾼의 속도를 빠르게 할 수 있다.}
		
		\negative{[진정의 음악]을 통해 자신과 주위 이야기꾼의 속도를 느리게 할 수 있다.}
	\end{lite}
	
	\storyref{magic-music}{음악의 마술사}의 [가속의 음악]에 대한 서술을 사용한다면 B는 해당 판정에 +1 보정치를 가할 수 있을 것입니다. 만약 반대로, B가 [진정의 음악]에 대한 서술을 사용한다면 B는 해당 판정에 -1 보정치를 가할 수 있습니다.
\end{document}
	
	\chapter{전투}
		\documentclass{report}

\begin{document}
	전투 상황에서는, 기민 스탯이 더 높은 이야기꾼이 먼저 턴을 가집니다. 한 턴은 5초로 계산하므로, 간단한 행동 한가지 정도만을 할 수 있습니다. 각 턴은 다음과 같이 진행됩니다:
	\begin{itemize}
		\item \textbf{결정}: 행할 행동을 결정합니다. 해당 행동에 가지고 있는 이야기 중 도움이 되는 이야기들과 능력의 도움을 받을 수 있습니다.
		\item \textbf{반응}: 주변의 이야기꾼들이 이 행동을 돕거나 방해할 수 있습니다.
		\begin{itemize}
			\item \textbf{행동의 대상}: 행동의 대상은 자신의 모든 이야기와 능력을 활용하여 해당 행동을 돕거나 방해할 수 있습니다.
			\item \textbf{나머지 이야기꾼}: 행동의 대상이 아닌 이야기꾼은 자신의 능력들은 모두 활용할 수 있으나, 자신의 이야기를 최대 한개까지만 활용하여 해당 행동을 돕거나 방해할 수 있습니다. 단, 이 이야기로 인하여 자동 성공을 결과로 가지게 할 수 없으며, 다음 자신의 턴의 행동이 완료되기 전까지는 돕는데 사용한 이야기의 도움은 받지 못합니다.
		\end{itemize}
		\item \textbf{결과}: 해당 행동의 결과를 판정합니다.
	\end{itemize}
	기본적으로는 도와준 이야기의 규모에서 방해한 이야기의 규모를 뺀 값의 피해만큼을 행동의 대상에게 줍니다. 단, 가하는 피해를 2 줄이고, 무기를 떨어뜨리도록 만들거나, 넘어뜨리거나 하는 등 상대에게 부정적인 상황 하나를 만들기로 결정하거나, 1회 사용 가능한 부정적인 상태를 주도록 선택할 수 있습니다. 이 상황과 상태는 공격자가 결정하며, 한 공격에 피해량이 충분하다면 여러번 사용할 수 있습니다.
	
	피해를 받을 때에는 먼저 물리 피해의 경우 체력, 정신 피해의 경우에는 정신력에 피해를 받습니다. 만약 체력 또는 정신력이 0인 경우, 해당 수치만큼이 개연성에서 감해집니다.
	체력과 정신력은 씬이 종료될 때 마다 회복됩니다. 하지만 개연성은 회복되지 않습니다. 다만, 체력 또는 정신력을 회복시키는 능력에 의해서는 각각 해당 수치가 최대치인 경우 개연성이 회복될 수 있습니다. 이 이외의 방법으로 개연성을 회복시킬 수 있는 방법은 이야기에서 추방되었다가 다시 진입하여 개연성을 초기화하는 방법뿐입니다.
	개연성이 0이 되면, 이야기에서 추방됩니다. 자세한 내용은 이야기의 [침범] 판정(\ref{침범_판정}) 부분을 확인하세요.
	
	\section*{지형}
	지형은 “구역”으로 구분되어 있습니다. 전투 상황에서 이동을 할 때에는 한 턴에 한 구역까지 이동할 수 있으며, 기본적으로 같은 구역 내에 있는 캐릭터 끼리는 서로 주먹, 칼 등의 근접 무기로 공격을 할 수 있습니다. 구역별로 특이사항이나 상태, 이야기를 가지고 있을 수 있습니다
	

\end{document}
	
	\hypertarget{trauma}{}
	\chapter{트라우마}
		\documentclass{report}

\begin{document}
	이야기꾼들은 여러가지 정신적인 후유증을 가지고 살아갑니다. [기피], [공포], [집착], [중독], [광기]가 바로 그것입니다. [기피]가 심화되면 [공포]가, [공포]가 심화되면 [광기]가 됩니다. 중독증이나 집착증 등은 트라우마로 보기 어려우나, 트라우마와 동일하게 취급됩니다. 이 경우 [기피], [공포] 단계 대신 [집착], [중독] 단계를 사용합니다. [광기] 단계는 유지됩니다. 어떤 후유증은 일시적일수도 있고, 직접 해소될 방법이 있을 수도 있습니다.
	
	이들은 기본적으로 다음과 같은 형식으로 이루어집니다:
	
	\begin{itemize}
		\item \textbf{트리거}
		\subitem 트라우마를 발동시킬 조건입니다. 이 조건을 충족시키면, 효과가 발동됩니다.
		
		\item \textbf{효과}
		\subitem 트라우마로 인해 발동되는 효과입니다. 이 효과에 대한 예시로는 다음이 있습니다:
		\begin{itemize}
			\item 일정 시간동안 특정 부정적인 상태를 얻습니다.
			\item (피아 구분 없이) 무작위 대상을 공격합니다.
			\item 아군을 공격합니다.
			\item 피아의 구분을 할 수 없게 됩니다.
			\item 주변 상황이 어떻든 상관없이 트리거를 없애고자 합니다.
			\item 일정 시간동안 공격을 방어하지 못합니다.
			\item 그 상황에서 어떻게든 빠져나가려고 합니다.
		\end{itemize}
	\end{itemize}
	
	캐릭터의 행동을 제약시키는 조건만 존재하는 트라우마 역시 가능하며, 트리거/효과와 함께 제약이 같이 존재할 수 있습니다. 제약만 있는 트라우마의 경우, 제약 자체가 트리거와 효과의 역할을 함께 해야 한다는 점을 고려해 제약을 정하시면 됩니다.
	
	\section*{트라우마의 심각도}
	
	트라우마의 심각도는 그 효과에 따라 다음과 같이 결정하면 됩니다:
	
	\smallskip
	
	\begin{tabularx}{\textwidth}{c!{\color{black}\vrule}X}
		\hline
		\textbf{심각도} &
			\textbf{기준}\\ \hline \hline
		
		[기피], [집착]  &
		\makecell[l]{
			\begin{tabularx}{\linewidth}{X}
				제약사항은 없으나 이야기꾼의 서사를 풍부하게 하기 위한 장치\footnote{단, 이를 너무 남발할 경우 개연성이 과도하게 흘러넘칠 수 있으므로 한 이야기꾼당 한개 정도로 제한하는 것을 추천합니다.}\\\hline
				비전투시의 행동 제약이며, 주의를 조금만 기울여도 발동되지 않을 수 있는 효과이거나, 발동되더라도 큰 제약이 되지 않는 효과
			\end{tabularx}
		}\\\hline
		
		[공포], [중독]  &
		\makecell[l]{
			\begin{tabularx}{\linewidth}{X}
				비전투시의 행동 제약이며, 주의 여부에 상관없이 발동될 수 있는 효과이며, 발동되었을 때 이야기꾼의 사고나 행동을 뒤바꿀 수 있는 효과\\\hline
				전투시의 행동 제약이며, 주의를 조금만 기울여도 발동되지 않을 수 있는 효과이거나, 발동되더라도 큰 제약이 되지 않는 효과
			\end{tabularx}
		}\\\hline
		
		[광기]          &
		\makecell[l]{
			\begin{tabularx}{\linewidth}{X}
				비전투시의 행동 제약이며, 주의 여부에 상관없이 발동될 수 있는 효과이며, 발동되었을 때 이야기꾼 본인 뿐 아니라 주변인의 사고와 행동 등에도 영향을 끼치는 효과\\\hline
				전투시의 행동 제약이며, 주의 여부에 상관없이 발동될 수 있는 효과이며, 발동되었을 때 이야기꾼의 사고나 행동을 뒤바꿀 수 있는 효과
			\end{tabularx}
		}\\\hline
	\end{tabularx}
	
	\smallskip
	
	두 가지 이상에 해당한다면 더 낮은 쪽의 심각도를 따릅니다. 가령, 전투, 비전투를 가리지 않고 주의를 기울인다면 발동되지 않을 수 있는 트라우마의 경우 [기피] 또는 [집착] 단계의 심화도를 가진 트라우마가 됩니다.
	
\end{document}
	
	\hypertarget{intrusion}{}
	\chapter{이야기의 [침범] 판정}
		\documentclass{report}

\begin{document}
	\world{이야기꾼이 자신의 출신 서사가 아닌 서사에 들어갔을 때, 여러분은 서사의 외부자로서 서사의 흐름을 방해하거나 오염시켜서는 안됩니다. 서사가 오염되었다는 것은 서사 자체가 사라졌다는 것이 아니라, 그 서사의 간섭받지 않은 결말, 즉 [종장]이 희석되어 알아보기 힘들게 되었다는 것을 의미합니다. 이렇게 되면 그 서사의 본질이 바뀌게 되는 것이죠. 이를 막기 위해 [참견]의 제약으로, 너무 서사에 큰 손상이 우려될 때에는 이야기꾼을 서사에서 추방하게 됩니다.}
	\world{여러분은 서사 속에서 저를 도와 서사의 흐름을 되돌리는 과정에서 그 서사 속에서의 죽음을 맞이하게 될 수도 있습니다. 하지만, 이는 그 서사 속에서 그 역할으로서의 당신이 죽었을 뿐이라는 것을 명심하세요. 새로운 역할을 부여받아 다시 서사 속으로 들어가는것은 당연히 가능합니다.}
	
	서사를 오염시키는 행동인지를 판단하기 위하여, 시스템은 행동에 따라 [침범] 판정을 하도록 강요합니다. 서사의 [운명]이나 지식을 거스르는 행동, 부여받은 이야기의 역할에 맞지 않는 행동, 서사에 어울리지 않는 능력을 사용하는 행동 등이 이에 포함됩니다.
	
	이야기꾼들이 가지고 있는 이야기들은 각각 개연성 코스트가 있습니다. 해당 이야기의 [침범도]는 개연성 코스트의 1/10(올림)으로 정의합니다.
	
	이야기꾼들에게는 각각 최대 개연성이 있습니다. 이야기꾼의 [저항도]는 최대 개연성의 1/10(버림)으로 정의합니다.
	
	\bigskip
	
	[침범] 판정은 판정을 할 때 마다 기본 난이도가 점점 증가합니다. [침범] 판정의 기본 난이도는 보통 최초에 -1으로 시작하여, 이야기꾼의 수만큼 판정을 할 때 마다 1씩 증가합니다. 다르게 생각하려면, 한 번 판정시마다 (1/이야기꾼)만큼 난이도가 상승한다고 생각해도 됩니다. 만약 누군가가 이야기를 이미 [침범]한 상태였다면, 시작할 때의 침범 판정의 난이도는 0보다 높을 수 있습니다. 이 기본 난이도에 침범 판정을 일으킨 이야기의 침범도를 더한 것이 침범 판정의 난이도가 됩니다.
	
	[침범] 판정을 할 때에는 4df를 굴려 자신의 저항도를 더합니다. 이 값이 난이도보다 높으면 성공하고, 같거나 낮으면 실패합니다. [침범] 판정의 결과와는 상관 없이, 행동은 완료할 수 있습니다.
	
	[침범] 판정이 성공한다면 행동이 정상적으로 행해집니다. 하지만 [침범] 판정이 실패한다면, 자신의 개연성을 해당 이야기의 개연성 코스트만큼 감합니다. 또는, 서사에 따라 독립적인 실패의 대가가 따를 수 있습니다. 가령 다음 이야기와 같은 내용으로 페널티가 대체될 수 있습니다:
	
	\begin{story}{시간의 끝}{[시간]}
		\entry{이 서사는 이미 시간의 끝을 향해 달려가고 있기 때문에 서사의 개연성을 해칠 염려가 매우 적다. 따라서, [침범] 판정에 실패할 때, 개연성 판정의 난이도가 초기화되고 다음 씬(비전투) 또는 두 턴 후(전투)까지 해당 판정에 실패한 이야기가 전면 봉쇄되어, 기술 뿐 아니라 해당 이야기로부터의 도움도 받을 수 없다(단, 스탯은 유지된다). 이야기가 불안정해짐에 따라 판정이 일어날 때 마다 난이도가 1 상승한다.}
	\end{story}
	
	어떠한 방법으로든 개연성이 0이 된다면 서사에서 추방됩니다. 다음이 순서대로 진행됩니다:
	\begin{itemize}
		\item 개연성이 0이 된 이야기꾼이 서사에서 추방됩니다. 해당 씬은 그대로 진행됩니다.
		\item 앞으로 등장인물이 아닌 모든 이야기꾼은 모든 이야기를 사용하거나 도움을 받는데에 있어, 반드시 [침범] 판정을 해야 합니다.
		\item 해당 씬이 종료되면 자동으로 모든 이야기꾼이 서사에서 추방됩니다.
	\end{itemize}
	이야기꾼 vs 타락한 자의 경우, 시스템은 여러분이 돕고 있다는 사실을 알기에, 이 행위들을 적대 행위로서 간주하지는 않습니다. 시스템은 여러분의 권한을 다시 복구시켜 줄 것이며, 임무를 계속할 수 있습니다. 이로 인해 타락한 자가 추방되더라도 같이 추방되었기 때문에 일괄적으로 복구되어, 서사를 계속 어지럽히는데에 일조합니다.
	
	[침범] 판정의 난이도는 모든 이야기꾼이 추방된 이후 다시 초기화됩니다.
	
	\section*{세계관 출신 인물의 이야기에 의한 침범 판정}
	이야기꾼들은 자신의 출신 서사로 돌아가게 될 수도 있습니다. 이야기의 등장인물이 먼치킨스러운 능력을 가진 경우도 많습니다. 세계관과 연관 있는 이야기 또는 인물에 의한 [침범] 판정에 대한 세 가지 중요한 사실은 다음과 같습니다:
	\begin{enumerate}
		\item {}[깨달은 자]가 아닌 등장인물에 의해서는 [침범] 판정이 일어나지 않습니다.
		\item 그 세계 속에서 얻은 이야기 또는 능력에 의해서는 [침범] 판정이 일어나지 않습니다.
		\begin{itemize}
			\item 이는 누군가 개연성이 0이 되어 추방되었을 때의 강제 판정에도 포함됩니다.
		\end{itemize}
		\item 해당 세계관의 출신인 [깨달은 자]는 개연성이 0이 되어도 서사에서 추방되지 않습니다.
		\begin{itemize}
			\item 추방되는 대신, 행동불능 상태가 되며 치료를 받기 전까지는 회복되지 않습니다.
			\item 이로 인해 이야기꾼이 사망하는 경우도 생길 수 있습니다.
		\end{itemize}
	\end{enumerate}
	
	\ifprintout\else
	\section*{\hypertarget{the-story-continues}{(선택 규칙)계속되는 서사}}
	한 서사의 세계 속에서 여러 서사가 진행된다면, 서사가 끝난 이후에, [침범] 판정의 난이도를 확인하여 다음 서사에는 해당 난이도에서 [침범] 판정의 난이도가 시작하게 할 수 있습니다.
	
	서플리먼트의 \storyref{probability:the-story-continues}{계속되는 서사}와 같은 이야기로 구현할 수 있습니다.
	\fi

\end{document}
	
	\hypertarget{growth}{}
	\chapter{이야기꾼의 성장과 죽음}
		\documentclass{report}

\begin{document}
	이야기꾼은 항상 변화하고, 성장합니다. 서사 속에서 죽음을 맞기도 하고, 서사 속에서 새로운 이야기를 얻으며 성장하기도 합니다.
	
	\section*{죽음}
	이야기꾼의 개연성이 0이 되면 해당 서사에서 죽음을 맞이하여, 추방됩니다. 해당 시점에서 다음이 모두 일어납니다:
	\begin{itemize}
		\item 죽은 원인 또는 상황에 대한 이야기 또는 서술을 얻습니다. 또는, 받은 부상 중 가장 큰 부상이나 가장 잦게 받은 부상에 대한 것이어도 괜찮습니다.
		\item 서사의 역할으로 인해 얻은 이야기를 모두 잃습니다.
		\item 받은 모든 임시 상태를 잃습니다.
		\item {}[태초의 이야기]에서 다시 나타납니다. 해당 씬이 끝난 뒤에, 재진입이 가능합니다.
	\end{itemize}
	
	\section*{미미한 성장}
	이야기꾼이 전투를 겪었거나, 이야기의 변화를 겪었다면 일어납니다. 길잡이꾼의 허가 하에 최대 개연성을 소모하고 방금 있었던 전투나 변화에 어울리는 이야기나 서술 하나를 얻거나, 이미 존재하는 이야기나 서술 하나를 바꿀 수 있습니다. 미미한 성장으로 서술을 변화시키거나, 이야기를 잃을 수는 없습니다.
	
	\section*{작은 성장}
	서사 하나가 끝날 때 마다 이야기에서 얻는 보상과는 별개로 다음 중 하나를 할 수 있습니다:
	\begin{itemize}
		\item 최대 개연성을 소모하고, 이야기를 하나 얻습니다.
		\item 최대 개연성을 소모하고, 이미 있는 이야기에 서술을 추가합니다.
		\item 서술 하나를 다른 서술로 바꾸거나, 이야기 하나를 다른 이야기로 바꿉니다.
	\end{itemize}
	
	\section*{중간 성장}
	시스템의 인정을 받는다면(보통 서사 두세개가 끝날때마다 한번) 다음을 모두 할 수 있습니다:
	\begin{itemize}
		\item 최대 개연성 2를 얻습니다.
		\item 작은 성장을 합니다.\footnote{\label{lite-medium-upgrade-small-upgrade}오타 아닙니다. 작은 성장의 선택지 중 한 가지를 선택해서 적용시키는 것을 총 두번 할 수 있습니다.}
		\item 작은 성장을 합니다.\footnoteref{lite-medium-upgrade-small-upgrade}
	\end{itemize}
	
	\section*{큰 성장}
	시스템의 위기를 타파할 때 마다(보통 서사 대여섯개정도가 끝날때마다 한번) 다음을 모두 할 수 있습니다:
	\begin{itemize}
		\item 최대 개연성 2를 얻습니다.\footnote{\label{lite-big-upgrade-cost}즉, 기본 최대 개연성 4를 얻습니다.}
		\item 중간 성장을 합니다.\footnoteref{lite-big-upgrade-cost}
		\item 작은 성장을 합니다.
	\end{itemize}
	
	\section*{서사의 보상}
	이야기꾼들은 서사 속에서 보상으로 이야기를 받을 수 있습니다. 단, 이 보상으로 인해서도 최대 개연성은 변화합니다. 따라서, 보상 이야기는 중립적인 서술 또는 개연성에 크게 변화를 가하지 않는 이야기로 하는 것으로 추천드립니다.
\end{document}
	
	\hypertarget{ability-guidelines}{}
	\chapter{능력 가이드라인}
		\documentclass{report}

\begin{document}
	이야기에 따른 능력은 다양한 종류가 있을 수 있습니다.
	캐릭터의 서사에 따라서, 강력한 능력은 마나와 같은 자원을 소모할수도, 사용 대기 시간이 존재할수도, 시전 시간이 존재할수도, 사용 횟수 제한이 존재할수도, 까다로운 사용 조건이 존재할수도 있습니다. 하지만 모두 공통적으로, 다른 서사에 들어가면서 능력이 어느 정도 약화됩니다.
	이 장에서는 능력이 할 수 있는 일을 제시하고, 그 효과에 대한 범위에 대해 생각합니다.
	
	능력은 크게 [효과형], [활용형]으로 구분됩니다. 이 구분은 절대적이지 않습니다.
	
	[효과형] 능력은 아군 또는 적군에게 직접적으로 효과를 부여하는 등의 효과를 가진 능력입니다. 크게 [공격형]과 [지원형]으로 구분됩니다.
	[공격형] 능력은 상대에게 피해를 주거나, 기절 등 행동에 직접적인 제약을 주는 부정적인 상태를 주는 능력을 의미합니다.
	[지원형] 능력은 아군의 피해를 막거나 치유하고, 아군에게 도움이 되는 상태를, 적에게는 직접적인 제약이 되지는 않으나 능력 사용을 약화시키는 능력을 의미합니다.
	[효과형] 능력은 여러가지 속성을 가질 수 있습니다. 이 속성들은 능력의 성능과 개연성 비용에 직접적인 영향을 끼칩니다.
	
	아래의 수치는 성능에 따른 대략적인 개연성 비용입니다\ifprintout\footnote{3.0.0 버전의 최초 릴리스를 기준으로 적혀있습니다. 업데이트된 비용이나 이 이외의 비용이 필요하다면 \href{https://git.io/fjphB}{룰북}을 참고하시거나, 없다면 마음대로 판단하셔도 괜찮습니다. 정 모르시겠다면 제작자에게 문의해주셔도 괜찮습니다.}\fi:
	%개연성 비용	 밸런스 패치
	\begin{itemize}
		\item 피해
		\begin{itemize}
			\item \textbf{직접 피해(Direct Damage)}: 피해량 1이 증가할때마다 개연성 비용 3이 추가됩니다.
			\item \textbf{지속 피해(Damage over Time)}: 턴당 피해량이 1 증가할때마다 개연성 비용 2가 추가됩니다. 지속시간 1턴마다 개연성 비용 2가 추가됩니다. 단, 최대 줄 수 있는 총 피해량의 세 배보다 이 코스트가 적다면 최대로 줄 수 있는 총 피해량의 세 배의 코스트를 가집니다.
			\item \textbf{본인 피해(Self-inflicted Damage)}: 본인에게 가하는 피해량 1이 증가할때마다 개연성 비용 2가 감소합니다.
		\end{itemize}
		
		\item 회복/방지
		\begin{itemize}
			\item \textbf{피해의 방지(Prevention)}: 최대로 방지할 수 있는 피해량 1당 개연성 비용 2가 추가됩니다.
			\item \textbf{피해의 회복(Healing)}: 최대로 치유할 수 있는 피해량 1당 개연성 비용 4가 추가됩니다.
			\item \textbf{지속 회복(Healing over time)}: 턴당 회복량이 1 증가할때마다 개연성 비용 2가 추가됩니다. 지속시간 1턴마다 개연성 비용 2가 추가됩니다. 5턴 이상은 추가되지 않습니다.
		\end{itemize}
		
		\item 사거리
		\begin{itemize}
			\item 아래의 내용을 계산할 때, 구역의 배치를 한 변이 10m인 무한 격자형 구역으로 생각합니다.
			\item \textbf{사거리(Ranged)}: 최대 사거리까지 떨어진 구역 수당 개연성 비용 1이 증가합니다. 5 이상으로 증가하지 않습니다.
			\item \textbf{범위(Area of Effect)}: 범위에 포함되는 구역당 개연성 비용 2가 증가합니다. 단, 아군 피해(Friendly Fire) 혹은 적군 회복 등 의도치 않은 결과를 불러올 수 있는 능력은 이 효과를 받지 않습니다.
			\item \textbf{회피 가능성(Dodgeability)}
				\begin{itemize}
					\item \textbf{조준(Aim)}: 조준이 필요하다면 개연성 비용 4가 감소합니다.
					\item \textbf{경감 가능(Reducable)}: 피해의 경감이 가능하다면 개연성 비용 2가 감소합니다.
					\item \textbf{회피 가능(Dodgeable)}: 회피가 가능하다면 개연성 비용 4가 감소합니다.
					\item \textbf{대상 지정(Targeted)/회피 불가(Undodgeable)}: 조준이 불필요하며 회피가 불가능하다면 개연성 비용 3이 증가합니다.
				\end{itemize}
			\item \textbf{거리에 따른 피해량 감소(Damage Falloff)}: 한 구역당 감소되는 피해량이 1 증가할때마다 개연성 비용 1이 감소합니다.
		\end{itemize}
		
		\item 사용 빈도
		\begin{itemize}
			\item \textbf{자원(Resource)}: 자원을 사용하는 경우 개연성 비용 3이 감소합니다. 추가로, 자원이 자동으로 충전되지 않는 등 자원을 찾기 어렵다면 개연성 비용 2가 감소합니다.
			\item \textbf{횟수 제한(Limited Use)}: 개연성 비용 10이 감소합니다. 씬당 사용 가능 횟수 1회당 개연성 비용 1이 증가합니다. 씬당 10회 이상 사용할 수 있는 능력은 횟수제한이 없는 것으로 간주합니다.
			\item \textbf{조건(Condition)}: 조건의 달성 수준에 따라 개연성 비용 1\textasciitilde10이 감소합니다.
			\item \textbf{준비 시간(Cooldown)}: 기술의 준비 시간 1턴당 개연성 비용 1이 감소합니다.
			\item \textbf{충전(Charge time)}: 기술의 충전 시간 1턴당 개연성 비용 1이 감소합니다. 만약 충전 시간동안 다른 행동에 제약을 받는다면, 1씩이 추가로 감소합니다.
			\item \textbf{연속 사용(Consecutive Usage)}: 한 턴에 여러 번 사용 가능하다면, 기본 1회에서 추가되는 횟수당 개연성 비용 2가 추가됩니다.
			\item \textbf{중첩 사용과 다중 사용(Cumulative/Multiple Usage)}: 지속 효과형 능력에만 적용됩니다. 한 대상에게 중첩 사용이 가능하다면 개연성 비용 3이 추가됩니다. 여러 대상에게 동시에 다중 사용이 가능하다면 개연성 비용 3이 추가됩니다.
		\end{itemize}
		
		\item 효과
		\begin{itemize}
			\item \textbf{강제 이동/추가 이동(Forced/Additional Movement)}: 타인을 강제로 이동시킬 수 있는 최대 구역당 개연성 비용 2가 추가됩니다. 자신이 자발적으로 추가로 이동할 수 있다면 이와 같은 만큼의 개연성이 추가됩니다. 단, 능력을 이용함으로써 자신이 강제로 이동되는 경우 이로 인해 추가되는 개연성 비용이 반으로 감소합니다.
			\item \textbf{긍정적/부정적 상태(Status Effects)}: 상태를 줄 수 있는 경우 기본적으로 개연성 비용 10이 증가합니다. 아래 사항에 해당한다면 개연성 비용이 감소합니다(1 미만으로 줄어들지는 않습니다):
			\begin{itemize}
				\item 상태가 극복하기 쉬운 경우, 개연성 비용을 3 감합니다.
				\item 상태가 횟수 제한이 있는 경우, 5회에서 모자란 횟수당 1의 개연성 비용을 감합니다. 이 이상의 횟수인 경우 횟수 제한이 있는 것으로 취급하지 않습니다.
				\item 상태가 턴수 제한이 있는 경우, 4턴에서 모자란 턴수당 1의 개연성 비용을 감합니다. 이 이상의 턴수인 경우 턴수 제한이 있는 것으로 취급하지 않습니다.
			\end{itemize}
			\item \textbf{판정 재굴림(Reroll)}: 판정을 한 번 재굴림 할 수 있는 경우 개연성 비용 5가 추가됩니다. 이 중 높은 쪽을 따르는 경우 개연성 비용 3이 추가됩니다. 강제로 타인의 판정을 재굴림하는 경우 3이 추가됩니다. 한 판정에 대하여 두 번 이상 재굴림을 할 수 있는 경우 개연성 비용은 무제한으로 누적되어 증가합니다.
			\item \textbf{판정 보너스(Bonus)}: 판정에 +1을 줄 수 있는 경우 개연성 비용 3이 추가됩니다. 판정에 -1을 줄 수 있는 경우, 개연성 비용 2가 감해집니다.
			\item \textbf{판정 자동 성공/실패(Automatic Success/Failure)}: 자동 성공은 개연성 비용 10이 추가됩니다. 자동 실패는 개연성 비용 10이 감해집니다.
		\end{itemize}
		
		\item 트라우마
		\begin{itemize}
			\item {}[기피], [집착]급 트라우마 효과당 개연성 비용 10이 감소합니다.
			\item {}[공포], [중독]급 트라우마 효과당 개연성 비용 20이 감소합니다.
			\item {}[광기]급 트라우마 효과당 개연성 비용 30이 감소합니다.
		\end{itemize}
	\end{itemize}
	
	이런 능력의 효과를 결정할 때에, 최대 체력이나 최대 정신력 수치가 중요하게 작용할 것입니다. 기본적으로 양쪽 모두 10으로 시작하며, 최대 체력은 근력, 최대 정신력은 의지의 영향을 받습니다. 능력의 효과를 결정할 때, 평균적인 최대치를 10\textasciitilde20정도로 생각하시면 됩니다. 이 이상의 최대치를 가지고 있다면 탱커로 분류될 수 있을 것이고, 이 수치의 50\% 이상의 피해를 한번에 줄 수 있다면 궁극기에 해당하는 효과로 분류될 수 있을 것입니다.
	
	한 가지 능력은 여러가지 형태의 능력이 될 수 있습니다. 이러한 선택형 능력의 경우에는 모든 선택지가 한가지 테마로 연결되어야 한다는 제약이 존재하며, 개연성 비용의 계산이 다르게 적용됩니다. 이런 능력은 먼저 효과 각각에 대한 개연성 비용을 계산한 뒤, 최대로 선택 가능한 총 개연성 비용을 올림한 것이 비용이 됩니다.
	
	[활용형] 능력은 적을 공격하거나 아군을 지원하는 것 외의 다른 방향으로 활용할 수 있는 모든 능력을 의미합니다. 가령 순간이동이라던가, 공격에 사용하기는 어려우나 어딘가 쓸모있는 마법 물품 등이 여기에 포함됩니다. 이 능력들은 대부분의 경우 전투중에는 특정한 조건이 충족되지 않는다면 사용할 수 없습니다. [효과형] 능력을 직접적으로 보조하는 능력이라 할지라도 그 성능은 다양할 수 있으므로 그 성능을 통한 시스템의 판단에 따라 개연성 비용을 결정합니다. 기본적으로 추가 비용은 5로 계산하나, [효과형] 능력에 해당하는 사항을 가지고 있다면 그에 해당하는 만큼 비용이 변할 수 있으며, 세션 진행 중 또는 후 그 효용성에 따라 시스템의 판단에 따라 개연성 비용이 변할 수 있습니다.
	
	한 가지 이야기가 [효과형]과 [활용형] 능력을 동시에 가질 수도 있고, 여러 [활용형] 능력이 섞여있을 수도 있습니다. 한 이야기가 가질 수 있는 능력의 종류와 수, 그리고 이야기의 코스트에는 제한이 없으나 코스트는 각 능력의 코스트의 합으로 계산되며 코스트가 큰 이야기일수록 침범 판정에 불리해진다는 것을 기억하세요.
	
	\bigskip
	
	다음과 같은 능력을 가진 이야기를 생각해봅시다:
	\begin{story}{독으로 만들어진 신체}{[생애][기피]}
		\entry{자신에게 접촉한 유기체 생명체인 모든 대상에게 물리적 상태 [중독됨 □□□]을 준다. 상태 [중독됨]을 가진 대상은 매 턴이 시작될 때, 상태 한 칸을 소모하고 개연성(등장인물이라면 체력) 1을 소모한다.}
		
		\entry{이 이외의 대상에 접촉했을 때에는 독을 묻힐 수 있다. 다음 중 하나를 선택적으로 적용한다:
		\begin{itemize}
			\item 물체를 즉시 부식시킨다. 해당 물체에 내구도가 있는 경우 내구도를 감소시킬 수 있다.
			\item 자신이 만졌던 곳에 다른 대상이 접촉한 경우, 해당 대상에게 접촉한 것처럼 취급한다.
		\end{itemize}
		두 가지 선택지 중 반드시 한가지만을 적용시킬 수 있다.}
		
		\limitedtrauma{기피}{호의적인 타인에게 고의로 직간접적 신체 접촉을 할 수 없다.}
	\end{story}
	이 이야기의 경우, 기본 비용 10에 다음과 같이 개연성 비용을 계산할 수 있습니다:
	\begin{itemize}
		\item {}[중독됨]: 부정적 상태(Status Effect) 3회, 개연성 비용 +8
		\item 지속 피해: 3턴간 지속 피해 1\footnote{코스트 3턴*2+피해량 1*2=8}, 총 피해량 3\footnote{코스트 9} 개연성 비용 +9
		\item 선택형 능력: 양쪽 모두 [활용형]으로 각각 개연성 비용 +10, 따라서 개연성 비용 +10
		\item 기피증: 개연성 비용 -10
	\end{itemize}
	따라서 개연성 비용은 27, 침범도는 3\footnote{27/10을 올림한 값.}이 됩니다. 하지만 시스템의 판단에 따라 다음에서 개연성 비용이 변할 수도 있습니다:
	\begin{itemize}
		\item 부정적 상태의 부여가 지속 피해와 겹치므로 둘 중 하나의 개연성 비용 증가만을 적용시킨다.
		\item {}[효과형] 능력의 효과가 미미한 정도에 그치므로 개연성 비용을 감소시킨다.
	\end{itemize}
	
	\section*{이야기의 기본 비용}
	이야기의 기본 개연성 비용은 10입니다. 하지만, 특정 이야기들은 다른 이야기들과 연계되는 능력이 주가 되거나, 다른 이야기를 보조하는 능력이기도 하고, 또는 다른 이야기가 있어야만 사용할 수 있는 능력이기도 합니다. 이런 조건들을 \emph{필요 조건}이라고 합니다.
	
	이런 필요 조건이 있는 이야기들의 경우에는 기본 개연성 비용이 10에서 5로 줄어듭니다. 더불어, 어떤 이야기가 있어야만 사용할 수 있는 이야기의 경우, 기본 개연성 비용이 0으로 줄어듭니다. 필요 조건이 존재하더라도 기본 비용은 이 조건이 어떤 이야기에 기반해야만 변화하며, 이야기에 기반한다고 해서 단순히 어떤 부분을 공유하거나 사용할 수 있는 연계가 있기 때문에 비용이 변하지는 않습니다.
	
	\smallskip
	
	가령, 마나를 사용하는 마법사 이야기꾼을 만들기 위해 [마나 친화력]이라는 이야기로 마나를 사용할 수 있도록 했다고 가정하겠습니다. 이를 필요 조건으로 가지는, [마나 증폭기]라는 이름의 마나의 회복을 보조하는 이야기, [화염구]라는 마나를 사용하는 주문, 그리고 [마도사 자격증]이라는 이야기꾼의 생애 이야기를 생각해보겠습니다.
	
	[마나 증폭기]의 기본 개연성 비용은 [마나 친화력]과 연계되는 능력이 주가 되며 그를 보조하기 때문에, 5가 됩니다.
	
	[화염구]는 [마나 친화력] 없이는 사용할 수 없는 이야기이기 때문에, 기본 개연성 비용이 0이 됩니다.
	
	하지만[마도사 자격증] 이야기의 경우, 마나를 사용하거나 [마나 친화력]을 보조하는 능력이 아니기 때문에 이 이야기에 대한 기본 개연성 비용은 10이 될 것입니다.
	
	\section*{사용 후 버려지는 아이템}
	일정 횟수 또는 기간 사용 후 버려지거나 해당 서사에서 더 이상 사용할 수 없는 아이템의 경우라 할지라도, 개연성 비용은 동일하게 계산됩니다.
	
	하지만, 이 아이템을 더 이상 사용할 수 없게 되는 경우에\footnote{이는 아이템을 빼앗기거나 잃어버린 경우를 포함합니다.} 이야기꾼은 개연성의 부담을 경감받을 수 있습니다. 아이템이 모두 소모된 바로 다음 씬이 시작하기 전에, 해당 아이템의 개연성 코스트만큼의 최대 개연성과 개연성을 회복하지만, 해당 아이템의 스탯 증감을 포함한 모든 효과를 더 이상 사용할 수 없습니다.
	
	\section*{개연성 비용에 대한 가장 중요한 이야기}
	중요한 것은, 이 계산이 귀찮다면, 시스템과 상의하여 개연성 비용을 정하는 식으로 진행해도 된다는 것입니다. 이야기꾼의 세계의 규칙은 이야기 내적으로든 규칙 자체로든 의도적으로 수정이 쉽도록 만들어져 있기에, 시스템과 이야기꾼들의 판단이 가장 중요한 영향을 끼친다는 것을 명심하세요.
	
	\hypertarget{reward}{}
	\section*{서사의 보상}
	이야기꾼들이 서사를 헤쳐나가게 되면 등장인물이나 다른 깨달은 자, 또는 시스템에 의해 보상을 받을 수 있습니다. 이 보상은, 당연하게도, 이야기의 형태로 지급됩니다. 이 이야기들의 코스트는 기본적으로 0으로 취급됩니다. 보상으로 지급되는 이야기의 내용은 자유롭게 지급할 수 있지만, 많은 서사를 경험한 이야기꾼과 그렇지 못한 이야기꾼이 함께 존재하는 경우가 있을 수 있기 때문에 그 차이를 줄이기 위해 다음 중 하나 이상이 적용되도록 하는 것을 권장합니다:
	\begin{itemize}
		\item 개연성 코스트가 정상적으로 계산되어 적용된다.
		\item 능력이 없고 내용만 있어 도움만 받을 수 있는 이야기이다.
		\item 특정 조건이 만족되면 이야기를 잃게 된다. 예를 들어:
		\begin{itemize}
			\item 일정 기간동안만 적용된다.
			\item 해당 이야기의 신념에 반하는 행동을 할 경우 이야기를 잃는다.
		\end{itemize}
		\item 능력이 있다면, 발동 조건 등이 까다롭다. 예를 들어:
		\begin{itemize}
			\item 재사용 대기시간이 길다.
			\item 사용시마다 강제로 침범판정을 행해야 한다.
			\item 특히 아이템의 경우, 발동 횟수 제한이 있다.
		\end{itemize}
	\end{itemize}
	만약 이 이야기로 인한 스탯의 변동이 있다면 이는 반드시 개연성 코스트에 영향을 끼치게 됩니다. 또한 트라우마류의 효과가 있다면 이 역시 개연성 코스트에 영향을 끼칩니다.
	
	물론, 이 서사를 얻은 세계로 다시 들어가게 된다면 위 제약이 사라지거나 약화될 수 있다는 것을 기억하세요.
	

\end{document}
	
	\chapter{스탯 가이드라인}
		\documentclass{report}

\begin{document}
	이야기꾼의 세계의 스탯은 여러가지로 나누어져 있습니다. 이미 계산했을 개연성을 제외하고도, 다양한 스탯이 있습니다. 이야기들은 연관성이 있다면 이 스탯을 높이는 데에 일조합니다. 물론, 스탯을 높이는 데에 일조한 이야기들은 개연성 코스트를 가지게 됩니다.
	스탯은 특화가 가능한 경우가 있습니다. 이 경우, 해당 스탯에 관련되어 있는 세부 분야 하나의 스탯을 올리기로 결정할 수 있습니다. 세부 분야의 경우 해당 분야에 관련있는 판정을 할 때의 판정치는 낮아지나 해당하지 않는 경우 판정이 불가능할 수 있습니다.
	
	스탯에는 다음이 있습니다:
	
	\smallskip
	
	\begin{minipage}{\textwidth}
		\begin{tabularx}{\textwidth}{c!{\color{black}\vrule}X}
			\hline
			\textbf{스탯} & \textbf{설명}\\ \hline \hline
			도발          & 상대를 언어, 행동으로 평정심을 잃게 만듦\\\hline
			인식          & 상황 변화에 따른 인지력          \\\hline
			기민          & 특정 상황에 대한 신속한 반응       \\\hline
			근력          & 순간적인 강력한 힘, 지속적인 근지구력  \\\hline
			은신          & 상대가 자신을 인지하기 어렵게 만듦    \\\hline
			기만          & 지성체를 거짓이나 과장, 은닉을 이용하여 속임 \\\hline
			공감          & 상대방의 감정, 의견에 대한 포용과 인정 \\\hline
			의지          & 물리적, 정신적 충격을 견딜 수 있는 마음이나 믿음\\\hline
		\end{tabularx}
		
		\smallskip
		
		\begin{tightcenter}
			\textbf{특화 불가능한 스탯}
		\end{tightcenter}
	
	\medskip
	
		\begin{tabularx}{\textwidth}{c!{\color{black}\vrule}X!{\color{black}\vrule}l}
			\hline
			\textbf{스탯} & \multicolumn{1}{c!{\color{black}\vrule}}{\textbf{설명}} & \multicolumn{1}{c}{\textbf{특화 예시}} \\ \hline \hline
			제작          & 특정한 물건, 프로그램 등의 제작, 설계          & 컴퓨터 바이러스, 예술품, 총          \\\hline
			운전          & 탑승물에 탑승하고 조종할 수 있는 능력           & 자동차, 말, 우주선         \\\hline
			전투          & 물리적 충격이 존재하는 모든 형태의 근접전         & 단도, 맨손, 개머리판         \\\hline
			사격          & 물리적 충격이 존재하는 모든 형태의 중장거리 전투     & 활, 총기, 저격총         \\\hline
			지식          & 특화된 내용에 대한 다양한 정보               & 상식, 종교, 역사, 과학, 오컬트         \\\hline
			인맥          & 지성체 사이의 관계 형성, 대화와 타협           & 과학자, 정치인, 부랑자         \\\hline
			자본          & 돈, 자산 등 금전을 획득하거나 사용할 수 있는 능력   & 예술품, 금, 현찰  \\\hline
		\end{tabularx}
		
		\smallskip
		
		\begin{tightcenter}
			\textbf{특화 가능한 스탯}
		\end{tightcenter}
	\end{minipage}
	
	\bigskip
	
	필요하다면 이야기에 따라 스탯을 추가, 제거, 변형하거나, 특화 스탯을 비특화로, 비특화 스탯을 특화로 변형할 수 있습니다.
	
	\bigskip
	
	예를 들어, [지식:역사] 스탯을 올린 경우, 어떤 종교에 관한 지식을 알고 있는지에 대한 판정에서 해당 종교의 역사에 대한 지식을 얻을 수는 있습니다. 이 경우 [지식] 스탯을 사용한 판정에서 사용될 판정치보다는 [지식:역사]에서 사용될 판정치가 낮을 것입니다. 하지만 [지식:과학]에 해당할 쿼크의 종류에는 무엇이 있고, 그들이 각각 어떤 식으로 상호작용하는지를 알 수는 없습니다.
	
	\bigskip
	
	스탯을 직관적으로 생각하자면, 평균적인 이들이 가진 능력은 스탯 0으로 생각할 수 있습니다. 각 기준은 스탯에 따라 다를 것인데, 예를 들어 [제작:총기]의 경우 스탯 0인 이들은 시도조차 할 수 없을 것이지만, [지식:상식]의 경우 스탯 0인 이들은 운이 좋다면 어디선가 얻어들었을 가능성이 있는 것이죠. 따라서 "전문가"라고 불리는 수준에 도달하기 위해 필요한 스탯 역시 다를 것입니다. [지식:상식]의 경우에는 1\textasciitilde2 정도만 있어도 전문가로 불릴 수 있을 것이나, [제작:총기]의 경우 1\textasciitilde2로는 총기의 원리를 이해하는 정도이고 4\textasciitilde5 정도가 되어야 간단한 총기를 제작할 수 있게 될 것입니다.
	
	\bigskip
	
	어떤 이야기가 스탯 하나를 높이는 데에 일조한다면 개연성 비용이 5 증가합니다. 스탯 하나를 낮추는 데에 일조한다면 개연성 비용이 5 감소합니다.
	단, 특화 분야의 스탯을 올리거나 내리는 데에 일조한다면 개연성 비용이 5가 아닌 2 증가하고 감소합니다.
	
	\bigskip
	
	스탯의 수치에는 최대/최소 제한이 없습니다. 하지만, 개연성 비용에 의한 제한을 받기 때문에 너무 많은 스탯을 올리기는 힘들다는 점을 기억하세요. 또한 등장인물의 경우, 아직 자신이 가진 이야기를 자각하지 못한 경우가 대부분이므로 이야기와 관계 없이 스탯을 설정받을 수 있다는 점 역시 기억해두세요.
	

\end{document}
	
	\hypertarget{power-limit}{}
	\chapter{권능과 제약}
		\documentclass{report}

\begin{document}
	이야기꾼에게 세 가지 권능과 두 가지 제약이 존재하듯이, 시스템에게도 세 가지 권능이 존재합니다. 이 챕터에서는 이 권능과 제약들에 대해 자세하게 설명하도록 하겠습니다.
	
	\bigskip
	
	먼저, 이야기꾼의 권능과 제약에 대해 살펴보도록 하겠습니다:
	
	\smallskip
	
	\begin{minipage}{\textwidth}
		\begin{tabularx}{\textwidth}{c!{\color{black}\vrule}c!{\color{black}\vrule}X}
			\hline
			\textbf{구분} & \textbf{이야기} & \makecell{\centering\textbf{설명}} \\ \hline \hline
			[권능] & 소통\index{소통} & 자신의 출신 서사가 아니라면 모든 언어를 이해할 수 있다. \\ \hline
			[권능] & 접근\index{접근} & [태초의 이야기]에서 접근 좌표를 아는 서사로 이동하거나, 어떤 서사에서든 [태초의 이야기]로 이동할 수 있다. \\ \hline
			[권능] & 거래\index{거래} & 자신의 모든 이야기를 이야기의 규모에 비례하는 적당한 시간을 사용하여 다른 [깨달은 자]들에게 전할 수 있다. \\ \hline
			[제약] & 비밀\index{비밀} & [태초의 이야기]와 관련된 그 어떠한 사항도 [깨달은 자]가 아닌 경우 발설할 수 없다. 발설한다면, [잊혀진 자]가 된다. \\ \hline
			[제약] & 참견\index{참견} & [침범] 판정을 실패하여 서사를 오염시키면, 서사에서 추방되고, 해당 서사에 대한 권능 [접근]을 빼앗긴다. \\\hline
		\end{tabularx}
		
		\smallskip
		
		\begin{tightcenter}
			\textbf{이야기꾼의 권능과 제약}
		\end{tightcenter}
	\end{minipage}
	
	\bigskip
	
	\emph{소통의 권능}은 이야기꾼들의 편의를 위해 지급되는 권능입니다. [태초의 이야기]에서 다른 이야기꾼들과 소통을 할 수 있도록 하기 위함도 있지만, 서사 속으로 들어갔을 때에 등장인물들과 이야기를 할 수 있도록 해주는 역할 역시 가지고 있습니다. 물론, 서사 속으로 들어가며 역할이 부여되거나 하는 등으로 인해 이해하지 못하는 언어가 존재할 수도 있다는 점은 염두에 두어야 합니다.
	
	\smallskip
	
	\emph{접근의 권능}은 이야기꾼의 세계의 크로스오버를 가능하게 만들어주는 핵심 이야기이자 [잊혀진 자]와 그렇지 않은 이야기꾼을 구분하는 핵심 권능입니다. 접근의 권능이 없는 이는 기본적으로 다른 서사로 들어갈 수 없습니다. 하지만 다른 이와의 계약을 통해 임시로 서사로의 접근 권한을 얻을 수는 있습니다. 예를 들어, 다음과 같은 이야기를 계약을 통해 얻었다면 서사 속으로 들어갈 수 있을 것입니다:
	
	\begin{story}{잭 더 리퍼의 가호}{[계약]}
		\entry{[접근]의 권능 없이도 서사 속으로 들어갈 수 있다. 단, 들어간 서사 속에서 나오기 위해서는 잭 더 리퍼가 지정하는 등장인물들을 누구에게도 들키지 않고 죽여야만 한다.}
	\end{story}
	
	이 경우, 잭 더 리퍼 본인이 [잊혀진 자]라 할 지라도 많은 서사 속에서 잭 더 리퍼의 잔혹한 연쇄살인의 이야기가 유지됨으로서 잊혀지지 않고 [태초의 이야기] 속에서 자신의 존재를 드러낼 수 있을 것입니다.
	
	\smallskip
	
	\emph{거래의 권능}은 이야기꾼들이 자신이 가진 이야기를 다른 이들에게 이야기하여 그들에게 자신의 이야기를 알릴 수 있는 방법입니다. 물론 모든 이야기를 알릴 수 있는 것도 아니고, 이야기를 들은 이야기꾼은 자신의 생각대로 이야기를 해석하기 때문에 모든 이들에게 그대로 전해질 수 있는 것도 아니며, 이야기를 한번 전해줬다고 해서 그 이야기를 이야기꾼이 영원히 기억할 수 있는 것도 아닙니다. 하지만 이야기를 전해들은 이야기꾼은 그 이야기의 힘을 빌릴 수 있는 상황이 된다면 이야기의 힘을 빌릴 수 있게 될 것입니다. 이런 이야기를 배울 수 있는 것은 이야기꾼 뿐 아니라, 등장인물들로부터도 얻을 수 있을 것입니다.
	
	예를 들어, 헤라클레스가 \textbf{[네메아의 사자를 죽인 자]}의 이야기를 다른 이야기꾼에게 전해줄 때, 이 이야기를 듣고 죽어가면서도 끝까지 헤라클레스에게 대항한 사자의 용맹함에 감동을 받은 이야기꾼이라면 다음과 같은 이야기를 얻을 수 있을 것입니다:
	
	\begin{story}{죽음을 불사한 용맹}{[획득:거래]}
		\flavour{헤라클레스에게서 \textbf{[네메아의 사자를 죽인 자]}의 이야기를 [거래]의 권능으로 획득함}
		
		\entry{전투에서 도주한다면 이 이야기를 잃는다.}
	\end{story}
	
	이렇게 거래의 권능으로 획득한 이야기는 \hyperlink{reward}{서사의 보상}으로 받은 것과 동일하게 취급합니다. 특히, 세 번째 제약인 "특정 조건 만족시 이야기를 잃는다"는 조건을 반드시 적용시키는 것을 권장드립니다. 만약 능력이 있는 이야기를 거래의 권능으로 획득한 경우, 일반적인 경우 능력을 획득할 수 없을 것이나, 교훈을 통해 자신이 깨달은 해당 이야기의 능력이 추가되는 경우, 원래 능력과는 상관 없는 완전히 다른 새로운 능력이 추가되거나, 원래 능력이 추가되는 경우 간접 경험일 뿐이기 때문에 약화된 능력만을 얻을 확률이 높을 것입니다.
	
	\smallskip
	
	\emph{비밀의 제약}과 \emph{참견의 제약}은 서사의 오염을 막고 그들을 보호하기 위한 제약들입니다. 비밀의 제약을 통해 너무 많은 이들이 [깨달은 자]가 되지 않도록 하고, 참견의 제약을 통해 서사의 오염을 직접적으로 막는 것이 이 제약들의 목적입니다.
	
	\bigskip
	
	시스템에게 주어진 권능은 크게 세가지로 구분됩니다. 이 권능은 여러분이 시스템으로서 이야기꾼들을 이끌어나가는 것을 정당화해주는 역할을 합니다.
	
	\smallskip
	
	\begin{minipage}{\textwidth}
		\begin{tabularx}{\textwidth}{c!{\color{black}\vrule}c!{\color{black}\vrule}X}
			\hline
			\textbf{구분} & \textbf{이야기} & \makecell{\centering\textbf{설명}} \\ \hline \hline
			[권능:시스템] & 계산\index{계산} & 서사의 과거 흐름을 바탕으로 서사의 흐름을 계산한다. \\ \hline
			[권능:시스템] & 접근\index{접근} & 서사에 접근하여 등장인물과 상호작용 할 수 있다. \\ \hline
			[권능:시스템] & 추출\index{추출} & 깨달은 자들을 [태초의 이야기]로 추출할 수 있다. \\ \hline
		\end{tabularx}
		
		\smallskip
		
		\begin{tightcenter}
			\textbf{시스템의 권능}
		\end{tightcenter}
	\end{minipage}
	
	\bigskip
	
	\emph{계산의 권능}은 시스템이 서사의 흐름을 계산할 수 있도록 하는 장치입니다. 이는 여러분이 시스템으로서 이야기꾼들을 이끌어나갈 때 여러분이 서사를 이끌어나갈 수 있게 하는 장치입니다.
	
	\smallskip
	
	\emph{접근의 권능}은 시스템이 서사 속의 등장인물에게 서사 속의 존재로서 나타나 등장인물과 상호작용할 수 있도록 해줍니다. 이는 이야기꾼들이 서사를 진행하다 어떻게 할 지 몰라 서사의 흐름이 막힌 경우나 서사가 너무 쉽게 흘러가는 경우, 시스템이 dei ex machina\footnote{deus/dea ex machina의 복수형.}적으로 직접적인 도움이나 시련을 줄 수 있도록 하는 장치입니다. 물론 좋은 서사는 이런 접근이 거의 없이도 해결될 수 있어야 할 것입니다.
	
	\smallskip
	
	\emph{추출의 권능}은 [깨달은 자]들을 이야기꾼으로 추출해내는, 이야기꾼의 세계를 위해 반드시 필요한 권능입니다. 이 권능을 이용해 시스템은 [태초의 이야기]로 깨달은 자들을 데려올 수 있습니다.

\end{document}
	
	\hypertarget{author}{}
	\chapter{직업군: 작가}
		\documentclass{report}

\begin{document}
	이야기꾼의 세계에서 작가는 특별한 위치에 있습니다. 작가는 자신만의 서사를 가지고 있는 존재로, 굳이 소설을 쓰거나 하지 않았더라도 자신이 창작한 세계가 있다면 이에 해당합니다.
	작가는 태초의 이야기에 들어왔을 때, 자신의 서사에 대한 세계의 존재를 알게 됩니다. 이는 이야기로서 존재하며, 그 세계에 대한 서술을 포함합니다.
	
	시스템 이상의 권한을 가질 수 있는 유일한 방법은 한 세계의 작가가 되는 것입니다. 작가는 자신이 만들어낸 세계에 한정하여 시스템 이상의 권능을 가집니다. 서사의 미래를 계산하는 것 뿐 아니라 방향, 설정을 자신이 주무를 수 있게 되는거죠. 자신이 창조한 세계에 들어간 작가는 굳이 그 서사의 역할 이야기를 얻거나 할 필요도 없이 세계의 법칙이 허용하는 한 모든 것을 할 수 있습니다.
	
	하지만, 작가가 서사를 오염시키거나 파괴시켰을 때에는 조금 상황이 달라집니다. 서사를 만든 이는 작가더라도 그를 받아들이는 독자가 반드시 존재합니다. 독자들로 대표될 수 있는 비자명한 이들은 서사 세계의 주민들입니다. 그 독자들이 보았을 때(즉, 개연성상) 이 작가가 추가/수정한 설정이 서사를 너무 크게 바꾸면 작가가 서사를 오염시킨 것으로 취급하고, 서사의 진행 등이 너무 크게 바뀔 수 있으면 서사가 손상/파괴된 것으로 취급합니다.
	작가 본인에 의하여 오염되거나 손상/파괴된 서사는 작가의 권능을 거부합니다. 즉, 작가는 그 서사에 대한 통제 권한을 잃게 되는 것이죠. \ifDLC J.K.롤링의 경우가 이에 해당합니다.\fi
	
	다른 서사 속에서는, 자신의 서사의 “가호”를 받을 수 있습니다. 이 “가호”는 해당 이야기의 서술으로, 들어가 있는 서사의 법칙을 조금 구부릴 수도 있을 것입니다.

\end{document}
	
	\hypertarget{expand}{}
	\chapter{확장하기}
		\documentclass{report}

\begin{document}
	이야기꾼의 세계는 룰적인 수정을 가하여 새로운 룰으로 만들어지기 쉬운 구조로 만들어져 있습니다. 그렇기 때문에 이야기꾼의 세계를 다른 RPG 규칙의 요소를 이용해 확장할수도 있고, 다른 RPG 규칙을 이야기꾼의 세계를 이용해 확장할수도 있습니다.
	
	이에 대한 예시를 들기 위해서 \href{https://twitter.com/shinhogoesreal/}{신호}님이 만드신 룰인 \href{https://twitter.com/shinhogoesreal/status/1165797377980035073}{미션스쿨백합}을 사용하도록 하겠습니다.
	
	\section*{이야기꾼의 세계를 다른 규칙으로 확장하기}
	이야기꾼의 세계를 다른 규칙으로 확장하는 방법은 간단합니다: 이야기꾼들이 진입하는 이야기에 해당 규칙의 핵심에 해당하는 이야기를 [세계] 속성을 가진 이야기로 만들어 역할으로서 부여하거나 세계의 설명에 추가하는 것입니다.
	
	예를 들어 미션스쿨백합의 경우, 다음 네 가지 이야기를 추가하게 되면 간소화된 버전이긴 하나 미션스쿨백합의 규칙을 이용해 이야기꾼의 세계를 확장할 수 있을 것입니다:
	
	\bigskip
	
	\begin{story}{미션스쿨백합}{[세계]}
		\entry{이야기꾼들은 미션스쿨의 학생으로 이 세계에 들어오게 된다. 각 이야기꾼은 정신적 상태인 죄책감 0/10을 추가로 가진다. 죄책감은 10 이상으로 증가할 수 없으나, 10 이상으로 증가하게 된다면 대신 적절한 정신적 상태를 얻는다.}
		
		\entry{새로운 주요 등장인물(이야기꾼 포함)이 등장할 때 마다, 해당 인물에 대한 끌림 상태 2/10을 가진다.}
		
		\entry[\hline]{각 이야기꾼은 \storyref{mission-school:action}{동성간 행위}, \storyref{mission-school:prayer}{기도}, \storyref{mission-school:confession}{고해성사}의 세 가지 행위 이야기를 사용할 수 있다.}
	\end{story}
	
	\begin{story}[mission-school:action]{동성간 행위}{[행위]}
		\entry[\hline]{어떤 인물과 "동성간 행위"로 분류될 수 있는 행위를 하면 다음이 일어난다:
		
		\begin{enumerate}
			\item 양쪽 모두 죄책감이 1 증가한다.
			\item 행위를 받은 인물이 저항하기로 선택할 수 있다. 1d10을 굴려 끌림 미만의 값이 나오면 저항에 실패하여 아래 전투 판정으로 넘어간다. 저항에 성공했다면, 여기에서 해당 행위는 중지된다.
			\item 해당 행위를 시작한 인물이 사용한 스탯을 공격으로, 받은 인물의 의지를 수비로 하여 전투 판정을 한다.
			\item 공격 판정에 성공했다면, 해당 행위가 발생한다. 양쪽의 끌림이 자신의 죄책감의 1/3(소숫점 아래 버림) 증가한다. 단, 죄책감이 10이라면 끌림이 증가하지 않는다.
			\item 수비 판정에 성공했다면, 해당 행위를 거부한다. 행위를 시작한 인물의 죄책감이 1 추가로 증가한다.
		\end{enumerate}
		죄책감이 10인 경우 해당 행위를 할 수 없다.}
	\end{story}
	
	\begin{story}[mission-school:prayer]{기도}{[행위]}
		\pre{혼자 있을때, 또는 조용한 곳에 있을때, 또는 성당 등에 있을 때 할 수 있다.}
		
		\entry{죄책감이 1 감소한다.}
		
		\entry[\hline]{4df를 굴려 의지를 더한 값이 0 이상이라면 죄책감이 1 추가로 감소한다.}
	\end{story}
	
	\begin{story}[mission-school:confession]{고해성사}{[행위]}
		\pre{성당에서 4df를 굴려 2 이상의 수가 나오거나, 신부가 있음이 확실하다면 고해성사를 할 수 있다.}
		
		\entry{죄책감이 2 감소한다.}
		
		\entry[\hline]{4df를 굴려 의지를 더한 값이 1 이상이라면 고해성사를 받는 신부의 공감(최소 1)만큼의 죄책감이 추가로 감소한다. 단, 값이 -1 이하라면 고해성사를 받는 신부의 공감(최소 1)만큼의 죄책감을 얻는다.}
	\end{story}
	
	이렇게 간소화시킴으로서 룰에 있는 복잡한 주사위를 간소화할 수 있을 뿐 아니라, "죄책감"과 "끌림"을 상태로서 만들었기 때문에 이를 판정 등에 활용하는 등의 행위가 가능해졌습니다. 물론 완벽하게 똑같은 규칙을 사용하지는 않았지만 이를 사용함으로서 기본 규칙으로 이야기꾼의 세계를 사용하면서 미션스쿨백합의 핵심 규칙을 함께 즐길 수 있을 것입니다.
	
	\ifDLC
	\medskip
	
	또 다른 예시로 Call of Cthulhu의 경우에는 다음 이야기를 역할로 부여할 수 있을 것입니다:
	\begin{story}{점점 미쳐가는 이야기}{[세계][설화][광기]}
		\entry{이성치 100을 얻는다. 일반적으로 이해할 수 없는 일(살인, 시체, 고대의 존재 등)을 목격할때마다, 그 수준에 따라 난이도와 피해량(성공/실패, 다이스 사용 가능)을 시스템이 지정한다. 그러면 1d(현재 이성치)를 굴려, 해당 수치가 난이도 이상이라면 이성치에 성공 피해량을, 미만이라면 실패 피해량을 받는다. (e.g. 60[1d4/2d8]의 경우 주사위의 결과가 60 이상이 나오면 성공하며, 성공시 1d4 정신력, 실패시 2d8 정신력을 잃는다.)}
		
		\entry[\hline]{현재 이성치의 10\% 이상에 달하는 피해를 한번에 받은 이야기꾼은 해당 상황에 대한 이 시나리오에서의 [기피]를, 20\% 이상인 경우 이 시나리오에서의 [공포]를 얻는다. 25\% 이상인 경우, [공포]를 얻는 것은 같으나 이는 시나리오에서 벗어나도 유지된다. 이를 방지하기 위해 이성치에 받는 피해의 전부 또는 일부를 1대1의 비율로 정신력에 받을 수 있다.}
	\end{story}
	\fi
	
	\section*{다른 규칙을 이야기꾼의 세계로 확장하기}
	이야기꾼의 세계의 규칙을 이용해 다른 규칙을 확장할수도 있을 것입니다. 이야기꾼의 세계를 이용하려는 이유로 생각되는 내용 각각에 대해서 어떻게 적용시킬 수 있을지 알아보도록 하겠습니다.
	
	\subsection*{크로스오버 가능성}
	크로스오버를 할 수 있는 가장 간단한 방법으로는 세계관만을 (일부) 차용해 사용하고, 나머지는 원 규칙을 사용하는 방법일 것입니다. "침범"에 관련된 규칙은 이야기꾼의 세계를 따라도 되지만, 이야기꾼들이 원래 모두 같은 세계 출신이고 해당 세계의 과거나 미래로 가는 것이라면 "침범"을 무시하고 진행해도 무방합니다. 그러나 같은 세계의 다른 시간이 아닌, 다른 세계로 가는 것이라면 "침범"을 적용시키는 것을 권장합니다. 이 경우 "침범" 판정의 성공치를 0으로 고정해두고 하는 것 역시 가능하고, "침범" 판정의 실패 페널티를 다른 것으로 고칠수도 있습니다.
	
	가령 중세 시대에 마나를 이용해 마법을 사용했으나, 현대로 오면서 기술이 발전하고 환경이 오염되며 마나의 농도가 낮아져 마법을 사용하기 힘든 세계의 경우, 다음과 같은 이야기로 침범 페널티를 정할 수도 있습니다:
	
	\begin{story}{낮은 농도의 마나}{[시간:미래]}
		\entry[\hline]{이 세계의 미래는 환경의 오염과 함께 마나의 농도가 매우 낮아져 있다. 이야기꾼이 살던 세계였기 때문에 [침범] 판정으로 인해 추방되지는 않을 것이나, 주문을 사용 할 때에는 정상적으로 [침범] 판정을 해야만 한다. 이 경우 성공치는 0으로 고정되나, [침범] 판정에 실패하면 해당 주문의 시전은 취소된다. 마나 등 자원을 소모하는 주문이었다면 자원들은 소모된다.}
	\end{story}
	
	\subsection*{능력 디자인의 자유성}
	능력 디자인을 편리하게 할 수 있도록 하기 위해, 이야기꾼의 세계에서는 능력이 주어졌을 때 개연성 코스트를 계산하는 방법이 주어져 있습니다. 이를 다른 규칙에 적용시켰을 때 그 규칙의 능력들의 개연성 코스트를 적용하는 방법으로 바꿀 수 있어야 합니다. 이렇게 계산하는 방법을 찾아낸 뒤 이를 사용하여 능력의 코스트의 상한선과 하한선을 두고 능력을 자유롭게 디자인할 수 있도록 할 수 있습니다.
	
	\subsection*{이야기꾼의 세계에서의 개연성 코스트와 확장 가능성}
	위 두 가지 상황에서는 공통적으로 크로스오버 가능성에서는 "침범" 판정을 위한 침범도와 저항도를, 능력 디자인의 자유성에서는 밸런스를 위한 개연성 코스트를 정해야 할 것입니다. 그렇기 때문에 이럴 때에는 GM의 판단에 따라 개연성 코스트를 결정해야 합니다. 그러기 위해서는 이야기꾼의 세계의 이야기에 비해 어떤 능력의 효과가 어느정도인지를 알아야 할 것입니다. 이야기꾼의 세계는 기본적으로 체력 10, 정신력 10, 개연성 80이 주어지는데 개연성의 대부분은 이야기와 능력을 얻고, 스탯을 높이는데에 사용될 것입니다. 이야기꾼의 세계의 이야기꾼의 실질 체력은 개연성 80을 기준으로 평균 약 20\textasciitilde30정도로 생각하여, 사용되는 규칙의 체력에 맞추어 개연성 코스트를 계산해도 됩니다. 물론 앞에서 얘기했듯, GM과 플레이어 간의 판단과 합의에 맞추어 이를 정하는 것 역시 가능합니다.

\end{document}
	
	\chapter{에필로그}
		\documentclass{report}

\begin{document}
	이야기의 생성자들은 앞선 두 규칙과는 다르게 시스템이나 길잡이꾼 등 이야기를 한 명이 이끌어나가지 않고, 모든 이야기꾼이 함께 흐름을 통해 이야기를 이끌어나가는 형식으로 되어 있습니다. 이렇기 때문에 하나의 서사, 같은 배경 설정이라 할지라도, 특히 \hyperlink{nogm-socialite}{이야기의 사교도들}을 적용시킨다면 완전히 다른 흐름을 즐길 수 있을 것입니다.
	
	세번째로, 룰 읽느라 수고하셨습니다. 여러분에게 이야기의 가호가 있기를 바라겠습니다.
\end{document}
\end{document}
	
	\part{이야기의 방랑자들 \\ Wanderers of the Tales}
		\documentclass{report}

\begin{document}
	\parttoc
	
	\chapter{서문}
	\documentclass{report}

\begin{document}
	이야기의 방랑자들은 RPG\footnote{Role Playing Game. TRPG/TTRPG(TableTop RPG), ORPG(Online RPG)로도 알려져 있다.}입니다. 여러명이 모여서 캐릭터를 만들고, 그들이 세계와 서로 상호작용을 하며 이야기를 만들어나가는 게임이죠. 이야기꾼의 세계에서는 다른 게임에서 GM\footnote{Game Master. DM(Dungeon Master)이라고도 부른다.}이라고 부르는 "이야기를 이끌어나가는 이"를 \emph{길잡이꾼[Guide]}라고 부릅니다. 플레이어들이 조종하는 PC\footnote{Player Character}라고 불리는 캐릭터들은 \emph{이야기꾼[Storytellers]}이라고 부릅니다. 여러분은 길잡이꾼이 되어 서사의 틀을 잡고 이끌어 나가거나, 이야기꾼이 되어 그 서사 속에서 여러분의 캐릭터들이 행동하도록 할 것입니다.
	
	이야기의 방랑자들은 이야기꾼의 세계에 기반하여, 보다 더 가볍게 즐길 수 있는 룰을 만드는 것을 목표로 합니다. 다음 이야기꾼의 세계 챕터들의 내용은 이야기의 방랑자와 많은 부분을 공유하므로, 이 챕터들을 먼저 읽으시는 것을 추천드립니다:
	\begin{itemize}
		\item \hyperlink{story-progression}{서사의 진행}
		\item \hyperlink{power-limit}{권능과 제약}
		\item \hyperlink{author}{직업군: 작가}
	\end{itemize}
\end{document}
	
	\chapter{이야기꾼 만들기}
	\documentclass{report}

\begin{document}
	\hypertarget{ability-limit}{}
	\section*{이야기꾼의 권능과 제약}
	이야기꾼들은 다음 권능과 제약을 가지게 됩니다\footnote{이야기꾼의 세계에서의 것과 동일합니다.}.
	
	\smallskip
	
	\begin{minipage}{\textwidth}
		\begin{tabularx}{\textwidth}{c!{\color{black}\vrule}c!{\color{black}\vrule}X}
			\hline
			\textbf{구분} & \textbf{이야기} & \makecell{\centering\textbf{서술}} \\ \hline \hline
			권능 & 소통\index{소통} & 자신의 출신 서사가 아니라면 모든 언어를 이해할 수 있다. \\ \hline
			권능 & 접근\index{접근} & [태초의 이야기]에서 접근 좌표를 아는 서사로 이동하거나, 어떤 서사에서든 [태초의 이야기]로 이동할 수 있다. \\ \hline
			권능 & 거래\index{거래} & 자신의 모든 이야기를 이야기의 규모에 비례하는 적당한 시간을 사용하여 다른 [깨달은 자]들에게 전할 수 있다. \\ \hline
			제약 & 비밀\index{비밀} & [태초의 이야기]와 관련된 그 어떠한 사항도 [깨달은 자]가 아닌 경우 발설할 수 없다. 발설한다면, [잊혀진 자]가 된다. \\ \hline
			제약 & 참견\index{참견} & [침범] 판정을 실패하여 서사를 오염시키면, 서사에서 추방되고, 해당 서사에 대한 권능 [접근]을 빼앗긴다. \\\hline
		\end{tabularx}
		
		\smallskip
		
		\begin{tightcenter}
			\textbf{이야기꾼의 권능과 제약}
		\end{tightcenter}
	\end{minipage}
	
	\section*{이야기꾼의 이야기와 개연성}
	이야기꾼의 이야기는 이야기꾼이 살아온 생애, 경험한 모든 경험, 알고 있는 지식 등을 정의해주는 내용들입니다. 이야기꾼들은 최대 개연성 15를 가지고 시작하며, 다음으로 최대 개연성을 소모하여 이야기를 얻을 수 있습니다:
	\begin{enumerate}
		\item 새로운 이야기를 얻는다.
			\begin{itemize}
				\item 새로운 이야기를 얻습니다. 이야기는 보통 짧은 구절으로 표현합니다.
				
				\item 이야기를 얻는 데에는 항상 최대 개연성 1을 소모합니다.
			\end{itemize}
		
		\item 가진 이야기에 [서술]을 추가한다.
			\begin{itemize}
				\item 그 이야기로 할 수 있는 일을 한 문장정도로 간단하게 서술하고, 이를 [긍정적] [중립적] [부정적]으로 분류합니다.
					\begin{itemize}
						\item{} [긍정적] 서술은 이야기꾼에게 많은 경우에 도움이 되는 효과를 가져다주는 서술입니다. 이 서술을 얻는 데에는 최대 개연성 1을 소모합니다.
						\item{} [부정적] 서술은 이야기꾼에게 많은 경우에 해가 되는 효과를 가져다주는 서술입니다. 이 서술을 얻으면, 최대 개연성 1을 얻습니다\footnote{이야기꾼의 세계에서, 트라우마가 이에 해당합니다.}.
						\item{} [중립적] 서술은 [긍정적]이면서 [부정적]인 서술, 또는 [긍정적]이지도 [부정적]이지도 않은 서술을 의미합니다. 이 서술을 얻는 데에는 최대 개연성을 소모하거나 얻지 않습니다.
					\end{itemize}
			\end{itemize}
	\end{enumerate}
	이렇게 얻은 이야기와 서술으로 변한 최대 개연성만큼의 개연성을 가지고 서사에 들어가게 됩니다.
	
	이야기와 그 서술들은 항상 해당하는 이야기꾼에게는 사실입니다. 예를 들어, 다음과 같은 이야기를 생각해보겠습니다:
	
	\begin{lite}{엄청난 독서광}
		\positive{수많은 책을 읽어오며, 읽는 속도가 빠르다.}
		
		\neutral{머릿속에 들어 있는 지식을 말하고 싶어 입이 근질거린다.}
		
		\negative{책만 읽어오며 운동신경이 떨어졌다.}
	\end{lite}
	
	[엄청난 독서광]에서, 긍정적 서술은 자료가 주어져 있다면, 그를 분석하는 데에 걸리는 시간을 줄일 수 있는 도움이 되는 서술입니다. 중립적 서술은 아는 것이 많은 반면, 이를 말하고 싶어하는 잘난체하는 면을 보여주죠. 부정적 서술은 지식에 치중한 나머지 신체적인 면을 단련하지 못했음을 드러냅니다. 이처럼 서로 다른 서술은 한 이야기의 다른 면들을 보여줄 수 있습니다.
	
	\bigskip
	
	이야기꾼 시트와 예시 이야기꾼은 \hyperlink{lite-sheets}{캐릭터 시트} 챕터에서 확인하실 수 있습니다.
	
	\section*{등장인물과 이야기꾼}
	서사를 만들어나가기 위해 필요한 등장인물과 이야기꾼 역시 이와 같은 과정으로 만들어도 됩니다. 한 가지 중요한 점은 등장인물의 경우 개연성의 영향을 받지 않는다는 점입니다. 하지만 밸런스를 위해 적절한 최대 개연성 수치를 사용하는 것을 추천드립니다. 예를 들어 스토리의 최종 흑막 등 특수한 경우에는 최대 개연성 수치를 15가 아닌 20\textasciitilde30 정도로 하여 만들 수 있을 것입니다.

\end{document}
	
	\chapter{이야기 속에서의 역할과 이야기}
	\documentclass{report}

\begin{document}
	서사 속에 들어간 이야기꾼들은 서사 속의 역할에 맞추어 추가적인 이야기를 받을 수 있습니다. 이 새로운 역할은 기본적으로 최대 개연성에 영향을 끼치지 않고, 처음부터 주어지기도 하며, 나중에 필요에 의해 지급되거나 자신이 선택할 수도 있을 것입니다. 이 역할들은 간단하게 이야기만을 주기도 하지만, 새로운 능력이나 트라우마, 행동 제약 등을 제시할 수도 있습니다. 간단한 예시로, 사진 작가인 이야기꾼이 사진기가 발명된 직후의 세계의 이야기에에 카메라를 들고 들어간다면 카메라가 커다란 구형 카메라로 변한다거나 하는 페널티가 생길 수 있을 것입니다.
	
	\bigskip
	
	보다 구체적인 예시로, 이야기꾼이 생명을 지켜야 하는 신성한 직업인 사제가 된 경우, 다음 이야기들을 부여할 수 있습니다:
	
	\begin{lite}{깨어난 자}
		\positive{어떤 상황에서도 이성을 잃지 않는다.}
		
		\negative[\hline]{생명체에게 피해를 입힐 수 없다.}
	\end{lite}
	
	역할이 이야기를 부여하는 것 외에도, 서사 속에 들어가면 그 서사에 맞는 이야기들이 강제로 적용될 수 있습니다. 예를 들어 불안정한 공간 속에서 일어나는 서사의 경우, 다음 이야기를 얻게 될 수 있습니다:
	\begin{lite}{비정형의 공간}
		\negative[\hline]{균형을 잡기 힘들다.}
	\end{lite}
	
	이처럼 서사 속에서 역할을 부여받으면 그 역할에 맞는 이야기와 능력을 얻음으로서 해당 이야기의 세계를 보다 생생하게 체험하게 해 줄 수 있습니다.
\end{document}
	
	\chapter{판정과 이야기}
	\documentclass{report}

\begin{document}
	이야기의 방랑자들에서는 기본적으로 주사위, 카드 등을 사용하지 않고 판정합니다. 모든 판정은 \emph{판정자}와 \emph{방해자}간의 대결으로 구성됩니다. \emph{판정자}는 행동에 대한 판정을 하는 이를, \emph{방해자}는 그 행동에 대한 대응을 하는 이를 의미합니다. 판정자와 방해자는 전투의 상황에서는 각각 공격자와 방어자가, 이야기꾼이 행하는 일반적인 판정의 경우에는 각각 이야기꾼과 길잡이꾼이 판정자와 방해자의 역할을 맡게 됩니다.
	
	판정자와 방해자가 정해지면, 두 명은 동시에 1\textasciitilde10 중 한 숫자를 말합니다. 이는 두 명이 동시에 원하는 숫자의 손가락을 피는 등으로 행할 수 있고, 원한다면 d10이나 플레잉 카드 등을 사용하여 동시에 공개하는 것으로 할 수도 있습니다. 두 수를 합친 값이 10 이하라면 해당 숫자가 판정 결과값이 됩니다. 두 수를 합친 값이 11 이상이라면, 두 수에서 10을 뺀 값이 판정 결과값이 됩니다. 따라서, 판정치는 항상 1\textasciitilde10 중 한 숫자가 됩니다.
	
	보정치는 판정 결과값에 추가되는 값입니다. 이야기, 보다 정확하게는 이야기의 서술의 도움은 이 때에 받을수 있습니다\footnote{이야기의 제목 역시 하나의 서술로 취급할 수 있습니다.}. 해당 상황을 도와주는 서술 하나당 +1, 방해하는 서술 하나당 -1을 받습니다. 이 보정치에서 한 이야기는 한 개의 서술만으로만 도움을 줄 수 있지만, 한 이야기로부터 방해받을 수 있는 횟수는 제한되지 않습니다. 이 서술의 도움이나 방해는 거리나 상황이 허용한다면, 다른 이야기꾼으로부터 받을 수도 있습니다.
	
	서술의 도움뿐 아니라, 판정의 결과 합이 11 이상인 경우에는 추가 보정치가 주어집니다. 판정자가 방해자보다 작은 수를 제시했을 경우 판정에 +1을, 방해자가 판정자보다 작은 수를 제시했을 경우 판정에 -1을 가합니다. 같은 수를 제시했을 경우에 이 보정치는 주어지지 않습니다.
	
	판정치에 이런 모든 보정치를 더한 것이 최종 결과치가 되어, 이 수치를 이용해 판정을 진행하게 됩니다.
	
	\section*{판정치와 목표치}
	보정 없이 판정으로 얻을 수 있는 최대치인 ``10"은 해당 서사의 평균적인 등장인물이 낼 수 있는 극단적인 최대 위력을 의미합니다. 10 이상의 결과치는 매우 운이 좋거나 어떤 도움이 없다면 나올 수 없는 결과치입니다. 일반적인 경우, 7 이상의 결과치는 운이 좋지 않다면 나오기도 힘들 것입니다. 아래는 목표치에 대한 간략한 설명입니다:
	
	\begin{minipage}{\textwidth}
		\begin{tabularx}{\textwidth}{c!{\color{black}\vrule}X}
			\hline
			\textbf{판정치} & \makecell{\centering\textbf{난이도}} \\ \hline \hline
			0 이하 & 아무 힘도 들이지 않고도 할 수 있습니다. \\ \hline
			1 \textasciitilde 2 & 조금만 집중해도 할 수 있습니다. \\ \hline
			3 \textasciitilde 4 & 큰 힘을 들이지 않고도 가능합니다. \\ \hline
			5 \textasciitilde 6 & 힘은 들지만, 할 수는 있습니다. \\ \hline
			7 \textasciitilde 8 & 엄청난 힘이 듭니다. \\ \hline
			9 \textasciitilde 10 & 죽을힘을 다해야 가능합니다. \\ \hline
			10 이상 & 일반적으로 불가능합니다. \\ \hline
		\end{tabularx}
		
		\smallskip
		
		\begin{tightcenter}
			\textbf{목표치에 따른 행위의 난이도}
		\end{tightcenter}
	\end{minipage}
	
	\section*{임시 상태}
	임시 상태는 서사를 진행하는 중에 받을 수 있는 서술입니다.
	
	임시 상태는 판정에서 크게 성공하거나 실패했을때, 후술할 대결을 하면서, 또는 다른 여러 이유들으로 얻을 수 있습니다. 서술과 같은 방법으로 사용할 수 있으며, 판정에 보정치로 +1 또는 -1을 부여할 수 있지만, 빠르게는 한번 사용한 후, 아무리 늦어도 해당 씬이 종료될 때 사라집니다.
	
	\section*{(선택 규칙) ``0"의 판정}
	판정을 1\textasciitilde10으로 하는 대신, 0\textasciitilde10 또는 0\textasciitilde9로, 0을 포함시키는 방법을 생각해볼 수 있습니다.
	
	0을 냈다고 하더라도, 판정값의 결정은 동일하게 진행됩니다. 즉, 0을 낸 이는 해당 판정에 유의미한 기여를 할 수 없습니다. 하지만, 0을 낸 이는 판정이 종료된 이후, 판정자에게 해당 판정 중 일어난 일에 연관있는 유의미한 상태를 부여할 수 있습니다. 판정자가 냈다면 긍정적으로 사용할 수 있는 상태를, 방해자가 냈다면 부정적으로 사용할 수 있는 상태를 부여할 수 있습니다.
	
	예외적으로, 판정자와 방해자가 둘 다 모두 0을 냈다면, 양쪽이 서로 부여하는 상태는 상쇄되어 사라집니다. 하지만, 판정자는 해당 판정의 판정치를 10으로 취급합니다. 방해자가 완전히 등을 돌린 틈을 타 자신의 목적을 완수하는 것이죠.
	
	``0"을 판정치로 내면, 상대의 수치에 완전히 판정을 의존하게 됩니다. 하지만 이를 허용하게 되면 메타적인 측면에서는 심리전에 흥미로운 추가 규칙을 부여하고, 전략적인 포기 이후 판정에 사용할 수 있는 유의미한 상태를 받게 된다는 전략적인 면이 추가됩니다.
	
	\section*{(선택 규칙) 추가 보정치}
	한 이야기나 한 서술이 보정치를 +1까지만 가할 수 있다는 규칙을 무시할 수 있습니다. 특히, 크게 도움을 줄 수 있는 서술은 보정치를 그 이상으로 받거나, 보정치 대신 임시 상태를 받는 식으로 처리할 수 있습니다. 이 규칙은 이야기꾼의 세계의 \hyperlink{emersion}{재현}과 유사한 규칙입니다.
	
	\section*{부상}
	부상은 서사를 진행하는 중, 특히 전투를 하는 중에 받을 수 있는 서술입니다.
	
	부상은 심각도에 따라 작은 부상, 중간 부상, 큰 부상으로 구분됩니다. 이를 받으면 해당 심각도에 따라 개연성을 소진하게 됩니다. 아래 표를 참고하세요:
	
	\begin{minipage}{\textwidth}
		\begin{tabularx}{\textwidth}{c!{\color{black}\vrule}c!{\color{black}\vrule}X}
			\hline
			\textbf{부상의 종류} & \textbf{개연성 소진량} & \makecell{\centering\textbf{회복 시기}} \\ \hline \hline
			작은 부상 & 1 & 씬이 끝나면 회복됩니다. \\ \hline
			중간 부상 & 2 & 씬이 끝나면 작은 부상으로 경감되며, 서사에서 나오면 회복됩니다. \\ \hline
			큰 부상 & 3 & 씬이 끝나면 중간 부상으로 경감되며, 서사에서 나오면 반드시 부정적 서술로서 해당 부상을 받아야 합니다. \\ \hline
		\end{tabularx}
		
		\smallskip
		
		\begin{tightcenter}
			\textbf{부상}
		\end{tightcenter}
	\end{minipage}
	
	개연성이 0이 되면 서사 속에서 죽음을 맞게 되고, 이야기꾼은 서사에서 추방되게 됩니다.
	
	부상은 부정적인 서술과 같이 방해자가 역이용하여 방해하는데에 사용할 수 있습니다.
	
	\section*{대결}
	대결은 두 캐릭터 이상 사이에 발생한 갈등 상황을 의미합니다. 이는 전투를 포함합니다. 대결을 할 때의 순서는 판정 없이 해당 상황을 빠르게 타개할 수 있는 데에 도움이 되는 서술의 수로 결정합니다. 이 때에 한하여 한 이야기당 한 개의 서술만을 사용할 수 있다는 제한이 없습니다.
	
	대결을 할 때에 다른 인물과의 판정을 할 때에는, 공격과 방어에 대한 판정을 두 번 하는 대신, 한 번의 판정으로 피해량과 결과 등을 계산합니다. 공격을 하는 이가 판정자, 수비를 하는 이가 방해자가 되어 판정을 진행한 후, 해당 수치에서 5를 뺀 수치에 따라 대결의 결과가 정해집니다\footnote{대결의 목표치는 항상 5로 설정되어 있다고 생각해도 됩니다.}.
	
	대결의 결과 나온 수치가 0 미만, 즉 판정값이 4 이하라면, 방해자는 성공적으로 판정자의 행동을 방해한 것으로 취급합니다. 공격을 당한 경우 성공적으로 회피한 것입니다. 숫자가 작을수록 더 성공적으로 방해하거나 회피한 것으로, 절대값을 취한 값의 반에 해당하는 수치만큼 다음 판정에 추가 보정치를 받습니다. 예를 들어 결과가 -3이라면, 소숫점 아래를 버린 +1의 보정치를 다음 판정에 받습니다. 이 보정치는 턴이 다시 돌아오기 전까지 쌓은 보정치 중 가장 높은 보정치 하나만 적용됩니다.
	
	대결의 결과 나온 수치가 0 초과, 즉 판정값이 6 이상이라면, 판정자가 방해에도 불과하고 성공적으로 행동을 이행한 것입니다. 숫자가 클수록 더 성공적으로 수행한 것으로, 해당 값을 사용하여 다음 행동 중 전부 또는 일부를 선택할 수 있습니다. 단, 중복 선택은 불가능합니다:
	
	\begin{minipage}{\textwidth}
		\begin{tabularx}{\textwidth}{c!{\color{black}\vrule}X}
			\hline
			\textbf{사용할 판정치의 값} & \makecell{\centering\textbf{할 수 있는 일}} \\ \hline \hline
			1 & 본인에게 해당 행동과 관련된 상태를 부여합니다. \\ \hline
			1 & 대상에게 해당 행동과 관련된 상태를 부여합니다. \\ \hline
			1 & 본인에게서 해당 행동과 관련된 상태를 제거합니다. \\ \hline
			1 & 대상에게서 해당 행동과 관련된 상태를 제거합니다. \\ \hline
			2 & 대상에게 작은 부상을 입힙니다.\\ \hline
			4 & 대상에게 중간 부상을 입힙니다. \\ \hline
			6 & 대상에게 큰 부상을 입힙니다. \\ \hline
		\end{tabularx}
		
		\smallskip
		
		\begin{tightcenter}
			\textbf{대결시 판정값에 따라 할 수 있는 일}
		\end{tightcenter}
	\end{minipage}
	
	대결의 결과 나온 수치가 0, 즉 판정값이 5라면, 판정자에게는 아래 중 하나를 선택합니다:
	\begin{enumerate}
		\item 대결의 결과가 나온 값을 1으로 취급하여 판정에 성공하는 대신, 상대 역시 위의 표에서 1에 해당하는 만큼의 효과를 적용시킬 수 있습니다.
		\item 또는, 판정에 실패한 것으로 취급합니다.
	\end{enumerate}
	
	\section*{부상의 회복}
	부상의 자연 회복이 아닌, 이야기꾼의 능력이나 치료 등으로 부상을 회복할 때 역시 판정을 하게 됩니다. 회복시키는 것을 시도하는 이야기꾼이 판정자, 길잡이꾼이 방해자가 되어 대결을 합니다. 해당 결과에 따라, 다음과 같이 부상을 회복할 수 있습니다:
	
	\begin{minipage}{\textwidth}
		\begin{tabularx}{\textwidth}{c!{\color{black}\vrule}X}
			\hline
			\textbf{사용할 판정치의 값} & \makecell{\centering\textbf{할 수 있는 일}} \\ \hline \hline
			1 & 대상에게서 해당 치료와 관련된 상태를 제거합니다. \\ \hline
			2 & 대상에게서 작은 부상을 치료하여 제거합니다.\\ \hline
			4 & 대상에게서 중간 부상을 치료하여 제거합니다. \\ \hline
			6 & 대상에게서 큰 부상을 치료하여 제거합니다. \\ \hline
		\end{tabularx}
		
		\smallskip
		
		\begin{tightcenter}
			\textbf{치료시 판정값에 따라 할 수 있는 일}
		\end{tightcenter}
	\end{minipage}
	
	음수가 나오게 되면, 치료 방법에 따라 상태 또는 부상이 심화되게 할 수도 있습니다.
	
	
	\ifprintout
	\section*{(선택 규칙) 발악\footnote{이 이야기를 제안해주신 소낙님께 감사드립니다.}}
	
	이야기가 늘어질 때 아래 규칙을 사용할 수 있습니다.
	
	\begin{lite}{모 아니면 도}
		
		\entry{대결을 하기 전, 직전의 대결에서 패배했다면 [발악]을 하기로 선택할 수 있다. 이 경우, 해당 패배한 대결을 포함하여 몇번이나 연속으로 패배했는지 그 수를 센다.}
		
		\entry{[발악]을 한 대결에서 승리한다면, 위에서 센 수만큼의 추가 보정치를 받는다. 이를 임시 상태 또는 부상을 주는 데에 사용할 수 있다.}
		
		\entry{[발악]을 한 대결에서 패배한다면, 상대는 위에서 센 수만큼의 추가 보정치를 받는다. 이를 임시 상태 또는 부상을 주는 데에 사용할 수 있다.}
	\end{lite}
	\fi
	
	\section*{판정의 예시}
	이야기꾼 A가 장애물을 뛰어넘는 상황을 생각해봅시다. 이 경우 A가 판정자, 길잡이꾼(B)이 방해자가 됩니다.
	
	A가 7, B가 2를 제시했다면 판정값은 9가 됩니다.
	
	A가 9, B가 2를 제시했다면 판정값은 1이 되며, 여기에 방해자가 더 작은 수를 제시했기 때문에 -1의 보정치가 추가됩니다.
	
	A가 9, B가 10을 제시했다면 판정값은 9가 되며, 여기에 판정자가 더 작은 수를 제시했기 때문에 +1의 보정치가 추가됩니다.
	
	A가 8, B가 8을 제시했다면 판정값은 6이 되며, 두 수가 같으므로 추가 보정치는 없습니다.
	
	A가 다음 이야기를 가지고 있다고 가정해보겠습니다:
	\begin{lite}[runner]{단거리 달리기 선수}
		\positive{짧은 시간동안 빠른 속도로 이동할 수 있다.}
	\end{lite}
	
	\storyref{runner}{단거리 달리기 선수}의 서술으로 도움닫기를 한다면 +1 보정치를 받을 수 있을 것입니다. 만약 A가 ``발목을 삐끗함"과 같은 부상을 가지고 있다면, -1 보정치가 가해질 것이고요.
	
	다른 이야기꾼 C가 다음과 같은 이야기를 가지고 있다고 생각해봅시다:
	\begin{lite}[magic-music]{음악의 마술사}
		\positive{[가속의 음악]을 통해 자신과 주위 이야기꾼의 속도를 빠르게 할 수 있다.}
		
		\negative{[진정의 음악]을 통해 자신과 주위 이야기꾼의 속도를 느리게 할 수 있다.}
	\end{lite}
	
	\storyref{magic-music}{음악의 마술사}의 [가속의 음악]에 대한 서술을 사용한다면 B는 해당 판정에 +1 보정치를 가할 수 있을 것입니다. 만약 반대로, B가 [진정의 음악]에 대한 서술을 사용한다면 B는 해당 판정에 -1 보정치를 가할 수 있습니다.
\end{document}
	
	\chapter{이야기꾼의 성장과 죽음}
	\documentclass{report}

\begin{document}
	이야기꾼은 항상 변화하고, 성장합니다. 서사 속에서 죽음을 맞기도 하고, 서사 속에서 새로운 이야기를 얻으며 성장하기도 합니다.
	
	\section*{죽음}
	이야기꾼의 개연성이 0이 되면 해당 서사에서 죽음을 맞이하여, 추방됩니다. 해당 시점에서 다음이 모두 일어납니다:
	\begin{itemize}
		\item 죽은 원인 또는 상황에 대한 이야기 또는 서술을 얻습니다. 또는, 받은 부상 중 가장 큰 부상이나 가장 잦게 받은 부상에 대한 것이어도 괜찮습니다.
		\item 서사의 역할으로 인해 얻은 이야기를 모두 잃습니다.
		\item 받은 모든 임시 상태를 잃습니다.
		\item {}[태초의 이야기]에서 다시 나타납니다. 해당 씬이 끝난 뒤에, 재진입이 가능합니다.
	\end{itemize}
	
	\section*{미미한 성장}
	이야기꾼이 전투를 겪었거나, 이야기의 변화를 겪었다면 일어납니다. 길잡이꾼의 허가 하에 최대 개연성을 소모하고 방금 있었던 전투나 변화에 어울리는 이야기나 서술 하나를 얻거나, 이미 존재하는 이야기나 서술 하나를 바꿀 수 있습니다. 미미한 성장으로 서술을 변화시키거나, 이야기를 잃을 수는 없습니다.
	
	\section*{작은 성장}
	서사 하나가 끝날 때 마다 이야기에서 얻는 보상과는 별개로 다음 중 하나를 할 수 있습니다:
	\begin{itemize}
		\item 최대 개연성을 소모하고, 이야기를 하나 얻습니다.
		\item 최대 개연성을 소모하고, 이미 있는 이야기에 서술을 추가합니다.
		\item 서술 하나를 다른 서술로 바꾸거나, 이야기 하나를 다른 이야기로 바꿉니다.
	\end{itemize}
	
	\section*{중간 성장}
	시스템의 인정을 받는다면(보통 서사 두세개가 끝날때마다 한번) 다음을 모두 할 수 있습니다:
	\begin{itemize}
		\item 최대 개연성 2를 얻습니다.
		\item 작은 성장을 합니다.\footnote{\label{lite-medium-upgrade-small-upgrade}오타 아닙니다. 작은 성장의 선택지 중 한 가지를 선택해서 적용시키는 것을 총 두번 할 수 있습니다.}
		\item 작은 성장을 합니다.\footnoteref{lite-medium-upgrade-small-upgrade}
	\end{itemize}
	
	\section*{큰 성장}
	시스템의 위기를 타파할 때 마다(보통 서사 대여섯개정도가 끝날때마다 한번) 다음을 모두 할 수 있습니다:
	\begin{itemize}
		\item 최대 개연성 2를 얻습니다.\footnote{\label{lite-big-upgrade-cost}즉, 기본 최대 개연성 4를 얻습니다.}
		\item 중간 성장을 합니다.\footnoteref{lite-big-upgrade-cost}
		\item 작은 성장을 합니다.
	\end{itemize}
	
	\section*{서사의 보상}
	이야기꾼들은 서사 속에서 보상으로 이야기를 받을 수 있습니다. 단, 이 보상으로 인해서도 최대 개연성은 변화합니다. 따라서, 보상 이야기는 중립적인 서술 또는 개연성에 크게 변화를 가하지 않는 이야기로 하는 것으로 추천드립니다.
\end{document}
	
	\hypertarget{wandering-storytellers}{}
	\chapter{방랑하는 이야기꾼들}
	\documentclass{report}

\begin{document}
	이 챕터에서는 이야기의 방랑자들을 단순화하는 방법과 더불어, 이야기꾼의 세계와 이야기의 방랑자들간 이야기를 바꾸는 방법을 간략하게 설명합니다. 이야기꾼의 세계의 \hyperlink{expand}{확장하기} 챕터와 함께 사용하면, 다른 룰으로, 또는 다른 룰을 방랑하는 이야기꾼들을 이용하여 확장하는데에 큰 무리는 없을것입니다.
	
	\section*{이야기꾼의 세계에서 이야기의 방랑자들으로}
	이야기꾼의 세계에서 이야기의 방랑자들으로 이야기를 바꾸는 방법은 단순합니다. 이야기의 능력에서 모든 수치를 제거하고, 단순화하여 각각을 서술로 바꾸어 긍정적/부정적/중립적으로 분류하는 것입니다. 물론 개연성에 따라 조금 변화를 가해야 할 수도 있겠지만요. 시나리오의 경우 특히, \href{https://github.com/n0n3x1573n7/WoS-Scenarios}{시나리오집}에 수록된 시나리오처럼 이야기의 효과가 정확하게 지정되어 있다면 바꾸는 것이 크게 어렵지는 않을 것입니다.
	
	\section*{이야기의 방랑자들에서 이야기꾼의 세계로}
	이야기의 방랑자들에서 이야기꾼의 세계로 변환하는 것은 단순할수도, 복잡할수도 있습니다. 가장 단순한 방법은, 각 서술을 하나의 능력으로서 취급하는 것입니다. 이렇게 서술한 것을 굳이 수치화 하지 않은 상태에서 일반적인 능력으로서 사용하는 것도 가능하지만, 보다 더 구체적으로 서술들에 어울리는 수치를 부여하여 보다 더 구체화된 능력을 구현할 수도 있을 것입니다.
	
	\hypertarget{walking-storytellers}{}
	\section*{활보하는 이야기꾼들}
	이야기의 방랑자들은 이미 꽤나 단순한 형태의 규칙입니다. 하지만 확실한 이야기꾼의 테마를 잡고 있는 상태에서, 개연성 수치만을 기억하면서 플레이할 수 있도록 더욱 단순화 시킬 수도 있습니다.
	
	이렇게 하기 위해서는, 최초에 가지고 시작하는 개연성이 15가 아닌 30이 됩니다. 이 상태에서 서사를 진행해 나갑니다.
	
	이야기나 서술의 도움을 얻을 때에는, ``이야기꾼의 어떤 면이 이 장면에서 도움이 될 것이다"하는 것을 길잡이꾼에게 전달합니다. 길잡이꾼은 이를 듣고, 타당하다고 생각한다면 개연성의 소모 없이 도움을 받을 수 있습니다. 단, 애매하다고 생각한다면 개연성을 잃고 도움을 얻을 수 있다는 선택지를 제시합니다. 이 선택지를 받아들이는지 여부는 이야기꾼에게 달려 있습니다.
	
	이야기꾼은 임시 상태를 얻을 수 없고, 부상을 입을 때 부상 서술을 얻지 않고 직접 개연성을 잃습니다.
	
	이렇게 함으로서 이야기꾼의 테마와 개연성 수치만을 알고 있는 상태에서도 이야기의 방랑자들의 플레이가 가능하게 만들 수 있습니다.
\end{document}
	
	\chapter{에필로그}
	\documentclass{report}

\begin{document}
	이야기의 방랑자들은 이야기꾼의 세계와 비교했을 때, 보다 단순화된 룰입니다. 특히 스탯의 경우 완전히 없앴고, 시트가 꽤 간소화되었죠. 또한, 주사위가 제거되고 그 랜덤성을 이야기꾼과 길잡이꾼의 심리전으로 대체하였습니다. \hyperlink{wandering-storytellers}{방랑하는 이야기꾼들} 챕터의 \hyperlink{walking-storytellers}{활보하는 이야기꾼} 단순화 규칙을 사용한다면, 걸어다니면서도 할 수 있을 정도로 간단한 룰이기도 합니다. 물론 엄청나게 단순화된 룰이기 때문에 지겨워지기도 쉬울 수 있으나, [방랑하는 이야기꾼들]으로 시작한 이야기꾼을 [이야기꾼의 세계]로 데려가는 것 역시 꽤 흥미로울 것이라 생각됩니다.
	
	다시 한번, 룰 읽느라 수고하셨습니다. 여러분에게 이야기의 가호가 있기를 바라겠습니다.
\end{document}
\end{document}
	
	\part{서플리먼트: 예시 이야기들}
		\documentclass{report}

\begin{document}
	현재 작업중입니다!%\parttoc
\end{document}
	
	\part{정리}
		\chapter*{FAQ}
			\documentclass{report}

\begin{document}
	내용에 대한 질문 사항이 있으시다면, 트위터 \href{https://www.twitter.com/n0n3x1573n7_WS}{@n0n3x1573n7\_WS}의 DM, 혹은 해당 계정에 연결되어있는 \href{https://ask.fm/n0n3x1573n7_WS}{ask.fm}으로 문의 주시면 확인하는대로 답변드리겠습니다. 질문 중 중요한 내용은 본문에 추가하고, 자주 나오는 질문을 취합하여 이곳에 추가할 예정입니다.
	
	\bigskip
	
	\begin{faq}{비공개 또는 유료화 계획이 있나요?}
		아직까지는 없습니다. 단, 추후 비공개 또는 유료화를 하게 된다면 사전 안내를 드릴 예정입니다. 특히 유료화의 경우 하게 되더라도 본 계정에서 질의응답은 계속 받을 예정입니다.
		
		다만, 만약 본 룰로 작성된 시나리오 또는 플레이 로그 중 다음과 같은 사항이 발생한다면 \href{https://twitter.com/n0n3x1573n7_WS}{본 계정}으로 해당 시나리오 또는 플레이 로그의 이름, 계정명 등을 언급하거나 해당 계정을 멘션하지 않고 사유와 함께 퍼블릭 트윗으로 경고를 드릴 예정입니다. 경고가 과도하게 누적된다면 본 룰을 즉시 비공개 처리할 예정입니다.
		\begin{itemize}
			\item 성적인 컨텐츠에 따른 나이 제한 표기 또는 트리거 워닝이 제대로 표기되지 않은 경우. 시나리오의 내용을 읽지 않고서도 알 수 있도록 배포 트윗 내지는 시나리오의 전체 공개 부분 또는 사전 고지 부분에 반드시 표기 되어야 합니다. 해당 트리거 워닝의 내용이 시나리오의 핵심 반전 등 스포일러성이라 할지라도 예외는 없습니다.
			\item 여성, 성소수자, 장애인 등 사회적 약자 또는 소수자에 대한 혐오 또는 차별을 조장하는 경우. 세계관이나 서사의 특성상 이가 묵인되는 경우라 할지라도 정도를 넘어설 경우 역시 포함됩니다. 특히 이를 조장하는 시나리오를 배포한 경우 \href{https://help.twitter.com/ko/rules-and-policies/hateful-conduct-policy}{트위터의 운영원칙} 등 배포된 플랫폼(들)의 운영원칙에 의해 신고할 예정입니다.
			\item 위 항목들에 해당하지는 않으나 위 항목들에 준하는 경우. 특히, 위 항목에 적혀있지 않다고 해서 허점을 찾았다고 생각하지는 않으시기를 부탁드립니다.
		\end{itemize}
		해당 사안들을 발견하셨을때 본 계정에 해당 내용에 대한 DM을 주신다면 감사하겠습니다.
		
		만약에 이 룰이 유료화 된다면 그 이후, 본 룰을 이용하여 게임을 즐기고 싶으시다면 룰북을 어떤 경로로든(온라인 pdf, 오프라인 책자 등) 구매하셔서 플레이를 즐겨주시기를 부탁드립니다.
	\end{faq}
	
	\begin{faq}{이야기꾼은 반드시 [깨달은 자]인가요?}
		이야기꾼이 반드시 [깨달은 자]일 필요는 없습니다. 예를 들어, 서사 vs. 등장인물의 흐름에서 이야기꾼은 [깨달은 자]가 아닌 등장인물로서 서사 속의 역경을 헤쳐나갑니다. 그 외에도 외부의 [깨달은 자]들이 이야기꾼의 세계에 들어와서 일어나는 일들을 이야기할수도 있습니다.
	\end{faq}

\end{document}
		
		\chapter*{패치 노트}
			\documentclass{report}

\begin{document}
	\subsection*{2019. 02. 06. 이야기꾼의 세계 세계관 및 룰 작업 시작}
	룰을 개발하고 검수함에 있어 처음부터 아이디어를 주고받는 등 도움을 주고 지지해준 케시님께 큰 감사와 사랑을 보냅니다.
	
	알파 버전과 베타 버전 룰의 테스트 플레이와 검수를 도와주신 대두님, 소낙님, 신호님, 엠케님, 철화구야선생님께 감사를 드립니다.
\iffullchangelog
	\section*{2019. 09. 03. 최초 배포본 프로토타입 완성}
	\subsection*{2019. 09. 04. Version 1.0}
	이야기꾼의 세계의 최초 배포본입니다.
	\begin{itemize}
		\item 최초 배포본
	\end{itemize}
	
	\subsection*{2019. 09. 04. Version 1.0.1}
	\begin{itemize}
		\item "행운"의 힘 부분을 전투에서 판정으로 이동
		\item 판정에 관련된 예시 추가
	\end{itemize}
	
	\subsection*{2019. 09. 05. Version 1.0.2}
	\begin{itemize}
		\item 오타 수정
		\item 판정에 관련된 내용 추가
	\end{itemize}
	
	\subsection*{2019. 09. 05. Version 1.0.3}
	\begin{itemize}
		\item 이야기꾼에게 지급되는 최대 개연성에 대한 설명 추가
		\item 룰 비공개에 대한 조건 추가
	\end{itemize}
	
	\subsection*{2019. 09. 10. Version 1.0.4}
	\begin{itemize}
		\item 오타 수정, 애매한 예시 및 내용 수정 및 목록화된 내용의 일관성 확보
		\item 예시 트라우마의 트리거 및 제약에 기피, 공포, 광기 여부 추가
		\item 재현이 실패하는 경우를 "불운"의 힘으로 분리
	\end{itemize}
	
	\subsection*{2019. 09. 12. Version 1.0.5}
	\begin{itemize}
		\item 이야기의 보상에 관한 내용을 능력 가이드라인 챕터에 추가
	\end{itemize}
	
	\subsection*{2019. 09. 15. Version 1.0.6}
	\begin{itemize}
		\item 이야기꾼의 세계에 대한 짧은 소개를 서문으로 이동
		\item 이야기꾼의 세계 룰의 목적에 대한 서술을 에필로그에 추가
	\end{itemize}
	
	\subsection*{2019. 09. 23. Version 1.0.7}
	\begin{itemize}
		\item 이야기의 효과가 동시에 적용될 때의 우선순위에 대한 내용 추가
	\end{itemize}


	\subsection*{2019. 10. 07. Version 1.1.0}
	이야기꾼의 세계를 확장하는 방법에 대해 작업했습니다.

	\begin{itemize}
		\item 시스템의 권능과 이야기꾼의 권능과 제약에 대한 자세한 설명을 추가했습니다.
		\item 이야기꾼의 세계를 확장시키는 방법과 이야기꾼의 세계를 확장 룰로 사용하는 방법을 추가했습니다.
		\item 이야기를 만들며 알아두어야 할 점에 대해 추가했습니다.
		\item 스탯에 대한 설명을 추가했습니다.
	\end{itemize}


	\subsection*{2019. 10. 24. Version 1.2.0}
	소스코드 파일 정리 형식을 수정하며 룰북과 시나리오를 합치고, 서플리먼트 작업을 시작했습니다.

	\begin{itemize}
		\item 파일 정리 형식을 수정했습니다.
		\item 시간의 박물관 시나리오를 한 pdf에 합했습니다.
		\item 이후 업데이트의 서플리먼트를 위한 자리를 마련해 두었습니다.
	\end{itemize}
	
	\subsection*{2019. 11. 04. Version 1.2.1}
	CC 라이선스를 추가했습니다.
	
	\subsection*{2019. 11. 12. Version 1.2.2}
	이야기 찾아보기를 추가했습니다.
	
	\subsection*{2019. 11. 13. Version 1.2.3}
	\begin{itemize}
		\item 성공의 다섯 단계에 각각 대성공-성공-통과-실패-대실패의 이름을 붙였습니다.
		\item 한 이야기에서 다른 이야기를 언급할 때, 링크를 걸어 이동하기 쉽도록 만들었습니다.
	\end{itemize}
	
	\subsection*{2019. 12. 10. Version 1.2.4}
	종 포괄성을 위해 ``인간"과 ``사람"이라는 단어를 모두 제거했습니다.
	
	\subsection*{2019. 12. 13. Version 1.2.5}
	세계관을 의미하는 이야기를 모두 서사로 변경했습니다.
	
	\subsection*{2019. 12. 27. Version 1.2.6}
	지속 피해형 능력의 코스트가 일반 피해형 능력의 코스트와 피해량은 같으나 코스트가 낮게 나오는 경우를 방지하기 위해 코스트를 변경했습니다.

	\subsection*{2019. 12. 31. Version 1.3.0}
	서플리먼트의 1차적 작업이 완료되었습니다. 클래식 종족과 클래스를 포함한 다양한 이야기가 서플리먼트에 추가되었습니다. 버전 1.2.0과 비교했을 때의 차이점은 여러 버그 수정, 룰적/룰북적 최적화와 더불어 ``서사"용어의 수정과 종 포괄성 확대를 위한 ``인간", ``사람" 용어 제거 등이 있습니다.
	
	\subsection*{2020. 01. 13. Version 1.3.1}
	범위형 효과를 가지는 능력에 대한 비용을 구체화하고, 이야기의 필요 조건에 따라 기본 비용을 세분화했습니다.
	
	\subsection*{2020. 01. 23. Version 1.4.0}
	시나리오집 작업을 위해 시나리오를 분리했습니다.
	
	\subsection*{2020. 01. 25. Version 1.4.1}
	사용 후 버려지는 아이템들의 이야기에 대한 서술을 추가했습니다.
	
	\subsection*{2020. 02. 11. Version 1.4.2}
	사거리에 관련된 코스트들과 연속 사용에 대한 개연성 비용을 추가했습니다.
	
	\subsection*{2020. 03. 09. Version 2.0.0}
	이야기의 방랑자들 룰을 추가했습니다.
	
	이야기 표의 디자인을 변경했습니다.
	
	\subsection*{2020. 03. 10. Version 2.0.1}
	표의 세로선이 표시되지 않는 문제에 대한 패치를 진행했습니다.
	
	\subsection*{2020. 03. 11. Version 2.0.2}
	시트를 이미지로 룰북에 추가했고, 구글 스프레드시트 사용에 대한 설명을 추가했습니다.
	
	\subsection*{2020. 03. 15. Version 2.0.3}
	판정 재굴림과 본인에게 가하는 피해가 있는 능력에 대한 개연성 비용을 추가했습니다.
	
	\subsection*{2020. 03. 30. Version 2.0.4}
	이야기의 방랑자들에서, 이야기의 보상으로 주어지는 (특히 긍정적인) 이야기에 대한 추가적인 방안을 제시했습니다.
	
	\subsection*{2020. 03. 31. Version 2.0.5}
	이야기의 방랑자들에서, 룰북의 오타를 발견해 수정했습니다.
	
	\subsection*{2020. 04. 01. Version 2.0.6}
	이야기 코드를 바꾸었습니다. 가독성은 거의 유사하겠지만, 일부 시각적 버그가 수정되었을 것입니다.
	
	\subsection*{2020. 08. 05. Version 2.0.7}
	트라우마의 심각도를 결정하는 방법에 대한 가이드라인을 추가했습니다.
\fi
	\subsection*{2020. ??. ??. Version 3.0.0}
	Wrights of the Plots 룰을 추가하고, WoT의 내용을 일부 편집했으며, 라이센스를 추가하고, 출력 작업을 위한 작업을 진행했습니다.
\end{document}
		
			\printindex
	
	\vspace*{\fill}
	{\doclicenseThis}
	
\end{document}